%!TEX root = main.tex

\documentclass{sig-alternate-05-2015}
% \usepackage{paralist}
% Load basic packages
\usepackage{balance}  % to better equalize the last page
\usepackage{graphicx} % for EPS, load graphicx instead
\usepackage{url}      % llt: nicely formatted URLs
\usepackage{amsmath}
\usepackage{color}
\usepackage{cancel}
\usepackage{listings}
%\usepackage{wrapfig}
\usepackage{graphicx}
\usepackage{subcaption}
\setlength{\abovecaptionskip}{10pt plus 3pt minus 2pt}
% \usepackage[english]{babel}
\usepackage{booktabs}
\usepackage{graphicx}
% \usepackage{subfigure}
% \usepackage{caption}
% \usepackage[scriptsize, it, IT]{subfigure}
% \usepackage[font={scriptsize,it}]{caption}
% \usepackage{tikz}
%\usepackage{MinionPro}
% \usetikzlibrary{arrows,matrix,positioning}
\usepackage[ruled,vlined,algonl,boxed]{algorithm2e}
\usepackage{algorithmic}
\usepackage{wrapfig}
% \usepackage{framed}
\usepackage{enumitem}
\usepackage{xspace}
\usepackage{xcolor}
\usepackage{colortbl}
% \usepackage{newtxmath}
\usepackage[T1]{fontenc}

\usepackage[nocompress]{cite}
\usepackage{microtype}

% llt: Define a global style for URLs, rather that the default one
\makeatletter
\def\url@leostyle{%
  \@ifundefined{selectfont}{\def\UrlFont{\sf}}{\def\UrlFont{\small\bf\ttfamily}}}
\makeatother
\urlstyle{leo}

\makeatletter
%\def\@copyrightspace{\relax}
\makeatother

% To make various LaTeX processors do the right thing with page size.
\def\pprw{8.5in}
\def\pprh{11in}
\special{papersize=\pprw,\pprh}
\setlength{\paperwidth}{\pprw}
\setlength{\paperheight}{\pprh}
\setlength{\pdfpagewidth}{\pprw}
\setlength{\pdfpageheight}{\pprh}

\newtheorem{definition}{Definition}%[section]
\newtheorem{proposition}[definition]{Proposition}
\newtheorem{lemma}[definition]{Lemma}
\newtheorem{remark}[definition]{Remark}
\newtheorem{corollary}[definition]{Corollary}
\newtheorem{claim}[definition]{Claim}
\newtheorem{theorem}[definition]{Theorem}
\newtheorem{heuristic}[definition]{Heuristic}
\newtheorem{example}[definition]{Example}
%\newtheorem{proof}[definition]{Proof}
\newtheorem{dimension}{Dimension}
\newcounter{prob}
\newtheorem{problem}[prob]{Problem}
\newtheorem{conjecture}[definition]{Conjecture}
\newtheorem{reduction}[definition]{Reduction}
\newtheorem{property}[definition]{Property}
\newtheorem{axiom}[definition]{Axiom}
% 
% \tikzset{
%     %Define standard arrow tip
%     >=stealth',
%     %Define style for boxes
%     punkt/.style={
%            rectangle,
%            rounded corners,
%            draw=black, very thick,
%            text width=6.5em,
%            minimum height=2em,
%            text centered},
%     % Define arrow style
%     pil/.style={
%            ->,
%            thick,
%            shorten <=2pt,
%            shorten >=2pt,}
% }


% \setitemize{noitemsep,topsep=0pt,parsep=0pt,partopsep=0pt}

% \def\compactify{\itemsep=0pt \topsep=0pt \partopsep=0pt \parsep=0pt}
% \let\latexusecounter=\usecounter
% \newenvironment{CompactItemize}
%   {\def\usecounter{\latexusecounter}
%    \begin{itemize}[noitemsep,topsep=0pt,parsep=0pt,partopsep=0pt,leftmargin=*]}
%   {\end{itemize}\let\usecounter=\latexusecounter}
% \newenvironment{CompactEnumerate}
%   {\def\usecounter{\compactify\latexusecounter}
%    \begin{enumerate}[leftmargin=*]}
%   {\end{enumerate}\let\usecounter=\latexusecounter}


%%%  What is this?  Use enumitem instead
% 
% \newcommand{\squishlist}{
%    \begin{list}{$\bullet$}
%     { \setlength{\itemsep}{0pt}
%       \setlength{\parsep}{2pt}
%       \setlength{\topsep}{6pt}
%       \setlength{\partopsep}{0pt}
%       \leftmargin=25pt
% \rightmargin=0pt
% \labelsep=5pt
% \labelwidth=10pt
% \itemindent=0pt
% \listparindent=0pt
% \itemsep=\parsep
%     }
% }
% \newcommand{\squishend}{\end{list}}
% 
% 
% \newcommand{\squishframe}{\vspace{-6pt}
% \begin{framed} 
% \vspace{-6pt}}
% 
% \newcommand{\frameend}{\vspace{-6pt}
% \end{framed}
% \vspace{-6pt}}


% create a shortcut to typeset table headings
% \newcommand\tabhead[1]{\small\textbf{#1}}




% Make sure hyperref comes last of your loaded packages, 
% to give it a fighting chance of not being over-written, 
% since its job is to redefine many LaTeX commands.
\usepackage[pdftex]{hyperref}
\hypersetup{
  colorlinks=false,
  linkcolor=darkred,
  citecolor=darkgreen,
  urlcolor=darkblue
}
% \usepackage{cleveref} % After hyperref, listings



% Avoid widows and orphans
\widowpenalty=500
\clubpenalty=500

% % Aggressive figure placement
% \renewcommand{\topfraction}{0.9}
% \renewcommand{\bottomfraction}{0.8}
% \setcounter{topnumber}{2}
% \setcounter{bottomnumber}{2}
% \setcounter{totalnumber}{4}
% \setcounter{dbltopnumber}{2}
% \renewcommand{\dbltopfraction}{0.9}
% \renewcommand{\textfraction}{0.07}
% \renewcommand{\floatpagefraction}{0.7}
% \renewcommand{\dblfloatpagefraction}{0.7}

\definecolor{light-gray}{gray}{0.95}
\definecolor{mid-gray}{gray}{0.85}
\definecolor{darkred}{rgb}{0.7,0.25,0.25}
\definecolor{darkgreen}{rgb}{0.15,0.55,0.15}
\definecolor{darkblue}{rgb}{0.1,0.1,0.5}
\definecolor{blue}{rgb}{0.19,0.58,1}

\newcommand{\red}[1]{\textcolor{red}{#1}}
\newcommand{\green}[1]{\textcolor{green}{#1}}
\newcommand{\blue}[1]{\textcolor{blue}{#1}}
\newcommand{\orange}[1]{\textcolor{orange}{#1}}
\newcommand{\darkred}[1]{\textcolor{darkred}{#1}}
\newcommand{\darkgreen}[1]{\textcolor{darkgreen}{#1}}
\newcommand{\darkblue}[1]{\textcolor{darkblue}{#1}}

\makeatletter
\setlength{\@fptop}{0pt}
\makeatother


\usepackage{url}
\def\UrlBreaks{\do\/\do-}

\newcommand{\papertext}[1]{#1}
\newcommand{\techreport}[1]{#1}

\definecolor{commentColor}{HTML}{0000FF}
\newcommand{\newtext}[1]{{\color{commentColor}{#1}}}

\newcommand{\xxx}[1]{{\fontsize{13pt}{13pt}\selectfont\textcolor{red}{#1}}}
\newcommand{\codesize}{\fontsize{7}{8}}
\newcommand{\stitle}[1]{\vspace{0.5em}\noindent\textbf{#1}}
\newcommand{\calF}[0]{$\cal{F}$}

\newcommand{\ind}{\hspace{\algorithmicindent}}

\newcommand{\deprecate}[1]{\noindent{\color{light-gray}{#1}}}

\newcommand{\prob}{{\sc Log-Corruption}\xspace}
\newcommand{\exact}{{\sc EXACTSOL}\xspace}
\newcommand{\qfix}{{\sc SingleQueryFix}\xspace}
\newcommand{\density}{{\sc DENSITY}\xspace}


\newcommand{\milpall}{\textsc{MILP-NAIVE}\xspace}
\newcommand{\milptuple}{\textsc{MILP-COMPL}\xspace}
\newcommand{\milptuplestopearly}{\textsc{MILP-COMPL-STOPEARLY}\xspace}
\newcommand{\milpadvtuple}{\textsc{MILP-ADV-TUPLE}\xspace}
\newcommand{\milpadvall}{\textsc{MILP-ADV-ALL}\xspace}
\newcommand{\heurstic}{\textsc{HEURISTIC}\xspace}


\makeatletter
\def\maketag@@@#1{\hbox{\m@th\normalfont\normalsize#1}}
\DeclareRobustCommand*\textsubscript[1]{
          \@textsubscript{\selectfont#1}}
        \def\@textsubscript#1{
          {\m@th\ensuremath{_{\mbox{\fontsize\sf@size\z@#1}}}}}
\makeatother

\newcommand{\sysname}{\textsc{QueryFix}}
\newcommand{\sys}{QFix\xspace}
\newcommand{\naive}{\emph{basic}\xspace}
\newcommand{\tslice}{\sys-{\it tuple}\xspace}
\newcommand{\qslice}{\sys-{\it query}\xspace}
\newcommand{\aslice}{\sys-{\it attr}\xspace}
\newcommand{\incremental}{\sys-{\it inc}\xspace}
\newcommand{\dt}{DecTree\xspace}
\newcommand*{\Scale}[2][4]{\scalebox{#1}{$#2$}}%

\setlength\floatsep{0.8\baselineskip plus 3pt minus 2pt}
% \setlength\textfloatsep{0.9\baselineskip plus 3pt minus 2pt}
\setlength\intextsep{1\baselineskip plus 3pt minus 2 pt}

\begin{document}

\CopyrightYear{2017} 
\setcopyright{acmlicensed}
\conferenceinfo{SIGMOD'17,}{May 14 - 19, 2017, Chicago, Illinois, USA}
\isbn{978-1-4503-4197-4/17/05}\acmPrice{\$15.00}
\doi{http://dx.doi.org/10.1145/3035918.3035925}
%Authors, replace the red X's with your assigned DOI string.

\clubpenalty=10000 
\widowpenalty = 10000
%Authors, replace the red X's with your assigned DOI string.

% for 
\title{{\sys}: Diagnosing Errors through Query Histories}

\numberofauthors{3}
\author{
  \alignauthor Xiaolan Wang\\
    \affaddr{School of Computer Science}\\
    \affaddr{University of Massachusetts}\\
    \email{xlwang@cs.umass.edu}\\
  \alignauthor Alexandra Meliou\\
  \affaddr{School of Computer Science}\\
    \affaddr{University of Massachusetts}\\
    \email{ameli@cs.umass.edu}\\
  \alignauthor Eugene Wu\\
    \affaddr{Computer Science}\\
    \affaddr{Columbia University}\\
    \email{ewu@cs.columbia.edu}\\
}




{
\maketitle
}

\begin{abstract}
    \looseness -1
Data-driven applications rely on the correctness of their data to
function properly and effectively. Errors in data can be incredibly
costly and disruptive, leading to loss of revenue, incorrect
conclusions, and misguided policy decisions. While data cleaning tools
can purge datasets of many errors before the data is used,
applications and users interacting with the data can introduce new
errors. Subsequent valid updates can obscure these errors and
propagate them through the dataset causing more discrepancies. Even
when some of these discrepancies are discovered, they are often
corrected superficially, on a case-by-case basis, further obscuring
the true underlying cause, and making detection of the remaining
errors harder.

In this paper, we propose \sys, a framework that derives explanations
and repairs for discrepancies in relational data, by analyzing the
effect of queries that operated on the data and identifying potential
mistakes in those queries. \sys is flexible, handling scenarios where
only a subset of the true discrepancies is known, and robust to
different types of update workloads. We make four important
contributions: (a) we formalize the problem of diagnosing the causes
of data errors based on the queries that operated on and introduced
errors to a dataset; (b)
we develop exact methods for deriving diagnoses and fixes for
identified errors using state-of-the-art tools; (c) we present several
optimization techniques that improve our basic approach without
compromising accuracy, and (d) we leverage a tradeoff between accuracy
and performance to scale diagnosis to large datasets and query logs,
while achieving near-optimal results. We demonstrate the effectiveness
of \sys through extensive evaluation over benchmark and synthetic
data.

\end{abstract}

\everypar{\looseness=-1}
%!TEX root=main.tex


%!TEX root = ../main.tex

\section{Introduction}
\label{s:intro}

Poor data quality is estimated to cost the US economy more than \$600 billion
per year~\cite{eckerson2002} and erroneous price data in retail databases
alone cost the US consumers \$2.5 billion each year~\cite{Fan2008}. 
\ewu{
It is possible to use tools to fix existing errors in the database.
However, databases are dynamic -- applications and users constantly generate and execute queries that modify the database.
Any errors in the queries can easily propagate in complex ways and spread errors throughout the database that are hard to reason about.
Fixing the errors that have been found in the current data is not enough because not all of the errors may have been found because the source of the error could be buried deep in the past.
No tools exist to both {\it find} the erroneous queries and provide {\it fixes/explanations}.
}
While data cleaning tools can purge datasets of many errors before the data is
used, applications and users interacting with the data can introduce new
errors. Mistakes in data entry and erroneous updates often affect datasets in
complex ways and result in data errors that are obscure and hard to trace and
correct. Traditional data cleaning approaches are not well-suited for this
purpose: While they provide general-purpose tools to identify and rectify
anomalies in the data, they are not designed to diagnose the causes of these
errors, which are rooted in erroneous updates.

Improving data quality and correcting data errors has been an important focus
for data management research. Yet, handling new errors, introduced during
regular database interactions, has received little attention. Integrity
constraints~\cite{Khoussainova2006} can guard against some improper updates,
but only if the data falls outside rigid, predefined ranges. Certificate-based
verification~\cite{Chen2011} is impractical as it requires users to answer
challenge questions before allowing the updates, and it is not applicable to
updates initiated by applications.


\ewu{
In this paper, we argue for a new type of data error diagnosis called Data Fracking 
that targets data modification queries deep in the database transaction log.
In contrast to existing data cleaning and explanation approaches that aim to
detect, fix, or explain errors in the current database, Data Fracking both identifies the
root causes of the errors as past query transactions that {\it introduced} the errors
into the database, and proposes modifications to those anomalous queries that
fix the current errors.  By introducing a {\it new dimension} to data error diagnosis,
data fracking can lead to the identification of additional discrepancies that would
have otherwise remained undetected.
}
In this paper, we present a specialized data cleaning framework, \sys, that
specifically targets errors in update workloads. In contrast to traditional
data cleaning techniques that aim to identify errors in the data directly, the
goal of \sys is to \emph{explain} how the errors occurred. Our goal is not
simply to provide explanations; identifying how an error was introduced to a
dataset, can lead to the identification of additional discrepancies that would
have otherwise remained undetected.

\begin{example}[Incorrect insurance premiums]
    After negotiations with health insurance companies, an employer achieved
    reductions in the premium rates for the upcoming year, for employees at
    levels 1-4. The administrator who updated the employee database mistakenly
    implemented the new policy for employees at levels 1-3, due to the
    incorrect predicate \texttt{level < 4}.
    
    In subsequent months, level 4 employees who noticed that their premiums
    had not been reduced as expected, notified HR of the error. HR corrected
    employees' records on a case-by-case basis, obscuring the real cause of
    the problem, and making it harder to identify and correct the error for
    other employees.
    
    To make matters worse, subsequent queries that calculated employee
    withholdings, resulted in misestimation of the company's revenue and
    incorrect allocation of bonuses. Correcting errors in the employee
    premiums is no longer sufficient, as these errors have already had a
    larger effect on the data.
    
        \alex{Needs to be more realistic and convincing.  Ideally grounded on a real scenario.  Do we have more specific info for the use-cases?}
\end{example}

\ewu{In reality, the problems presented to HR were simply a small sample
of the errors stemming from the anomalous query.  Once HR fixes the errors,
there are still a potentially large number of employees whose premiums are incorrect that have
simply not complained to HR yet.
it is thus desirable to, given HR's fixes, identify candidate queries that may have been the source
of the error, propose fixes, and subsequently identify the other employees whos premiums are in error.

This problem is fundamentally challenging because ??? and must solve several challenging subproblems.
1) rollback  2) fix generation  3) fix picking (if there are multiple possible fixes) 4) performance.
}



% these characterize the problem
\ewu{Would say these nicely characterize the problem, not necessarily challenges.}
Diagnosing data errors stemming from incorrect updates raises three major
challenges that make existing data cleaning techniques not applicable.



\begin{description}[leftmargin=*, topsep=0mm, itemsep=0mm]
    
    \item[Obscurity.] Detection of the resulting errors in the data often
    leads to partial fixes that further complicate the eventual diagnosis and
    resolution of the problem. For example, a transaction implementing a
    change in the state tax law updated tax rates using the wrong rate,
    affecting a large number of consumers. This causes a large number of
    complaints to a call center, but each customer agent usually fixes each
    problem individually, which ends up obscuring the source of the problem.
    
    \item[Large impact.] Erroneous queries cause errors in a large scale. The
    potential impact of the errors is high, as manifested in several
    real-world cases~\cite{Yates10, Grady13, sakalerrors}. Further, errors
    that remain undetected for a significant amount of time can instigate
    additional errors, even through valid updates. This increases both their
    impact, and their obscurity.
    
    \item[Systemic errors.] The errors created by bad queries are
    \emph{systemic}: they have common characteristics, as they share the same
    cause. The link between the resulting data errors is the query that
    created them; cleaning techniques should leverage this connection to
    diagnose and fix the problem. Diagnosing the cause of the errors, will
    achieve systematic fixes that will correct all relevant errors, even if
    they have not been explicitly identified.
    
\end{description}
% 
\sys addresses these challenges by analyzing the queries that operated on a
dataset in an efficient and scalable manner. More concretely, we make the
following contributions:


% The goal of this paper is to design effective query
% diagnosis techniques and identify possible fixes for query errors. We
% model the problem assuming a log of update workloads over a database,
% and a set of complaints that identify errors in the final database
% state. We organize our contributions as follows:

\ewu{really like this organization}.

\begin{itemize}[leftmargin=*, topsep=0mm, itemsep=0mm]      
    \item We formalize the Data Fracking problem of diagnosing a set of errors using log
    histories of updates that operated on the data. Given a set of 
    \emph{complaints} as representations of data discrepancies in the current
    state of a dataset, \sys identifies the queries in the log that require the  minimal
    amount of changes that would resolve all of the complaints (Section~\ref{???}).
      
    \item We provide an exact error-diagnosis solution through a non-trivial
    transformation of the problem to a mixed integer program (MIP) that
    analyzes the data provenance of the erroneous tuples at the attribute
    level. Our approach employs state-of-the-art MIP solvers to identify
    optimal diagnoses that are guaranteed to resolve all complaints, and is
    tolerant to incomplete information (missing complaints)
    (Section~\ref{???}).
    
    \item While modern solvers can handle large numbers of variables and
    constraints, eventually, they fail to scale to very large datasets and
    large query logs. We present several performance optimizations that allow
    our diagnostic methods to scale, in many cases without affecting the
    quality of the produced solutions (Section~\ref{???}).
    
    \item We extend our framework to also handle false positives: complaints
    that mistakenly identify data as erroneous. We define the notion of
    complaint \emph{density}, which is a query-driven measure of closeness of
    a complaint to other complaints. The main intuition of our approach is
    that complaints of low density are likely false positives and thus can be
    safely ignored (Section~\ref{???}).
    
    \item We experimentally evaluate the effectiveness and efficiency of our
    methods against real-world and synthetic datasets and query logs. We
    demonstrate that \xxx{... to be completed when we know what we have}
    (Section~\ref{???}).
\end{itemize}


\deprecate{
While exploring data, its natural to come across surprising or unexpected data.
For example, visual data analysis explores the current state of the database and users may be surprised by outliers in a visualization.
Similarly, enterprise customers (e.g., billing) may find outliers in their monthly bills and be surprised by the amount they are asked to pay.

When presented with these surprises, users want to better understand the reasons behind the anomalies.
A recent wave of research focuses on deriving predicate-based explanations for outliers for statistical aggregation queries.
For example, if the user wants to understand why the total sales in the past few months have gone up, these systems can general explanations such as ``most related to customers in California between the ages of 12 to 18.''
However, these approaches simply generate predicates that describe {\it current state} of the database, and do not resolve {\it how} the anomalous data came to be.

Specifically, the user may also be interested to understand which past database modification was responsible for these explanations.
Describe why this makes sense to want.  In this form of the problem, we are interested in historical database queries whose modifications, when propogated to the current database state, 
The goal is to provide diagnostic tools that can peer into past transactions.


In this paper, we approach anomaly explanation from the persepctive of the query log and seek to
both {\it identify}  historical database modification queries that most likely caused user complaints 
in the current state of the database, and suggest replacement queries that will resolve these complaints.
We call this problem the {\it Query-based Complaint-Satisfaction Problem}.

Given a database query log and a set of {\it complaints} (e.g., tuple 1's attribute B should be 20\% lower) about records in the current state of the database,
we seek to identify the subset of queries in the log that, by modifying their parameters and propogating the new effects of the queries, 
will best resolve the complaints.  

One way to solve this problem is to try modifying the most recent query until it fixes the complaints.  
If not, then try the second most recent query.  
The problem with this approach is the number of possible modifications is unbounded.

Our contributions include

\begin{enumerate}
\item Developing and formalizing the problem of Query-oriented explanation in contrast to data-oriented explanation
\item Prove that the general problem is impossible.
\item Designing alogirthems to solve the problem for complete complaint sets
\item Extending the algorithms to support incomplete complaint sets
\item Extending to support multiple queries
\end{enumerate}

%%%%%%%%%%%%%%% Clean up.... %%%%%%%%%%%%%%

}


\section{{\Large\textbf{\sys}}: architecture}


\begin{figure}[t]
    \centering
        \includegraphics[scale=0.35]{figures/architecture}
    \caption{\sys processes data anomalies in the form of complaints and analyzes logged query histories to identify the causes of error. In the heart of the system, transformation algorithms express the diagnosis problem as a mixed integer program, and optimization modules ensure that the MIP programs can be evaluated efficiently.}
    \label{fig:architecture}
\end{figure}


\alex{Do we want to actually describe the architecture in a small section, or should we just include the figure in the front page?}



% %!TEX root = ../main.tex

\section{Real-world examples}
\label{s:usecase}

\alex{I am not convinced that these deserve a section.  It would make sense if we actually applied our system on them and had something to say about the derived diagnoses and fixes.  As they are, I would use as examples in the introduction instead.}

\subsection{Telco}

Call centers get customer complaints about billing all the time.
For example, there was a change in the tax law, but the transactions
that updated the taxes didn't change, so they would get lots of
customer complaints.  Each customer agent usually fixes each problem
individually, however, it's a system problem to all of Kansas that
usually doesn't get caught until late in the game.  In this case
it's a faulty update to a numerical value whose effects compound
over time.  There is external domain knowledge to isolate the
culprits to tax-related transactions, and sensitivity analysis can
be useful.

\subsection{Insurance}

A colleague recently switched to medicare but because of a problem
with the transactions to switch him from his previous to new
insurance, it's caused him 6 months of headaches.  3 months ago he
learned that a bunch of other people have been complaining about
the same issues regarding the switch.  In this case it's a categorical
change (from one insurance to another) that has more obvious impact.



%!TEX root = ../main.tex

\section{Modeling abstractions}
\label{sec:abstractions}

In this section, we introduce a running example inspired from the use-case of
Example~\ref{ex:taxes}, and describe the model abstractions that we use to
formalize the diagnosis problem.


% \ewu{Add text to say false positive complaints are an orthogonal problem.}

%!TEX root = ../main.tex
\begin{example}[Salary Update Error]\label{ex:telco}
  A manager updates the tax amount with $30\%$ tax rate for high income employees that earn more than $\$87500$.
  She submits this through a form in the salary accounting application, 
  but due to keyboard slip, incorrectly types $\$85700$ for the income threshold.  
  Later queries that insert new paychecks, compute tax calculations,
  aggregate department salaries end up propogating this error throughout other records in the database, leading to
  employee dissatifaction.  Figure~\ref{fig:example} illustrates a concrete example where $Q_1$, $Q_2$, and $Q_3$ are executed on an 
  initial salary database $D_0$.  The error in $Q_1$ that incorrectly sets some of the tax rates is propogated to other fields in the table.
\end{example}

\begin{figure}[t]
    \centering
        \includegraphics[width=0.45\textwidth]{figures/example2}
    \caption{$Q_1$ updates the tax amount with $30\%$ tax rate 
      for high income employees using an incorrect predicate.  
      The error is propogated by $Q_2$ to the $pay$ field in the database.
      Finally, a benign insert query $Q_3$ inserts correct salary information. 
      The final database state contains a mixture of incorrect and correct salary data.
    }
    \label{fig:example}
\end{figure}


In this example, by the time errors in the database have been detected, 
perhaps by employees that report their encorrect paystubs, it is difficult 
to both identify all of the other errors in the database, and to trace these errors back to the erroneous query to correct it.
This example can occur in any data processing systems where manual input is used to generate queries that modify the database --
this could be in the form of adhoc queries executed by a system administrator, or web-based forms that construct queries based
on user input, or even stored procedures that use human input to fill the parameter values.


\begin{example}\label{ex:taxes2}
Figure~\ref{fig:example} demonstrates an example tax bracket adjustment in the
spirit of Example~\ref{ex:taxes}. The adjustment sets the tax rate to 30\% for
income levels above \$87,500, and is implemented by query $q_1$. A digit
transposition mistake in the query, results in an incorrect owed amount for tuples
$t_3$ and $t_4$. Query $q_2$, which inserts a tuple with slightly higher
income than $t_3$ and $t_4$ and the correct (lower) tax rate, obscures this mistake.
This mistake is further propagated by query $q_3$, which calculates the pay 
check amount based on the corresponding
tax rate and income. 
\iffalse    
Figure~\ref{fig:example} demonstrates an example tax bracket adjustment in the
spirit of Example~\ref{ex:taxes}. The adjustment sets the tax rate to 30\% for
income levels above \$87,500, and is implemented by query $q_1$. A digit
transposition mistake in the query, results in an incorrect tax rate for tuples
$t_3$ and $t_4$. Query $q_2$ that calculates the amount owed based on the corresponding
tax rate and income propagates the error to other fields. The mistake is
further obscured by query $q_3$, which inserts a tuple with slightly higher
income than $t_3$ and $t_4$ and the correct (lower) tax rate.
\fi
\end{example}
\vspace*{-0.07in}
While traditional data cleaning techniques seek to identify and correct the
erroneous values in the table \emph{Taxes} directly, our goal is to diagnose
the problem, and understand the reasons for these errors. In this case, the
reason for the data errors is the incorrect predicate value in query $q_1$.

In this paper, we assume that we know \emph{some} errors in the dataset, and
that these errors were caused by erroneous updates. The errors may be
obtained in different ways: traditional data cleaning tools may identify
discrepancies in the data (e.g., a tuple with lower income has higher tax
rate), or errors can be reported directly from users (e.g., customers
reporting discrepancies to customer service). \emph{Our goal is not to correct
the errors directly in the data, but to analyze them as a ``symptom'' and provide a
diagnosis.} The diagnosis can produce a targeted treatment: knowing how the
errors were introduced guides the proper way to trace and resolve them.



\begin{figure}[t]
\centering
{\small
\begin{tabular}{ll}
    \toprule
    \textbf{Notation} & \textbf{Description}\\
    \midrule
    $\mathcal{Q}$& The sequence of executed update queries (log)\\ 
             & $\mathcal{Q}=\{q_1, \dots, q_n\}$ \\
    $D_0$    & Initial database state at beginning of log\\
    $D_n$    & End database state (current) $D_n=\mathcal{Q}(D_0)$\\
    $D_i$    & Database state after query $q_i$: $D_i=q_i(\dots q_1(D_0))$\\
    $c: t\mapsto t^*$ & Complaint: $\mathcal{T}_c(D) = (D_n\setminus\{t\})\cup\{t^*\}$\\
    $\mathcal{C}$ & Complaint set $\mathcal{C}=\{c_1,\dots,c_k\}$\\
    $\mu_q(t)$  & Modifier function of $q$ (e.g., \texttt{SET} clause)\\
    $\sigma_q(t)$   & Conditional function of $q$ (e.g., \texttt{WHERE} clause)\\
    $t_{new}$   & Tuple values introduced in an \texttt{INSERT} query\\
    $\mathcal{Q}^*$& Log repair\\
    $d(\mathcal{Q}, \mathcal{Q}^*)$ & Distance functions between two query logs\\
    \bottomrule
\end{tabular}
}
\vspace{-2mm}
\caption{Summary of notations used in the paper. }
\label{tbl:notation}
\end{figure}

\subsection{Error modeling}
\label{sec:model}

In our setting, the diagnoses are associated with errors in the queries that
operated on the data. In Example~\ref{ex:taxes2}, the errors in the dataset
are due to the digit transposition mistake in the WHERE clause predicate of
query $q_1$. Our goal is to infer the errors in a log of queries
automatically, given a set of incorrect values in the data. We proceed to
describe our modeling abstractions for data, queries, and errors, and how we
use them to define the diagnosis problem.

\subsubsection*{Data and query models}
\label{sec:models}

\noindent
\emph{Query log ($\mathcal{Q}$):}
We define a query log that update the database 
as an ordered sequence of \texttt{UPDATE}, \texttt{INSERT}, and
\texttt{DELETE} queries $\mathcal{Q}=\{q_1,\dots,q_n\}$, that have
operated on a database $D$. In the rest of the paper, we use the term
\emph{update queries}, or just \emph{queries}, to refer to any of the queries in $\mathcal(Q)$,
including insertion and deletion queries.

\smallskip
\noindent
\emph{Query ($q_i$):} We model each query as a function over a
database $D$, resulting in a new database $D'$. For \texttt{INSERT}
queries, $D'=q(D)=D\cup\{t_{new}\}$.
We model \texttt{UPDATE} and \texttt{DELETE} queries as follows:  
\begin{align*}
    D'=q(D)= &\{\mu_{q}(t)\;|\;t\in D, \sigma_{q}(t)\}%\\
    % &
    \cup\{t\;|\;t\in D, \neg\sigma_{q}(t)\}%\\
    % &\cup\{t_{new}\;|\;q\in\texttt{INSERT}\}
\end{align*}
% 
In this definition, the modifier function $\mu_q(t)$ represents the query's update equations, and it transforms a tuple by either deleting it ($\mu_q(t)=\bot$) or changing the values of some of its attributes.
The conditional function $\sigma_q(t)$ is a boolean function that represents the query's condition predicates.  In the example of Figure~\ref{fig:example}:
\begin{align*}
    &\mu_{q_1}(t)=(t.income, t.income*0.3, t.pay)\\
    &\sigma_{q_1}(t)=(t.income\ge 85700)\\
    &\mu_{q_3}(t)=(t.income, t.owed, t.income-t.owed)\\
    &\sigma_{q_2}(t)=\texttt{true}
\end{align*} 
\iffalse
\begin{align*}
    &\mu_{q_1}(t)=(30, t.income, t.owed)\\
    &\sigma_{q_1}(t)=(t.income\ge 85700)\\
    &\mu_{q_2}(t)=(t.rate, t.income, \frac{t.income\cdot t.rate}{100})\\
    &\sigma_{q_2}(t)=\texttt{true}
\end{align*} 
\fi
Note that in this paper, we only consider query without sub-query or aggregation. 
% 
% As an insertion query, $q_3$ has $\sigma_{q_3}(t)=\texttt{false}$ and $t_{new}=(25, 85800, 21450)$.
% \ewu{why does it return false?  should it only be true if input is $\bot$?}


\smallskip
\noindent
\emph{Database state ($D_i$):}
We use $D_i$ to represent the state of a database $D$ after the application of
queries $q_1$ through $q_i$ from the log $\mathcal{Q}$. $D_0$ represents the
original database state, and $D_n$ the final, or current, database state. Out
of all the states, the system only maintains $D_0$ and $D_n$. In practice,
$D_0$ can be a checkpoint: a state of the database that we assume is correct;
we cannot diagnose errors before this state. The intermediate states can be
derived by executing the log: $D_i=q_i(q_{i-1}(\dots q_1(D_0)))$. We also
write $D_n=\mathcal{Q}(D_0)$ to denote that the final database state $D_n$ can
be derived by applying the sequence of queries in the log to the original
database state $D_0$.

\smallskip
\noindent
\emph{True database state ($D_i^*$):}
Queries in $\mathcal{Q}$ are possibly erroneous, introducing errors in the
data. There exists a sequence of \emph{true} database states $\{D_0^*,
D_1^*\dots, D_n^*\}$, with $D_0^*=D_0$, representing the database states that
would have occurred if there had been no errors in the queries.
The true database states are unknown; our goal is to find and correct the errors in $\mathcal{Q}$ and retrieve the correct database state $D_n^*$.

For ease of exposition, in the remainder of the paper we assume that the
database contains a single relation with attributes $A_i,\ldots,A_m$,
but the single table is not a requirement in our framework.


\subsubsection*{Error models}

Following the terminology in Examples~\ref{ex:telco}
and~\ref{ex:taxes}, we model a set of identified or user-reported
data errors as \emph{complaints}. A complaint corresponds to a
particular tuple in the final database state $D_n^*$, and identifies
that tuple's correct value assignment. We formally define complaints
below:

\begin{definition}[Complaint]
    A complaint $c$ is a mapping between two tuples: $c: t\mapsto t^*$, such that $t$ and $t^*$ have the same schema, $t\in D_n\cup\{\bot\}$, and $t\neq t^*$. A complaint defines a
    transformation $\mathcal{T}_c$ on a database state $D$: $\mathcal{T}_c(D)
    = (D\setminus\{t\})\cup\{t^*\}$.
\end{definition}

In the example of Figure~\ref{fig:example}, two complaints are reported on the final database state $D_3$: 
$c_1: t_3\mapsto t_3^*$ and
$c_2: t_4\mapsto t_4^*$, 
where $t_3^*=(86000,21500,64500)$ and $t_4^*=(86500,21625,64875)$. 
%where $t_3^*=(25,86000,21500)$ and $t_4^*=(25,86500,21625)$.  
For both these cases, each complaint denotes a \textbf{value correction} for a tuple in $D_3$.  Complaints can also model the \textbf{addition} or \textbf{removal} of tuples: $c: \bot\mapsto t^*$ means that $t^*$ should be added to the database, whereas $c: t\mapsto \bot$
means that $t$ should be removed from the database.


\smallskip
\noindent
\emph{Complaint set ($\mathcal{C}$):}
We use $\mathcal{C}$ to denote the set of all known complaints
$\mathcal{C}=\{c_1,\dots,c_k\}$, and we call it the \emph{complaint set}.
Each complaint in $\mathcal{C}$ represents a transformation (addition,
deletion, or modification) of a tuple in $D_n$. We assume that the
complaint set is consistent, i.e., there are no two complaints that
propose different transformations to the same tuple $t\in D_n$.
Applying all these transformations to $D_n$ results in a new database
instance
$D_n'=\mathcal{T}_{c_1}(\mathcal{T}_{c_2}(\dots\mathcal{T}_{c_k}(D_n)))$.\footnote{Since
the complaint set is consistent, it is easy to see that the order of
transformations is inconsequential.} $\mathcal{C}$ is \emph{complete}
if it contains a complaint for each error in $D_n$. In that case,
$D_n'=D_n^*$. In our work, we do not assume that the complaint set is
complete, but, as is more common in practice, we only know a subset of
the errors (incomplete complaint set). Further, we focus our analysis
on \emph{valid} complaints; we briefly discuss dealing with invalid
complaints (complaints identifying a correct value as an error) in
Section~\ref{sec:noise}, but these techniques are beyond the scope of this paper.

\smallskip
\noindent
\emph{Log repair ($\mathcal{Q}^*$):}
The goal of our framework is to derive a diagnosis as a log repair
$\mathcal{Q}^*=\{q_1^*,\dots, q_n^*\}$, such that
$\mathcal{Q}^*(D_0)=D_n^*$. In this work, we focus on errors produced
by incorrect parameters in queries, so our repairs focus on altering
query constants rather than query structure. Therefore, for each query
$q_i^*\in\mathcal{Q}^*$, $q_i^*$ has the same structure as $q_i$
(e.g., the same number of predicates and the same variables in the \texttt{WHERE} clause), 
but possibly different parameters. For example, a good log repair for the
example of Figure~\ref{fig:example} is
$\mathcal{Q}^*=\{q_1^*,q_2,q_3\}$, where $q_1^*$=\texttt{UPDATE Taxes
SET owed=income*0.3 WHERE income >= 87500}.


\subsubsection*{Problem definition}

We now formalize the problem definition for diagnosing data
errors using query logs. A diagnosis is a log repair
$\mathcal{Q}^*$ that resolves all complaints in the set $\mathcal{C}$
and leads to a correct database state $D_n^*$.

\begin{definition}[Optimal diagnosis]\label{def:problem}
    Given database states $D_0$ and $D_n$, a query log $\mathcal{Q}$ such that $\mathcal{Q}(D_0)=D_n$, a set of complaints $\mathcal{C}$ on $D_n$,  and a distance function $d$, the optimal diagnosis is a log repair $\mathcal{Q}^*$, such that:
    \begin{itemize}[itemsep=0pt, parsep=0pt]
        \item $\mathcal{Q}^*(D_0)=D_n^*$, where $D_n^*$ has no errors
        \item $d(\mathcal{Q}, \mathcal{Q}^*)$ is minimized
    \end{itemize}
\end{definition}

More informally, we seek the minimum changes to the log $\mathcal{Q}$
that would result in a clean database state $D_n^*$. Obviously, a
challenge is that $D_n^*$ is unknown, unless we know that the
complaint set is complete. 

In Section~\ref{sec:sol}, we describe our basic method, 
which
uses a constraint programming formulation that expresses this
diagnosis problem as a mixed integer linear program (MILP). 
% We justify
% using this constraint formulation as opposed to methods, such as
%classification, that can analyze one query at a time in Section~\ref{sec:heuristic}. We show that the latter, heuristic
% approach is flawed, and one needs to encode the constraints in the entire log.
In Section~\ref{sec:opt}, we present several optimization
techniques that extend the basic method, allowing \sys to 
(1)~handle
cases of incomplete information (incomplete complaint set), and
(2)~scale to large data and log sizes. Specifically, the proposed \sys algorithm with incremental
repair and tuple slicing optimization (Section~\ref{sec:incremental}), can
handle $10\times$ compared to the basic MILP approach. 

Although we emphasis on solving the query logs with multiple queries, 
an alternative approach, which leveraging on classification tools and linear system equations for 
fixing the WHERE/SET clauses, applies for simpler scenario where the query log only contains one query. 
In Appendix~\ref{sec:heuristic}, we provide detailed description and experimental comparison 
for this alternative approach and demonstrate that it is not compatible with the \sys even in the simple 
single query scenario. 

% \begin{figure}[t]
% \centering
% \includegraphics[width = 0.75\columnwidth]{figures/probtransform}
% \caption{Graphical depiction of the diagnosis problem in our \sys framework.  $D_0$, $D_n$, $\mathcal{Q}$, and $\mathcal{C}$ are given, and \sys uses them to derive the log repair $\mathcal{Q}^*$.
% \alex{not sure if this figure is actually useful.}}
% \label{f:probtransform} 
% \end{figure}


% \deprecate{
% \subsection{Naive Formulation}
% 
% The most general version of the problem
% (depicted in Figure~\ref{f:probtransform}) is to find a sequence of
% transformations $T$ that insert, delete, and/or modify queries in $Q_{seq}$ 
% such that the resulting sequence, $Q'_{seq} = T(Q_{seq})$, resolves the user's complaint set. 
% 
% However this problem is ill-defined because there exist an unbounded set of transformations that
% can resolve the user's complaint set.  A naive solution is to append to the query log a statement
% that deletes all the records in the database, followed by a query that insert all of the correct records.
% Unfortunately this naive solution does not help explain the complaints in any way!
% 
% \subsection{Constraints}
% 
% For this reason, we constrain the set of possible transformations $\mathcal{T}$ to the following:
% 
% \begin{itemize}
% \item delete query
% \item modify insert statement constants
% \item modify constants in WHERE clause
% \end{itemize}
% 
% Our transformations don't include adding new queries, synthesizing arbitrary queries, or modifying the
% number of clauses in a WHERE condition.  We apply these restrictions because we believe it is more likely
% for the user to mis-type a constant value as opposed to having an error in the query structure.
% 
% Futhermore we define a distance metric between two query logs in order to evaluate
% the qulatiy of a transformation.
% \xxx{define $\mathcal{T}$ here.}
% 
% 
% 
% \subsection{Problem Statements}
% 
% In this paper, we present three variants of this problem.
% 
% \begin{problem}[Prob-Complete]\label{prob:complete}
% Given $C = P_{D_n, D^*_n}$, $Q_{seq}$, and the sequence of database states $D_0,\ldots,D_n$, 
% identify a sequence of transformations $T$ such that:
% \begin{itemize}
% \item $T(Q_{seq})(D_0) = C(D_n)$
% \item $|T| = 1$
% \item $T$ metric is minimized
% \end{itemize}
% \end{problem}
% 
% This variation of the problme relaxes the constraint that the complaint set must be complete, and allows
% for both false positives as well as false negatives.  The goal is the same, however the constraints are relaxed:
% 
% \begin{problem}[Prob-Incomplete]\label{prob:incomplete}
% Given $C$ where $acc_C < 1$, $Q_{seq}$, and the sequence of database states $D_0,\ldots,D_n$, 
% identify a sequence of transformations $T$ such that:
% \begin{itemize}
% \item $T(Q_{seq})(D_0) = D^*_n$
% \item $T$ metric is minimized.
% \item $|T| = 1$
% \end{itemize}
% \end{problem}
% 
% Finally, we extend the problem to allow transformations with one or more operations.
% 
% \begin{problem}[Prob-MultiQ]\label{prob:multi}
% Given $C$ where $acc_C < 1$, $Q_{seq}$, and the sequence of database states $D_0,\ldots,D_n$, 
% identify a sequence of transformations $T$ such that:
% \begin{itemize}
% \item $T(Q_{seq})(D_0) = D^*_n$
% \item $T$ metric is minimized.
% \end{itemize}
% \end{problem}
% 
% 
% 
% 
% \subsection{A Naive Approach}
% 
% \begin{itemize}
% \item roll back complaints to penultimate state using algebraic expressions 
% \item perturb each expression in query until the query result matches correct state
% \item if an expression cannot be found, iterate
% \end{itemize}
% 
% 
% Not clear how to roll back complaints
% 
% Ways to perturb query expressions is unbounded
% 
% }
%!TEX root = ../main.tex

\subsection{A plausible (but bad) alternative}
\label{sec:dt}

% \sys analyzes query logs and complaints by producing a mathematical
% formulation of the constraints that need to be satisfied. The
% constraint problem can then be evaluated by dedicated external tools.

The MIP models generated by \sys can grow large as the sizes of the
data and the log increase. However, modeling all present constraints
from the beginning to the end of the log is necessary; in this
section, we examine alternative, simpler models that process a single
query at a time, and demonstrate why they are insufficient.

\smallskip
\noindent
\textbf{WHERE repairs through classification:}
The \texttt{WHERE} clause of an update query is equivalent to a
rule-based binary classifier that splits tuples into two groups:
(1)~tuples that satisfy the conditions in the \texttt{WHERE} clause
and (2)~tuples that do not. A mistake in a query predicate can then
result in misclassification: some tuples get classified into the wrong
group, which in turn translates to errors in the data. Therefore,
repairing the mistake corresponds to repairing the imprecise
classification. This works as follows: For an incorrect query $q$, let
$D_0$ be the database state before $q$, and $D_1^*$ the \emph{correct}
database state that should result after $q$.
We use each tuple $t \in D_0$ as an element in the input training data
for the classifier where the values (of each attribute) of $t$ define
the feature vector and the label for $t$:
	\[
    label(t)= 
    \begin{cases}
    true ,& D_0.t \neq D_1^*.t\\
    false,              & \text{otherwise}
    \end{cases}
\]
We then train a classifier, such as decision trees \cite{???} to learn
the correct classification rules rules for the \texttt{WHERE} clause.


\smallskip
\noindent
\textbf{SET repairs:}
After repairing the \texttt{WHERE} clause through learning a
rule-based classifier, some complaints may still persist. This
indicates a possible error in the \texttt{SET} clause. The errors can
be modeled and solved by constructing a simple linear system of
equations: For each expression in the \texttt{SET} clause we create a
linear equation, using unknown variables to represent any parameters
in the \texttt{SET} expression. Solving for these variables then
provides a repair for the \texttt{SET} expression.


\smallskip
\noindent
\textbf{Why it does not work:}
The na\"ive approach that we just described is heuristic in nature. It
is simple and fast, but it can only process a single incorrect query.
This results in several shortcomings that make it insufficient in
practice:
\begin{itemize}[itemsep=1pt, leftmargin=5mm]
    
\item In principle, one could attempt to apply this technique to the
entire log one-query-at-a-time. However, this is not possible in
practice: to learn a classifier on the \texttt{WHERE} clause of query
$q_i$, one needs to know the correct classification output, which
corresponds to $D_i^*$. Unfortunately, even with a complete complaint
set, which can derive the correct database $D_n^*$, there is no
obvious way to ``rollback'' this state to derive $D_i^*$.

\item The classifier may derive a clause that is structurally very
different from the original one (different attributes or number of
conditions). This is problematic in general, as it corresponds to a
larger-scale mistake in the query, which is a less likely scenario.

\item Classifiers try to avoid overfitting, which is problematic for
queries with high selectivity (e.g., single-tuple updates), as the
classifier is unlikely to generate any rules.

\end{itemize}


Therefore, while examining one query at a time superficially appears
to be a reasonable and efficient alternative, the reality is that one
has to model all constraints and transformations through the entire
log history. In the following section, we propose several
optimizations to our initial approach that make scaling to large data
and log sizes feasible. 
% \red{Add graph comparing naive and d-trees here.}


\iffalse
In this
section, we examine alternative, simpler models that process a single
query at a time, and demonstrate why they are insufficient.

\noindent
\textbf{WHERE repairs through classification:}
The \texttt{WHERE} clause of an update query is equivalent to a
rule-based binary classifier that splits tuples into two groups:
(1)~tuples that satisfy the conditions in the \texttt{WHERE} clause
and (2)~tuples that do not. Thus, by training a classifier,
such as decision trees \cite{quinlan1987} to learn
the correct classification rules rules for the \texttt{WHERE} clause.

\noindent
\textbf{SET repairs:}
This alternative approach constructs 
a simple linear system of equations to solve the parameters in the \texttt{SET}
when errors persist after fixing the \texttt{WHERE} clause:
For each expression in the \texttt{SET} clause we create a
linear equation, using unknown variables to represent any parameters
in the \texttt{SET} expression. 
  \begin{figure}[h]
  \centering
    \includegraphics[width = .6\columnwidth]{figures/heuristicacc}
    \vspace*{-.1in}
    \caption{Heuristic Approach vs. \sys on Single-Query. }
    \label{f:heuristic_acc} 
  \end{figure}
  \vspace*{-0.1in}
  
The na\"ive approach that we just described is heuristic in nature. It
is simple and fast, but it can only process a single incorrect query. As shown in
Figure~\ref{f:heuristic_acc}, the F-1 score of na\"ive heuristic approach is less 
than 0.6 while \sys maintain high accuracy in solving single query problem with
above 0.9 F-1 score across all database sizes. 
\fi

%!TEX root = ../main.tex

\section{A MILP-based Solution}
\label{sec:sol}

In this section, we introduce an exhaustive solver-based approach to 
resolve the errors reflected in the complaint set.
This approach constructs a mixed-integer linear 
programming (MILP) problem~\cite{milp} by linearizing and parameterizing the 
corrupted query log over the tuples in the database. 
Briefly, an MILP is a linear program~\cite{} where only a subset of the undetermined variables
are required to be integers, while the rest are real valued.

Our general strategy is to model each query as a linear equation 
that computes the output tuple values from the inputs and to transform the
equation into a set of of linear constraints.   
In addition, the constant values in the queries are parameterized
into a set of undetermined variables, while the database state is encoded 
as constraints on the initial and final tuple values.
Finally, the undetermined variables are used to construct an objective function
that prefers value assignments that minimze both the amount that the queries change and
the number of non-complaint tuples that are affected. \alex{I was under the impression that we stick to using the non-complaint tuples as constraints.}

The rest of this section will first describe the process of linearizing a single query
and translating it into a set of constraints.  We then extend the process to the entire
query log and finally define the objective function.
Subsequent sections introduce optimizations that both
improve the speed and quality of the results, as well as harness the trade-off between the two. 


% \xlw{Essentially, 
% we convert the updating process of the query log into a set of 
% constraints, transform values in queries (in the query log)
% as undetermined variables, and 
% using these values to form the objective function. Thus, the problem of 
% deriving a log repair is converted into a mixed-integer linear 
% programming (MILP) optimization problem. }\\
% {\xlw{need to introduce MILP problems a little bit here, so readers know what undetermined
% variables are and why things need to be linearized}}






\subsection{Encoding a Single Query}%Linearizing \& Parameterizing Single Query}
\label{sec:linearize}

\if{0}
  We model a query $q_i$ as a conditional function $f_{q_i}(t)$ that takes as input a tuple $t$
  and returns its next state $t'$.  $f_{q_i}$ is applied to each 
  tuple $t \in \mathcal{D}_{i-1} \cup \{\bcancel{t}\}$ in the input relation along with a special
  non-existant tuple $\bcancel{t}$. \ewu{maybe fold this into the data model.}
  By treating the query as a function, we are able to encode its effects into a set
  of linear inequality constraints.  We call this process the linearization and 
  parameterization of a query.

  \begin{definition} [Conditional Function]
  \label{def:cond}
    The conditional function for query $q$ is:
    \[
      f_{q_i}(t)= 
      \begin{cases}
      f_{q_i.\mu} (t) ,& \text{if } f_{q_i.\sigma} (t)\\
      t,              & \text{otherwise}
      \end{cases}
  \]
  where the \textit{update function} $f_{q_i.\mu}$ models a set of \textit{update equation(s)};
  and the \textit{condition function} $f_{q_i.\sigma}$ models a set of \textit{logical expression(s)} in 
  disjunctive or conjunctive form.
  \end{definition} 


  % We linearize and parameterize a query $q$ by considering its effects on the 
  % targeted table $R(A_1, ..., A_m)$: we treat the query $q$ as a 
  % conditional function over each tuple $t\in R$ and convert the effects of $q$ 
  % over $t$ into a set of linear inequality constraints. 
  % \subsubsection{Query as a Conditional Function}
  % The effect of query $q$ over a tuple $t$ can be expressed in a conditional
  % function $f_q(t)$ as the following:

  Conditional functions can describe the common classes of update queries:
  \begin{enumerate}
  \item \texttt{UPDATE}: $f_{q_i.\mu}(t)$, $f_{q_i.\sigma}(t)$ model the \texttt{SET}
        and \texttt{WHERE} clauses.  For example, $f_{q_i.\mu}(<t.a, t.b>) = <t.a + 1, 2>$ and
        $f_{q_i.\sigma}(t) = (t.a > 20)$ corresponds to the query 
        \texttt{UPDATE D SET a = a + 1, b = 2 WHERE a > 20}.

  \item \texttt{INSERT}: $f_{q_i.\mu}(t)$ returns the inserted tuple, while 
        $f_{q_i.\sigma}(t) = (t = \bcancel{t})$ evaluates to true when it is executed over
        the special nonexistant tuple.
        %is a boolean variable reflects 
        %the existence of tuple $t$; $\vee_{t\in R} f_{q.\sigma}(t)$ represents 
        %the existence of this insert query.  

  \item \texttt{DELETE}: $f_{q_i.\mu}(t) = \bcancel{t}$ returns a nonexistant tuple whenever
        the predicate encoded in $f_{q_i.\sigma}(t)$ evaluates to true.
        For example, the query \texttt{DELETE FROM T WHERE a < 20} represents 
        $f_{q_i.\sigma}(t) = (t.a < 20)$.
        
        % the deleted values for 
        % each attribute and $f_{q.\sigma}(t)$ reflects the expressions in where clause 
        % (similar to $f_{q.\sigma}(t)$ in \texttt{UPDATE} query).
  \end{enumerate}

  Finally, we parameterize $q_i$ by replacing all numeric constants in the
  conditional function with undetermined variables.   Consider the conditional
  function for \texttt{UPDATE} above: the constants $1$, $2$ in $f_{q_i.\mu}$
  as well as $20$ in $f_{q_i.\sigma}$ will be transformed into undetermined
  variables \texttt{v1, v2, v3} that are solved by the MILP solver.

  % We parameterize query $q$ by replacing all numeric values in the above conditional 
  % function into undetermined variables (Example~\ref{ex:parameterize}).
  % \begin{example}\label{ex:parameterize}
  % Consider a \texttt{UPDATE} query $q$:
  % \texttt{UPDATE R SET A$_1$ = 3 WHERE A$_2$ $\leq$ 10},
  % the conditional function of this query is: 
  % \[
  %     f_q(t)= 
  % \begin{cases}
  %     f_{q.\mu}(t) = \{t.A_1 = 3\} ,& \text{if } f_{q.\sigma}(t) = \wedge\{t.A_2 \leq 10\}\\
  %     t,              & \text{otherwise}
  % \end{cases}
  % \]
  % The numeric variables in query $q$ including \texttt{3} in $f_{q.\mu}(t)$ and \texttt{10}
  % in $f_{q.\sigma}(t)$. Thus, we parameterize query $q$ by replacing the value \texttt{3, 10}
  % by two undetermined variables \texttt{var1, var2}. 
  % \end{example} 
\fi



% \subsubsection{Constructing Linear Inequality Constraints}

MILP problems require that the constraints are expressed as a set of linear
(in)equality constraints, thus our task is to linearize the parameterized
conditional function.   We will individually describe this linearization process 
to represent the value of a single attribute $A_j$ and a single tuple $t$ for 
\texttt{UPDATE}, \texttt{INSERT}, and \texttt{DELETE} queries; 
extending the process to the rest of the attributes is a straighforward exercise.
% Now that we have modeled $q_i$ as a parameterized conditional function, 
% we must linearize $f_{q_i}$ into a set of linear (in)equality constraints
% for the MILP problem.



\stitle{\textbf{UPDATE:} }
Recall from Section~\ref{sec:models} that query $q_i$ can be modeled as
the combination of a modifying function $\mu_{q_i}(t)$ and conditional function $\sigma_{q_i}(t)$.
We can re-express the value of attribute $A_j$ in the updated tuple $t'$ as the following equation:
{\scriptsize
\begin{eqnarray}
\label{eq:linearization}
t'.A_j = x\otimes \mu_{q_i}(t).A_j + (1-x)\otimes t.A_j 
\end{eqnarray} 
}


\noindent Where the binary variable $x$ is introduced to represent the output of 
the conditional function:
{\scriptsize
\begin{eqnarray}
\label{eq:x}
x = \sigma_{q_i}(t)
\end{eqnarray}
}
We then introduce two variables $u.A_j$ and $v.A_j$ to represent 
the respective components in Equation~\ref{eq:linearization}, where
$M$ representes the upper bound of $t.A_j$'s domain:
{\scriptsize 
\begin{eqnarray}
\label{eq:uv}
u.A_j &\leq & \mu_{q_i}(t).A_j \nonumber\\
u.A_j &\leq & xM \nonumber\\ 
u.A_j &\geq & \mu_{q_i}(t).A_j - (1-x)M \nonumber \\\nonumber \\
v.A_j &\leq & t.A_j \nonumber\\
v.A_j &\leq & (1-x)M \nonumber\\
v.A_j &\geq & t.A_j - xM
\end{eqnarray}
}
The first set of conditions forces $u.A_j = \mu_{q_i}(t).A_j$ if $x=1$, and $0$ otherwise.  
The second set forces $v.A_j = t.A_j$ if $x=0$, and $0$ otherwise.  
Now, Equation~\ref{eq:linearization} is simply a linear equation:
{\scriptsize\begin{eqnarray}
\label{eq:tnew}
t.A_j' = u.A_j + v.A_j
\end{eqnarray}}



\stitle{\textbf{INSERT:}}
An insert query adds a new tuple $t_{new}$ to the database.  If the query were 
corrupted, then the inserted values may need to be repaired.  Thus, we allocate
an undetermined variable for each of the inserted tuple's attributes:
{\scriptsize
\begin{eqnarray}
\label{eq:insert}
t'.A_j = x \otimes t_{new}.A_j + (1-x) \otimes v.A_j 
\end{eqnarray}
}
\noindent Where $x$ is an undetermined binary variable that represents whether
the query is incorrect.  If it is, then it takes the value of an undetermined real 
variable $v.A_j$.


\stitle{\textbf{DELETE:}}
A delete query removes a set of tuples from the database.  
Since the MILP problem doesn't have a way to express a non-existant value, 
we encode a deleted tuple by setting its attributes to a value
outside of the attribute domain $M^+$.  In this way, subsequent conditional functions
on the attribute will return false, so it will not have an effect on subsequent queries encoded
in the MILP problem:
{\scriptsize
\begin{eqnarray}
\label{eq:delete}
t'.A_j &=& x \otimes M^+ + (1-x) \otimes t.A_j \nonumber \\
x &=& \sigma_{q_i}(t)
\end{eqnarray}
}


\stitle{\textbf{Putting it All Together:}}

% )(\ref{eq:uv})(\ref{eq:tnew})(\ref{eq:insert})(
The above constraints (\ref{eq:x}--\ref{eq:delete})
form the main structure of MILP subproblem for a single attribute $A_j$ of a single tuple $t$.
In this incomplete form, all of the variables including the binary variables $x$,
the real-valued attribute values (e.g., $u.A_j$),
and the real-valued constants in $\mu_{q_i}$ and $\sigma_{q_i}$ are all undetermined
and need to be assigned values by a MILP solver.  
To linearize $q_i$, we can simply apply the same procedure for each attribute in the 
query and every tuple in the database.
We describe the above linearization process as the function 
$Linearize(q, t)$, which takes a query $q$ and tuple $t$ as input.

The final step is to assign concrete values to the starting and ending attribute values 
$t.A_j$ and $t'.A_j$ based on the starting database state $D_{i-1}$ and the ending database 
state that has been transformed by $\mathcal{C}$, $D'_i$.
In this way, only the query parameters and the binary $x$ variables are undetermined;
a solution to the MILP formulation will assign values to those undetermined variables
such that the resulting query fixes the complaints correctly.

The next subsection will describe how to extend the above encoding procedure to the 
entire query log, and how to incorporate an objective function that encourages solutions
that minimize the amount of changes to the query log.  







\subsection{Encoding and Repairing the Query Log}
\label{sec:milp}

We are now ready to describe the procedure (Algorithm~\ref{alg:basic}) to encode 
the full query log into a MILP problem, and solve the MILP problem to derive $\mathcal{Q}^*$.
The algorithm takes as input the query log $\mathcal{Q}$, 
the initial and final (dirty) database states 
$\mathcal{D}_{0, n}$, and the complaint set $\mathcal{C}$, and outputs a fixed query 
log $\mathcal{Q}^*$.  

We first call \textit{Linearize} on each tuple in $\mathcal{D}_0$ and each query in $\mathcal{Q}$, 
and add the result to a set of constraints \textit{milp\_cons}.
Similar to the single-query procedure, \textit{AssignVals} additionally 
adds constraints to fix the values of the inputs to $q_0$ and 
the outputs of $q_n$ to their respective values in
$\mathcal{D}_0$ and $\mathcal{C}(\mathcal{D}_n)$.
Next, we add constraints to take into account the fact that the output of 
query $q_i$ is the input of $q_{i+1}$ (\textit{ConnectQueries}).
This function simply equates $t'$ from the linearized result for $q_i$ to the $t$ input 
for the linearized result of $q_{i+1}$.

Finally, we use the undetermined variables in \textit{milp\_cons}, along with $\mathcal{Q}$,
to encode the distance function $d$ described in Section~\ref{def:obj} 
into the MILP objective function, and submit it to a MILP solver.
\textit{ConvertQLog} replaces the constants in the query log with the 
assigned values for the undetermined variables in the solution, and constructs
the fixed query log $\mathcal{Q}^*$.
\ewu{explicitly mention each function in algorithm.}



% Using the linearization method in Section~\ref{sec:linearize}, we can further
% linearize the entire query log by converting every tuples in the table $R$. 
% During the linearization, we further parameterize each
% query in the query log $\mathcal{Q}$ in order by derive the log repair. 
% The linearized and parameterized query log
% should start from and end at clean database states.
% To achieve this, we add constraints by assigning the true initial and end database 
% states' values (based on complaints) to the corresponding variables. 
% Following the above steps (Algorithm~\ref{alg:basic}), we convert the 
% query log into a collection of constraints with newly introduced variables. 
% Some of these variables are introduced to linearize the query log 
% and the rest usually represent the numeric values in the query log
% during the paramerization process. 
% The latter set of variables often involved in the objective function 
% according to the pre-defined distance function $d$ 
% (as described in Definition~\ref{def:problem}). 


\begin{algorithm}[htbp]
\caption{$QueryFix_{exh}$ based on MILP formulation.}
\label{alg:basic}
\scriptsize
\begin{algorithmic}
\REQUIRE {$\mathcal{Q}, D_0, D_n, \mathcal{C}$}
%\ENSURE {$\mathcal{Q^*}$}
\STATE $milp\_cons \leftarrow \emptyset$
\FOR {each $t$ in $R$}
\FOR {each $q$ in $\mathcal{Q}$}
\STATE $milp\_cons \leftarrow milp\_cons \cup Linearize(q, t)$
\ENDFOR
\STATE $milp\_cons \leftarrow milp\_cons \cup AssignVals(D_0.t, D_n.t, \mathcal{C})$
\FOR {each $i$ in $\{0,\ldots,N-1\}$}
\STATE $milp\_cons \leftarrow milp\_cons \cup ConnectQueries(q_i, q_{i+1})$
\ENDFOR
\ENDFOR 
\STATE $milp\_obj \leftarrow EncodeObjective(milp\_cons, \mathcal{Q})$
\STATE $solved\_vals \leftarrow MILPSolver(milp\_cons, milp\_obj)$
\STATE $\mathcal{Q}^* \leftarrow ConvertQLog(Q, solved\_vals)$
\STATE Return $\mathcal{Q}^*$
\end{algorithmic}
\end{algorithm}

% By constructing linear (in)equality constraints and
% defining a objective function, we convert the problem into
%  a mixed-integer linear programming (MILP) problem that can be 
%  solved by MILP solvers. By solving this MILP problem, we collect the
% corrections for the parameterized variables and form the log repair. 

% describe how to linearize the whole querylog with provided
% database states info. 


















\section{The Objective Function}

The solution of a MILP problem is designed to minimize an objective function defined by the user.
A natural object function is to simply use the distance function $d(\mathcal{Q}, \mathcal{Q}^*)$ 
(Definition~\ref{def:problem}) so that the proposed solution
does not deviate significantly from the original query log e.g., does not modify too many queries.  
This has the added benefit of interpretability, since the user is presented with less diagnoses.

The general form of $d(\mathcal{Q}, \mathcal{Q}^*$ provides us with substantial flexibility in the precise
function implementation so that different function can potentially be customized towards different scenarios.
For example, when the query log only involves numeric attributes, 
the normalized Manhattan distance~\cite{manhattan} between the parameters 
in $\mathcal{Q}$ and $\mathcal{Q}^*$ is a natural objective function.
Alternatively, we may use the number of parameters that are different with the original ones 
in order to penalize repairs that modify too many parameters. 

Moving beyond the queries themselves, we may also augment the distance function
by penalizing the number and magnitude of the tuples that are modified by the repair.
In this paper, for simplicity, we use Manhattan distance between parameters 
in the query log as our objective function.  We use $q.param_i$ to denote the $i^{th}$ paramater of query $q$,
and $|q.param|$ to denote the total number of parameters in $q$:
\[d(\mathcal{Q}, \mathcal{Q}^*) = \sum_{i = 1} ^{n} \sum_{j = 1}^{|q_i.para|} |q_i.para_j - q_i.para_j^*|\]

% impact of the log repair $\mathcal{Q}^*$ can also be involved 
% in the distance function: one would prefer that $\mathcal{Q}^*$
% modifies as few tuples as possible, in which case the distance 
% function could be the number of tuples that modified by the log repair.
% By allowing this custmizable distance function, we increase the flexibility 
% of our system. \\
% 
% 
% the distance function is the Manhattan distance between parameters in 
% the query log $\mathcal{Q}$ and the log repair $\mathcal{Q}^*$; 
% 
% w.r.t the property of the query log and its
% impact over the database. 
% he
% \ewu{mention goal is to minimie the objective unction}
% talked about moving the t.A_j = CONSTANT constraints for non-complaints
% into obj functino as |t.A_j - CONSTANT|, however makes problems slower
% so only use soft constraints during second iteration.

\section{Optimizing the Basic Approach}
\label{sec:opt}

A large drawback of the MILP-based approach described in the previous section is
that it exhaustively encodes the combination of all tuples in the database and all queries
in the query log.  In this approach, the number of constraints grows quadratically with respect to
the database and the query log; the number of undetermined variables also increases quandratically.
Both of these properties typically .
In a simple experiment, shown in Figure~\ref{fig:querysize_vs_time}, we generated increasingly 
larger query logs for a database of $XXX$ tuples, encoded the problem using $\sys_{exh}$, and solved
the problem using an ILP-solver (CPLEX~\cite{cplex} in our experiments).
We find that the solver time increases exponentially as the number of encoded queries increases,
due exponential increase in the number of possible states for the undetermined variables.
Although the solver performance varies widely depending on the specific problem, this experiment 
illustrates the limitations of the naive approach.

\begin{figure}{h}
    \centering
        \includegraphics[width=0.4\textwidth]{figures/auctionmark_qsize_time}
    \caption{\# of queries vs. execution time on Auctionmark dataset. }
    \label{fig:querysize_vs_time}
\end{figure}

To resolve this limitation, we explored two classes of optimizations that seek to
reduce the number of encoded tuples and the number of linearized queries in the MILP problem.
In addition, incremental algorithm that is faster if the corruption is recent.  
We see in experiments that this is likely the case.




% In the previous section, we introduce the basic approach to derive
% the log repair by incorporating information from every query
% and every tuple into a single MILP problem. However, oftentimes, 
% this basic approach end up with 
% a huge problem as the query log size and table size increase. 
% In Figure~\ref{fig:querysize_vs_time}, 
% we observe that the total solver (IBM CPLEX) solving time 
% grows exponentially and 
% unpredictably as the query 
% log size or the table size increases. As a result, the basic approach does not scale over 
% large problems (large query log size and large table size).


linearize whole query log, so cost of adding an additional tuple is very high.
second iteration generally takes ~1 - 10%
for large databases knn cost is pretty high: ~

if the solver returns, it is always a super set of the clean range

only tuples modified by the fixed queries
- tuples already in the complaints (correct)
- not in complaints, but any of the originial queries modified it
- not in complaints, but no original queries modified it



\subsection{Reducing Tuples}
\label{sec:opt:tbsize}


\subsubsection{Adjusting Over-generalized Fixes}

fix is always the smallest possible, so it can only require a sceond iteration if the dirty and clean
ranges don't overlap at all.  

The basic solution, which suggests to linearize every tuple in the database into the MILP problem,
has two major disadvantages: 1. the system may end up with a large MILP problem that requires 
the solver to run forever to solve; 2. we can only linearize the every tuple
when the complaint set is complete, which is often hard to guarantee. Thus, in the first 
optimization, we propose a two-iteration approach handles both of these two problems 
by linearizing tuples in the complaint set in
the 1st iteration and refining the log repair in the 2nd iteration. \\
\subsubsection{1st iteration}
We use tuples in the complaints $\mathcal{C}$ to construct the MILP problem and derive a 
inaccurate log repair $\mathcal{Q}^*$\xlw {we need to find a term for this}. This inaccurate 
log repair resolves tuples in the complaints $\mathcal{C}$, but, at the same time, 
introduces noises by over correcting tuples that are not involved in the MILP problem. 
As shown in Example~\ref{ex:2nditer}, by resolving tuples in complaints $\mathcal{C}$, 
the system derives a log repair that introduces other errors. 
\begin{example}\label{ex:2nditer}
Including tuples in the complaints $\mathcal{C}$ to solve problem in Example~\ref{ex:taxes2} 
and using Euclidean distance of variables in the query as the objective function, the system 
derives a log repair $\mathcal{Q}^*$ with $q_1^*=$ \texttt{\small UPDATE Taxes SET rate = 30}
\texttt{\small WHERE income >= \color{red}{9500.0001} \color{black}{and income <=} \color{red}{90000}}. 
This log repair resolve tuples in the complaints. However, the fixed query $q_1$ has a much wider range 
and it it incorrectly modifies tuples that are not in the complaint, e.g., tuple $t_4$.
\end{example}
\ewu{why would this happen if we encoded all the tuples?  Doesn't the objective penalize this case?}


To avoid introducing such noises, we introduce the 
2nd iteration which targets on refining the log repair derived in 
the 1st iteration. 
\subsubsection{2nd iteration}
The goal for 2nd iteration is to optimize the impact of the log repair (tuples
modified by $\mathcal{Q}$). 
Let $\mathcal{Q}^*_{sub} = \mathcal{Q}^*-\mathcal{Q}$ as 
queries modified in the log repair, in the 2nd iteration, we 
construct a separate MILP problem by linearizing 
 queries between the first and the last query in
$\mathcal{Q}^*_{sub}$, parameterizing variables in 
the where clause for queries in $\mathcal{Q}^*_{sub}$,
and maximizing the log repair impact score. 

\begin{figure}{h}
    \centering
    \includegraphics[width=0.4\textwidth]{figures/2nditerationgroups}
    \caption{Graphical depiction of the group 2 and group 3. }
    \label{fig:groups}
\end{figure}

We define the impact score by dividing the 
impacted tuples $T$ into three groups and defining the 
impact score accordingly. \xlw {we need to fix these terms}. 

\smallskip

\noindent\textbf{Group 1}: tuples in the complaints $\mathcal{C}$. Impact of 
$\mathcal{Q}^*$ on this group
are guaranteed as correct. \textbf{Rule:} 
strictly satisfy, including these tuples as 
constraints in the 2nd iteration MILP problem. 

\smallskip

\noindent\textbf{Group 2:} tuples not in $\mathcal{C}$, but also modified 
by the original query log $\mathcal{Q}$. Impact of $\mathcal{Q}^*$ 
on the group 2 are likely to
be correct. \textbf{Rule:} adding these tuples into the 
rewarded tuple set $T_{reward}$. 
\ewu{what does "modified by the original query log Q mean?}

\smallskip

\noindent\textbf{Group 3:} tuples not in $\mathcal{C}$ and not modified 
by the original query log $\mathcal{Q}$. Impact of 
$\mathcal{Q}^*$ on the last group should penalized, thus 
we put them into $T_{penalize}$ (\textbf{Rule}).

\smallskip

The impact score, which is also the objective function for
the MILP problem, is $T_{reward} - T_{penalize}$. \ewu{what exactly is this objective function?  the number of tuples modified?}We further 
optimize the 2nd iteration to reduce the size of $T$ 
by searching for K-nearest-neighbors
of tuples in complaints $\mathcal{C}$. 
There are multiple benefits for the two-iteration approach:
\begin{itemize}
\item It minimizes the size of MILP problem in the 1st iteration. 
\item It refines the log repair while avoid over correcting tuples not in 
the complaints $\mathcal{C}$. 
\item It pptimizes the impacted tuples efficiently by controlling the queries 
and number of impacted tuples
involved in the 2nd MILP problem.
\item In addition, it handles cases when we don't have complete complaints
$\mathcal{C}$ (false negatives). 
\end{itemize}
\subsection{Optimization: log size}


\if{0}
\subsubsection{A Naive but Flawed approach}
\ewu{Better explained as: CPLEX searches through an exponential space of all possible combinations of MILP variables.  In a chunked approach, the solution of each chunk is one out of a potentially arbitrary number of possible solutions, thus it is easy to pick an incorrect one}
A natural idea to optimize the basic approach is 
to \textbf{chunk the query log} into
smaller, fixed size pieces and then solve each piece at a time: starting
from the most recent piece, the system linearizes and parameterizes queries 
in the current piece and derives a corresponding log repair; 
it then examines the other pieces iteratively
in the same way. Since complaints only provides
true values for the most recent database state, in order to avoid 
linearizing additional queries, 
we need to know \textbf{rollback} the true values of tuples 
until the last query in each query log piece. \\
However, rollback the database is non-easy. An ideal, precise rollback
algorithm would generate a set of valid ranges for each attribute of a tuple. 
But the size of valid ranges also grows exponentially with the number queries
we want to rollback, which, in turn, could not improve the system performance. 
On the other hand, an approximate, imprecise 
rollback algorithm would either make the rest of the problems
infeasible to solve (only maintain fixed number of valid ranges) 
or result in deriving 
incorrect log repairs (maintain the lower 
bound and upper bound among all valid ranges).
  

In order to improve the system performance without losing accuracy, we propose
the following two optimizations: query-slicing optimization 
based on provenance over queries and
attribute-slicing optimization based on provenance over 
attributes. 
\fi

\subsubsection{Query-Slicing}
\label{sec:opt:query}
The exhaustive parameterization of all queries in the query log, as described
in the basic approach, is typically unnecessary because many queries in the log
could not have affected the tuple attribute values in the complaint set.
To avoid such redundant computation, \textit{query-slicing} 
removes \texttt{UPDATE} queries whose \texttt{SET} clauses did not modify 
attributes that could possibly have affected the {\it complaint attributes} 
specified in the complaint set.

\ewu{changed INcorrect attributes to Complaint Attributes}

\begin{definition} [Complaint Attributes]
	The complaint attributes $\mathcal{A}(C)$ are attributes in 
	table $R$ with incorrect values: 
	\[\mathcal{A}(C) = \{A_i|A_i\in R, \exists t.A_i \neq t.A_i^*, t\in C\}\]
\end{definition} 

\ewu{Why don't we use lineage/provenance language?}
\begin{definition}[Query dependency\& impact]
    The \textbf{dependency}, $\mathcal{P}(q)$, of a query $q$
    is the set of 
    attributes involved in the condition function of $q$:
    \[\mathcal{P}(q) = \Pi_{f_{q}.\sigma}(R)\]
    The \textbf{direct-impact} of query $q$, denoted
    by $\mathcal{I}(q)$, is the set of attributes 
    involved in the update function of $q$:
    \[\mathcal{I}(q) = \Pi_{f_{q}.\mu}(R)\]
    The \textbf{full-impact}
    of $q$, $\mathcal{F}(q)$, propogates the $q$'s direct impact through
    the subsequent queries in the query log to describe {\it all} attributes
    that are affected by the changes caused by $q$'s \texttt{SET} clause.
    We describe its implications and calculation next.
    \ewu{Is this correct?}
\end{definition}


By comparing $\mathcal{F}(q)$ and $\mathcal{A}(C)$, we know whether
or not corrupting query $q$'s parameters could possibly have caused the complaint
set.  Specifically, when $|\mathcal{F}(q) \cap \mathcal{A}(C)|=|\mathcal{A}(C)|$, 
$q$ potentially affected all of the attributes referenced in the complaint set and is
a candidate for fixing; 
when $0 < |\mathcal{F}(q) \cap \mathcal{A}(C)|< |\mathcal{A}(C)|$, 
$q$ contributed to a subset of the attributes in the complaint set; 
and when $|\mathcal{F}(q) \cap \mathcal{A}(C)|=0$, $q$ is irrelevant 
and can be ignored as a candidate for fixing. 
We use $Rel\mathcal{(Q)}$ to denote the set of relevant queries. 

Algorithm~\ref{alg:fullimpact} describes how we find
$\mathcal{F}(q_i)$ for $q_i$ in the query log:

\begin{algorithm}[htbp]
\caption{$FullImpact$ algorithm for finding $\mathcal{F}(q)$.}
\label{alg:fullimpact}
\begin{algorithmic}
\REQUIRE {$\mathcal{Q}$, $q_i$}
% \ENSURE {$\mathcal{F}(\mathcal{Q})=\{\{\mathcal{F}(q_1)\}, ..., \{\mathcal{F}(q_n)\}\}$}
% \FOR {each $q_i$ in $q_n, ..., q_1$}
\STATE $\mathcal{F}(q_i) \leftarrow \mathcal{I}(q_i)$
\FOR {each $q_j$ in $q_{i+1}, ..., q_{n}$}
\IF {\red{$\mathcal{F}(q_i)\cap \mathcal{P}(q_j) \neq \emptyset$}}
\STATE $\mathcal{F}(q_i) \leftarrow \mathcal{F}(q_i) \cup \mathcal{F}(q_j)$
\ENDIF
\ENDFOR
\STATE $\mathcal{F}(\mathcal{Q}) \leftarrow \mathcal{F}(\mathcal{Q}) \cup {\red{\{\mathcal{F}(q_i)\}}}$
% \ENDFOR
\STATE Return $\mathcal{F}(\mathcal{Q})$
\end{algorithmic}
\end{algorithm}

Finally, we can compute $Rel\mathcal{(Q)}$



\subsubsection{I don't understand this}


In addition to pruning out irrelevant queries,
we also prune irrelevant attributes. \\
Let $Rel\mathcal{(Q)}$ as the set of 
relevant queries, we can find the relevant 
attributes as following:
\[Rel\mathcal{(A)} = \cup_{q_i \in Rel\mathcal{Q)}} 
(\mathcal{F}(q_i)\cup \mathcal{P}(q_i)) \]



\subsubsection{Incremental Computation}

\begin{figure}[t]
  \centering
  \includegraphics[width=.4\textwidth]{figures/placeholder}
  \caption{Solver time vs size of encoded query log.}
  \label{fig:badscaling}
\end{figure}

\ewu{Need to justify why we incremental algorithm steps back query at a time given Figure~\ref{fig:querysize_vs_time} suggests parameterizing >1 query is the same speed.}

Despite the previous optimizations, it is still expensive to parameterize all 
queries in $Rel\mathcal{(Q)}$ and solve for all parameterized values.
In particular, the cost of solving a MILP problem increases \ewu{cubically?}
with respect to the number of variables.
For example, Figure~\ref{f:badscaling} shows the time cost of the CPLEX solver
as we increase the number of query logs that are encoded and sent to the solver.  
The solid line is when the values in all of the queries are parameterized, while
the dashed line illustrates the time if only the oldest query is parameterized. 
These results suggest that it is {\it faster} to run a separate MILP problem for each 
suffix of the query log (e.g., $q_i, \ldots, q_m$) where $q_i$ is parameterized, 
rather than encode and parameterize the entire log.
Thus we use use an incremental approach, where we individually test each relevant query from the most recent
to the oldest:


\begin{algorithm}[htbp]
\caption{$QueryFix_{inc}$ algorithm.}
\label{alg:incalg}
\begin{algorithmic}
\REQUIRE {$Rel\mathcal{(Q)}$}
\STATE Sort $Rel\mathcal{(Q)}$ from most to least recent
\FOR {each $q_i \in Rel\mathcal{(Q)}$}
  \STATE $q_i^*$ $\leftarrow$ $QueryFix(\{q_j | j \ge i \wedge q_j \in Rel\mathcal{(Q)}\})$
  \IF {$q_i^* \neq \emptyset$}
    \STATE Return $q_i^*$
  \ENDIF
\ENDFOR
\end{algorithmic}
\end{algorithm}



A benefit of this approach is that complaints are more likely to be the result of
recent queries rather than very old queries, and our experiments in Section~\ref{exp:}
speak towards this point.





\section{Noisy Complaint Sets}
\label{sec:noise}

As described in the problem setup (Section~\ref{sec:model}),
complaints sets may themselves have two types of error.
The first type are incomplete complaint sets, which are missing complaints
that have not been included in the sets.
In this case, naively encoding the query log and database as in $QueryFix_{exh}$
will compose a problem that is likely to be infeasible to solve.  
For example, consider the query $\textrm{UPDATE T SET a = 1 WHERE b} \in [0,5]$,
the corrupted query $\textrm{UPDATE T SET a = 1 WHERE b} \in [10,15]$,
and initial database state $t_1=(a=2, b=0), t_2=(2, 3), t_3=(2, 5), t_4=(2, 10), t_5=(2, 13)$.
The complete complaint set is thus $c_1((2,0), (1,0)), c_2((2,3), (1,3)), c_3((2,5), (1,5)), c_4((1,10), (2,10)), c_5((1,13), (2,14))$.
However, if the user only submits $c_{1,3,4,5}$ to \sys, then \sys would interpret $(2,3)$
as the correct state of $t_2$ and the correct repair $b \in [0, 5]$ would 
be interpreted as introducing a new error.   Thus the solver will simply return
an infeasibility error and not produce a candidate repair.

In turns out that the Tuple Slicing technique (Section~\ref{sec:tbsize})
can ameliorate this issue.  By only encoding the tuples in the incomplete complaint set,
the encoded problem does not have constraints on the query's effect on other tuples in the database.
In other words, it allows the result to generalize to tuples not in the complaint set.
The second iteration of the MILP execution then uses a soft constraint on the number of
non-complaint tuples that are affected by the repair in order to address the possibilty of over-generalization.

The second type of error are is false positives -- either erroneous complaints that are 
not actually errors, or complaints whose clean tuple $t^*$ is incorrect.
The MILP-based approach suffers from the same infeasibility challenges as in the incomplete complaint set case.
One option is to remove these errors by using one of numerous outlier detection algorithms~\cite{}.
Ultimately, however, we view this as an orthogonal pre-processing step that is worthy of its own study.
Thus, in our experiments, we focus on incomplete complaints sets and assume that
there are not erroneous complaints.


\if{0}
  \subsubsection{False Negatives}
  False negatives are cases when we don't have the full complaint sets, but
  what's provided in the complaint sets are guaranteed as correct. The 
  two-iteration approach in Section~\ref{sec:opt:tbsize} can handle 
  such cases. Refer to Section~\ref{sec:opt:tbsize} for detail.



  \subsubsection{False Positives}
  False positives are cases when we have incorrect information in the complaint sets 
  : they can be a falsely reported tuple which are actually 
  correct, or incorrect suggestions for the erroneous tuples. 
  Including such false positives in the MILP problem may result in a problem
  that is infeasible to solve. Thus, we want to detect and prune these false positives 
  beforehand. However, false positives are hard to 
  detect when there is a lack of information. 
  For example, if we only have few tuples in the complaint
  set, we cannot make any claim about which of them is a false positive 
  complaint. Thus, in this paper, we only considering false positive
  case when the ratio of the number of false positives to the number
  of correctly reported complaints is small.
  A obvious trend for false positives is that they normally are very 
  different or conflict
  with other complaints (Example~\ref{ex:false_positive_1}).
  \begin{example} \label{ex:false_positive_1}
  In Example~r\ref{ex:taxes2}, say there is 
  a false positive complaint on tuple $t_5$ which suggests the correct value
  for $t_5$ at database state $D_3$ should be 
  \{\textbf{ID}:$t_5$, \textbf{rate}: \color{red}{30}
  \color{black}{, \textbf{income}: \$5000, \textbf{owned}
  : }\color{red}{\$1500}\color{black}{\}}. To resolve this complaint, we have
  to guarantee the lower bound of the income range in $q_1$ as 5000. 
  However, the other complaints, $t_1, t_2, t_3$, suggest to
  move this lower bound to at least 9500.0001. In this case, the complaint
  $t_5$ conflicts with the all the other complaints and we may thus
  claim that $t_5$ is likely to be a false positive complaint. 
  \end{example}
  In this section, we introduce a \textbf{Pre-Processing} process that detects 
  false positives effectively. This pre-processing process first searches the 
  best log repair for each complaint separately, it then constructs 
  a bipartite graph between complaints and their impacted tuples and
  searches for the densest sub-graph of the bipartite graph, and finally 
  prunes complaints that are not in the densest sub-graph.
  \begin{itemize}
  \item Using algorithm described in Section~\ref{sec:opt} to solve each 
  complaint individually. Denote the log repair for complaint $c_i$ 
  as $\mathcal{Q}^*(c_i)$, and the impacted tuples of this log repair as
  $T_{c_i}$.
  \item Construct a bipartite graph $G = (\mathcal{C}, T, E)$, where 
  $T = \cup_{i} T_{c_i}$. Note that tuple with same primary key but modified 
  differently are treated as two separate vertices in the bipartite graph. 
  \item Search for densest sub-graph $G' = (\mathcal{C}', T', E')$ in $G$ 
  [] \xlw{cite some papers} and prune complaints in $\mathcal{C} - \mathcal{C}'$. 
  Density[] of a graph $G' = (\mathcal{C}', T', E')$
  is defined by $density(G') = \frac{|E'|}{|\mathcal{C}'|+|T'|}$. 
  \end{itemize}
  For achieve better performance, we can sample tuples
  from the database uniformly when constructing the bipartite graph $G$. 
  We demonstrate how to use this \textbf{Pre-Processing} 
  approach to prune false positive complaint(s) in Example~\ref{ex:false_positive_2}. 
  \begin{example}
  \label{ex:false_positive_2}
  Let's construct the bipartite graph between complaints $t_1, t_2, t_3, t_5$ 
  and tuples in the table for Example~\ref{ex:false_positive_1}. 
  $t_1$ suggests to move the range of income in $q_1$ 
  as $(9000, 10000]$. Similarly, $t_2, t_3, t_5$ suggest $[8570, 90000]$, 
  $[8570, 86000]$, and $[500, 10000]$ respectively. Let's assume tuples are
  uniformly distributed in these ranges. The bipartite graph is shown in 
  Figure~\ref{f:fp}. The density for $t_2, t_3$ is 1.9202 ($\frac{313}{163}$);
  density for $t_2, t_3, t_1$ is 1.9207 ($\frac{315}{164}$); density for 
  $t_1, t_2, t_3, t_5$ is 1.8232 ($\frac{330}{181}$). Thus, $t_5$ is pruned. 
  \begin{figure}[ht]
  \centering
  \includegraphics[width = 0.85\columnwidth]{figures/falsepositive_example}
  \caption{Bipartite graph of complaints and tuples for Example~\ref{ex:taxes2}. }
  \label{f:pf} 
  \end{figure}
  \end{example}

\fi

%\section{Heuristics}
\label{sec:heurstic}

\subsection{Attribute Slicing}

To further improve efficiency, we propose the attribute-slicing heuristic 
(Algorithm~\ref{alg:heu}) that may lose accuracy. This attribute-slicing
heuristic iteratively
generates the log repair by splitting the relevant 
attributes $Rel\mathcal{(A)}$ into
groups and fixing parameters involved 
in each group separately through a 
much smaller MILP problem. By 
controlling the number of attributes 
in each group, we bound the size of 
each MILP problem in the repair process.

\begin{algorithm}[htbp]
\caption{$AttributeSlicing$ heuristic.}
\label{alg:heu}
\begin{algorithmic}
\REQUIRE {$\mathcal{Q}, D_0, D_n, \mathcal{C}$}
\ENSURE {$\mathcal{Q^*}$}
\STATE $Rel\mathcal{(A)} \leftarrow FindRelAttr(\mathcal{Q}, D_n, \mathcal{C})$
\STATE $Rel\mathcal{(Q)} \leftarrow FindRelQuery(\mathcal{Q}, D_n, \mathcal{C})$
\STATE $\mathcal{G} \leftarrow SplitAttr(Rel\mathcal{(A)})$
\STATE $solved\_vals \leftarrow \emptyset$
\FOR {each $\partial (Rel\mathcal{(A)})$ in $\mathcal{G}$}
\STATE $\mathcal{Q'} \leftarrow PartialCF(\mathcal{Q}, \partial (Rel\mathcal{(A)}))$
\STATE $solved\_vals \leftarrow BasicApproach(\mathcal{Q'}, solved\_vals, ...)$
\ENDFOR
\STATE $\mathcal{Q}^* \leftarrow ConvertQLog(Q, solved\_vals)$
\STATE Return $\mathcal{Q}^*$
\end{algorithmic}
\end{algorithm}

In order to construct the sub MILP problem 
for a attribute group, we rewrite the conditional 
function, $f_q(t)$, into 
the partial conditional function format, $\partial (f_q(t))$.

\begin{definition} [Partial Conditional Function]
	The partial conditional function for a query $q$ over 
	attribute set $\partial(Rel\mathcal{(A)})$ is:
	\[
    \partial (f_q(t))= 
\begin{cases}
    \partial (f_{q.\mu} (t)) ,& \text{if } \partial (f_{q.\sigma} (t))\\
    t,              & \text{otherwise}
\end{cases}
\]
where
\begin{eqnarray*}
\partial (f_{q.\mu} (t)) &=& \{f|f\in f_{q.\mu}(t), \Pi_{f}(R) 
\in \partial (Rel\mathcal{(A)})\}\\
\partial (f_{q.\sigma} (t)) &=& \cup_{f \in f_{q.\sigma} (t)} \partial(f)
\end{eqnarray*}
Note that $\partial(f)$ is a transformation over a function 
(logical expression) $f$ defined
as following:
\[
\partial(f) = 
\begin{cases}
x,\ if\ \Pi_{f}(R) \cap \partial(Rel\mathcal{(A)}) = \emptyset\\
f, otherwise
\end{cases}
\]
$x$ is a newly introduced boolean variable. 
\end{definition} 
By doing so, we can easily construct the sub MILP problem 
in the same way as in the basic approach. Note that 
after each iteration, the solved variables, 
especially boolean variable $x$(s), are reused to provide 
constraint(s) for attributes that have not been examined. 
We demonstrate the process in Example~\ref{ex:heurstic}.

\begin{example}\label{ex:heurstic}
In this example, we demonstrate how to use Attribute-slicing heuristic to
solve the problem in Example~\ref{ex:taxes}. According to
Section~\ref{sec:opt}, we have:
\begin{eqnarray*}
\mathcal{F}(q_1) &=& \{rate, owed \} \\
\mathcal{F}(q_2) &=& \{owed\} \\
\mathcal{F}(q_3) &=& \{ID, rate, income, owed\}\\
Rel(\mathcal{A}) &=& \{ID, rate, income, owed\}
\end{eqnarray*}
Let us split the relevant attributes into groups: \\ 
$\{\{rate\}, \{owed\}, \{income\}, \{ID\}\}$.\\
Consider the first attribute group $\{rate\}$, we 
first rewrite the query log into $\partial(f_q(t))$ as the following:\\
\begin{minipage}{0.7\textwidth}
    \begin{minipage}[t]{0.2\textwidth}
        \begin{align*}
            f_{q_1.\mu} &=& \{\{rate = 30\}\}; \\
f_{q_2.\mu} &=& \emptyset;  \\
f_{q_3.\mu} &=& \{\{rate = 25\}\};
        \end{align*}
    \end{minipage}
    \hspace{4em}
    \begin{minipage}[t]{0.2\textwidth}
        \begin{align*}
           f_{q_1.\sigma} &=& x_1\\
            f_{q_2.\sigma} &=& true\\
           f_{q_1.\sigma} &=& x_3
        \end{align*}
    \end{minipage}
\end{minipage}

\smallskip

\indent We construct a sub MILP problem by linearizing
this query log over tuple $\{t_1, t_2, t_3, t_4\}$, parameterizing
the numeric variables, $30 and 25$, into $var1, var2$,
and setting the objective function as the minimum Euclidean
distance between modified numeric variable with original ones.
This sub MILP problem provides the following values: $var1 = 30, 
var2 = 25$;$t_1.x_1 = false, t_1.x_3 = false$; $ t_2.x_1 = true, 
t_2.x_3 = false$; $t_3.x_1 = false, t_3.x_3 = false$;
$t_4.x_1 = false; t_4.x_3 = true$.

\smallskip

In next iteration, we focus on attribute group $\{owed\}$ and find
the only numeric variable in $q_2$ should be $100$.
We then examine attribute group $\{income\}$, since we solved
$t_3.x_1 = false$ in the first iteration, the numeric 
variable $85700$ in the where clause of $q_1$ has to be 
changed into $86000+\epsilon$, where $\epsilon$
is a small number less than the minimum gap among values in
attribute $income$. Finally, we check attribute group
$\{ID\}$, and derive a log repair by revise $q_1$ as
\texttt{UPDATE Taxes Set rate = 30 WHERE income >= 86000+$\epsilon$ }. 
\end{example}







\section{Implementation}

\sys is implemented as a Java-based middleware in front of PostgreSQL.

Talk about tricks to encode problem intto CPLEX/decision tree?



%!TEX root = ../main.tex

%
% Notes: 
%
%  exact implementation
%  comparison with rollback at batch N
%  relaxing the solvers
%  dealing with false positives
%
%  running on TPC-C/sanjay's.  equality is easier
%
% Need names for
%  * rollback
%  * exact solution
%  * fixing an individual/batch of queries
%
\section{Experiments}

In this paper, we have described several heuristic algorithms based on decision trees 
and bounding box algorithms, as well as an exact, albeit less-scalable,
constraint-based solution to the \prob problem.  In addition, we 
introduced  several extensions to the CPLEX-based solution that 
1) improve the scalability of the system and 2) tolerate false positives and 
negatives in the input complaint set.
Our goals in the evaluation is to understand these trade-offs in
controlled synthetic scenarios, as well as study the effectiveness
in typical database query workloads based on widely used benchmarks.

To this end, our experiments are organized as follows: First, 
establish the quality limitations of existing heuristics and the need for a formal, 
constraint-based algorithm (\exact).  Second, we study how each of the 
optimizations described in Section~\ref{s:optimiztaions} improves algorithm scalability.
Third, we introduce different forms of error in the input complaint sets and study the 
effectiveness of our noise-handling heuristics.  Finally, we use two
representative database transaction benchmarks, TPC-C XXX~\cite{tpcc} and AuctionMark~\cite{auctionmark}
to study \sys in realistic scenarios.




%
% NOTE: figures are named <experimentsection>_<subsection>_<xaxis>.pdf
%

\subsection{Experimental Setup}


\begin{table}[t]\small
  \centering
  \begin{tabular}{c|l|c}
  {\bf Param} & {\bf Description} & {\bf Default} \\
  $V_d$  & Domain range of the attributes  & $[0, 100]$ \\
  $N_D$  & \# tuples in final database & $1000$ \\
  $N_a$  & \# attributes in database & $10$ \\
  $N_w$  & \# predicates in \texttt{WHERE} clauses & $1$ \\
  $N_s$  & \# \texttt{SET} clauses & $1$ \\
  $N_q$  & \# queries in query log & $100$ \\
  $idx$  & Index (backwards from most recent) & $\{0, 25, 50,$ \\
         & of corrupted query & $100, 200, 250 \}$ \\ %$\frac{N_q}{2}$ \\
  $r$    & Range size of \texttt{UPDATE} queries & 10 \\
  $s$    & Zipf $\alpha$ param of query attributes, & $1$ \\ 
  $set$  & Constant vs relative \texttt{SET} clauses. & const \\ \end{tabular}
         %& power low distribution $P(v) = v^{-s}$ & \\\end{tabular}
  \label{t:params}
  \caption{Experimental Parameters}
\end{table}


\begin{table}[t]\small
  \centering
  \begin{tabular}{c|l}
  {\bf Param} & {\bf Description} \\
  $TP$ & True positive rate: \% of true complaints corrected \\
  $FP$ & False positive rate: \% of errors introduced\\
  $t_{prep}$ & Time to construct CPLEX problem \\
  $t_{send}$ & Time to send CPLEX problem to solver \\
  $t_{solve}$ & Time for solver to generate a solutions\\
  $t_{total}$ & End-to-end execution time \\ 
  $d_{measure}$ & \red{Some sort of distance measure} \\\end{tabular}
  \label{t:metrics}
  \caption{Metrics Compared}
\end{table}



Each of our experiments follows a standard procedure.  
We generate a sequence of queries using a synthetic query generator or 
the benchmark program, and corrupt the query log as described below. 
We then execute the original and corrupt query logs on an initial (possibly empty) database,
and perform a tuple-wise comparison between the resulting database states 
to generate a true complaint set.  
We then add noise to the complaint set by 1) picking random tuples not in the true
complaint set to add false positive complaints, and 2) removing true complaints to simulate false negatives.
Finally, we execute the evaluated algorithms on the complaints and compare the fixed
query log with the true query log, as well as the fixed and true
final database states to measure performance and accuracy metrics.

Tables~\ref{t:params} and~\ref{t:metrics} summarize the key parameters that
we vary throughout our experiments, as well as the metrics we use to evaluate
the quality of the algorithm solutions, respectively.  The rest of this section
outlines our datasets, workloads, and parameters used to generate the synthetic query log.


\subsubsection{Datasets and Workloads}

This subsection describes the query and data generation process in greater detail.

% Anant's workload?

\stitle{TPC-C} We use the data and query workload over the {\it
CUSTOMER} table in TPC-C~\cite{}.  We generate a database at scale
1 with one warehouse, and keep only the queries that modify the
{\it CUSTOMER} table.  We then randomly perturb a subset of the
queries to generate the corrupted query log.
\xlw{Describe how corrupted}

\stitle{AuctionMark} \xlw{Describe benchmark and how queries are corrupted}



\stitle{Synthetic} 
We generate an initial database of $N_D$ random tuples.  
The schema contains $N_a=5$ attributes $a_1\ldots a_5$, whose values are
picked from $V_d$ uniformly at random, along with a primary key $id$.
We then generate a sequence of $N_q$ queries containing a mixture of \texttt{INSERT} queries,
point \texttt{UPDATE} queries, and range \texttt{UPDATE} queries.  
The \texttt{UPDATE} queries have the following respective forms, where \verb|?| is a query parameter 
and \verb|r| is the size of the range.

{\scriptsize
\begin{verbatim}
  UPDATE SET (a_i = ?),.. WHERE a_j = ? AND ...
  UPDATE SET (a_i = ?),.. WHERE a_j in [?, ?+r] AND ...
\end{verbatim}
}

The $set$ parameter controls whether the \texttt{UPDATED} queries set attributes to random constant values ({\it const}),  
or increment attributes by a random amount ({\it rel}).  The \texttt{WHERE} clauses form a conjunction.
In addition we varied a skew parameter $s$, which determines the attributes referenced in the \texttt{WHERE}
and \texttt{SET} clauses.  Each attribute in  a query is picked from a zipfian~\cite{zipf} distribution
with exponent $1+s$.  This allows our experiments to vary between a near-uniform distribution, where each attribute is
equally likely to be picked, and a skewed distribution where nearly all attributes are the same.
\texttt{INSERT} queries simply insert values picked uniformly at random from $V_d$.  

\xlw{Replace with what we actually do:
We first consider three different homogenous query logs: \texttt{INSERT} only ($p_I = 1$), 
\texttt{PK} update only ($p_I = 0, p_{pk} = 1$), and \texttt{RANGE} update only ($p_I = 0, p_{pk} = 0$).
These query logs help us understand \sys's performance characteristics for each query type individually.  
Finally, we investigate heterogenous mixtures of the three query types to simulate varying amounts of real settings.
}

\noindent {\it Effect of Index of Corrupted Query:}
A key parameter for our experiments is the location of the corrupted query ($idx$).  
This parameter deteremines the number of queries \sys must consider when searching for a fix,
and affects the size of the complaint set.  Both of these characteristics directly impact \sys's 
runtime performance. For this reason, it is undesirable to randomly pick and corrupt queries
throughout the query log, as the performance and accuracy results may not be comparable. 
To better understand the relationship between $idx$ and the size of the complaint set, we ran
simulations using a database with $20$ attributes, and a query log of size $1000$ 
that varied $idx$ for query logs containing o
nly $set = const$ and $set = rel$ \texttt{UPDATE} queries.
In addition, we varied the skew $s$ and range $r$ parameters to study how they affect this relationship.


  \begin{figure}[h]
  \centering
  \includegraphics[width = 3.5in]{figures/qidxsimulation/qidx_v_ncomplaints_20attrs_const}
  \caption{.}
  \label{f:qidx_v_ncomplaints} 
  \end{figure}

  \begin{figure}[h]
  \centering
  \includegraphics[width = 3.5in]{figures/qidxsimulation/qidx_v_ncomplaints_20attrs_rel}
  \caption{.}
  \label{f:qidx_v_ncomplaints} 
  \end{figure}



Figure~\ref{f:qidx_v_ncomplaints} compares the location of the corrupted query 
with the complaint set.  We find 


  \begin{figure}[h]
  \centering
  \includegraphics[width = 3.5in]{figures/qidxsimulation/numinrange}
  \caption{.}
  \label{f:numinrange} 
  \end{figure}



Within each query log, we independently vary the log size, the
database size, the complexity of the \texttt{WHERE} clause predicates, and
location of the query, and the number of attributes that are corrupted
(by replacing the constant with a random value within $[0, 100]$).






\subsection{Heuristics and Baseline Experiments}

\subsection{Skew}


\subsection{Scalabalitiy}




\subsection{Handling Noise}

In this set of experiments, we add false positive and false negative complaints to the input
complaint set and evaluate our noise handling algorithms (Section~\ref{sec:noise}).
To generate false positives ($FP$), we select a random set of tuples in the database, and for each one, 
pick a random attribute and set its XXX to a random value within $V_d$.
To generate false negatives ($FN$), we select a random subset of the true complaint set.


\begin{figure}[h]
\centering
  \begin{subfigure}[t]{.48\columnwidth}
  \includegraphics[width = .95\columnwidth]{figures/placeholder}
  \caption{.}
  \label{f:falsepositive} 
  \end{subfigure}
  \begin{subfigure}[t]{.48\columnwidth}
  \includegraphics[width = .95\columnwidth]{figures/placeholder}
  \caption{.}
  \label{f:falsenegative} 
  \end{subfigure}
  \caption{cap}

\end{figure}



Figure~\ref{f:falsepositive} shows \sys performance and quality as we vary $FP \in \{\}$.

Figure~\ref{f:falsenegative} varies $FN \in \{\}$.


\subsection{Benchmark Experiments}














\iffalse
\subsubsection{Algorithms}

\sys fixes query logs in two distinct steps: first, we filter the query log using 
provenance information and roll back the queries to compute the $ideal$ states of the database.
Second, we apply the solver algorithms (Section~\ref{}) to speculatively fix the query.

We consider the roll-back algorithm in batches of $n_{rollback}$.

\begin{itemize}
\item Combined:  combine rollback and query fixing in a single CPLEX problem
\item $R_n-CPLEX$: 
\item DT
\item Box,Density
\end{itemize}



\subsubsection{Comparison}

\begin{itemize}
\item Query By Example algorithm
\item Quoc's ConQueR
\end{itemize}

Conditions, given a database ${\cal D}$ and query log $qlog$:

\begin{itemize}
\item {\bf $N_\mathcal{D}$: } Size of the database (number of tuples)
\item {\bf $N_{dim}$:} Dimensionality of the database.
\item {\bf $N_\mathcal{Q}$:} Vary number of queries in $qlog$.
\item {\bf $N_{pred}$:} The number of predicates in each UPDATE query's WHERE condition.
\item {\bf $N_{ins}$: } When corrupting the log, the number of values in INSERT to corrupt.
\item {\bf $N_{set}$: } When corrupting the log, the number of clauses in SET that are corrupted.
\item {\bf $N_{where}$: } When corrupting the log, the number of attributes in the WHERE clause that are corrupted.
\item {\bf $idx \in [0, 1]$: } The index of the query in the query log that was corrupted as a percentage of the query log.  
      For example $Idx = 0.0$ is the oldest query in the log, whereas $Idx = 1.0$ is the most recent query position.
\item {\bf $p_{I}$: } Percentage of INSERT queries in the query log (as compared to UPDATEs).
\item {\bf $p_{pk}$: } Percentage of UPDATE queries with primary key filter clauses as compared to range clauses over non-primary key attributes.
\item {\bf $p_{FP}$: } Percentage of false positives in the complaint set.
\item {\bf $p_{FN}$: } Percentage of false negatives in the complaint set.
\end{itemize}



\subsection{Exact Experiments}

\begin{itemize}
\item $N_\mathcal{D} \in \{10, 100, 1000\}$
\item $N_q \in \{10, 20, 50, 100\}$
\item $N_{dim} = 4$
\item $N_{pred} \in \{1, 2, 3\}$
\item $N_{where} = \{1, 2\}$
\item $Idx = \{0, 0.5, 1\}$
\end{itemize}

\subsection{Rollback and \qfix Microexperiments}


The first set of experiments seeks to understand the effectiveness of the database rollback
algorithm.  We use a synthetic dataset (\ewu{describe}) and execute the rollback algorithm
while varying the query batch size.

\subsection{End-to-end experiments}

\subsubsection{Complete Complaint Set}

\subsubsection{Complaint Set with Noise}

\begin{itemize}
\item $N_\mathcal{D} \in \{10, 100, 1000\}$
\item $N_q \in \{10, 20, 50, 100\}$
\item $N_{dim} = 4$
\item $N_{pred} \in \{1, 2, 3\}$
\item $N_{attrs} = \{1, 2\}$
\item $Idx = \{0, 0.5, 1\}$
\end{itemize}

\subsection{TPC-C Experiment}


\deprecate{
  \subsection{Single-Query Log}

  In the first set of experiments, we evaluate the simplest case where there
  is a single update query.  In each experiment, we vary the DBSize,
  NClauses, as well as the number of clauses in the query that have
  been corrupted and report the metrics described above.  We first 
  compare the learning algorithms on a complete complaint set, then evaluate them
  using incomplete complaint sets with varying percentages of false positive and negative complaints.

  \subsubsection{Complete Complaints}

  {\it Vary DBSize

  Vary NClauses, corrupt 1 and 2 clauses
  }

  We found that CPLEX and BBOX identify the correct fix, however their
  running times are significantly higher than DTree.  This is because
  CPLEX is an exact solution, as compared to DTree, whose poor early
  splitting decisions can adversely affect the final tree structure.

  \subsubsection{Incomplete Complaints}


  {\it 
  Vary DBSize

  Vary NClauses, corrupt 1 and 2 clauses

  Each line is plots has different perc FP
  }

  We first increased the number of false positives in the complaint set (no false negatives).
  Figure~\ref{f:single_incomplete_fp} shows how the fix quality and running time vary as the
  percentage of false positives increases.   Compared variations of CPlex and Bounding box with varying
  density thresholds (?).

  Each line varies perc FN

  We then varied the number of false negatives while keeping the percentage of false positives fixed at 5\%.
  Figure~\ref{f:single_incomplete_fn} 


  \subsection{Increased Query Log Size}

  In the following set of experiments, we increase the number of
  queries in the log while varying ?.  The number of corrupted queries
  is still one.  In these experiments, we set the DBSize to 10000,
  the default NClauses to 4, and the number of corrupted clauses to
  2.   We first show results for varying the false positives and
  negatives in the complaint set and comparing the algorithms described
  in Section~\ref{s:incomplete-algs}.  We then evaluate the efficacy
  of of provenance-based query log filtering, which reduces the running
  time without affecting the result quality.

  To generate the false-positives, we randomly sample without replacement
  from the tuples in the database that are not in the true complaint
  set.

  \subsection{False Positive}

  Vary false positives (1 graph)


  \subsection{False Negative}

  Vary false negatives (1 graph)

  \subsubsection{Filtering Queries}

  We also compared the provenance-based filtering techniques in the above experiments
  to measure their effectiveness at reducing the running time.  We varied the complexity of the update 
  WHERE clauses to control the amount that queries in the log overlap in their updates.  The query log contained 50 update WHERE queries.
  The quality of the suggested fixes were the same,  As the clauses became less complex, the likelihood 
  of overlap increased, and increased the amount of queries that affected the complaint sets.


  \begin{figure}[h]
  \centering
  \includegraphics[width = 2in]{figures/complete_qfilter_complexity}
  \caption{Varying query complexity.}
  \label{f:complete_qfilter_complexity} 
  \end{figure}



  \subsection{Multi-Query}

  Using a previous experimental configuration, we varied the number of queries that are corrupted.  Figures~\ref{}
  show the quality and running times of the results for a query log of size 1000 and dbsize of 100k.  
  As we can see, the cost increases quadratically with each corrupted query and the accuracy of the proposed fixes increases marginally.  
  This is because XXX.

  We focus on two scenarios.  {\bf Try multiple corruptions that don't overlap (provenance-wise) with each other}.  This is the condition of multiple silo'd that
  corrupt their own logs.  Then {\bf Try multiple corrupted queries where one query modifies the updated state of the other}.  This shows that
  it is really hard and hopefully we get close?



  To better understand the algorithm, we plot the quality metrics after each fixed query and measure how quickly \sys converges to the final result. 
  This suggests that an incremental approach where the user can set a threshold to stop the algorithm may be effective.



  \subsection{Real Transactional Workload}

  We used the web application workload, and evaluated our alogirthms with artifically injected corruptions.
  We compared two types of corruptinos.  In figure~\ref{f:real_existing}, we randomly picked a single existing 
  query and corrupted its value.  If the query was an INSERT, we randomly pick a value and perturbed it.
  If the transaction was an UPDAET, we randomly varied the SET or WHERE clauses.   We re-ran this
  100 times and plot the average and standard deviation of the results.

}
\fi


\section{Summary and discussion}

In this paper, we presented \sys, a framework for diagnosing and
repairing data errors in the queries that operate on the data.
Datasets are typically dynamic; so even if a dataset starts clean,
updates on the data can introduce new errors. With \sys, we can
analyze query logs to trace reported errors to the queries that
introduced them. This allows us in turn to identify additional errors
in the data that may have been missed and gone unreported.

We proposed a basic algorithm, \naive, that uses non-trivial transformation rules to
encode information from the data and the query log as a MILP problem. We further improve 
the \naive algorithm by two types of optimizations: slicing-based optimizations that reduce problem
size without compromising the accuracy and 
incremental optimization that analyzes a single query at a time. 
Our experiments show that the latter
optimization can achieve significant scaling gains for single-query
errors, without significant reduction in accuracy.

To the best of our knowledge, \sys is the first approach to diagnosis
and repair of errors in queries. Obviously, correcting such errors in
practice poses additional challenges. 
The current version of \sys focuses on linear functions, so it may not
be able to encode complex updates that involve non-linear
computations. Future versions can target extended models and solvers
to represent richer scenarios. Finally, in our future work, we plan to
investigate additional methods of scaling the constraint analysis, as
well as techniques that can adapt the benefits of single-query
analysis to errors in multiple queries.







%!TEX root = ../main.tex

\section{Related Work}
\label{s:related}

\sys tackles the problem of diagnosis and repair in relational query
histories (query logs). This is in contrast to traditional data
cleaning~\cite{rahm00} which focuses on identifying and correcting
data ``in-place.'' \sys does not aim to correct errors in the data
directly, but rather to find the underlying reason for reported errors
in the queries that operated on the data.



Scorpion explanation uses data to synthesize predicates that explain.

Why not work.

View construction and query by example.

Take a look at temporal databases, the CIDR/arxiv paper has a few good starting points. Aditya also dug these out a while back


The way we return results is more like online aggregation or anytime algorithms -- 
as you run longer, the suggested fixes improve because we examine more of the query log.

\url{http://citeseerx.ist.psu.edu/viewdoc/download?doi=10.1.1.49.3765&rep=rep1&type=pdf}

\url{http://people.cs.aau.dk/~csj/Thesis/pdf/chapter26.pdf}

\balance

% Balancing columns in a ref list is a bit of a pain because you
% either use a hack like flushend or balance, or manually insert
% a column break.  http://www.tex.ac.uk/cgi-bin/texfaq2html?label=balance
% multicols doesn't work because we're already in two-column mode,
% and flushend isn't awesome, so I choose balance.  See this
% for more info: http://cs.brown.edu/system/software/latex/doc/balance.pdf
%
% Note that in a perfect world balance wants to be in the first
% column of the last page.
%
% If balance doesn't work for you, you can remove that and
% hard-code a column break into the bbl file right before you
% submit:
%
% http://stackoverflow.com/questions/2149854/how-to-manually-equalize-columns-
% in-an-ieee-paper-if-using-bibtex
%
% Or, just remove \balance and give up on balancing the last page.
%


\newpage
{
% If you want to use smaller typesetting for the reference list,
% uncomment the following line:
% \small
%\bibliographystyle{acm-sigchi}
\bibliographystyle{abbrv}
\bibliography{main}
}

\techreport{%!TEX root=../main.tex

\appendix

\section{Heuristic Approaches}
\label{sec:heuristic}

In this
section, we examine alternative, simpler models that process a single
query at a time, and demonstrate why they are insufficient.

\noindent
\textbf{WHERE repairs through classification:}
The \texttt{WHERE} clause of an update query is equivalent to a
rule-based binary classifier that splits tuples into two groups:
(1)~tuples that satisfy the conditions in the \texttt{WHERE} clause
and (2)~tuples that do not. Thus, by training a classifier,
such as decision trees \cite{quinlan1987} to learn
the correct classification rules rules for the \texttt{WHERE} clause.

\noindent
\textbf{SET repairs:}
This alternative approach constructs 
a simple linear system of equations to solve the parameters in the \texttt{SET}
when errors persist after fixing the \texttt{WHERE} clause:
For each expression in the \texttt{SET} clause we create a
linear equation, using unknown variables to represent any parameters
in the \texttt{SET} expression. 
  \begin{figure}[h]
  \centering
    \includegraphics[width = .6\columnwidth]{figures/heuristicacc}
    \vspace*{-.1in}
    \caption{Heuristic Approach vs. \sys on Single-Query. }
    \label{f:heuristic_acc} 
  \end{figure}
  \vspace*{-0.1in}
  
The na\"ive approach that we just described is heuristic in nature. It
is simple and fast, but it can only process a single incorrect query. As shown in
Figure~\ref{f:heuristic_acc}, the F-1 score of na\"ive heuristic approach is less 
than 0.6 while \sys maintain high accuracy in solving single query problem with
above 0.9 F-1 score across all database sizes. 

  
\iffalse
\xlw{CUT FROM THIS POINT}
\section{Effect of Index of Corrupted Query}
\label{app:qidx}

A key parameter for our experiments is the location of the corrupted query ($idx$).  
\alex{Have we discussed anywhere yet that we focus on single errors?}
This parameter determines the number of queries \sys must consider when searching for a fix,
and affects the size of the complaint set.  
\alex{It won't be clear to the reader how this relates to the size of the complaint set.}
Both of these characteristics directly impact \sys's 
runtime performance. For this reason, it is undesirable to randomly pick and corrupt queries
throughout the query log, as the performance and accuracy results may not be comparable. 
To better understand the relationship between $idx$ and the size of the complaint set, we ran
simulations using a database with $20$ attributes, and a query log of size $1000$ containing
either all $set = const$ or $set = rel$ \texttt{UPDATE} queries.
We varied  $idx$ uniformly throughout the query log, and additionally varied
the skew $s$ and range $r$ parameters to study how they affect the size of the complaint sets.


  \begin{figure}[h]
  \centering
  \includegraphics[width = 3.5in]{figures/qidxsimulation/qidx_v_ncomplaints_20attrs_const}
  \caption{Query index vs complaint set size for $set = const$.}
  \label{f:qidx_v_ncomplaints_const} 
  \end{figure}


Figure~\ref{f:qidx_v_ncomplaints_const} plots a representative set of parameters.  We plot one point
for each corrupted query index that results in a complaint set with at least one complaint. 
These results highlight several interesting trends.  When queries do not overlap ($r = 1$, leftmost column),
the size of the complaint sets are relatively small, and their frequency is constant across the possible query indices.
However as the possibility of overlap increases (e.g., $r$ increases), more recent queries are more likely to result in
very large complaint sets (at times the size of the database).   
This effect is a symptom of the fact that queries with large ranges will set groups of tuples to the same value,
and over time, skew the distribution of tuple values to a small number of possible values.
Thus, more recent corruptions that affect a large cluster of similar tuples will result in a large complaint set.
We find that increasing the skew parameter also exacerbates this effect.  
In addition, high skew increases the likelihood that queries will share the same \texttt{WHERE} and \texttt{SET} clause 
attributes as a corrupted query, thus overwriting the error introduced by the corrupted query.  
This is why the frequency of non-empty complaint sets decreases significantly as $s$ increases.


\begin{figure}[h]
\centering
\includegraphics[width = 3.5in]{figures/qidxsimulation/qidx_v_ncomplaints_20attrs_rel}
\caption{Query index vs complaint set size for $set = rel$.}
\label{f:qidx_v_ncomplaints_rel} 
\end{figure}

In contrast to $set=const$ queries, Figure~\ref{f:qidx_v_ncomplaints_rel} executes the 
same experiment using $set=rel$ queries.  In this setting, we find that the trend is
reversed, and older corruptions tend to result in larger complaint sets.  This is because,
subsequent \texttt{UPDATE} queries increment or decrement the attribute value, rather than
overwriting it with a constant value.  The clustering of data values due to query overlap
then increases the number of other tuples affected.


\ewu{summarize findings and implications to experiments here.}
not all corruptions result in complaint sets.
In constant SET clause workloads, larger complaints sets are more likely to
result from more recent corrupted queries -- particularly if the queries are range updates or
the updated attributes are skewed.
For this reason, our experiments corrupt the query log at six positions 
$idx \in \{0, 25, 50, 100, 200, 250\}$ , relative 
to the most recent query (e.g., the most recent query, the $25^th$ most recent query, and so on).

% \begin{figure}[h]
% \centering
% \includegraphics[width = 3.5in]{figures/qidxsimulation/numinrange}
% \caption{.}
% \label{f:numinrange} 
% \end{figure}


As we observed from Figure~\ref{f:multiquery}, \milpall maintains high accuracy when errors
happen more recent, however it does not scale when the error locate further from the most
recent query. \milptuple scales better than \milpall, but ignoring tuples not 
in the complaint set apparently hurts the precision. \milptuplestopearly run times faster
than \milpall and \milptuple, however the aggressive strategy greatly reduce the 
precision. In the end, \milpadvtuple significantly improves the precision with very limited
time cost compare to \milptuple. \\
Based on these observations, we only include the performance of \milpadvtuple and \milpadvall
in the rest of the experiments. 

%!TEX root = ../main.tex

\subsection{A plausible (but bad) alternative}
\label{sec:dt}

% \sys analyzes query logs and complaints by producing a mathematical
% formulation of the constraints that need to be satisfied. The
% constraint problem can then be evaluated by dedicated external tools.

The MIP models generated by \sys can grow large as the sizes of the
data and the log increase. However, modeling all present constraints
from the beginning to the end of the log is necessary; in this
section, we examine alternative, simpler models that process a single
query at a time, and demonstrate why they are insufficient.

\smallskip
\noindent
\textbf{WHERE repairs through classification:}
The \texttt{WHERE} clause of an update query is equivalent to a
rule-based binary classifier that splits tuples into two groups:
(1)~tuples that satisfy the conditions in the \texttt{WHERE} clause
and (2)~tuples that do not. A mistake in a query predicate can then
result in misclassification: some tuples get classified into the wrong
group, which in turn translates to errors in the data. Therefore,
repairing the mistake corresponds to repairing the imprecise
classification. This works as follows: For an incorrect query $q$, let
$D_0$ be the database state before $q$, and $D_1^*$ the \emph{correct}
database state that should result after $q$.
We use each tuple $t \in D_0$ as an element in the input training data
for the classifier where the values (of each attribute) of $t$ define
the feature vector and the label for $t$:
	\[
    label(t)= 
    \begin{cases}
    true ,& D_0.t \neq D_1^*.t\\
    false,              & \text{otherwise}
    \end{cases}
\]
We then train a classifier, such as decision trees \cite{???} to learn
the correct classification rules rules for the \texttt{WHERE} clause.


\smallskip
\noindent
\textbf{SET repairs:}
After repairing the \texttt{WHERE} clause through learning a
rule-based classifier, some complaints may still persist. This
indicates a possible error in the \texttt{SET} clause. The errors can
be modeled and solved by constructing a simple linear system of
equations: For each expression in the \texttt{SET} clause we create a
linear equation, using unknown variables to represent any parameters
in the \texttt{SET} expression. Solving for these variables then
provides a repair for the \texttt{SET} expression.


\smallskip
\noindent
\textbf{Why it does not work:}
The na\"ive approach that we just described is heuristic in nature. It
is simple and fast, but it can only process a single incorrect query.
This results in several shortcomings that make it insufficient in
practice:
\begin{itemize}[itemsep=1pt, leftmargin=5mm]
    
\item In principle, one could attempt to apply this technique to the
entire log one-query-at-a-time. However, this is not possible in
practice: to learn a classifier on the \texttt{WHERE} clause of query
$q_i$, one needs to know the correct classification output, which
corresponds to $D_i^*$. Unfortunately, even with a complete complaint
set, which can derive the correct database $D_n^*$, there is no
obvious way to ``rollback'' this state to derive $D_i^*$.

\item The classifier may derive a clause that is structurally very
different from the original one (different attributes or number of
conditions). This is problematic in general, as it corresponds to a
larger-scale mistake in the query, which is a less likely scenario.

\item Classifiers try to avoid overfitting, which is problematic for
queries with high selectivity (e.g., single-tuple updates), as the
classifier is unlikely to generate any rules.

\end{itemize}


Therefore, while examining one query at a time superficially appears
to be a reasonable and efficient alternative, the reality is that one
has to model all constraints and transformations through the entire
log history. In the following section, we propose several
optimizations to our initial approach that make scaling to large data
and log sizes feasible. 
% \red{Add graph comparing naive and d-trees here.}


\iffalse
In this
section, we examine alternative, simpler models that process a single
query at a time, and demonstrate why they are insufficient.

\noindent
\textbf{WHERE repairs through classification:}
The \texttt{WHERE} clause of an update query is equivalent to a
rule-based binary classifier that splits tuples into two groups:
(1)~tuples that satisfy the conditions in the \texttt{WHERE} clause
and (2)~tuples that do not. Thus, by training a classifier,
such as decision trees \cite{quinlan1987} to learn
the correct classification rules rules for the \texttt{WHERE} clause.

\noindent
\textbf{SET repairs:}
This alternative approach constructs 
a simple linear system of equations to solve the parameters in the \texttt{SET}
when errors persist after fixing the \texttt{WHERE} clause:
For each expression in the \texttt{SET} clause we create a
linear equation, using unknown variables to represent any parameters
in the \texttt{SET} expression. 
  \begin{figure}[h]
  \centering
    \includegraphics[width = .6\columnwidth]{figures/heuristicacc}
    \vspace*{-.1in}
    \caption{Heuristic Approach vs. \sys on Single-Query. }
    \label{f:heuristic_acc} 
  \end{figure}
  \vspace*{-0.1in}
  
The na\"ive approach that we just described is heuristic in nature. It
is simple and fast, but it can only process a single incorrect query. As shown in
Figure~\ref{f:heuristic_acc}, the F-1 score of na\"ive heuristic approach is less 
than 0.6 while \sys maintain high accuracy in solving single query problem with
above 0.9 F-1 score across all database sizes. 
\fi

\fi
}









\end{document}
