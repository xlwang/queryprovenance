%!TEX root = ../main.tex

\section{Modeling abstractions}
\label{sec:abstractions}

In this section, we introduce a running example inspired from the use-case of
Example~\ref{ex:taxes}, and describe the model abstractions that we use to
formalize the diagnosis problem.





\begin{example}\label{ex:taxes2}
Figure~\ref{fig:example} demonstrates an example tax bracket adjustment in the
spirit of Example~\ref{ex:taxes}. The adjustment sets the tax rate to 30\% for
income levels above \$87,500, and is implemented by query $q_1$. A digit
transposition mistake in the query, results in an incorrect owed amount for tuples
$t_3$ and $t_4$. Query $q_2$, which inserts a tuple with slightly higher
income than $t_3$ and $t_4$ and the correct information, obscures this mistake.
This mistake is further propagated by query $q_3$, which calculates the pay 
check amount based on the corresponding
income and owed. 
\iffalse    
Figure~\ref{fig:example} demonstrates an example tax bracket adjustment in the
spirit of Example~\ref{ex:taxes}. The adjustment sets the tax rate to 30\% for
income levels above \$87,500, and is implemented by query $q_1$. A digit
transposition mistake in the query, results in an incorrect tax rate for tuples
$t_3$ and $t_4$. Query $q_2$ that calculates the amount owed based on the corresponding
tax rate and income propagates the error to other fields. The mistake is
further obscured by query $q_3$, which inserts a tuple with slightly higher
income than $t_3$ and $t_4$ and the correct (lower) tax rate.
\fi
\end{example}
\vspace*{-0.07in}
While traditional data cleaning techniques seek to identify and correct the
erroneous values in the table \emph{Taxes} directly, our goal is to diagnose
the problem, and understand the reasons for these errors. In this case, the
reason for the data errors is the incorrect predicate value in query $q_1$.

In this paper, we assume that we know \emph{some} errors in the dataset, and
that these errors were caused by erroneous updates. The errors may be
obtained in different ways: traditional data cleaning tools may identify
discrepancies in the data (e.g., a tuple with lower income has higher owed tax
amount), or errors can be reported directly from users (e.g., customers
reporting discrepancies to customer service). \emph{Our goal is not to correct
the errors directly in the data, but to analyze them as a ``symptom'' and provide a
diagnosis.} The diagnosis can produce a targeted treatment: knowing how the
errors were introduced guides the proper way to trace and resolve them.


%!TEX root=../main.tex

\begin{figure}[t]
\centering
{\small
\begin{tabular}{ll}
    \toprule
    \textbf{Notation} & \textbf{Description}\\
    \midrule
    $\mathcal{Q}$& The sequence of executed update queries (log)\\ 
             & $\mathcal{Q}=\{q_1, \dots, q_n\}$ \\
    $D_0$    & Initial database state at beginning of log\\
    $D_n$    & End database state (current) $D_n=\mathcal{Q}(D_0)$\\
    $D_i$    & Database state after query $q_i$: $D_i=q_i(\dots q_1(D_0))$\\
    $c: t\mapsto t^*$ & Complaint: $\mathcal{T}_c(D) = (D_n\setminus\{t\})\cup\{t^*\}$\\
    $\mathcal{C}$ & Complaint set $\mathcal{C}=\{c_1,\dots,c_k\}$\\
    $\mathcal{Q}^*$& Diagnosis or log repair\\
    $d(\mathcal{Q}, \mathcal{Q}^*)$ & Distance functions between two query logs\\
    \bottomrule
\end{tabular}
}
\caption{Summary of notations used in the paper. \alex{to update once finalized}}
\label{tbl:notation}
\end{figure}

\subsection{Error Modeling}
\label{sec:model}

In our setting, the diagnoses are associated with errors in the queries that
operated on the data. In Example~\ref{ex:taxes2}, the errors in the dataset
are due to the digit transposition mistake in the WHERE clause predicate of
query $q_1$. Our goal is to infer the errors in a log of queries
automatically, given a set of incorrect values in the data. We proceed to
describe our modeling abstractions for data, queries, and errors, and how we
use them to define the diagnosis problem.

\subsubsection*{Data and query models}
\label{sec:models}

\noindent
\emph{Query log ($\mathcal{Q}$):}
We define a query log that update the database 
as an ordered sequence of \texttt{UPDATE}, \texttt{INSERT}, and
\texttt{DELETE} queries $\mathcal{Q}=\{q_1,\dots,q_n\}$, that have
operated on a database $D$. In the rest of the paper, we use the term
\emph{update queries}, or just \emph{queries}, to refer to any of the queries in $\mathcal(Q)$,
including insertion and deletion queries.

\smallskip
\noindent
\emph{Query ($q_i$):} We model each query as a function over a
database $D$, resulting in a new database $D'$. For \texttt{INSERT}
queries, $D'=q(D)=D\cup\{t_{new}\}$.
We model \texttt{UPDATE} and \texttt{DELETE} queries as follows:  
\begin{align*}
    D'=q(D)= &\{\mu_{q}(t)\;|\;t\in D, \sigma_{q}(t)\}
    \cup\{t\;|\;t\in D, \neg\sigma_{q}(t)\}
\end{align*}
In this definition, the modifier function $\mu_q(t)$ represents the query's update equations, and it transforms a tuple by either deleting it ($\mu_q(t)=\bot$) or changing the values of some of its attributes.
The conditional function $\sigma_q(t)$ is a boolean function that represents the query's condition predicates.  In the example of Figure~\ref{fig:example}:
\begin{align*}
    &\mu_{q_1}(t)=(t.income, t.income*0.3, t.pay)\\
    &\sigma_{q_1}(t)=(t.income\ge 85700)\\
    &\mu_{q_3}(t)=(t.income, t.owed, t.income-t.owed)\\
    &\sigma_{q_2}(t)=\texttt{true}
\end{align*} 
\iffalse
\begin{align*}
    &\mu_{q_1}(t)=(30, t.income, t.owed)\\
    &\sigma_{q_1}(t)=(t.income\ge 85700)\\
    &\mu_{q_2}(t)=(t.rate, t.income, \frac{t.income\cdot t.rate}{100})\\
    &\sigma_{q_2}(t)=\texttt{true}
\end{align*} 
\fi

Note that in this paper, we do not consider sub-queries or aggregation. We clarify all assumptions later in this section.

\smallskip
\noindent
\emph{Database state ($D_i$):}
We use $D_i$ to represent the state of a database $D$ after the application of
queries $q_1$ through $q_i$ from the log $\mathcal{Q}$. $D_0$ represents the
original database state, and $D_n$ the final, or current, database state. Out
of all the states, the system only maintains $D_0$ and $D_n$. In practice,
$D_0$ can be a checkpoint: a state of the database that we assume is correct;
we cannot diagnose errors before this state. The intermediate states can be
derived by executing the log: $D_i=q_i(q_{i-1}(\dots q_1(D_0)))$. We also
write $D_n=\mathcal{Q}(D_0)$ to denote that the final database state $D_n$ can
be derived by applying the sequence of queries in the log to the original
database state $D_0$.

\smallskip
\noindent
\emph{True database state ($D_i^*$):}
Queries in $\mathcal{Q}$ are possibly erroneous, introducing errors in the
data. There exists a sequence of \emph{true} database states $\{D_0^*,
D_1^*\dots, D_n^*\}$, with $D_0^*=D_0$, representing the database states that
would have occurred if there had been no errors in the queries.
The true database states are unknown; our goal is to find and correct the errors in $\mathcal{Q}$ and retrieve the correct database state $D_n^*$.

For ease of exposition, in the remainder of the paper we assume that the
database contains a single relation with attributes $A_1,\ldots,A_m$,
but the single table is not a requirement in our framework.


\subsubsection*{Error models}

Following the terminology in Example~\ref{ex:taxes}, we model a set of identified or user-reported
data errors as \emph{complaints}. A complaint corresponds to a
particular tuple in the final database state $D_n^*$, and identifies
that tuple's correct value assignment. We formally define complaints
below:
\begin{definition}[Complaint]\label{def:complaint}
    A complaint $c$ is a mapping between two tuples: $c: t\mapsto t^*$, such that $t$ and $t^*$ have the same schema, $t, t^*\in D_n\cup\{\bot\}$, and $t\neq t^*$. A complaint defines a
    transformation $\mathcal{T}_c$ on the final database state $D_n$: $\mathcal{T}_c(D_n)
    = (D_n\setminus\{t\})\cup\{t^*\}$, which replaces $t$ in $D_n$ with $t^*$.
\end{definition}

In the example of Figure~\ref{fig:example}, two complaints are reported on the final database state $D_3$: 
$c_1: t_3\mapsto t_3^*$ and
$c_2: t_4\mapsto t_4^*$, 
where $t_3^*=(86000,21500,64500)$ and $t_4^*=(86500,21625,\linebreak 64875)$. 
For both these cases, each complaint denotes a \textbf{value correction} for a tuple in $D_3$.  Complaints can also model the \textbf{addition} or \textbf{removal} of tuples: $c: \bot\mapsto t^*$ means that $t^*$ should be added to the database, whereas $c: t\mapsto \bot$
means that $t$ should be removed from the database.


Note that in this definition, a complaint needs to specify the exact values
for all fields of a tuple. However, this is done for ease of exposition, and
it is not an inherent restriction in our approach. The definitions and our
algorithms can trivially generalize to handle complaints where only some of the
correct tuple values are known.


% \xlw{Ideally, a complaint should cover a \textbf{complete value correction} such that the exact fix value for all attributes are included. This is because a complete value correction helps reducing the searching space and improves the diagnosis quality. In this paper, we use complaints with complete value correction in our experiments, however, the proposed algorithm can be easily extended to handle problems with missing value correction. }

\smallskip
\noindent
\emph{Complaint set ($\mathcal{C}$):}
We use $\mathcal{C}$ to denote the set of all known complaints
$\mathcal{C}=\{c_1,\dots,c_k\}$, and we call it the \emph{complaint set}.
Each complaint in $\mathcal{C}$ represents a transformation (addition,
deletion, or modification) of a tuple in $D_n$. We assume that the
complaint set is consistent, i.e., there are no two complaints that
propose different transformations to the same tuple $t\in D_n$.
Applying all these transformations to $D_n$ results in a new database
instance
$D_n'=\mathcal{T}_{c_1}(\mathcal{T}_{c_2}(\dots\mathcal{T}_{c_k}(D_n)))$.\footnote{Since
the complaint set is consistent, it is easy to see that the order of
transformations is inconsequential.} $\mathcal{C}$ is \emph{complete}
if it contains a complaint for each error in $D_n$. In that case,
$D_n'=D_n^*$. In our work, we do not assume that the complaint set is
complete, but, as is more common in practice, we only know a subset of
the errors (incomplete complaint set). Further, we focus our analysis
on \emph{valid} complaints; we briefly discuss dealing with invalid
complaints (complaints identifying a correct value as an error) in
Section~\ref{sec:noise}, but these techniques are beyond the scope of this paper.

\smallskip
\noindent
\emph{Log repair ($\mathcal{Q}^*$):}
The goal of our framework is to derive a diagnosis as a log repair
$\mathcal{Q}^*=\{q_1^*,\dots, q_n^*\}$, such that
$\mathcal{Q}^*(D_0)=D_n^*$. In this work, we focus on errors produced
by incorrect parameters in queries, so our repairs focus on altering
query constants rather than query structure. Therefore, for each query
$q_i^*\in\mathcal{Q}^*$, $q_i^*$ has the same structure as $q_i$
(e.g., the same number of predicates and the same variables in the \texttt{WHERE} clause), 
but possibly different parameters. For example, a good log repair for the
example of Figure~\ref{fig:example} is
$\mathcal{Q}^*=\{q_1^*,q_2,q_3\}$, where $q_1^*$=\texttt{UPDATE Taxes
SET owed=income*0.3 WHERE income >= 87500}.


\subsubsection*{Problem definition}

We now formalize the problem definition for diagnosing data
errors using query logs. A diagnosis is a log repair
$\mathcal{Q}^*$ that resolves all complaints in the set $\mathcal{C}$
and leads to a correct database state $D_n^*$.
\begin{definition}[Optimal diagnosis]\label{def:problem}
    Given database states $D_0$ and $D_n$, a query log $\mathcal{Q}$ such that $\mathcal{Q}(D_0)=D_n$, a set of complaints $\mathcal{C}$ on $D_n$,  and a distance function $d$, the optimal diagnosis is a log repair $\mathcal{Q}^*$, such that:
    \begin{itemize}[itemsep=0pt, parsep=0pt, topsep=1pt]
        \item $\mathcal{Q}^*(D_0)=D_n^*$, where $D_n^*$ has no errors
        \item $d(\mathcal{Q}, \mathcal{Q}^*)$ is minimized
    \end{itemize}
\end{definition}

More informally, we seek the minimum changes to the log $\mathcal{Q}$
that would result in a clean database state $D_n^*$. Obviously, a
challenge is that $D_n^*$ is unknown, unless we know that the
complaint set is complete. 

\subsubsection*{Problem scope and solution outline}
In this work, we focus on handling data manipulation statements (\texttt{UPDATE}, \texttt{INSERT}, and
\texttt{DELETE} queries) with simple, basic query structures without subqueries, aggregations, or joins.
Expressions (in predicates and SET clauses) may be over linear combinations of constants and a single attribute, and we do not support arbitrary user defined functions.
We find that the queries that we focus on are applicable to a broad range of user-facing web applications (e.g., conference attendee scheduling, voting,  data entry) and OLTP benchmarks, and that complex query structures and operations are less common than in read-oriented analytical workloads.

Given a query log and a set of complaints, \sys proposes repairs as
modifications of values in one or more queries. \sys does not modify the
structure of queries, and makes the implicit assumption that the log starts
either with an empty or a clean database.
We demonstrate that \sys can solve problems with corruptions in multiple queries, but its scalability in this setting is limited (up to about 50 queries in the log).  For cases where corruptions are restricted to a single query, \sys can scale to large data and log sizes.

% \xlw{In this paper, we rely on complaint set with valid information and mainly target at solving workloads with a single error. At this point, we plan to focus our future work on solving more complex workloads with multiple errors and inconsistent complaints efficiently and effectively.} 
%We study the impact of the number of predicates in the WHERE clause in Section~\ref{sec:experiments:synth}.



In Section~\ref{sec:sol}, we describe our basic method, 
which
uses a constraint programming formulation that expresses this
diagnosis problem as a mixed integer linear program (MILP). 
Section~\ref{sec:opt} presents several optimization
techniques that extend the basic method, allowing \sys to 
(1)~handle cases of incomplete information (incomplete complaint set), and
(2)~scale to large data and log sizes. 
Specifically, the fully optimized, incremental algorithm (Section~\ref{sec:incremental}), can
handle query logs with hundreds of queries within minutes, while the performance of the basic approach collapses by $50$ queries.

\iffalse Due to space considerations, we omit discussion of alternative approaches that use classification tools and linear systems of equations.
These approaches are limited to a query log containing a single query, and are discussed and evaluated in more detail in our technical report~\cite{qfixarxiv}. 
\fi




