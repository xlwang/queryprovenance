%!TEX root = main.tex

\documentclass{sig-alternate-05-2015}
% \usepackage{paralist}
% Load basic packages
\usepackage{balance}  % to better equalize the last page
\usepackage{graphicx} % for EPS, load graphicx instead
\usepackage{url}      % llt: nicely formatted URLs
\usepackage{amsmath}
\usepackage{color}
\usepackage{cancel}
\usepackage{listings}
%\usepackage{wrapfig}
\usepackage{graphicx}
\usepackage{subcaption}
\setlength{\abovecaptionskip}{10pt plus 3pt minus 2pt}
% \usepackage[english]{babel}
\usepackage{booktabs}
\usepackage{graphicx}
% \usepackage{subfigure}
% \usepackage{caption}
% \usepackage[scriptsize, it, IT]{subfigure}
% \usepackage[font={scriptsize,it}]{caption}
% \usepackage{tikz}
%\usepackage{MinionPro}
% \usetikzlibrary{arrows,matrix,positioning}
\usepackage[ruled,vlined,algonl,boxed]{algorithm2e}
\usepackage{algorithmic}
\usepackage{wrapfig}
% \usepackage{framed}
\usepackage{enumitem}
\usepackage{xspace}
\usepackage{xcolor}
\usepackage{colortbl}
% \usepackage{newtxmath}
\usepackage[T1]{fontenc}

\usepackage[nocompress]{cite}
\usepackage{microtype}

% llt: Define a global style for URLs, rather that the default one
\makeatletter
\def\url@leostyle{%
  \@ifundefined{selectfont}{\def\UrlFont{\sf}}{\def\UrlFont{\small\bf\ttfamily}}}
\makeatother
\urlstyle{leo}

\makeatletter
%\def\@copyrightspace{\relax}
\makeatother

% To make various LaTeX processors do the right thing with page size.
\def\pprw{8.5in}
\def\pprh{11in}
\special{papersize=\pprw,\pprh}
\setlength{\paperwidth}{\pprw}
\setlength{\paperheight}{\pprh}
\setlength{\pdfpagewidth}{\pprw}
\setlength{\pdfpageheight}{\pprh}

\newtheorem{definition}{Definition}%[section]
\newtheorem{proposition}[definition]{Proposition}
\newtheorem{lemma}[definition]{Lemma}
\newtheorem{remark}[definition]{Remark}
\newtheorem{corollary}[definition]{Corollary}
\newtheorem{claim}[definition]{Claim}
\newtheorem{theorem}[definition]{Theorem}
\newtheorem{heuristic}[definition]{Heuristic}
\newtheorem{example}[definition]{Example}
%\newtheorem{proof}[definition]{Proof}
\newtheorem{dimension}{Dimension}
\newcounter{prob}
\newtheorem{problem}[prob]{Problem}
\newtheorem{conjecture}[definition]{Conjecture}
\newtheorem{reduction}[definition]{Reduction}
\newtheorem{property}[definition]{Property}
\newtheorem{axiom}[definition]{Axiom}
% 
% \tikzset{
%     %Define standard arrow tip
%     >=stealth',
%     %Define style for boxes
%     punkt/.style={
%            rectangle,
%            rounded corners,
%            draw=black, very thick,
%            text width=6.5em,
%            minimum height=2em,
%            text centered},
%     % Define arrow style
%     pil/.style={
%            ->,
%            thick,
%            shorten <=2pt,
%            shorten >=2pt,}
% }


% \setitemize{noitemsep,topsep=0pt,parsep=0pt,partopsep=0pt}

% \def\compactify{\itemsep=0pt \topsep=0pt \partopsep=0pt \parsep=0pt}
% \let\latexusecounter=\usecounter
% \newenvironment{CompactItemize}
%   {\def\usecounter{\latexusecounter}
%    \begin{itemize}[noitemsep,topsep=0pt,parsep=0pt,partopsep=0pt,leftmargin=*]}
%   {\end{itemize}\let\usecounter=\latexusecounter}
% \newenvironment{CompactEnumerate}
%   {\def\usecounter{\compactify\latexusecounter}
%    \begin{enumerate}[leftmargin=*]}
%   {\end{enumerate}\let\usecounter=\latexusecounter}


%%%  What is this?  Use enumitem instead
% 
% \newcommand{\squishlist}{
%    \begin{list}{$\bullet$}
%     { \setlength{\itemsep}{0pt}
%       \setlength{\parsep}{2pt}
%       \setlength{\topsep}{6pt}
%       \setlength{\partopsep}{0pt}
%       \leftmargin=25pt
% \rightmargin=0pt
% \labelsep=5pt
% \labelwidth=10pt
% \itemindent=0pt
% \listparindent=0pt
% \itemsep=\parsep
%     }
% }
% \newcommand{\squishend}{\end{list}}
% 
% 
% \newcommand{\squishframe}{\vspace{-6pt}
% \begin{framed} 
% \vspace{-6pt}}
% 
% \newcommand{\frameend}{\vspace{-6pt}
% \end{framed}
% \vspace{-6pt}}


% create a shortcut to typeset table headings
% \newcommand\tabhead[1]{\small\textbf{#1}}




% Make sure hyperref comes last of your loaded packages, 
% to give it a fighting chance of not being over-written, 
% since its job is to redefine many LaTeX commands.
\usepackage[pdftex]{hyperref}
\hypersetup{
  colorlinks=false,
  linkcolor=darkred,
  citecolor=darkgreen,
  urlcolor=darkblue
}
% \usepackage{cleveref} % After hyperref, listings



% Avoid widows and orphans
\clubpenalty=10000 
\widowpenalty = 10000

% % Aggressive figure placement
% \renewcommand{\topfraction}{0.9}
% \renewcommand{\bottomfraction}{0.8}
% \setcounter{topnumber}{2}
% \setcounter{bottomnumber}{2}
% \setcounter{totalnumber}{4}
% \setcounter{dbltopnumber}{2}
% \renewcommand{\dbltopfraction}{0.9}
% \renewcommand{\textfraction}{0.07}
% \renewcommand{\floatpagefraction}{0.7}
% \renewcommand{\dblfloatpagefraction}{0.7}

\definecolor{light-gray}{gray}{0.95}
\definecolor{mid-gray}{gray}{0.85}
\definecolor{darkred}{rgb}{0.7,0.25,0.25}
\definecolor{darkgreen}{rgb}{0.15,0.55,0.15}
\definecolor{darkblue}{rgb}{0.1,0.1,0.5}
\definecolor{blue}{rgb}{0.19,0.58,1}

\newcommand{\red}[1]{\textcolor{red}{#1}}
\newcommand{\green}[1]{\textcolor{green}{#1}}
\newcommand{\blue}[1]{\textcolor{blue}{#1}}
\newcommand{\orange}[1]{\textcolor{orange}{#1}}
\newcommand{\darkred}[1]{\textcolor{darkred}{#1}}
\newcommand{\darkgreen}[1]{\textcolor{darkgreen}{#1}}
\newcommand{\darkblue}[1]{\textcolor{darkblue}{#1}}

\makeatletter
\setlength{\@fptop}{0pt}
\makeatother
