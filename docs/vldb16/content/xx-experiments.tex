%!TEX root = ../main.tex

%
% Notes: 
%
%  exact implementation
%  comparison with rollback at batch N
%  relaxing the solvers
%  dealing with false positives
%
%  running on TPC-C/sanjay's.  equality is easier
%
% Need names for
%  * rollback
%  * exact solution
%  * fixing an individual/batch of queries
%
\section{Experiments}
\label{sec:experiments}

% \ewu{Should say this somewhere: 
% Updates are often a small fractice of a query workload -- typically $10\%$~\cite{} in most benchmarks --
% thus in practice \sys can work for much larger workloads.  In the experiments, we filtered the workloads to 
% only look at modification queries, and specifically UPDATE queries which are hard.
% }


In this section, we carefully study the performance and accuracy
characteristics of the naive MILP-based repair algorithm, as well as 
slicing-based optimizations that improve the latency of the system
, an incremental algorithm for single query corruptions that
further improves the latency, and 
the multi-pass tuple-slicing algorithm that tolerates incomplete complaint sets with minimal loss in accuracy.
Our goals in the evaluation is to understand these trade-offs in
controlled synthetic scenarios, as well as study the effectiveness
in typical database query workloads based on widely used benchmarks.

To this end, our experiments are organized as follows: First, 
we compare the exhaustive MILP algorithm against the different optimizations 
to highlight the value of different optimizations.  We then compare the
repair costs of \texttt{INSERT}, \texttt{DELETE}, or \texttt{UPDATE}-only query logs 
and find that the latter query type is by far the most complicated and costly to repair.
For the subsequent experiments, we focus solely on \texttt{UPDATE}-only synthetic workloads 
to understand how \sys responds to different query logs and databases.  
Finally, we evaluate the efficacy on real-world databases transaction benchmarks,
TPC-C and AuctionMark.
All experiments were run on 12x2.66 GHz  machines with 16GB RAM running CentOS release 6.6.



% establish the quality limitations of existing heuristics and the need for a formal, 
% constraint-based algorithm (\exact).  Second, we study how each of the 
% optimizations described in Section~\ref{s:optimiztaions} improves algorithm scalability.
% Third, we introduce different forms of error in the input complaint sets and study the 
% effectiveness of our noise-handling heuristics.  




%
% NOTE: figures are named <experimentsection>_<subsection>_<xaxis>.pdf
%

\subsection{Experimental Setup}


\iffalse
\begin{table}[t]\small
  \centering
  \begin{tabular}{@{}cll@{}}
  \toprule
  {\bf Param} & {\bf Description} & {\bf Default} \\ \midrule
  $V_d$  & Domain range of the attributes  & $[0, 100]$ \\
  $N_D$  & \# tuples in final database & $1000$ \\
  $N_a$  & \# attributes in database & $10$ \\
  $N_w$  & \# predicates in \texttt{WHERE} clauses & $1$ \\
  $N_s$  & \# \texttt{SET} clauses & $1$ \\
  $N_q$  & \# queries in query log & $300$ \\
  $idx$  & Index of corrupted query & $\{0, 25, 50,$ \\
         & (backwards from most recent) & $100, 200, 250 \}$ \\ %$\frac{N_q}{2}$ \\
  $r$    & Range size of \texttt{UPDATE} queries & 8 (tuples) \\
  $s$    & Zipf $\alpha$ param of query attributes, & $1$ \\ \bottomrule \end{tabular}
  %$set$  & Constant vs relative \texttt{SET} clauses. & const \\ 
         %& power low distribution $P(v) = v^{-s}$ & \\\end{tabular}
  \caption{Experimental Parameters}
  \label{t:params}
\end{table}
\fi


\iffalse
  \begin{table}[t]\small
    \centering
    \begin{tabular}{@{}cl@{}}
    \toprule
    {\bf Param} & {\bf Description} \\ \midrule
    $p$ & Precision: \% of repaired tuples that are correct. \\
    $r$ & Recall: \% of full complaint set repaired.\\
    % $t_{prep}$ & Time to construct CPLEX problem \\
    % $t_{send}$ & Time to send CPLEX problem to solver \\
    % $t_{solve}$ & Time for solver to generate a solutions\\
    $t_{total}$ & End-to-end execution time \\ 
    $d_{measure}$ & \red{Some sort of distance measure} \\ \bottomrule \end{tabular}
    \caption{Metrics Compared}
    \label{t:metrics}
  \end{table}
\fi




Each of our experiments follows a consistent procedure. 
We generate a sequence of queries using a synthetic query generator or 
the benchmark program, and corrupt the query log as described below. 
We then execute the original and corrupt query logs on an initial (possibly empty) database,
and perform a tuple-wise comparison between the resulting database states 
to generate a true complaint set.  
We then add noise to the complaint set by 1) picking random tuples not in the true
complaint set to add false positive complaints, and 2) removing true complaints to simulate false negatives.
Finally, we execute the evaluated algorithms on the complaints and compare the fixed
query log with the true query log, as well as the fixed and true
final database states to measure performance and accuracy metrics.

We compare the following algorithms:
\xlw{$\sys_{S \subseteq \{t,q,a,inc-x\}}$, where $S$ 
defines the set of optimizations present in the algorithm: 
the subscripts $t,q,a$ represent the
application of tuple, query and attribute slicing optimizations and $inc-x$ represents 
the incremental query repairs with step size {\it x}.
For example, $\sys_{tq,inc-1}$ uses both tuple and query slicing along with incremental repair with step size 1.}
\xlw{I removed the stop early and search all description here. Could be added later, current updated experiment do not contain results for this. }

We evaluate the algorithms along several metrics.  Performance is measured as wall clock
time between submitting a query log and the system terminating after retrieving a valid repair.  
We also measure the repair's precision (percentage of repaired tuples that were correctly fixed), 
the recall (the percentage of the full complaint set that was repaired), 
and the F1 measure (the harmonic mean of precision and recall).
We note that in the context of complete complaint sets, the recall will always be $1$ if \sys returns a result.
The recall can only degrade due to solver infeasibility (the solver does not find a valid assignment within a loose time bound), 
or in the context of incomplete complaint sets.   Thus our reported metrics are the average across multiple runs.
%Finally, Table~\ref{t:params} summarizes the key parameters that we vary throughout our experiments.  
We describe the experimental parameters in the context of the datasets and workloads below.



\subsubsection{Datasets and Workloads}

This subsection describes the query and data generation process in greater detail.

\stitle{Synthetic:} \label{sec:syntheticgen}
We generate an initial database of $N_D$ random tuples.  
The schema contains $N_a=5$ attributes $a_1\ldots a_5$, whose values are
picked from $V_d$ uniformly at random, along with a primary key $id$.
We then generate a sequence of $N_q$ \texttt{UPDATE} queries~\footnote{\scriptsize We focus on
\texttt{UPDATE} only query logs because they are the {\it predominant} cost 
in a query log.  Experiment~\ref{sec:indelup} compares \texttt{INSERT}, \texttt{DELETE} and \texttt{UPDATE}
only workloads to illustrate the differenc.} where 
\verb|?| is a randomly generated value and \verb|r| is the size of the range predicate: 
{\scriptsize
\begin{verbatim}
  UPDATE SET X,.. WHERE Y ...
  X = Constant: SET (a_i = ?),..
  X = Relative: SET (a_i = a_i + ?)
  Y = Point: WHERE a_j = ? AND ...
  Y = Range: WHERE a_j in [?, ?+r] AND ...
\end{verbatim}
}
\xlw{
$X \& Y$ represents the \texttt{UPDATE} query type. $X$ represents the set clause type which can be either constant update ($Constant$) or
relative update ($Relative$). Similarly, $Y$ represents the where clause type that can be either point update ($Poing$) or
range update ($Range$)}.

The \texttt{SET} clause sets the attribute to a random constant value in $V_d$ and 
the \texttt{WHERE} clauses form a conjunction.  
% The $set$ parameter controls whether the \texttt{UPDATE} queries set attributes to random constant values ({\it const}),  
% or relative values ({\it rel}) by incrementing by a random, possibly negative, value.  
In addition, the skew parameter $s$ determines the distribution attributes referenced in the \texttt{WHERE} and \texttt{SET} clauses.  
Each attribute in a query is picked from either a uniform distribution when $s=0$ or a zipfian distribution.
This allows our experiments to vary between a uniform distribution, where each attribute is
equally likely to be picked, and a skewed distribution where nearly all attributes are the same. \\
The \texttt{WHERE} clauses in \texttt{DELETE} queries are generated in an identical fashion, while
\texttt{INSERT} queries insert values picked uniformly at random from $V_d$.



\stitle{TPC-C:} We use the data and query workload over the {\it ORDER} table in TPC-C~\cite{tpcc}.  
We generated a database at scale 1 with one warehouse, and kept the queries that modify the
{\it ORDER} table. The initial table contains 6000 tuples and we ultimately generated a log with
2000 queries, where $940$ are \texttt{UPDATE}s and the rest are \texttt{INSERT}s.

\stitle{TATP:} TATP workload simulates the 
caller location system. We generate a database from {\it SUBSCRIBER} table
with 5000 tuples and $2000$ \texttt{UPDATE} queries.\\
Both TPC-C and TATP setups were generated using the OLTP-bench project~\cite{difallah2013oltp}.


\stitle{Corrupting Queries:} We corrupt query $q_i$ by replacing it with a randomly
generated query of the same type based on the procedures described above.
To standardize our procedures, we selected a fixed set of indexes $idx$
that are used in all experiments.  Appendix~\ref{app:qidx} presents
experiments that justify the rationale behind our selection.






% \begin{figure}[h]
% \centering
%   \begin{subfigure}[t]{\columnwidth}
%   \includegraphics[width = .95\columnwidth]{figures/tpcc_time}
%   \caption{Performance on TPC-C Benchmark}
%   \label{f:tpcc} 
%   \end{subfigure}
%   \begin{subfigure}[t]{\columnwidth}
%   \includegraphics[width = .95\columnwidth]{figures/auction_time}
%   \caption{Performance on AuctionMark Benchmark}
%   \label{f:auctionmark} 
%   \end{subfigure}
%   \caption{Benchmark Performance}
% \end{figure}



  \begin{figure*}[h]
  \centering
    \vspace*{-.2in}
    \begin{subfigure}[t]{.3\textwidth}
    \includegraphics[width = .99\columnwidth]{figures/multi_time}
    \vspace*{-.1in}
    \caption{Performance for multiple corruptions.}
    \label{f:multi_time} 
    \end{subfigure}
    \begin{subfigure}[t]{.3\textwidth}
    \includegraphics[width = .99\columnwidth]{figures/incrementalcompare_time}
    \vspace*{-.1in}
    \caption{Performance for incremental algorithms with or without tuple slicing.}
    \label{f:singlequeryinc_time} 
    \end{subfigure}
    \begin{subfigure}[t]{.3\textwidth}
    \includegraphics[width = .99\columnwidth]{figures/indelup_time}
    \vspace*{-.1in}
    \caption{Performance for different query types.}
    \label{f:indelup_time} 
    \end{subfigure}
    \\
    \begin{subfigure}[t]{.3\textwidth}
    \includegraphics[width = .99\columnwidth]{figures/multi_pr}
    \vspace*{-.1in}
    \caption{Accuracy for multiple corruptions.}
    \label{f:multi_acc} 
    \end{subfigure}
    \begin{subfigure}[t]{.3\textwidth}
    \includegraphics[width = .99\columnwidth]{figures/incrementalcompare_acc}
    \vspace*{-.1in}
    \caption{Accuracy for incremental algorithms with or without tuple slicing}
    \label{f:singlequeryinc_acc} 
    \end{subfigure}
    \begin{subfigure}[t]{.3\textwidth}
    \includegraphics[width = .99\columnwidth]{figures/indelup_pr}
    \vspace*{-.1in}
    \caption{F1-score for different query types.}
    \label{f:indelup_acc} 
    \end{subfigure}
    \vspace*{-.1in}
    \caption{Preliminary Experiments.}
  \end{figure*}


\subsection{Preliminaries}
The following set of experiments are designed to establish the rationale for 
our experimental settings in the subsequent experiments.  
Specifically, we compare different slicing-based optimizations of the exhaustive approach
as the number of corrupted queries increases.  
We then evaluate the scalability of the different exhaustive algorithms in the context of a single
corrupted query and highlight the value of the second MILP iteration when using tuple-slicing.
We then establish the difficulty of repairing \texttt{UPDATE} workloads as compared to other query types.


\stitle{Multiple Corrupt Queries:}
In this experiment, we compare the \xlw{non-incremental algorithm} with four types of algorithms
$\sys_S$, where $S \in \{t, q, a, tqa\}$.  We use the default settings with $1000$ tuples and
$50$ \texttt{update} queries.  We corrupt every tenth query starting from oldest query $q_0$,
up to $q_{40}$.  For example, when we corrupt $3$ queries, we corrupt $q_{0,10,20}$.
We find that the number of corruptions greatly affects both the scalabality (Figure~\ref{f:multi_time}) 
and the accuracy (Figure~\ref{f:multi_acc}) of the algorithms.  Specifically, as the number increases,
the number of possible assignments of the MILP parameters increases exponentially and the solver often takes
longer than our experimental time limit of $1000$ seconds and returns an infeasibility error.  
This is a predominant reason why the accuracy degrades past $3$ corruptions.  For example, 
when $4$ queries are corrupted and we ignore the infeasible executions, the average execution time is $300$ seconds
and the precision and recall are greater than $0.94$.  Unfortunately, with \xlw{$5$ corruptions (50 queries)}, 
all runs exceed the time limit.

\stitle{Single Query Optimization:}
\xlw{
In this experiment, we evaluate the efficacy of the incremental repair algorithm (Section~\ref{sec:incremental})
in the special case when only one query has been corrupted. We first compare the incremental algorithm $\sys_{inc-1}$ 
with or without
tuple slicing optimization, Figure~\ref{f:singlequeryinc_time} highlights the scalability limitation of the incremental 
algorithm without tuple slicing: with 50 queries $\sys_{inc-1}$ could easily runs exceed the 1000 seconds.  
We further compare the incremental algorithm with 4 different 
incremental step sizes $\sys_{t, S}$, where $S \in \{inc-1, inc-2, inc-4, inc-8\}$. 
Figure~\ref{f:singlequeryinc_acc} shows a significant drop in accuracy as we increase the incremental step since 
the failure rate of the MILP problem is positive proportional to the incremental step.}


\stitle{Query Type Experiment:}\label{sec:indelup}
\xlw{
Our final preliminary experiment evaluates the incremental, tuple-slicing algorithm 
$\sys_{t,inc-1}$ on insert, delete, or update-only workloads.
We increase the number of queries from $1$ to $200$ and corrupt the oldest query in the log.  
Figure~\ref{f:indelup_time} shows that while the cost of repairing insert workloads
remains relatively constaint, the cost for delete-only and update-only workloads increase as 
the corruption happen earlier in the query log while update-only increase much faster than delete-only.
\sys shows near perfect F1-score for all settings (Figure~\ref{f:indelup_acc}).}


{\it Takeaways: we find that the non-incremental \sys algorithms, even when using the optimizations,
have severe scalability limitations due to the large number of undetermined values -- this is not surprising
since MILP constraint solving is a known NP-hard problem.
In contrast, we find that the incremental algorithms are more effective at scaling to larger query log sizes.  
Furthermore, we show that \texttt{UPDATE}-only workloads are significantly more expensive to repair than other 
query types. \xlw{Based on the result, we consider 
algorithm $\sys_{t,inc-1}$ as default setting in the rest of the experiment and we only consider \texttt{UPDATE}-only workload in synthetic experiments in Section~\ref{sec:experiments:synth}}. 
}


\subsection{Benchmark}
\label{sec:experiments:benchmark}
Figure~\ref{f:benchmark} plots the performance of the incremental and optimized algorithm ($\sys_{t,inc-1}$)
on both benchmark applications.  In both workloads, the latency to derive a repair was consistenly less
than $5$ seconds, and for TPC-C, on the order of milliseconds.  The precision and recall measures were
$1$ both all settings.   One reason for the good performance is due to the fact that the benchmark queries
are pre-dominantly point update queries.  In this setting, each tuples affects a very small set of 
tuples, which results in $1$ or $2$ complaints on average.  In this case, the tuple and query slicing algorithms
reduce the total number of constraints for almost every setting to less than $100$.   \xlw{Rewrite needed after experiment is done. }

\begin{figure}[h]
\centering
  \includegraphics[width = .75\columnwidth]{figures/benchmark_time}
  \label{f:benchmark} 
  \vspace*{-.2in}
  \caption{Performance on Benchmarks \xlw{tpcc not done yet for 2000 queries. }}
  \vspace*{-.1in}
\end{figure}

{\it Takeaway: many workloads in practice are dominated by point update queries.  In these settings,
 \sys is very effective at reducing the number of constraints and can derive repairs in nearly
 interactive latencies.}
 

\subsection{Synthetic Experiments}


\label{sec:experiments:synth}
In the following set of synthetic experiments, we individually vary 
the parameters skewness $s$, log size $N_q$, database size $N_D$, and
query complexity $N_w$ in order to tease apart the parameters that most affect our metrics.
Motivated by our preliminary results, we focus on \texttt{UPDATE}-only query logs with a single corruption,
and on variations of the incremental repair algorithm.  Based on previous result, we only show results for 
$\sys_{t,inc-1}$ since when we set $N_a = 10$, performance of algorithms with or without query slicing or 
attribute slicing is identical compare to $\sys_{t,inc-1}$. 

  \begin{figure*}[h]
    \centering
    \begin{subfigure}[t]{.3\textwidth}
      \includegraphics[width = .99\columnwidth]{figures/dbsize_time}
      \vspace*{-.1in}
      \caption{Database Size.}
      \label{f:dbsize_time} 
    \end{subfigure}
        \begin{subfigure}[t]{.3\textwidth}
      \includegraphics[width = .99\columnwidth]{figures/pointrelv_time}
      \vspace*{-.1in}
      \caption{Query Complexity (Query Type).}
      \label{f:qidx_time} 
    \end{subfigure}
    \begin{subfigure}[t]{.3\textwidth}
      \includegraphics[width = .99\columnwidth]{figures/where_time}
      \vspace*{-.1in}
      \caption{Query Complexity (Where Size).}
      \label{f:where_time} 
    \end{subfigure}
    \iffalse 
    \\
    \begin{subfigure}[t]{.3\textwidth}
      \includegraphics[width = .99\columnwidth]{figures/dbsize_acc}
      \vspace*{-.1in}
      \caption{Database Size vs F1-score. }
      \label{f:dbsize_acc} 
    \end{subfigure}
        \begin{subfigure}[t]{.3\textwidth}
      \includegraphics[width = .99\columnwidth]{figures/pointrelv_acc}
      \vspace*{-.1in}
      \caption{Query Complexity (Query Type) vs F1-score.}
      \label{f:qidx_acc} 
    \end{subfigure}
    \begin{subfigure}[t]{.3\textwidth}
      \includegraphics[width = .99\columnwidth]{figures/where_acc}
      \vspace*{-.1in}
      \caption{Query Complexity (Where Size) vs F1-score.}
      \label{f:where_acc} 
    \end{subfigure}
    \fi
    \caption{Scalability Experiments (Performance)}
  \end{figure*}


\iffalse 
\stitle{Scalability - Log Size:}
Figures~\ref{f:logsize_time} and~\ref{f:logsize_acc} increase the query log size (each subplot) between $50, 100, 500$ and $1000$.
We compared the baseline $\sys_{t,inc}^{1st}$ with addition attribute, query or both slicing optimizations.
For readibility purposes, we have normalized the x-axis to show the distance of the corrupted query from the most recent query in the log
(e.g., $x=50$ denotes that $q_{N-50}$ was corrupted).
As expectied, the cost increases as the corruption is older in the query log since more queries need to be tested before the
true corruption is encountered.  There algorithms all execute at roughly the same performance, though there is a slight advantage to
using the query-slicing optimization. The optimization reduces the time to linearize the query log by roughly $20\%$, however
the CPLEX solver time contributes to over half the total time so the improvement is difficult to see.
All algorithms exhibit near perfect F1-scores.
\fi


\stitle{Scalability - DB Size:}
Figures~\ref{f:dbsize_time} and~\ref{f:dbsize_acc} vary the database size from $100$ to $100$k tuples while setting the other parameters to their default settings.
For simplicity, we plot the performance and accuracy of corruptions at two query indexes: $50$ and $200$.
The more recent corruption ($200$) is executed in nearly constant time as the database size increases, with a perfect F1-score.
In contrast, the older corruption increases in cost as the database size increases.  One reason is because as the database size increases, the
number of complaints also increases.  We have found that the performance of the solver in dependent on the number of complaints (which
increases the size of the MILP problem).  This is verified by the fact that the cost of linearizing the query log increases linearly with the 
database size, from $100$ to $200$ seconds, however the solver time varies in the "v" shape shown in Figure~\ref{f:dbsize_time}
and dominates the cost.   Despite this, the F1-score remains high.


\stitle {Scalability - Query Complexity (update type): }
\xlw{Figure~\ref{f:qidx_time} compares the performance of $\sys_{t,inc-1}$ in solving three types \texttt{UPDATE}-only workloads: {\it Constant/Point, Constant/Range, Relative/Range} while increase the number of queries \sys need to solve from $1$ to $250$ (by adjusting the corrupt query index). We found that \sys solves \texttt{UPDATE} with point updates much faster
than with range updates since point updates greatly reduce the searching space of the MILP problem. 
}


\stitle{Scalability - Query Complexity (where clause size):}
Our final scalability experiment (Figures~\ref{f:where_time}, Figures~\ref{f:where_acc})
varies the complexity of the update queries by increasing the number of predicates in the \texttt{WHERE} clause, 
while keeping the overall selectivity constant.   We find that the performance improves as the number of predicates increases.
This counter-intuitive result is due to the reduction in the complaint set size as the complexity increases.




\stitle{Skew:} We now study the effects of attribute skew on the algorithms.
We vary the skew parameter from $0$ (uniform) to $1e-7, 0.5, 1$. When $s=1$, nearly every attribute is on $A_0$.
Figure~\ref{f:skew_time} shows that the performance is weakly related to the skewness.  
This result, in particular when $idx=50$, is misleading for a similar reason as the database size scalability experiments.
Though the cost of linearizing the problem increases with the skew, the solver time behaves 
unpredictably, and dominates the shape of the curve.  This effect is less noticable when $idx > 50$
due to the smaller size of the linearized MILP problem.
The F1-score for all settings remains near $1$.


% 
% Our hypothesis is that higher skew results in a larger number of relevant queries, thus limiting the effectiveness of query slicing.
% In addition, recall that each query sets all tuples within the same range to the same value.  
% Since increasing the skew makes it more likely that both the \texttt{SET} and \texttt{WHERE} clauses reference the same attributes,
% it causes the tuples to cluster around the values, thus increasing the selectivity of the queries.
% This results in a larger complaint set, which 
% 
% with high skew, each query is more likely to modify the same tuples, thus increasing the number of
% 
% We generate query log at 4 different skewness levels from uniform to 
% super skewed with $s = 0, 1e-7, 0.5, 1$. As we can see as query log
% is more skewed, the execution time for solving same amount of queries 
% increase. This is also due to the fact that increasing skewness also
% increase the query dependency, which, in turn, increase the searching
% space in the MILP problem.
 
  \begin{figure*}[h]
    \vspace*{-.1in}
    \centering
    \begin{subfigure}[t]{.23\textwidth}
    \includegraphics[width = \columnwidth]{figures/skew_time}
    \caption{Skew vs Time}
    \label{f:skew_time} 
    \end{subfigure}
    \begin{subfigure}[t]{.23\textwidth}
    \includegraphics[width = \columnwidth]{figures/skew_acc}
    \caption{Skew vs F1-score}
    \label{f:skew_acc} 
    \end{subfigure}
    \begin{subfigure}[t]{.23\textwidth}
    \includegraphics[width = .9\columnwidth]{figures/noise_fn_time}
    \caption{Time vs False Negative}
    \label{f:falsenegative_time} 
    \end{subfigure}
    \begin{subfigure}[t]{.23\textwidth}
    \includegraphics[width = .9\columnwidth]{figures/noise_fn_acc}
    \caption{F1-score vs False Negative}
    \label{f:falsenegative_acc} 
    \end{subfigure}
    \vspace*{-.15in}
    \caption{Skew and False Negative experiments. }
  \end{figure*}





\stitle{Incomplete Complaint Sets:}
Our final experiment (Figures~\ref{f:falsenegative_time} and~\ref{f:falsenegative_acc}) evaluates the fase positive rate in incomplete complaint sets.
We increase the rate from $0$, meaning a full complaint set, to $0.75$, where 
$75\%$ of the full complaint set is missing.  
We find that reduciing the size of the submitted complaint set naturally improves the repair performance,
however the repair quality may suffer if the corruption occured in a very old query.


%   \begin{figure}[h!]
%     \centering
%     \begin{subfigure}[t]{\columnwidth}
%     \includegraphics[width = .9\columnwidth]{figures/noise_fn_time}
%     \caption{Time vs Incomplete Complaint Size}
%     \label{f:falsenegative_time} 
%     \end{subfigure}
%     \begin{subfigure}[t]{\columnwidth}
%     \includegraphics[width = .9\columnwidth]{figures/noise_fn_acc}
%     \caption{F1-score vs Incomplete Complaint Size}
%     \label{f:falsenegative_acc} 
%     \end{subfigure}
%     \label{f:falsenegative}
%     \caption{Incomplete Complaint Experiments}
%   \end{figure}
% 

{\it Takeaways: we find that the performance of the repair algorithm heavily depends on the 
number of complaints in the complaint set.  This is expected, since each complaint requires 
an additional linearization of all relevant queries.  In addition, since CPLEX is a black-box
solver, its performance can vary unpredictably -- though its performance weakly correlates to the 
size of the MILP problem.  We also find that the quality of our repairs 
is high when the corruption is between the most recent to the $100$th most recent query.  However,
when the corruption is very old, it can be very difficult to identify the perfect fix -- 
in particular because a more recent query may provide a partial, though over-generalized fix.}

% \stitle{Incremental Results}
% INcremental algorithm, incremental results














