%!TEX root = ../main.tex

\section{A MILP-based Solution}
\label{sec:sol}

In this section, we introduce a \emph{basic} solver-based approach to 
resolve the errors reflected in the complaint set.
This approach constructs a mixed-integer linear 
programming (MILP) problem by linearizing and parameterizing the 
corrupted query log over the tuples in the database. 
Briefly, an MILP is a linear program where only a subset of the undetermined variables
are required to be integers, while the rest are real-valued.

\looseness -1
Our general strategy is to model each query as a linear equation 
that computes the output tuple values from the inputs and to transform the
equation into a set of of linear constraints.   
In addition, the constant values in the queries are parameterized
into a set of undetermined variables, while the database state is encoded 
as constraints on the initial and final tuple values.
Finally, the undetermined variables are used to construct an objective function
that prefers value assignments that minimize both the amount that the queries change and
the number of non-complaint tuples that are affected. 

The rest of this section will first describe the process of linearizing a single query
and translating it into a set of constraints.  We then extend the process to the entire
query log and finally define the objective function.
Subsequent sections introduce optimizations that both
improve the speed and quality of the results, as well as harness the trade-off between the two. 


% \xlw{Essentially, 
% we convert the updating process of the query log into a set of 
% constraints, transform values in queries (in the query log)
% as undetermined variables, and 
% using these values to form the objective function. Thus, the problem of 
% deriving a log repair is converted into a mixed-integer linear 
% programming (MILP) optimization problem. }\\
% {\xlw{need to introduce MILP problems a little bit here, so readers know what undetermined
% variables are and why things need to be linearized}}






\subsection{Encoding a Single Query}%Linearizing \& Parameterizing Single Query}
\label{sec:linearize}

\if{0}
  We model a query $q_i$ as a conditional function $f_{q_i}(t)$ that takes as input a tuple $t$
  and returns its next state $t'$.  $f_{q_i}$ is applied to each 
  tuple $t \in \mathcal{D}_{i-1} \cup \{\bcancel{t}\}$ in the input relation along with a special
  non-existant tuple $\bcancel{t}$. \ewu{maybe fold this into the data model.}
  By treating the query as a function, we are able to encode its effects into a set
  of linear inequality constraints.  We call this process the linearization and 
  parameterization of a query.

  \begin{definition} [Conditional Function]
  \label{def:cond}
    The conditional function for query $q$ is:
    \[
      f_{q_i}(t)= 
      \begin{cases}
      f_{q_i.\mu} (t) ,& \text{if } f_{q_i.\sigma} (t)\\
      t,              & \text{otherwise}
      \end{cases}
  \]
  where the \textit{update function} $f_{q_i.\mu}$ models a set of \textit{update equation(s)};
  and the \textit{condition function} $f_{q_i.\sigma}$ models a set of \textit{logical expression(s)} in 
  disjunctive or conjunctive form.
  \end{definition} 


  % We linearize and parameterize a query $q$ by considering its effects on the 
  % targeted table $R(A_1, ..., A_m)$: we treat the query $q$ as a 
  % conditional function over each tuple $t\in R$ and convert the effects of $q$ 
  % over $t$ into a set of linear inequality constraints. 
  % \subsubsection{Query as a Conditional Function}
  % The effect of query $q$ over a tuple $t$ can be expressed in a conditional
  % function $f_q(t)$ as the following:

  Conditional functions can describe the common classes of update queries:
  \begin{enumerate}
  \item \texttt{UPDATE}: $f_{q_i.\mu}(t)$, $f_{q_i.\sigma}(t)$ model the \texttt{SET}
        and \texttt{WHERE} clauses.  For example, $f_{q_i.\mu}(<t.a, t.b>) = <t.a + 1, 2>$ and
        $f_{q_i.\sigma}(t) = (t.a > 20)$ corresponds to the query 
        \texttt{UPDATE D SET a = a + 1, b = 2 WHERE a > 20}.

  \item \texttt{INSERT}: $f_{q_i.\mu}(t)$ returns the inserted tuple, while 
        $f_{q_i.\sigma}(t) = (t = \bcancel{t})$ evaluates to true when it is executed over
        the special nonexistant tuple.
        %is a boolean variable reflects 
        %the existence of tuple $t$; $\vee_{t\in R} f_{q.\sigma}(t)$ represents 
        %the existence of this insert query.  

  \item \texttt{DELETE}: $f_{q_i.\mu}(t) = \bcancel{t}$ returns a nonexistant tuple whenever
        the predicate encoded in $f_{q_i.\sigma}(t)$ evaluates to true.
        For example, the query \texttt{DELETE FROM T WHERE a < 20} represents 
        $f_{q_i.\sigma}(t) = (t.a < 20)$.
        
        % the deleted values for 
        % each attribute and $f_{q.\sigma}(t)$ reflects the expressions in where clause 
        % (similar to $f_{q.\sigma}(t)$ in \texttt{UPDATE} query).
  \end{enumerate}

  Finally, we parameterize $q_i$ by replacing all numeric constants in the
  conditional function with undetermined variables.   Consider the conditional
  function for \texttt{UPDATE} above: the constants $1$, $2$ in $f_{q_i.\mu}$
  as well as $20$ in $f_{q_i.\sigma}$ will be transformed into undetermined
  variables \texttt{v1, v2, v3} that are solved by the MILP solver.

  % We parameterize query $q$ by replacing all numeric values in the above conditional 
  % function into undetermined variables (Example~\ref{ex:parameterize}).
  % \begin{example}\label{ex:parameterize}
  % Consider a \texttt{UPDATE} query $q$:
  % \texttt{UPDATE R SET A$_1$ = 3 WHERE A$_2$ $\leq$ 10},
  % the conditional function of this query is: 
  % \[
  %     f_q(t)= 
  % \begin{cases}
  %     f_{q.\mu}(t) = \{t.A_1 = 3\} ,& \text{if } f_{q.\sigma}(t) = \wedge\{t.A_2 \leq 10\}\\
  %     t,              & \text{otherwise}
  % \end{cases}
  % \]
  % The numeric variables in query $q$ including \texttt{3} in $f_{q.\mu}(t)$ and \texttt{10}
  % in $f_{q.\sigma}(t)$. Thus, we parameterize query $q$ by replacing the value \texttt{3, 10}
  % by two undetermined variables \texttt{var1, var2}. 
  % \end{example} 
\fi



% \subsubsection{Constructing Linear Inequality Constraints}

MILP problems express constraints as a set of linear inequalities. Our
task is to derive such a mathematical representation for each query in
$\mathcal{Q}$. Starting with the functional representation of a
query (Section~\ref{sec:models}), we describe how each query type,
\texttt{UPDATE}, \texttt{INSERT}, and \texttt{DELETE}, can be
transformed into a set of linear constraints over a tuple $t$ and an
attribute value $A_j$.

% We will individually describe this linearization process 
% to represent the value of a single attribute $A_j$ and a single tuple $t$ for 
% \texttt{UPDATE}, \texttt{INSERT}, and \texttt{DELETE} queries; 
% extending the process to the rest of the attributes is a straighforward exercise.
% Now that we have modeled $q_i$ as a parameterized conditional function, 
% we must linearize $f_{q_i}$ into a set of linear (in)equality constraints
% for the MILP problem.


\smallskip
\noindent
\texttt{UPDATE:}
Recall from Section~\ref{sec:models} that query $q_i$ can be modeled
as the combination of a modifier function $\mu_{q_i}(t)$ and
conditional function $\sigma_{q_i}(t)$. First, we introduce a binary
variable $x_{q_i, t}$ to indicate whether $t$ satisfies the
conditional function of $q_i$: $x_{q_i, t}=1$ if
$\sigma_{q_i}(t)=\texttt{true}$ and $x_{q_i, t}=0$ otherwise. In a
slight abuse of notation:
% 
% {\scriptsize
\begin{align}
\label{eq:x}
x_{q_i, t} = \sigma_{q_i}(t)
\end{align}
% }
Next, we introduce real-valued variables for the attributes of $t$.
We express the updated value
of an attribute using semi-modules, borrowing from the models of
provenance for aggregate operations~\cite{Amsterdamer2011}. A
semi-module consists of a commutative semi-ring, whose elements are
scalars, a commutative monoid whose elements are vectors, and a
multiplication-by-scalars operation that takes a scalar $x$ and a
vector $u$ and returns a vector $x \otimes u$. A similar formalism has
been used in literature to model hypothetical data
updates~\cite{tiresias}.

Given a query $q_i$ and tuple $t$, we express the value of attribute $A_j$ in the updated tuple $t'$ as follows:
% {\scriptsize
\begin{align}
\label{eq:linearization}
t'.A_j = x_{q_i, t}\otimes \mu_{q_i}(t).A_j + (1-x_{q_i, t})\otimes t.A_j 
\end{align} 
% }
In this expression, the $\otimes$ operation corresponds to regular
multiplication, but we maintain the $\otimes$ notation to indicate
that it is a semi-module multiplication by scalars. This expression
models the action of the update: If $t$ satisfies the conditional
function ($x_{q_i, t}=1$), then $t'.A_j$ takes the value
$\mu_{q_i}(t).A_j$; if $t$ does not satisfy the conditional function
($x_{q_i, t}=0$), then $t'.A_j$ takes the value $t.A_j$.
In our running example, the rate value of a tuple $t$ after query $q_1$ would be expressed as:
$t'.owed = x_{q_1, t}\otimes (t.income*0.3) + (1-x_{q_1, t})\otimes t.owed$.
%$t'.rate = x_{q_1, t}\otimes 30 + (1-x_{q_1, t})\otimes t.rate $
Equation~\eqref{eq:linearization} does not yet provide a linear
representation of the corresponding constraint, as it contains
multiplication of variables. To linearize this expression, we adapt a
method from~\cite{tiresias}: We introduce two variables $u.A_j$ and
$v.A_j$ to represent the two terms of
Equation~\eqref{eq:linearization}: $u.A_j=x_{q_i, t}\otimes
\mu_{q_i}(t).A_j$ and $v.A_j=(1-x_{q_i, t})\otimes t.A_j$. Assuming a
number $M$ as the upper bound of the domain of $t.A_j$, we get the
following constraints:
% 
% {\scriptsize
\begin{align}
\label{eq:uv}
u.A_j &\!\leq\! \mu_{q_i}(t).A_j   &v.A_j &\!\leq\! t.A_j &\nonumber\\
u.A_j &\!\leq\! x_{q_i, t}M        &v.A_j &\!\leq\! (1\!-\!x_{q_i, t})M &\\
u.A_j &\!\geq\! \mu_{q_i}(t).A_j \!-\! (1\!-\! x_{q_i, t})M \phantom{i} 
&v.A_j &\!\geq\! t.A_j \!-\! x_{q_i, t}M &\nonumber
\end{align}
% }%
% 
The set of conditions on $u.A_j$ ensure that $u.A_j = \mu_{q_i}(t).A_j$ if $x_{q_i, t}=1$, and $0$ otherwise. Similarly, 
the conditions on $v.A_j$ ensure that $v.A_j = t.A_j$ if $x_{q_i, t}=0$, and $0$ otherwise.  
Now, Equation~\eqref{eq:linearization} becomes linear:
 % {\scriptsize
\begin{align}
\label{eq:tnew}
t.A_j' = u.A_j + v.A_j
\end{align}
% }

\noindent\texttt{INSERT:}\looseness -1
An insert query adds a new tuple $t_{new}$ to the database.  If the query were 
corrupted, then the inserted values need repair. We use a binary variable $x$ to model whether the query is correct.  Each attribute of the newly inserted tuple ($t'.A_j$) may take one of two values: the value specified by the insertion query ($t_{new}.A_j$) if the query is correct ($x=1$), or an undetermined value ($u.A_j$) if the query is incorrect ($x=0$).  Thus, similar with Equation~\eqref{eq:linearization}, we write:
 % {\scriptsize
\begin{eqnarray}
\label{eq:insert}
t'.A_j = x \otimes t_{new}.A_j + (1-x) \otimes v.A_j 
\end{eqnarray}
 % }%


\smallskip
\noindent
\texttt{DELETE:}
A delete query removes a set of tuples from the database.  
Since the MILP problem doesn't have a way to express a non-existent value, 
we encode a deleted tuple by setting its attributes to a value
outside of the attribute domain $M^+$.  In this way, subsequent conditional functions
on the attribute will return false, so it will not have an effect on subsequent queries encoded
in the MILP problem:
 % {\scriptsize
\begin{eqnarray}
\label{eq:delete}
t'.A_j &=& x_{q_i, t} \otimes M^+ + (1-x_{q_i, t}) \otimes t.A_j \\
x_{q_i, t} &=& \sigma_{q_i}(t)\nonumber 
\end{eqnarray}
 % }%
This expression is further linearized using the same method as Equation~\eqref{eq:uv}.

\smallskip
\noindent
\textbf{Putting it all together.}
% 
% )(\ref{eq:uv})(\ref{eq:tnew})(\ref{eq:insert})(
The constraints defined in Equations \eqref{eq:x}--\eqref{eq:delete}
form the main structure of the MILP problem for a single attribute
$A_j$ of a single tuple $t$. To linearize a query $q_i$ one needs to
apply this procedure to all attributes and tuples. This process is
denoted as $Linearize(q, t)$ in Algorithm~\ref{alg:basic}. Our MILP
formulation includes three types of variables: the binary variables
$x_{q_i, t}$, the real-valued attribute values (e.g., $u.A_j$), and
the real-valued constants in $\mu_{q_i}$ and $\sigma_{q_i}$. All these
variables are undetermined and need to be assigned values by a MILP
solver.

% % I removed this paragraph because I think that (1) it is not needed,
% % and (2) it is incorrect. We cannot get D'_i from C, as C is defined on
% % D_n.
% The final step is to assign concrete values to the starting and ending attribute values 
% $t.A_j$ and $t'.A_j$ based on the starting database state $D_{i-1}$ and the ending database 
% state that has been transformed by $\mathcal{C}$, $D'_i$.
% In this way, only the query parameters and the binary $x_{q_i, t}$ variables are undetermined;
% a solution to the MILP formulation will assign values to those undetermined variables
% such that the resulting query fixes the complaints correctly.

Next, we extend this encoding to the entire query log,
and incorporate an objective function encouraging solutions
that minimize the overall changes to the query log.







\subsection{Encoding and Repairing the Query Log}
\label{sec:milp}

We proceed to describe the procedure (Algorithm~\ref{alg:basic}) that encodes 
the full query log into a MILP problem, and solves the MILP problem to derive $\mathcal{Q}^*$.
The algorithm takes as input the query log $\mathcal{Q}$, 
the initial and final (dirty) database states 
$\mathcal{D}_{0, n}$, and the complaint set $\mathcal{C}$, and outputs a fixed query 
log $\mathcal{Q}^*$.  

\looseness -1
We first call \textit{Linearize} on each tuple in $\mathcal{D}_0$ and
each query in $\mathcal{Q}$, and add the result to a set of
constraints \textit{milp\_cons}. The function \textit{AssignVals} adds
constraints to set the values of the inputs to $q_0$ and the outputs
of $q_n$ to their respective values in $\mathcal{D}_0$ and
$\mathcal{T}_\mathcal{C}(\mathcal{D}_n)$. Additional constraints
account for the fact that the output of query $q_i$ is the input of
$q_{i+1}$ (\textit{ConnectQueries}). This function simply equates $t'$
from the linearized result for $q_i$ to the $t$ input for the
linearized result of $q_{i+1}$.

Finally, \emph{EncodeObjective} augments the program with an objective
function that models the distance function between the original query
log and the log repair ($d(\mathcal{Q},\mathcal{Q}^*)$). In the
following section we describe our model for the distance function,
though other models are also possible. Once the MILP solver returns a
variable assignment, \textit{ConvertQLog} updates the constants in the
query log based on this assignment, and constructs the fixed query log
$\mathcal{Q}^*$.
% \ewu{explicitly mention each function in algorithm.}



% Using the linearization method in Section~\ref{sec:linearize}, we can further
% linearize the entire query log by converting every tuples in the table $R$. 
% During the linearization, we further parameterize each
% query in the query log $\mathcal{Q}$ in order by derive the log repair. 
% The linearized and parameterized query log
% should start from and end at clean database states.
% To achieve this, we add constraints by assigning the true initial and end database 
% states' values (based on complaints) to the corresponding variables. 
% Following the above steps (Algorithm~\ref{alg:basic}), we convert the 
% query log into a collection of constraints with newly introduced variables. 
% Some of these variables are introduced to linearize the query log 
% and the rest usually represent the numeric values in the query log
% during the paramerization process. 
% The latter set of variables often involved in the objective function 
% according to the pre-defined distance function $d$ 
% (as described in Definition~\ref{def:problem}). 


\begin{algorithm}[t]
\caption{$Basic:$ The MILP-based approach.}
\label{alg:basic}
\scriptsize
\begin{algorithmic}[1]
\REQUIRE {$\mathcal{Q}, D_0, D_n, \mathcal{C}$}
%\ENSURE {$\mathcal{Q^*}$}
\STATE $milp\_cons \leftarrow \emptyset$
\FOR {each $t$ in $R$}
\FOR {each $q$ in $\mathcal{Q}$}
\STATE $milp\_cons \leftarrow milp\_cons \cup Linearize(q, t)$
\ENDFOR
\STATE $milp\_cons \leftarrow milp\_cons \cup AssignVals(D_0.t, D_n.t, \mathcal{C})$
\FOR {each $i$ in $\{0,\ldots,N-1\}$}
\STATE $milp\_cons \leftarrow milp\_cons \cup ConnectQueries(q_i, q_{i+1})$
\ENDFOR
\ENDFOR 
\STATE $milp\_obj \leftarrow EncodeObjective(milp\_cons, \mathcal{Q})$
\STATE $solved\_vals \leftarrow MILPSolver(milp\_cons, milp\_obj)$
\STATE $\mathcal{Q}^* \leftarrow ConvertQLog(Q, solved\_vals)$
\STATE Return $\mathcal{Q}^*$
\end{algorithmic}
\end{algorithm}

% By constructing linear (in)equality constraints and
% defining a objective function, we convert the problem into
%  a mixed-integer linear programming (MILP) problem that can be 
%  solved by MILP solvers. By solving this MILP problem, we collect the
% corrections for the parameterized variables and form the log repair. 

% describe how to linearize the whole querylog with provided
% database states info. 

















