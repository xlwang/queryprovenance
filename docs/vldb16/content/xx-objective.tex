%!TEX root=../main.tex

\subsection{The Objective Function}

The optimal diagnosis problem (Definition~\ref{def:problem}) seeks a
log repair $\mathcal{Q}^*$, such that the distance
$d(\mathcal{Q},\mathcal{Q}^*)$ is minimized. In this section, we
describe our model for the objective function, which assumes numerical
parameters and attributes. This assumption is not a restriction of the
\sys framework.
Handling other data types, such as categorical values comes down to defining an appropriate distance function, which can then be directly incorporated into \sys.

In our experiments, we use the normalized Manhattan
distance~\cite{manhattan} between the parameters in $\mathcal{Q}$ and
$\mathcal{Q}^*$. We use $q.param_i$ to denote the $i^{th}$ parameter
of query $q$, and $|q.param|$ to denote the total number of parameters
in $q$: \[d(\mathcal{Q}, \mathcal{Q}^*) = \sum_{i = 1} ^{n} \sum_{j =
1}^{|q_i.para|} |q_i.para_j - q_i.para_j^*|\]

Different choices for the objective function are also possible. For
example, one may prioritize the total number of changes incurred in
the log, rather than the magnitude of these changes. However, a
thorough investigation of different possible distance metrics is
beyond the scope of our work.

% The solution of a MILP problem is designed to minimize an objective function defined by the user.
% A natural object function is to simply use the distance function $d(\mathcal{Q}, \mathcal{Q}^*)$ 
% (Definition~\ref{def:problem}) so that the proposed solution
% does not deviate significantly from the original query log e.g., does not modify too many queries.  
% This has the added benefit of interpretability, since the user is presented with less diagnoses.
% 
% The general form of $d(\mathcal{Q}, \mathcal{Q}^*$ provides us with substantial flexibility in the precise
% function implementation so that different function can potentially be customized towards different scenarios.
% For example, when the query log only involves numeric attributes, 
% the normalized Manhattan distance~\cite{manhattan} between the parameters 
% in $\mathcal{Q}$ and $\mathcal{Q}^*$ is a natural objective function.
% Alternatively, we may use the number of parameters that are different with the original ones 
% in order to penalize repairs that modify too many parameters. 
% 
% Moving beyond the queries themselves,  it is possible to augment the objective function 
% to penalize the number and magnitude of changes to the database.
% Such a modification would require removing the hard constraints that set the
% attribute values $t.A_j$ to database values and instead add their weighted difference
% to the objective function.  The drawback of this approach is the increased number of
% undetermined variables, which severely affects performance in practice.
% 
% For these reasons, in this paper, we use Manhattan distance between parameters 
% in the query log as our objective function.  We use $q.param_i$ to denote the $i^{th}$ paramater of query $q$,
% and $|q.param|$ to denote the total number of parameters in $q$:
% \[d(\mathcal{Q}, \mathcal{Q}^*) = \sum_{i = 1} ^{n} \sum_{j = 1}^{|q_i.para|} |q_i.para_j - q_i.para_j^*|\]

% impact of the log repair $\mathcal{Q}^*$ can also be involved 
% in the distance function: one would prefer that $\mathcal{Q}^*$
% modifies as few tuples as possible, in which case the distance 
% function could be the number of tuples that modified by the log repair.
% By allowing this custmizable distance function, we increase the flexibility 
% of our system. \\
% 
% 
% the distance function is the Manhattan distance between parameters in 
% the query log $\mathcal{Q}$ and the log repair $\mathcal{Q}^*$; 
% 
% w.r.t the property of the query log and its
% impact over the database. 
% he
% \ewu{mention goal is to minimie the objective unction}
% talked about moving the t.A_j = CONSTANT constraints for non-complaints
% into obj functino as |t.A_j - CONSTANT|, however makes problems slower
% so only use soft constraints during second iteration.
