%!TEX root = ../main.tex

\section{Summary and discussion}

\ewu{The general problem of data errors is highly complex, largely due to its highly contextual nature.
We believe that an approach that explains and repairs such errors based on operations made by the application 
or user is a promising approach towards incorporating contextual hints into the analysis process.
}

Towards this goal, we presented \sys, the first framework to diagnose and
repair errors in the queries that operate on the data.
Datasets are typically dynamic: even if a dataset starts clean,
updates may introduce new errors. \sys can
analyze query logs to trace reported errors to the queries that
introduced them. This in turn helps identify additional errors
in the data that may have been missed and gone unreported.

We proposed a basic algorithm, \texttt{basic}, that uses non-trivial transformation rules to
encode information from the data and the query log as a MILP problem. We further improve 
\texttt{basic} with two types of optimizations: 
!) slicing-based optimizations that reduce the problem
size without compromising, and can improve, the accuracy and 
2) an incremental approach that analyzes a single query at a time. 
Our experiments show that the latter
optimization can achieve significant scaling gains for single-query
errors, without significant reduction in accuracy.

To the best of our knowledge, \sys is the first formalization and solution to the diagnosis
and repair of errors using past executed queries. 
Obviously, correcting such errors in practice poses additional challenges. 
\ewu{The initial version of \sys described in this paper focuses on a constrained problem consisting of
simple (no subqueries, UDFs, aggregations, nor joins)
single query transactions with query clauses composed of linear functions, and
complaint sets withoun false positive complaints.
In future work, we hope to extend our techniques to relax these limitations.
In addition, we plan to investigate additional methods of scaling the constraint analysis, 
as well as techniques that can adapt the benefits of single-query analysis to errors in multiple queries.
}



