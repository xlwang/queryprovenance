
\subsection{The Objective Function}

The optimal diagnosis problem (Definition~\ref{def:problem}) seeks a
log repair $\mathcal{Q}^*$, such that the distance
$d(\mathcal{Q},\mathcal{Q}^*)$ is minimized. In this section, we
describe our model for the objective function, which assumes numerical
parameters and attributes. This assumption is not a restriction of the
\sys framework.
Handling other data types, such as categorical values comes down to defining an appropriate distance function, which can then be directly incorporated into \sys.

In our experiments, we use the normalized Manhattan
distance (in linearized format in the MILP problem) 
between the parameters in $\mathcal{Q}$ and
$\mathcal{Q}^*$. We use $q.param_i$ to denote the $i^{th}$ parameter
of query $q$, and $|q.param|$ to denote the total number of parameters
in $q$: \[d(\mathcal{Q}, \mathcal{Q}^*) = \sum_{i = 1} ^{n} \sum_{j =
1}^{|q_i.param|} |q_i.param_j - q_i.param_j^*|\]
Different choices for the objective function are also possible. For
example, one may prioritize the total number of changes incurred in
the log, rather than the magnitude of these changes. However, a
thorough investigation of different possible distance metrics is
beyond the scope of our work.


