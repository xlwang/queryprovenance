%!TEX root = ../main.tex


\section{Summary and discussion}

The general problem of data errors is complex, and exacerbated by its highly contextual nature.
We believe that an approach to explain and repair such data errors, based on operations performed by the application or user, is a promising step towards incorporating contextual hints into the analysis process.

Towards this goal, \sys is the first framework to diagnose and
repair errors in the queries that operate on the data.
Datasets are typically dynamic: even if a dataset starts clean,
updates may introduce new errors. \sys 
analyzes OLTP query logs to trace reported errors to the queries that
introduced them. This in turn helps identify additional errors
in the data that may have been missed and gone unreported.

We proposed \texttt{basic} which uses non-trivial transformation rules to
encode the data and query log as a MILP problem. We further 
presented two types of optimizations: 
(1)~slicing-based optimizations that reduce the problem
size and often improve, rather than compromise accuracy, and 
(2)~an incremental approach that analyzes one query at a time. 
Our experiments show that the latter significantly increases the scalability and latency of repairing single-query corruptions---at interactive speeds for OLTP benchmarks such as TPC-C---without significant reduction in accuracy.

To the best of our knowledge, \sys is the first formalization and solution to the diagnosis
and repair of errors using past executed queries. 
Obviously, correcting such errors in practice poses additional challenges. 
The initial version of \sys described in this paper focuses on a constrained problem consisting of
simple (no subqueries, complex expressions, UDFs, aggregations, nor joins)
single-query transactions with clauses composed of linear functions, and
complaint sets without false positives.
In future work, we hope to extend our techniques to relax these limitations towards more complex query structures and towards support for CRUD-type web application logic.
In addition, we plan to investigate additional methods of scaling the constraint analysis, 
as well as techniques that can adapt the benefits of single-query analysis to errors in multiple queries. 
%\newtext{In addition, we plan to investigate additional methods of scaling the constraint analysis and we expect
%the improvement of MILP solvers and hardwares, so that \sys can break its current bottleneck and solve harder problems effectively and efficiently. }


% \ewu{Say that this was largely simulated, and we want to move towards more realistic applications that have more complex query expressions, and how the queries interact with (simple) application logic is e.g., CRUD web applications}

