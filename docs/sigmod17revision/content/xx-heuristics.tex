\section{Heuristics}
\label{sec:heurstic}

\subsection{Attribute Slicing}

To further improve efficiency, we propose the attribute-slicing heuristic 
(Algorithm~\ref{alg:heu}) that may lose accuracy. This attribute-slicing
heuristic iteratively
generates the log repair by splitting the relevant 
attributes $Rel\mathcal{(A)}$ into
groups and fixing parameters involved 
in each group separately through a 
much smaller MILP problem. By 
controlling the number of attributes 
in each group, we bound the size of 
each MILP problem in the repair process.

\begin{algorithm}[htbp]
\caption{$AttributeSlicing$ heuristic.}
\label{alg:heu}
\begin{algorithmic}
\REQUIRE {$\mathcal{Q}, D_0, D_n, \mathcal{C}$}
\ENSURE {$\mathcal{Q^*}$}
\STATE $Rel\mathcal{(A)} \leftarrow FindRelAttr(\mathcal{Q}, D_n, \mathcal{C})$
\STATE $Rel\mathcal{(Q)} \leftarrow FindRelQuery(\mathcal{Q}, D_n, \mathcal{C})$
\STATE $\mathcal{G} \leftarrow SplitAttr(Rel\mathcal{(A)})$
\STATE $solved\_vals \leftarrow \emptyset$
\FOR {each $\partial (Rel\mathcal{(A)})$ in $\mathcal{G}$}
\STATE $\mathcal{Q'} \leftarrow PartialCF(\mathcal{Q}, \partial (Rel\mathcal{(A)}))$
\STATE $solved\_vals \leftarrow BasicApproach(\mathcal{Q'}, solved\_vals, ...)$
\ENDFOR
\STATE $\mathcal{Q}^* \leftarrow ConvertQLog(Q, solved\_vals)$
\STATE Return $\mathcal{Q}^*$
\end{algorithmic}
\end{algorithm}

In order to construct the sub MILP problem 
for a attribute group, we rewrite the conditional 
function, $f_q(t)$, into 
the partial conditional function format, $\partial (f_q(t))$.

\begin{definition} [Partial Conditional Function]
	The partial conditional function for a query $q$ over 
	attribute set $\partial(Rel\mathcal{(A)})$ is:
	\[
    \partial (f_q(t))= 
\begin{cases}
    \partial (f_{q.\mu} (t)) ,& \text{if } \partial (f_{q.\sigma} (t))\\
    t,              & \text{otherwise}
\end{cases}
\]
where
\begin{eqnarray*}
\partial (f_{q.\mu} (t)) &=& \{f|f\in f_{q.\mu}(t), \Pi_{f}(R) 
\in \partial (Rel\mathcal{(A)})\}\\
\partial (f_{q.\sigma} (t)) &=& \cup_{f \in f_{q.\sigma} (t)} \partial(f)
\end{eqnarray*}
Note that $\partial(f)$ is a transformation over a function 
(logical expression) $f$ defined
as following:
\[
\partial(f) = 
\begin{cases}
x,\ if\ \Pi_{f}(R) \cap \partial(Rel\mathcal{(A)}) = \emptyset\\
f, otherwise
\end{cases}
\]
$x$ is a newly introduced boolean variable. 
\end{definition} 
By doing so, we can easily construct the sub MILP problem 
in the same way as in the basic approach. Note that 
after each iteration, the solved variables, 
especially boolean variable $x$(s), are reused to provide 
constraint(s) for attributes that have not been examined. 
We demonstrate the process in Example~\ref{ex:heurstic}.

\begin{example}\label{ex:heurstic}
In this example, we demonstrate how to use Attribute-slicing heuristic to
solve the problem in Example~\ref{ex:taxes}. According to
Section~\ref{sec:opt}, we have:
\begin{eqnarray*}
\mathcal{F}(q_1) &=& \{rate, owed \} \\
\mathcal{F}(q_2) &=& \{owed\} \\
\mathcal{F}(q_3) &=& \{ID, rate, income, owed\}\\
Rel(\mathcal{A}) &=& \{ID, rate, income, owed\}
\end{eqnarray*}
Let us split the relevant attributes into groups: \\ 
$\{\{rate\}, \{owed\}, \{income\}, \{ID\}\}$.\\
Consider the first attribute group $\{rate\}$, we 
first rewrite the query log into $\partial(f_q(t))$ as the following:\\
\begin{minipage}{0.7\textwidth}
    \begin{minipage}[t]{0.2\textwidth}
        \begin{align*}
            f_{q_1.\mu} &=& \{\{rate = 30\}\}; \\
f_{q_2.\mu} &=& \emptyset;  \\
f_{q_3.\mu} &=& \{\{rate = 25\}\};
        \end{align*}
    \end{minipage}
    \hspace{4em}
    \begin{minipage}[t]{0.2\textwidth}
        \begin{align*}
           f_{q_1.\sigma} &=& x_1\\
            f_{q_2.\sigma} &=& true\\
           f_{q_1.\sigma} &=& x_3
        \end{align*}
    \end{minipage}
\end{minipage}

\smallskip

\indent We construct a sub MILP problem by linearizing
this query log over tuple $\{t_1, t_2, t_3, t_4\}$, parameterizing
the numeric variables, $30 and 25$, into $var1, var2$,
and setting the objective function as the minimum Euclidean
distance between modified numeric variable with original ones.
This sub MILP problem provides the following values: $var1 = 30, 
var2 = 25$;$t_1.x_1 = false, t_1.x_3 = false$; $ t_2.x_1 = true, 
t_2.x_3 = false$; $t_3.x_1 = false, t_3.x_3 = false$;
$t_4.x_1 = false; t_4.x_3 = true$.

\smallskip

In next iteration, we focus on attribute group $\{owed\}$ and find
the only numeric variable in $q_2$ should be $100$.
We then examine attribute group $\{income\}$, since we solved
$t_3.x_1 = false$ in the first iteration, the numeric 
variable $85700$ in the where clause of $q_1$ has to be 
changed into $86000+\epsilon$, where $\epsilon$
is a small number less than the minimum gap among values in
attribute $income$. Finally, we check attribute group
$\{ID\}$, and derive a log repair by revise $q_1$ as
\texttt{UPDATE Taxes Set rate = 30 WHERE income >= 86000+$\epsilon$ }. 
\end{example}






