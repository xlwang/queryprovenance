\section*{Comments by Reviewer 1}

\noindent
\textbf{Comment \#1:} Slicing soundness
\begin{quote}
\reviewer{
What are the soundness/correctness properties needed for the slicing
heuristics to be applicable/useful?

\dots

Unclear whether (or why) slicing techniques are ``sound'', or whether they
would be useful on realistic data rather than synthetic / randomly generated
benchmark data.

\dots

p6. Why is tuple slicing sound? It sounds like it will amount to modeling only
the tuples that are subject to a complaint...

\dots

p7. Likewise, I'm not convinced query or attribute slicing are safe
optimizations.
\dots  
this requires proof

}
\end{quote}

We have added discussion in Sections~\ref{sec:opt:tbsize},
\ref{sec:opt:query}, and~\ref{sec:opt:attslice} on the soundness of the three
slicing methods.   


In the general case, tuple slicing is a heuristic method. It decomposes a
large MILP problem into two, typically much smaller, MILP problems. It is
effective in practice and greatly helps improve \sys performance, especially
when the ratio of the complaint set size and the database size is small. In
general, this heuristic can result in incorrect repairs. This is possible if a
query $q$ succeeds an erroneous query $q_e$, and $q$ overwrites all changes by
$q_e$. Under a more complex scenario, where tuples are dependent to 
each other (as mentioned in comments ``p6...''), tuple-slicing is more likely 
to make mistakes as it under-specify many dependent tuples that are 
not in the complaint set. 
However, \emph{tuple slicing is sound in certain settings}, namely,
when the complaint set is complete and if corruptions are limited to a single
query. In these cases, by using incremental repair
(Section~\ref{sec:incremental}) and by disallowing all non-complaint tuples in
the refinement step (e.g., by restricting the value of the objective function
in the refinement MILP to zero), the solver will be forced to pick the correct
repair.

To conclude, \emph{tuple-slicing} is heuristic method that often helps   
improving execution efficiency of \sys in solving workloads with diverse properties. 
Although we don't have any realistic workload data, by examining various  
simulation benchmarks in different application domains~\cite{oltpbench}, we found most of them
share the same properties as we evaluated in Section~\ref{sec:experiments:benchmark}. 
Thus, we believe tuple-slicing does not limit to synthetic / randomly
generated benchmark workloads, instead, it could be applied on realistic workloads as well. 



The query and attribute slicing optimizations are always sound, in the sense
that \sys will produce the same repairs when these optimizations are applied
as when they are not. These slicing methods remove from the problem any
queries and attributes, respectively, that could not have have had an impact
to the incorrect values in the complaints. More specifically,
\emph{query-slicing} computes the \emph{full-impact} of each query, which is a
form of forward provenance: starting at query $q$, and tracing the log forward
(toward more recent queries) we keep track of all attributes that may have
been modified by $q$. If the full-impact of $q$ does not intersect with any
incorrect attributes in the complaint set, it is guaranteed that $q$ does not
affect any of the errors reported in the complaint set. Similarly, attribute
slicing summarizes all attributes that may have been modified by or define
predicates in queries in the history. Any attributes that are not on this list
can be safely ignored. Thus, applying query or attribute slicing will never
reduce the accuracy of \sys. We formalize this result in
Lemma~\ref{lem:sound}.


\alex{Add details, not just references to different responses, as it is hard
for reviewers to go back and forth.}


\comskip

\noindent
\textbf{Comment \#2:} Correction
\begin{quote}
\reviewer{
Please correct definition 6.

\dots

Suppose we have three queries \dots

}
\end{quote}

\alex{The definition needs to be corrected, and more details should be added
here.}

We apologize for the unclear definition and we have clarified it in Definition~\ref{eq:dependency}.
The impact of a query covers a set of attributes that may updated by 
the query. We trace the query 
history to search and expand the impact of a query. More specifically, given a query 
$q_i$, we compute its full-impact $\mathcal{F}(q_i)$ by explicitly 
comparing with each query in the query history in their execution order. 

\noindent In the example reviewer suggested:\\
\textit{\indent $q_1$ writes t.A; \\
\indent $q_2$ reads t.A and writes u.B; \\
\indent $q_3$ reads u.B and writes v.C.}\\
When computing the full-impact of query $q_1$, we start from the impact of $q_1$
on $q_1$, $\mathcal{F}_1(q_1) = \{t.A\}$; and then we move to the next query $q_2$. 
Since $\mathcal{F}_1(q_1)  \cap\mathcal{P}(q_2) \neq \emptyset$, we extend the impact 
of $q_1$ on $q_2$ as $\mathcal{F}_2(q_1) = \{t.A, u.B\}$. Similarly, with 
$\mathcal{F}_2(q_1)  \cap\mathcal{P}(q_3) \neq \emptyset$, we conclude the full-impact
of $q_1$ as $\{t.A, u.B, v.C\}$.


\comskip

\noindent
\textbf{Comment \#3:} Experiments on large datasets
\begin{quote}
\reviewer{
Do the experiments substantiate the claim of scalability to ``large''
databases of 100k records? If so the experimental evaluation needs to explain
this more clearly. If not the introduction must avoid overselling this point.

\dots

Experimental evaluation doesn't live up to claims in introduction (5-6K
records vs. 100K)

\dots

the experiments only seem to consider much smaller databases (5000-6000
tuples).

}
\end{quote}

We apologize for the apparent inconsistency between the text and the graphs in
the experiments. In our original submission, the graphs depicting the
experiments on larger datasets were in the appendix
(Appendix~\ref{sec:heuristic}).

For the revision, we further extended our experiments and augmented
Figure~\ref{f:attr100} to datasets of up to $100K$ records. This experiment
uses the most complex setting (\texttt{UPDATE} queries with \textit{range}
\texttt{WHERE} clause), and shows that \sys is efficient for datasets of
$100K$ records, even when the query corruption is not recent.


\comskip

\noindent
\textbf{Comment \#4:} Why $M^+$ in \texttt{DELETE} query?
\begin{quote}
\reviewer{
It's also not entirely clear why it is correct to use a ``large/unused'' value $M^+$ for attributes of deleted tuples.

\dots

p5. Handling DELETE using $M^+$. \dots I'm not sure I understand why this the
case and if so, why it is correct. \dots for example if the test is ``t.A > 5'' then setting t.A to $M^+$ (assuming it's bigger than 5) will not invalidate the test.

}
\end{quote}

\alex{Make sure to address both points!}

In this paper, we introduce a \emph{``ghost''} value, 
$M^-$\footnote{For clarification we rename the former variable $M^+$ as $M^-$ to distinguish
from $M$ and emphasis its relationship with $M$ since $M^- \leq M$.}, outside of the attribute domain
to encode the deleted tuple.
Since $M^-$ is outside of the attribute domain, any subsequent conditional functions will
evaluate to false, so subsequent queries do not affect ghost tuples. There are
nuances to how $M^-$ is set. It needs to be sufficiently large, for the MILP
problem to prioritize a modification to the \texttt{WHERE} clause of the
\texttt{DELETE} query ($\sigma_{q_i}(t) = 0/1$), compared to a modification of
the \texttt{SET} clause of an \texttt{UPDATE} query to the ghost value
($\mu_{q_i}(t.A_j) = M^-$). However, it should be $M^- \leq M $ to ensure the
constraints remain feasible (Equation~\ref{eq:uv}). These considerations
ensure that the ghost assignment will not affect subsequent queries.

Note that for workloads with both \texttt{DELETE} and \texttt{UPDATE} queries.
The assignment of $x_{q_i, t}$ in a \texttt{UPDATE} query 
requires the following changes:
$x_{q_i, t} = x'_{q_i, t} \otimes 0 + (1-x'_{q_i, t}) \otimes \sigma_{q_i}(t)$ and
$x'_{q_i, t}= (t.A_j = M^-) \wedge (t.A_j^* = M^-)$.
This arrangement ensures a deleted tuple will not be updated by the following \texttt{UPDATE} queries. 
In the example in reviewer mentioned in $p5$, suppose a tuple is deleted and values in all attributes are assigned to $M^-$. 
The boolean variable $x'_{q_i, t}$ is set to $1$ and $x_{q_i, t} $ to $0$ for any following \texttt{UPDATE} queries regardless
of their \texttt{WHERE} predicate. Thus, \texttt{UPDATE} queries won't updates values for already deleted tuples. 

\comskip

\noindent
\textbf{Comment \#5:} Assumptions
\begin{quote}
\reviewer{
Problem specification seems restrictive (users need to provide complete,
correct rows as complaints).

\dots

p3. Please clarify whether a complaint needs to specify exact fix values for all fields \dots 
}
\end{quote}

We have edited the text in Section~\ref{sec:abstractions} to emphasize all assumptions used by \sys.  We summarize the main points here.

We have also clarified in Sections~\ref{sec:abstractions} and~\ref{sec:opt}
the role of two additional restrictions in our framework: completeness of the
complaint set, and single-query corruptions. Under these assumptions, \sys can
take advantage of powerful optimizations. However, \sys is not restricted to
settings where these assumptions hold, and can handle cases of multiple query
corruptions and incomplete complaints. The effectiveness and efficiency of
\sys can be limited in these settings, mostly due to limitations of the MILP
solvers. Improvements in MILP technologies will also improve \sys's
capabilities. In our evaluation, we study these cases empirically, and we have
included experiments with incomplete complaints and multiple query corruptions
(Section~\ref{sec:experiments:hardprob}).

In addition, in the paper we use exact fix values for all fields of a complaint, we
only do so for ease of exposition, and this is actually not a limitation in
our techniques. We have added clarification about this in
Section~\ref{sec:model}.



\comskip