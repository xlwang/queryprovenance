%!TEX root = ../main.tex

\noindent
\textbf{Comment \#6:} Additional corrections and clarifications

\smallskip

We thank the reviewer for the many detailed comments. We have corrected the
typos, and reworded confusing sentences. We also provide some further notes on
some of the detailed comments below.

\begin{quote}
\reviewer{
p2. Although it is just an example, it does seem like questionable design for
the tax rates for different salary levels to be stored in the query rather
than in the database.
}
\end{quote}

The example is emulating a form-based entry of the tax-rate information, but
we agree that in practice the design would be much more complex. We chose a
substantial simplification to make the example easy to follow and to allow us
to more easily highlight the contributions of our work.

% \begin{quote}
% \reviewer{
% p2. The figures in the example seem wrong (ironically) \dots
% }
% \end{quote}
% 
% Thank you for noting this mistake. We corrected the figure to have the
% correct numbers.


\begin{quote}
\reviewer{
p3. This is explained later, but please clarify that the expressions used in
mu and sigma are from a limited language (e.g. arithmetic).
}
\end{quote}

\alex{Have we done anything about this?} \xlw{The explanation and definition is
quite close. We give def and then description. Maybe we should switch the order?}


\begin{quote}
\reviewer{
p3. What does notation ``$D \setminus\{t\}$'' mean (in def 3) if $t = \bot$? No-op? Also,
t* should be allowed to range over $D \cup \{\bot\}$ also to allow for deletion
(as discussed on the next page).
}
\end{quote}

If $t = \bot$, this means that $t$ does not exist in $D$, and $D
\setminus\{t\}$ is indeed a no-op. We have revised
Definition~\ref{def:complaint} based on the reviewer's suggestion.

% The transformation of 
% a database state $\mathcal{T}_c(D_n)$ excludes the complaint tuple $\{t\}$ (``$\setminus$'' is the set minus operation) 
% and replace it with correct tuple $t^*$. More specifically, when $t = \bot$, it means tuple $t$ does not exist in the original
% database and should be added into the database; similarly when $t^* = \bot$, the tuple $t$ is in
% the original database while it should be removed. 

\begin{quote}
\reviewer{
p4. The paper often refers to ``data manipulation queries'', which sounds
self-contradictory to me (a query reads, an update writes). Perhaps just
``updates'' instead of ``data manipulation queries'' ?
}
\end{quote}

We have revised the term to ``data manipulation statements'' (which include
\texttt{UPDATE}, \texttt{INSERT}, and \texttt{DELETE} queries). We prefer to
avoid the term ``update queries'' in order to distinguish it from the specific
\texttt{UPDATE} queries.


\begin{quote}
\reviewer{
p5. How strongly does the performance of the MILP solver depend on M (the
upper bound chosen on the value of a given numeric field)? Presumably,
choosing the largest machine-representable number would lead to overflow
problems?
}
\end{quote}

We tried different choices of M but did not observe any effect on the solver
performance. We indeed avoided setting $M$ to a too high number to avoid
overflow problems. In general, we just need to select a value greater than the
values in the data domain.

% We set $M$ to a value that is outside of the domain range in order to 
% avoid generating conflict constraints. 
% For example, attribute \textit{human age} has a domain range of $[0, 200]$, 
% in which case $M$ must be set to a value that is above $200$. 
% Besides, in order to maintain the feasibility for workloads with \texttt{DELETE}
% queries, $M$ must be set to a value that is greater than or equal to $M^-$ (we have explained
% the setting of $M^-$ in Section~\ref{sec:linearize}). 
% In conclude, we typically set $M$ to a sufficiently large number which can be 
% much smaller than the upper bound of a numeric field. Otherwise, we may
% have overflow problems.

% \begin{quote}
% \reviewer{
% p5. In (5), should v be u?
% }
% \end{quote}
% 
% We thank the reviewer for noting this, we have corrected this typo in the paper. 

\begin{quote}
\reviewer{
p7. ``Figure 3 showed the exponential cost'' - the scaling from 40 to 60 to 80 looked linear to me, just with a high coefficient.}
\end{quote}

The trend in our Figure~\ref{fig:querysize_vs_time} was confusing because we
failed to point out that we used a timeout of 1000sec, and \naive failed to
return an answer within that time for problems with 80 or more queries. We
removed that datapoint to avoid confusion, and used log-scale for the y-axis
to make the exponential trend clear.

% We apologize for not explaining it clear. In Figure~\ref{fig:querysize_vs_time}, \naive fails to produce an answer
% for problems with $\geq 80$ queries within 1000 sec as we terminate \naive once it reaches a time limit of 1000 sec. 
% In fact, the execution cost of \naive shows exponential growth with $10$, $20$, $40$, and $60$ queries. 


% \begin{quote}
% \reviewer{
% p10. Fig. 8 - please add 0.75 (the largest x-axis value) to the x-axis, better yet, label the x-axis with the rate values that correspond to data points.
% }
% \end{quote}
% 
% We thank the reviewer for the detailed suggestion and we have replaced Figure~\ref{f:falsenegative} with suggested x-axis labels. 

\begin{quote}
\reviewer{
p10. ``may suffer if the corruption occured in a very old query'' - does it (and are you saying fig. 8 supports this claim)?

p10. ``may over-generalize'' - again, are you speculating or interpreting the results here?
}
\end{quote}

We have revised the discussion of Figure~\ref{f:falsenegative} to make the wording more precise.
We find that reducing the number of reported complaints lowers the
runtime; however, we observe a small reduction in repair quality (precision
and recall in Figure~\ref{f:falsenegative_acc}) for recent corruptions and a
significant drop for older ones. This is expected: the less information we
have on the problem (fewer complaints), the lower our ability to fix it
correctly. In the extreme case where very few complaints are available, the
problem can be under-specified, and the true repair harder to identify.

\begin{quote}
\reviewer{
p10, footnote: this seems like a strong assumption! I guess this means that dealing with multiple errors that interact with each other is future work...
}
\end{quote}

\sys can solve problems with corruptions in multiple queries, but its
scalability in this setting is limited (up to about 50 queries in the log).
Assuming that \sys is used frequently enough to ensure that errors don't
remain undetected for too long, this may be an acceptable limitation. For
cases where corruptions are restricted to a single query, the incremental
repair algorithm allows \sys to scale to large data and log sizes. Indeed, we
hope to develop similarly scalable methods for the case of multiple
corruptions as well.

We  clarified this and other assumptions in Section~\ref{sec:abstractions}. 


%%%%%%%%%%%%%%%%%%%%%%%%%%%%%%%%%%%%%%%%%%%%%
