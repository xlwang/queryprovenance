%!TEX root = ../main.tex

\section{Introduction}

\label{s:intro}


In spite of the growing importance of big data, sensors, and automated
data collection, manual data entry continues to be a primary source of
high-value data across organizations of all sizes, industries, and
applications: sales representatives manage lead and sales data through
SaaS applications~\cite{salesforce}; human resources, accounting, and
finance departments manage employee and corporate information through
terminal or internal browser-based applications~\cite{sap}; driver
data is updated and managed by representatives throughout local DMV
departments~\cite{dmv,dmvsystem}; consumer banking and investment data
is managed through web or mobile-based
applications~\cite{betterment,chase}. In all of these examples, the
database is updated by translating form-based human inputs into
INSERT, DELETE or UPDATE query parameters that run over the backend
database---in essence, these are instances of OLTP applications that
translate human input into stored procedure parameters. Unfortunately,
numerous studies~\cite{kandel2012,krishnan2016hilda,Barchard20111834},
reports~\cite{citibank,Yates10,Grady13,Robeznieks05} and citizen
journalists~\cite{iquantnyc} have consistently found evidence that
human-generated data is both error-prone, and can significantly
corrupt downstream data analyses~\cite{iquantnycnypd}. Thus, even if
systems assume that data import pipelines are error-free, queries of
human-driven applications continue to be a significant source of data
errors, and there is a pressing need for solutions to diagnose and
repair these errors. Consider the following representative toy example
that we will use throughout this paper:

\begin{example}[Tax bracket adjustment]\label{ex:taxes}
Tax brackets determine tax rates for different income levels and are
often adjusted. Accounting firms implement these changes to their
databases by appropriately updating the tax rates of their customers.
Figure~\ref{fig:example} shows a simplified tax rate adjustment
scenario and highlights how a single error to the predicate in update
query $q_1$ can introduce errors in the \texttt{owed} attribute, and a
benign query $q_3$ can then propagate the error to affect the
\texttt{pay} attribute.
\end{example}

%!TEX root = ../main.tex


\begin{figure*}[t]
    \begin{minipage}[t]{0.28\textwidth}
         \vspace{0pt} 
         \centering
        \begin{tabular}{llll}
            \multicolumn{4}{l}{\emph{Taxes}: $D_0$}\\
            \toprule
            \textbf{ID}  & \textbf{income}    & \textbf{owed} & \textbf{pay} \\
            \midrule
            $t_1$   & \$9500    & \$950		& \$8550 \\
            $t_2$   & \$90000   & \$22500 	& \$67500\\
            $t_3$   & \$86000   & \$21500	& \$64500\\
            $t_4$   & \$86500   & \$21625	& \$64875\\
            \bottomrule
            \\
        \end{tabular}
    \end{minipage}
    \begin{minipage}[t]{0.43\textwidth}
         \vspace{0pt} 
         \centering
        \begin{tabular}{|p{1ex}l|}
            \multicolumn{2}{l}{\emph{Query log}: $\mathcal{Q}$}\\
            % \toprule
            \hline
            
            $q_1$: & \texttt{\small UPDATE Taxes SET owed=income*0.3}\\
            	   & \texttt{\small WHERE income>=\color{red}{85700}}\\
            
            $q_2$: & \texttt{\small INSERT INTO Taxes}\\ 
                   & \texttt{\small VALUES (25, 85800, 21450)}\\
                   
            $q_3$: & \texttt{\small UPDATE Taxes SET pay=income-owed}\\ 
            \hline
            % \bottomrule
        \end{tabular}
    \end{minipage}
    \begin{minipage}[t]{0.28\textwidth}
         \vspace{0pt} 
         \centering
        \begin{tabular}{llll}
            \multicolumn{4}{l}{\emph{Taxes}: $D_4$}\\
            \toprule
            \textbf{ID}  & \textbf{income}    & \textbf{owed}  & \textbf{pay}\\
            \midrule
            $t_1$ & \$9500    & \$950		& \$8550\\
            $t_2$   & \$90000   & \$27000	& \$63000\\
            \rowcolor{mid-gray}
            $t_3$ & \$86000   & \color{red}{\$25800} & \color{red}{\$60200}\\
            \rowcolor{mid-gray}
            $t_4$	 & \$86500	  & \color{red}{\$25950}	& \color{red}{\$60550}\\
            $t_5$	 & \$87000	  & \$21750	& \$65250\\
            \bottomrule
        \end{tabular}
    \end{minipage}

    \caption{A recent change in tax rate brackets calls for a tax rate of 30\% for those with income above \$87500.  The accounting department issues query $q_1$ to implement the new policy, but the predicate of the WHERE clause condition transposed two digits of the income value.
\iffalse  
As a result, the owed amount of $t_3$ and $t_4$ were calculated incorrectly. 
This mistake is obscured by $q_2$ which insert a tuple with correct income and owed amount and later further propagated to additional filed(s) by query $q_3$ which calculates the pay check amount based on the corresponding income and (incorrect) owed values.
\fi
    % \alex{I am confused about what happened with this example.  The numbers in the figure don't match the description in Example 3.  This doesn't see to be used later (yet).  Before I fix it, I want to check whether there is a reason for the particular changes...}}\xlw{I changed this example for later optimization sec.6 and noise sec.7 handling explanation. There are two goals: 1. we will demonstrate why we need 2 iterations to reduce noise when we only encode tuples in the complaint set for efficiency; 2. we will show if how we detect and prune false positive complaint $t_5$ by constructing complaint-tuple bipartite graph and searching densest sub-graph. 
    }
    \label{fig:example}
\end{figure*}

This type of data entry error can be found throughout information management systems. 
In 2012, there were nearly 90,000 local governments in the United States, each responsible for tracking taxation information such as the tax rate, regulatory penalties, and the amount each citizen owes. 
Government employees manage this information using form-based accounting systems~\cite{reutersmanagement} and are ultimately susceptible to form entry error.  
A cursory search of the news finds numerous reports of how data entry errors resulted in utility companies overcharging cities by millions of dollars~\cite{powercompany}, a state attorney prosecuting the wrong people~\cite{mosby}, and the Boston Police paying tens of millions of dollars for a new records system to combat data entry errors that ``riddled'' the previous system, which contained ``highly scrutinized stop-and-frisk information''~\cite{bostonpolice}.
In addition, numerous articles describe how clerical errors by local governments~\cite{twinfalls} and the CRA~\cite{cra} affect citizen tax bills and returns calculations.

% Most recently in 2015, a tax mistake, affecting taxpayers in 37 states impacted thousands of lower-income Americans~\cite{nyt-tax-article}: about 20\% of the forms mailed by the federal government contained incorrect premiums on health insurance plans, which in turn affected the amount of tax credits each taxpayer could claim.  
% 
% Although the federal government found the root cause in this high
% profile case \alex{This is not quite true; they just mailed out corrected forms, but didn't find out what caused this (according to the article).  How do we want to say this?}, 
In real-world cases like these,
data errors are typically identified and reported by
individuals to departments that do not have the resources nor the
capability to deeply investigate the reports. Instead, the standard
course of action is to correct mistakes on a case-by-case basis for
each complaint. As a result, unreported errors can remain in the
database indefinitely, and their cause becomes harder to trace as
further queries modify the database, propagate the errors, and obscure
their root cause. There is a need for tools that can use the error
reports to diagnose and identify the anomalous queries (root causes)
in the database.

% that may have affected the erroneous records (i.e., query lineage) \alex{is this a recognized term, or did we come up with it?} \alex{also, do we make a distinction between the query log and the query lineage? It's not clear here.}

\looseness -1
In this paper, we present \sys, a diagnosis and repair system for data
errors caused by anomalous DML queries in OLTP applications. Given a
set of reported errors (\emph{complaints}) about records in the current database state, \sys
analyzes the sequence of historical queries executed on the database, 
filters them to those that may have affected the erroneous records,  and
generates diagnoses by identifying the specific subset of queries that
most likely introduced the errors. Alongside these diagnoses, \sys also
proposes repairs for the erroneous queries; these repairs can correct the
reported errors, as well as potentially identify and fix 
additional errors in the data that would have otherwise remained
undetected. To derive these diagnoses and repairs, we must address three
key characteristics, which make this problem both difficult to solve
and unsuitable to existing techniques:
\begin{description}[leftmargin=0mm, topsep=0mm, itemsep=0mm, parsep=1mm]
\item[Obscurity.] Handling data errors directly often leads to partial
fixes that further complicate the eventual diagnosis and resolution of
the problem. For example, a transaction implementing a change in the
state tax law updated tax rates using the wrong rate, affecting a
large number of consumers. This causes a large number of complaints to
a call center, but each customer agent usually fixes each problem
individually, which ends up obscuring the source of the problem.

\item[Large impact.] Erroneous queries cause errors at a large scale.
The potential impact of the errors is high, as manifested in several
real-world cases~\cite{Yates10, Grady13, sakalerrors}. Further, errors
that remain undetected for a significant amount of time can instigate
additional errors, even through valid updates. This increases both
their impact, and their obscurity.

\item[Systemic errors.] The errors created by bad queries are
\emph{systemic}: they have common characteristics, as they share the
same cause. The link between the resulting data errors is the query
that created them; cleaning techniques should leverage this connection
to diagnose and fix the problem. Diagnosing the cause of the errors
will achieve systemic fixes that will correct all relevant errors,
even if they have not been explicitly identified. 

\end{description}
% 
Traditional approaches to data errors take two main forms. The first
uses a combination of detection algorithms (e.g., human reporting,
outlier detection, constraint violations) to identify a candidate set
of error values that are corrected through
human-based~\cite{haasclamshell,Gokhale:2014wv,Kandel:2011vj} or
semi-automated means (e.g., denial constraints~\cite{ChuIP13}, value
imputation). Unfortunately, this has the following problems: (a)~it targets the symptom (incorrect
database state) rather than the underlying cause (incorrect queries),
(b)~it can be expensive to perform, (c)~it may introduce errors if
the automated corrections are not perfect~\cite{cleaningbenchmark}, and (d)~it may make it harder to identify other data affected by the bad query.

The second form attempts to prevent data errors by guarding against erroneous updates.  For example, integrity constraints~\cite{Khoussainova2006} reject some improper updates, but only if the data falls outside rigid, predefined ranges.  In addition, data entry errors such as in the tax example will satisfy the integrity constraints and not be rejected, despite being incorrect.  Certificate-based verification~\cite{Chen2011} is less rigid, but it is impractical and non-scalable as it requires users to answer challenge questions before allowing each update.

% \looseness -1
\sys is complementary to existing techniques: it does not prevent errors from entering the database, and its primary goal is not to identify errors in the data.  Rather, given some reported data errors, \sys analyzes query histories to determine how the errors entered the database. Determining the root cause of data errors can in turn help identify additional data errors that are due to the same cause, and which would have potentially remained unidentified.  Specifically, in this paper, we make the following contributions:

\begin{itemize}[leftmargin=*, topsep=0mm, itemsep=0mm]

\item We formalize the problem of \emph{Query Explanation}: diagnosing a set of data errors using the log of update queries over the database.  Given a set of \emph{complaints} as representations of data discrepancies in the current database state, \sys determines how to resolve all of the complaints with the minimum number of changes to the query log (Section~\ref{sec:abstractions})

\item We illustrate how existing synthesis, learning, and cleaning-oriented techniques have difficulty scaling beyond a query log containing a single query.   We then introduce an exact error-diagnosis solution using a novel mixed integer linear programming (MILP) formulation that can be applied to a broad class of OLTP applications.  This approach uses state-of-the-art solvers to identify optimal diagnoses that are guaranteed to resolve all complaints without introducing new errors to the data (Section~\ref{sec:sol}).

\item We present a suite of optimizations that reduce the problem size without affecting the quality of the proposed repairs.  Further, we propose a pragmatic incremental algorithm tailored to cases when the user is looking for individual corrupt queries (in contrast to sets of corruptions), and show how these optimizations can scale to large datasets (100K records, Figure~\ref{f:heuristic_time}) and query logs (up to 2K DML statements, Figure~\ref{f:tpcctatp}), and tolerate incomplete information such as unreported errors (Section~\ref{sec:experiments:hardprob}).

\item We perform a thorough evaluation of the data and query log characteristics that influence \sys's trade-offs between performance and accuracy.  We compare the baseline and optimized algorithms under a controlled, synthetic setting and demonstrate that our optimizations improve response times by up to $40\times$ and exhibit superior accuracy.   We also evaluate QFix on common OLTP benchmarks and show how QFix can propose fully accurate repairs within milliseconds on a scale 1 TPC-C workload with 2000 queries (Section~\ref{sec:experiments:benchmark}).

\end{itemize}
