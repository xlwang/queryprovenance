%!TEX root = ../main.tex

%%%%%%%%%%%%%%%%%%%%%%%%%%%%%%%%%%%%%%%%%%%%%
\section*{Comments by Reviewer 2}

\noindent
\textbf{Comment \#1:} Improvements
\begin{quote}
\reviewer{
W1) Given that the paper considers a novel problem, I believe it could trigger
quite some follow-up work. However, to do so, I personally think that further
technical details or detailed theoretical discussion are necessary.
 
}
\end{quote}

We have strengthened our technical contributions with a discussion of the
theoretical guarantees of our slicing optimization methods. We demonstrate
that query and attribute slicing are sound: they improve the efficiency of
\sys without compromising accuracy. The third slicing technique, tuple
slicing, is heuristic in the general case, and could potentially introduce
incorrect repairs. However, it is highly effective in practice, and unlikely
to make mistakes in many problem settings. In particular, we also show that
\emph{tuple slicing is sound in certain settings}, namely, when the complaint
set is complete and if corruptions are limited to a single query. In these
cases, by using incremental repair (Section~\ref{sec:incremental}) and by
disallowing all non-complaint tuples in the refinement step (e.g., by
restricting the value of the objective function in the refinement MILP to
zero), the solver will be forced to pick the correct repair.


\comskip

\noindent
\textbf{Comment \#2:} Evaluation improvements
\begin{quote}
\reviewer{
W2) I find the experimental evaluation rather preliminary and would appreciate
experiments on larger data sets or some real data set.
}
\end{quote}

In the revision, we have augmented our experimental evaluation in three ways:
(1)~We extend the experiment in Figure~\ref{f:attr100} to larger data sizes,
up to $100K$ records. This experiment uses the most complex setting
(\texttt{UPDATE} queries with \textit{range} \texttt{WHERE} clause), and shows
that \sys is efficient for datasets of $100K$ records, even when the query
corruption is not recent.
(2)~We analyze the repairs produced by \sys to measure how frequently it
selects an incorrect query for a repair. Our evaluation examines different
types of workload and varies corruption ages, data size, and incompleteness of
complaints (Figure~\ref{fig:truerate}).
(3)~We study the performance of \sys under variations of query interactions
and include additional experiments varying query selectivity in
Appendix~\ref{app:selectivity}. 

We agree that testing \sys on real-world data is highly desirable. We were not
able to find a suitable public dataset. We used publicly available benchmarks
as the next best choice.
