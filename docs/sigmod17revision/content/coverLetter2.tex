%!TEX root = ../main.tex

\twocolumn[
\begin{center}
    {\LARGE \textbf{Cover Letter}}
\end{center}
\bigskip
]

\definecolor{commentColor}{HTML}{0000FF}

\newcommand{\reviewer}[1]{\itshape{{\color{commentColor} #1}}}
\newcommand{\comskip}{\bigskip}


\renewenvironment{quote}
{\vspace{-1mm}\list{}{\rightmargin=0cm \leftmargin=0cm}%
\item\relax}
{\endlist}

We thank the reviewers and meta-reviewer for their insightful comments on our
SIGMOD 2017 submission and their suggestions on how to strengthen our work. We
used their comments, questions, and suggestions to revise and improve our
paper. We provide here a summary of the revisions based on the reviews. We
quote reviewer comments in blue, above our responses.

\section*{Comments by Meta-Reviewer}

\noindent
\textbf{Comment \#1:} Slicing soundness
\begin{quote}
\reviewer{
Discuss and clarify the soundness of slicing techniques.
}
\end{quote}

We propose tuple, query, and attribute-slicing optimization to improve \sys's
performance.

\emph{Tuple-slicing} is a \textbf{heuristic} that breaks down a large MILP problem 
into two, often much smaller, MILP problems. It works effectively in practice and 
greatly helps improving \sys's performance, especially when the ratio of complaint 
set size and the database size is small. However, it it possible that \emph{tuple-slicing} 
fixes the incorrect query and introduce noises to the query log repair. 
In addition, \emph{tuple-slicing} suffers from higher risk when applying to workloads 
with nested updating queries where tuples are no longer independent to each other, 
e.g., query update one tuple by reading from another tuple. This is because queries 
may read from non-complaint tuples and update tuples in the complaint. Without including 
variables from non-complaints, the constructed MILP problem become loosely constrained 
and noises are more likely to be introduced. 
 
However, for problems with complete complaint set, by further enforcing \textbf{no non-complaint tuples}
in the refinement step (disallow above zero objective-function value for the refinement MILP problem), 
we ensure the correctness of tuples that are not in the complaint. In this paper, since we 
mainly focus on problems with single corruption and complete complaint set, 
\emph{tuple-slicing} is an effective method to improve the performance of the \emph{basic} 
approach, without compromising, and often improving, repair quality.  
 
Here we show a simple example when \emph{tuple-slicing} may fail:

\indent \textit{$q_1$ writes $t_1$ (incorrect query); \\
\indent $q_2$ writes $t_2$;\\
\indent $q_3$ writes $t_1, t_3$.}\\
A workload with three queries where $q_1$ is the incorrect query updates tuple $t_1$. 
Another query $q_3$ covers the impact of $q_1$ and updates tuple $t_1$ and $t_3$. 
When only considering tuples in the complaint set, \sys may fix $q_3$ instead of $q_1$ 
while modifying an additional tuple $t_3$. In this case, \emph{tuple-slicing} introduces 
noises, however, such noise could be identified by the non empty non-complaint set. In case
where the $t_1$ is guaranteed as the only erroneous tuple, such noise could be further avoided 
by enforcing empty non-complaint set.
 
\emph{Query-slicing} and \emph{attribute-slicing} remove irrelevant queries and attributes 
according to whether or not they would affect the incorrect values in the complaints. More specifically,
\emph{query-slicing} generates the \textbf{full-impact} of each query by tracing forward the 
query history and searching for all attributes that may modified by the current query. Thus, 
when the \textbf{full-impact} of a query does not intersect with incorrect attributes in the 
complaint set, the query is guaranteed irrelevant to the errors reported in the complaint set. 
Similarly, given a set of candidate queries that may contribute to the errors (the whole
query history without \emph{query-slicing}), \emph{attribute-slicing} summarizes all attributes 
that may modified by these queries and removes attributes that are not in the previous list.
Thus, both \emph{query-slicing} and \emph{attribute-slicing} are guaranteed to be correct. 
 
\comskip

\noindent
\textbf{Comment \#2:} Clarifications
\begin{quote}
\reviewer{
Clarify the assumptions, their interaction, and why they are realistic in practice. clarify that the system is not repairing the update queries, but the affected data (i.e., query is not correct and errors can still happen).}
\end{quote}

In this paper, we propose a framework \sys that identifies errors in query history and generates query log repair. 
We assume errors are only in \textbf{query data instead their structures}. In addition, we focus on relatively 
``simple'' OLTP queries with \textbf{no subqueries, aggregation, joins, and user-defined functions (UDFs)}. 
We concentrate on a setting where no errors are exist in the data before importing into the database, i.e., 
the initial database state always starts from empty or a clean checkpoint. \sys is not restricted to consistent 
complaint set and we evaluate \sys's performance with incomplete complaint set in Figure~\ref{f:falsenegative}. 
We leave the improvement on other types of inconsistency (e.g., false positive complaints, incomplete correct
values) into our future work. Besides, \sys does not require single query corruption either, however, \sys 
suffers from severe scalability limitation using current MILP solvers. We expect future improvement on \sys
as well as MILP solvers and hardware to solve problems with multiple corruptions efficiently. 

We also summarize all these assumptions in Section~\ref{sec:abstractions}.

\alex{We should give some details here.}

\comskip

\noindent
\textbf{Comment \#3:} Larger data sizes
\begin{quote}
\reviewer{
Add experiment with larger data size, or clarify why they cannot be done.}
\end{quote}

% We extend our experiment with larger database sizes in Figure~\ref{f:attr100}
% with up to $100k$ tuples.
We apologize for the inconsistency between our scalability statement in the Introduction, 
which actually refers to our experiments in Appendix~\ref{sec:heuristic}, and the experiments 
in our main content. 

In this revision, we further extend our experiments and augment
Figure~\ref{f:attr100} to database with up to $100k$ records. 
We demonstrate that even with the most complex
setting, \texttt{UPDATE} queries with \textit{range} \texttt{WHERE} clause,
\sys scales to databases with $100k$ records when the corruption age is as old
as $250$.

\alex{We should add a couple more sentences here to apologize for the
apparent inconsistency, and say that the 100k experiments were in the
appendix, but they are now in the main paper. Did we just move an experiment,
or did we augment the presented results?}


\comskip

\noindent
\textbf{Comment \#4:} Error generation
\begin{quote}
\reviewer{
investigate how many of the errors exhibit interesting patterns (such as a
later update masking the effect of an earlier erroneous one). If there are
already lots of examples of this, compare effectiveness on the ``simple''
errors with the ``complex'' ones; otherwise, you should improve the test data
generator.
}
\end{quote}


To carefully evaluate \sys's performance over problems with different
properties, we introduce a synthetic data generator in Section~\ref{sec:setup}
with multiple adjustable parameters, including query type, workload and
database size, and query selectivity. With the help of these parameters, we
are able to control one of the key properties in the workload, the interaction
of the queries, and thus to better understand \sys's performance. The level of
query interactions greatly influences to the hardness of the problem as errors
may propagate differently under different scenarios.

There are many factors that may affect the level of query interactions. In
general, \texttt{UPDATE} queries are heavily interact with each other than
\texttt{INSERT} or \texttt{DELETE} queries since tuples persist in the
database and may be continuously updated by the following \texttt{UPDATE}
queries. We demonstrate this in Figure~\ref{f:indelup_time} and observe that
under the same corruption age, \texttt{UPDATE}-workload requires longer time
to solve than \texttt{DELETE} and \texttt{INSERT}-workloads. For
\texttt{UPDATE} queries, \textit{constant} \texttt{SET} clauses do not require
former attribute value(s), and thus have less query interaction compare to
queries with \textit{relative} \texttt{SET} clauses
(Figure~\ref{f:qidx_time}). Under default parameters, \texttt{UPDATE} queries
with \textit{point} \texttt{WHERE} clauses update less amount of tuples than
queries with \textit{range} \texttt{WHERE} clauses. Thus, the first type
oftentimes has lower level of query interaction and easier to solve in
practice (Figure~\ref{f:qidx_time}). For \texttt{UPDATE} queries with
\textit{range} \texttt{WHERE} clauses, we found that both the number of
attributes $N_a$ and query selectivity $s$ may strongly influence the level of
query interaction (Appendix~\ref{app:selectivity}). Under our default setting,
increase the number of attribute $N_a$ reduces the query interaction and thus
we observe that problems with larger number of $N_a$ are easier to solve in
Figure~\ref{f:attr}. On the other hand, increasing query selectivity $s$ means
more tuples are updated by each query and thus leads to higher level of query
interaction. Thus, higher query selectivity leads to longer solving time.



\comskip

\noindent
\textbf{Comment \#5:} Possible baselines
\begin{quote}
\reviewer{
Implementation of a baseline (described in the review) or explain why such
baselines would fail in the context, to convince readers that complex patterns
can be handled only by this new system.
}
\end{quote}

In this paper, \sys solves two problems: 1. it finds the root reason(s) in the
query history that causing database errors, and 2. it fixed the incorrectness
by proposing a query log repair. Existing works~\cite{Wu13, roy2014formal,
chalamalla2014,meliou2011tracing} mentioned by reviewer 3 study highly
corrected problems: in general, they all target at explaining erroneous or
undesired query results by tracing back data source input and reporting either
particular input tuples or common input patterns. After carefully analysis, we
conclude that none of these existing works could fix the incorrect queries
(problem 2) nor solve the error identification task (problem 1) effectively
and efficiently. We provide detailed analysis for each works in Reviewer 3's
\# comment 2.


%%%%%%%%%%%%%%%%%%%%%%%%%%%%%%%%%%%%%%%%%%%%%

\section*{Comments by Reviewer 1}

\noindent
\textbf{Comment \#1:} Slicing soundness
\begin{quote}
\reviewer{
What are the soundness/correctness properties needed for the slicing
heuristics to be applicable/useful?

\dots

Unclear whether (or why) slicing techniques are ``sound'', or whether they
would be useful on realistic data rather than synthetic / randomly generated
benchmark data.

\dots

p6. Why is tuple slicing sound? It sounds like it will amount to modeling only
the tuples that are subject to a complaint...

\dots

p7. Likewise, I'm not convinced query or attribute slicing are safe
optimizations.
\dots  
this requires proof

}
\end{quote}

\emph{Tuple-slicing} is a \textbf{heuristic} that breaks down a large MILP problem 
into two, often much smaller, MILP problems. It works effectively in practice and 
greatly helps improving \sys's performance, especially when the ratio of complaint 
set size and the database size is small. However, it it possible that \emph{tuple-slicing} 
fixes the incorrect query and introduce noises to the query log repair. In addition, 
under more complex settings where tuples are no longer independent 
(also mentioned in the review ``p6 ...''), \emph{tuple-slicing} 
is more likely to make mistakes. 

In cases where the complaint set is complete,
\sys is able avoid mistakes due to \emph{tuple-slicing} by enforcing empty non-complaint set
in the refinement step. Even with incomplete complaint set, one can adjust constrains 
in the refinement step on the non-complaint set size to control noise introduced by \emph{tuple-slicing}.
To conclude, \emph{tuple-slicing} is safe as long the complaint set is complete on both synthetic / randomly
generated benchmark workload and realistic workload. We plan to improve \sys for handling 
inconsistent complaints, including incomplete complaint set, in our future work.

\emph{Query-slicing} and \emph{attribute-slicing} remove irrelevant queries and attributes 
according to whether or not they would affect the incorrect values in the complaints. 
\sys conducts a throughout analysis by tracing forward the query history and identifies
queries that are not able to modify incorrect attributes in the complaint set. Since adjustment
to such queries has no impact on the erroneous values, it is safe to apply \emph{query-slicing}. 
Similar to \emph{query-slicing}, \emph{attribute-slicing} searches for all attributes that may
modified by queries in the workload (or a subset of queries after applying \emph{query-slicing}) and
avoids encoding attributes that are not in the list into the MILP problem. Thus, \emph{attribute-slicing}
is also safe. 

\alex{Add details, not just references to different responses, as it is hard
for reviewers to go back and forth.}


\comskip

\noindent
\textbf{Comment \#2:} Correction
\begin{quote}
\reviewer{
Please correct definition 6.

\dots

Suppose we have three queries \dots

}
\end{quote}

\alex{The definition needs to be corrected, and more details should be added
here.}

\xlw{Original def. 6 seems misleading, let's verify it.}

The \textbf{full-impact}, $\mathcal{F}(q_i)$, of a query $q_i$ includes all
attributes that may modified by the query $q_i$ and it is calculate by the
intermediate-impact of all its consecutive queries. Let us denote
$\mathcal{F}_j(q_i)$ as the intermediate-impact of query $q_i$ on query $q_j$
($j > i$) and $\mathcal{F}_j(q_i)$ is defined as follows. \[
\mathcal{F}_j(q_i)=\mathcal{F}_{j-1}(q_i)\bigcup_{\substack{\mathcal{F}_{j-1}(q_i)\cap
\mathcal{P}(q_j) \neq \emptyset}} \mathcal{I}(q_j), \] With initial impact
$\mathcal{F}_i(q_i) = \mathcal{I}(q_i)$. The full-impact of $q_i$,
$\mathcal{F}(q_i)$ is indeed the impact on the final query $q_n$,
$\mathcal{F}_n(q_i)$: \[ \mathcal{F}(q_i)=\mathcal{F}_n(q_i) =
\mathcal{I}_(q_i)\bigcup_{\substack{j = i+1 \\ \mathcal{F}_{j-1}(q_i)\cap
\mathcal{P}(q_j) \neq \emptyset}}^n \mathcal{I}(q_j), \]

For example, in the following query log (example mentioned in reviews):\\
\textit{\indent q1 writes t.A; \\
\indent q2 reads t.A and writes u.B; \\
\indent q3 reads u.B and writes v.C.}\\
\[\mathcal{F}_1(q_1) = \{t.A\};\] 
\[\mathcal{F}_2(q_1) = \{t.A\} \bigcup_{\mathcal{F}_1(q_1) \cap \{t.A\} \neq \emptyset} \{u.B\} = \{t.A, u.B\};\]
\[\mathcal{F}_3(q_1) = \{t.A, u.B\}\bigcup_{\mathcal{F}_2(q_2) \cap \{u.B\} \neq \emptyset} \{v.C\} = \{t.A, u.B, v.C\}.\]
Thus the full-impact of $q_1$ is $\{t.A, u.B, v.C\}$.




\comskip

\noindent
\textbf{Comment \#3:} Experiments on large datasets
\begin{quote}
\reviewer{
Do the experiments substantiate the claim of scalability to ``large''
databases of 100k records? If so the experimental evaluation needs to explain
this more clearly. If not the introduction must avoid overselling this point.

\dots

Experimental evaluation doesn't live up to claims in introduction (5-6K
records vs. 100K)

\dots

the experiments only seem to consider much smaller databases (5000-6000
tuples).

}
\end{quote}

Again, we apologize for inconsistency between our scalability statement and
experiments in the main text and our statement was based on experiments in
Appendix~\ref{sec:heuristic}.

In addition, we extend our scalability experiment in Figure~\ref{f:attr100} to
databases with $100k$ records and demonstrate that even with the most complex
setting, \texttt{UPDATE} queries with \textit{range} \texttt{WHERE} clause,
\sys scales to databases with $100k$ records when the corruption age is as old
as $250$.

\alex{Say more about the inconsistency between intro and evaluation.}


\comskip

\noindent
\textbf{Comment \#4:} Why $M^+$ in \texttt{DELETE} query?
\begin{quote}
\reviewer{
It's also not entirely clear why it is correct to use a ``large/unused'' value $M^+$ for attributes of deleted tuples.

\dots

p5. Handling DELETE using $M^+$. \dots I'm not sure I understand why this the
case and if so, why it is correct. \dots for example if the test is ``t.A > 5'' then setting t.A to $M^+$ (assuming it's bigger than 5) will not invalidate the test.

}
\end{quote}

\alex{Make sure to address both points!}

For clarification we rename the former variable $M^+$ as $M^-$ to distinguish
from $M$ and emphasis its relationship with $M$ as $M^- \leq M$. $M^-$ is set
to a large enough number outside the attribute domain in order to guarantee
that \sys would fix the incorrect \texttt{DELETE} query instead of its
consecutive \texttt{UPDATE} queries. Recall our MILP encoding in Equation 6,
to ensure that the attribute value of tuple is set to $M^-$ in a
\texttt{DELETE} query, one need to set $x_{q_i,t} = 1$ which only requires
variable adjustment in the \texttt{WHERE} clause such that $\sigma_{q_i}(t) =
1$. Alternatively, it may also feasible to manipulate the consecutive
\texttt{UPDATE} query such that the \texttt{SET} clause updates the tuple to
$M^-$ as $\mu_{q_i}(t) = M^-$ (according to Equation 4). Modifying
\texttt{DELETE} query or \texttt{UPDATE} query may all lead to feasible
solution, however, different objective-function values are associated with
them. By setting $M^-$ to a ``sufficiently large'' number, we force the
objective-function value of the \texttt{UPDATE} query modification exceedingly
large than the \texttt{DELETE} query's. Thus it is guaranteed that \sys will
always fix the incorrect \texttt{DELETE} query instead of its consecutive
\texttt{UPDATE} queries. In addition, we set $M^- \leq M$ to avoid generating
infeasibility conditions.

Note that for workloads with both \texttt{DELETE} and \texttt{UPDATE} queries
while there exist \texttt{UPDATE} queries without a \texttt{WHERE} clause. 
The assignment of $x_{q_i, t}$ in a \texttt{UPDATE} query 
(with no \texttt{WHERE} clause) requires the following change:
\begin{align}
\label{eq:x2}
x_{q_i, t} = x'_{q_i, t} \otimes 0 + (1-x'_{q_i, t}) \otimes \sigma_{q_i}(t) \nonumber \\
x'_{q_i, t}= (t.A_j = M^-) \wedge (t.A_j^* = M^-) 
\end{align}
This arrangement avoid generating invalid MILP problems where a deleted tuple with $(t.A_j = M^-) \wedge (t.A_j^* = M^-)$
is further updated by \texttt{UPDATE} queries due to the original assignment of $x_{q_i, t}  = 1$.

In the example in $p5$, suppose a tuple is deleted and values in all attributes are assigned to $M^-$. 
The boolean variable $x'_{q_i, t}$ is set to $1$ for any following \texttt{UPDATE} queries (regardless
of there \texttt{WHERE} predicate) and 
thus \texttt{UPDATE} queries won't updates values for deleted tuples. 

\comskip

\noindent
\textbf{Comment \#5:} Assumptions
\begin{quote}
\reviewer{
Problem specification seems restrictive (users need to provide complete,
correct rows as complaints).

\dots

p3. Please clarify whether a complaint needs to specify exact fix values for all fields \dots 
}
\end{quote}

While in the paper we use exact fix values for all fields of a complaint, we
only do so for ease of exposition, and this is actually not a limitation in
our techniques. We have added clarification about this in
Section~\ref{sec:model}.



\comskip

\noindent
\textbf{Comment \#6:} Additional corrections and clarifications

\smallskip

We thank the reviewers for the many detailed comments. We have corrected the
typos, and reworded confusing sentences. We also provide some further notes on
some of the detailed comments below.

\begin{quote}
\reviewer{
p2. Although it is just an example, it does seem like questionable design for
the tax rates for different salary levels to be stored in the query rather
than in the database.
}
\end{quote}

The example is emulating a form-based entry of the tax-rate information, but
we agree that in practice the design would be much more complex. We chose a
substantial simplification to make the example easy to follow and to allow us
to more easily highlight the contributions of our work.

\begin{quote}
\reviewer{
p2. The figures in the example seem wrong (ironically) \dots
}
\end{quote}

Thank you for noting this mistake. We corrected the figure to have the
correct numbers.


\begin{quote}
\reviewer{
p3. This is explained later, but please clarify that the expressions used in
mu and sigma are from a limited language (e.g. arithmetic).
}
\end{quote}

\alex{Have we done anything about this?} \xlw{The explanation and definition is
quite close. We give def and then description. Maybe we should switch the order?}


\begin{quote}
\reviewer{
p3. What does notation ``$D \setminus\{t\}$'' mean (in def 3) if $t = \bot$? No-op? Also,
t* should be allowed to range over $D \cup \{\bot\}$ also to allow for deletion
(as discussed on the next page).
}
\end{quote}

We correct our definition on range of $t$ and $t^*$ as $D \cup \{\bot\}$. The transformation of 
a database state $\mathcal{T}_c(D)$ excludes the complaint tuple $\{t\}$ and replace it with 
$t^*$ with correct values. If $t = \bot$, it means tuple $t$ does not exist in the original
database and should be added into the database; similarly if $t^* = \bot$, the tuple $t$ is in
the original database but should be removed. 

\alex{Have we done anything about this?}

\begin{quote}
\reviewer{
p4. The paper often refers to ``data manipulation queries'', which sounds
self-contradictory to me (a query reads, an update writes). Perhaps just
``updates'' instead of ``data manipulation queries'' ?
}
\end{quote}

We change it as ``data manipulation statements'' that include \texttt{UPDATE},
\texttt{INSERT}, and \texttt{DELETE} queries. We intentionally avoid naming 
them ``update queries'' in order to distinguish from the \texttt{UPDATE} query. 

\alex{This term is only used once in the paper. How about we just change it to
data manipulation statements?}


\begin{quote}
\reviewer{
p5. How strongly does the performance of the MILP solver depend on M (the
upper bound chosen on the value of a given numeric field)? Presumably,
choosing the largest machine-representable number would lead to overflow
problems?
}
\end{quote}

To guarantee the correctness of the \texttt{UPDATE} queries, $M$ must 
be assigned to a value outside the domain range of the corresponding attribute 
because constraints in ~\ref{eq:uv} will be self-contradictory otherwise. For example, 
attribute \textit{human age} has a domain range of $[0, 200]$, in which case $M$ must be 
set to a value that is above $200$. 
In addition, in order to maintain the feasibility for workloads with \texttt{DELETE}
queries, $M$ must be set to a value that is $\geq M^-$ (we explained
the setting of $M^-$ in Comment\# 4), but not necessary a 
upper bound of the given numeric field. 

\alex{Have we done anything about this?}


\begin{quote}
\reviewer{
p5. In (5), should v be u?
}
\end{quote}

We thank the reviewer for noting this, we correct this typo in the paper. 
\alex{(5) does not seem to have changed. If it does not need to be changed, we
should explain here why.} 


\begin{quote}
\reviewer{
p7. ``Figure 3 showed the exponential cost'' - the scaling from 40 to 60 to 80 looked linear to me, just with a high coefficient.}
\end{quote}
We apologize for not stating it clear. In Figure~\ref{fig:querysize_vs_time}, \naive fails to produce an answer
 for problems with $\geq 80$ queries within 1000 sec and \naive shows exponential cost for problems with $20$, $40$, and $60$ queries. 
\alex{What is the answer here?} \xlw{We add a sentence in the caption to clarify this. Instead of fixing problems with 80 queries in 1000 sec, 
baisc fails to derive a result in 1000sec. }


\begin{quote}
\reviewer{
p10. Fig. 8 - please add 0.75 (the largest x-axis value) to the x-axis, better yet, label the x-axis with the rate values that correspond to data points.
}
\end{quote}

We thank the reviewer for the detailed suggestion and we implement this in our main text. 

\begin{quote}
\reviewer{
p10. ``may suffer if the corruption occured in a very old query'' - does it (and are you saying fig. 8 supports this claim)?

p10. ``may over-generalize'' - again, are you speculating or interpreting the results here?
}
\end{quote}

We observe from Figure~\ref{f:falsenegative_acc} that \sys derives inaccurate log repairs when the corruption occured in 
very old query  \textit{corruption age = 250}). Also, if there are insufficient complaints, then it is possible that the 
repair will over-generalize to  ``repair'' the wrong records and lead to lower precision.

\alex{What have we done for this?} \xlw{We added some clarification in the text. }

\begin{quote}
\reviewer{
p10, footnote: this seems like a strong assumption! I guess this means that dealing with multiple errors that interact with each other is future work...
}
\end{quote}

We clarify this assumption in Section~\ref{sec:abstractions}. Currently, we cannot apply incremental algorithm 
on problems with multiple corruptions since errors may spread across the entire query history, \sys 
does not scale over problems with more than 40 queries. We leave further improvement in our future work.

\alex{We need to say something here.  Have we clarified this assumption early in the paper?}


%%%%%%%%%%%%%%%%%%%%%%%%%%%%%%%%%%%%%%%%%%%%%

\section*{Comments by Reviewer 2}

\noindent
\textbf{Comment \#1:} Improvements
\begin{quote}
\reviewer{
W1) Given that the paper considers a novel problem, I believe it could trigger
quite some follow-up work. However, to do so, I personally think that further
technical details or detailed theoretical discussion are necessary.
 
}
\end{quote}

We clarify the soundness of the proposed optimization and conclude that
\emph{tuple-slicing} is a highly effective heuristic that greatly improved
\sys's execution efficiency and unlikely to make mistakes in practice;
\emph{attribute, query-slicing} follow attribute updating lineage and thus
guarantee the correctness.

\comskip

\noindent
\textbf{Comment \#2:} Additional experiments
\begin{quote}
\reviewer{
W2) I find the experimental evaluation rather preliminary and would appreciate
experiments on larger data sets or some real data set.
}
\end{quote}

We improve our experimental evaluation in three ways: we first extend our
experiment with larger database sizes to up to $100k$ records in
Figure~\ref{f:attr100}; we also evaluate \sys's ability in identifying the
actual incorrect query in Appendix~\ref{app:index} and demonstrate that \sys
always fixes the actual incorrect query when the complaint set is complete; we
further study \sys's performance across different level of query interactions
and include additional experiments on query selectivity in
Appendix~\ref{app:selectivity}. Unfortunately, we are unable to get real data
set and we put such evaluation to our future work.


%%%%%%%%%%%%%%%%%%%%%%%%%%%%%%%%%%%%%%%%%%%%%

\section*{Comments by Reviewer 3}

\noindent
\textbf{Comment \#1:} Clarifying assumptions
\begin{quote}
\reviewer{
\dots there is a long list of assumptions scattered around the paper

\dots

The problem is indeed difficult \dots and assumptions must be put in place

\dots

how do the assumptions interact with each other in practice 

\dots

Is it realistic to assume a safe checkpoint? 

\dots

Is it realistic to have at least 25\% of the errors detected?
 
}
\end{quote}

\alex{We need to be detailed, and not just refer to another answer. This
reviewer is asking very specific questions, not just collecting the
assumptions to one place!}


\comskip

\noindent
\textbf{Comment \#2:} Baseline -- Query errors as data
\begin{quote}
\reviewer{
Errors are indeed in the queries, but are ultimately data errors \dots
existing solutions \dots can be directly applied by modelling the query data
as source data + SELECT queries.

\dots

it would be great if the author can precisely explain why such baselines would
fail in their context.

}
\end{quote}

\alex{Good arguments here, but this needs to be cleaned up.}

In this paper, \sys solves two problems: 1. it finds the root reason(s) in the
query history that causing database errors, and 2. it fixed the incorrectness
by proposing a query log repair. Existing works mentioned above study highly
corrected problems: in general, they all target at explaining erroneous or
undesired query results by tracing back data source input and reporting either
particular input tuples or common input patterns. After carefully analysis, we
conclude that none of these existing works could fix the incorrectness nor
solve the error identification task effectively and efficiently.

In order to use these works to diagnose query history, we first convert the
\textsc{Optimal Diagnosis} problem in Definition 4 as \textsc{Source data +
SELECT queries} problem: Given database states $D_0$ and $D_n$, a query log
$\mathcal{Q} = \{q_i\}$ such that $\mathcal{Q}(D_0) = D_n$, and a desired
database state $D_n^*$ with no errors, we model the \textsc{Source data +
SELECT queries} problem by creating a table $P$ for under-determined
parameters in $\mathcal{Q}$ and table $D_0$ as source tables; and a single
nested SELECT query $q_n'$ for queries in $\mathcal{Q}$. We demonstrate an
abstract example in Example~\ref{fig:example}.

\begin{figure}[t]
The \textsc{Optimal Diagnosis} problem:\\
    \begin{minipage}[t]{0.1\textwidth}
         \vspace{0pt} 
         \centering
        \begin{tabular}{ll}
            \multicolumn{2}{l}{Table $D_0$}\\
            \toprule
            \textbf{A}  & \textbf{B}\\
            \midrule
			 1 & 1 \\
			 2 & 2 \\
			 3 & 5 \\
            \bottomrule
            \\
        \end{tabular}
    \end{minipage}
    \begin{minipage}[t]{0.2\textwidth}
         \vspace{0pt} 
         \centering
        \begin{tabular}{p{26ex}}
            \multicolumn{1}{l}{\emph{Query log}: $\mathcal{Q}$}\\
            $q_1$: \texttt{\small UPDATE T SET B=B+1}\\
            \texttt{\small WHERE A > 2 and a < \sout{5} {\color{red} 3}} \\
            $q_2$: \texttt{\small UPDATE T SET B=B+3}\\
                  \texttt{\small WHERE A > 0 and a < 4} \\
        \end{tabular}
    \end{minipage}
    \begin{minipage}[t]{0.16\textwidth}
         \vspace{0pt} 
         \centering
        \begin{tabular}{ll}
            \multicolumn{2}{l}{Table $D_2^*$}\\
            \toprule
            \textbf{A}  & \textbf{B}\\
            \midrule
			 1 & 1 \\
			 2 & 2 \\
			 3 & {\textbf{6}} \\
            \bottomrule
            \\
        \end{tabular}
    \end{minipage}
The \textsc{Source data + SELECT queries} problem: \\
\begin{minipage}[t]{0.1\textwidth}
         \vspace{0pt} 
         \centering
        \begin{tabular}{ll}
            \multicolumn{2}{l}{Table $D_0$}\\
            (same as above) \\
        \end{tabular}
    \end{minipage}
\begin{minipage}[t]{0.36\textwidth}
         \vspace{0pt} 
         \centering
        \begin{tabular}{llllll}
            \multicolumn{6}{l}{Table $P$}\\
            \toprule
            \textbf{$p_1$}  & \textbf{$p_2$} & \textbf{$p_3$} & \textbf{$p_4$}  & \textbf{$p_5$} & \textbf{$p_6$} \\
            \midrule
			 1 & 2 & 3 & 3 & 0 & 4\\
            \bottomrule
            \\
        \end{tabular}
    \end{minipage}\\
        \begin{minipage}[t]{0.22\textwidth}
         \vspace{0pt} 
         \centering
        \begin{tabular}{p{2ex}p{55ex}}
         \multicolumn{2}{l}{\emph{Queries}: }\\
        $q'_1$: &
        \texttt{\small SELECT $D_0.A$ AS A, $D_0.B$+$p_1$ AS B FROM $P, D_0$ } \\
        & \texttt{\small WHERE $D_0.A$ > $p_2$ AND $D_0.A$ < $p_3$ UNION ALL}\\
        & \texttt{\small SELECT $D_0.A$ AS A, $D_0.B$ AS B FROM $P, D_0$ }\\
        &\texttt{\small  WHERE $\neg$($D_0.A$ > $p_2$ AND $D_0.A$ < $p_3$)} \\
        $q'_2:$ &
         \texttt{\small SELECT $D_1.A$ AS A, $D_1.B$+$p_4$ AS B FROM $P, q_1'\ as\ D_1$ } \\
        & \texttt{\small WHERE $D_1.A$ > $p_5$ AND $D_1.A$ < $p_6$ UNION ALL}\\
        & \texttt{\small SELECT $D_1.A$ AS A, $D_1.B$ AS B FROM $P, q_1'\ as\ D_1$ }\\
        &\texttt{\small  WHERE $\neg$($D_1.A$ > $p_5$ AND $D_1.A$ < $p_6$)} \\
        \end{tabular}
    \end{minipage}
    \caption{An abstract example for R3.2. (For simplicity, we use $q_1'$ for the input sub-query when introducing $q_2'$.)}
\label{fig:example-cover}
\end{figure}

\noindent \textbf{[Roy et al. SIGMOD 2014, Wu et al. VLDB 2013, Chalamalla et al. SIGMOD 2014]} \\
TLDR: These papers quantify responsibility of input data to errors in the
output, however responsibility


These papers focus on detecting and summarizing the likely source tuples by
propagating ``responsibility'' from annotated output error tuples to sources.
~\cite{Wu13} targets at explaining a set of SELECT query results by a
conjunction of predicates on attributes where the SELECT clause in each query
is restricted to a single aggregate operator. Similarly, ~\cite{roy2014formal}
explains a group-by SQL query with a single aggregate function. Since there is
no obvious mapping between complaints and a particular aggregate function,
both algorithms cannot solve the error identification problem. Instead of
aggregation results, ~\cite{chalamalla2014} searches for explanations for data
quality rule violations. Even though complaints can be expressed as data
quality rules, the algorithm does not distinguish the actual input source
error (in $D_0$) and query parameter error (in $P$). Thus, in
Example~\ref{fig:example}, the explanation $D_0.A = 3$ on table $D_0$ is also
a valid explanation. Besides, ~\cite{chalamalla2014} relies on the causality
relationship between erroneous data and their lineage, which is invalid for
incorrect parameters in the update workload. Thus, this work cannot
effectively solve the error identification problem either.

\noindent \textbf{[Meliou et al. SIGMOD 2011]} \\
TLDR: The number of disjunctions in the generated SAT expression increases
exponentially with respect to the number of queries in the log. For instance,
if the database contained 100 tuples and 10 queries, the number of
disjunctions would be $\cdot 2^{10}$ and thus the number of lineage
expressions would be $100\cdot 2^{100}$.

The View-Conditioned Causality(VCC) work takes a set of input variables
$\mathbf{X}$, a transformation $\mathbf{\Phi} =\{\Phi_1, \dots , \Phi_m\}$
computing the output values $\mathbf{z}$, and a ground truth
$\hat{\mathbf{z}}$, and detects input variables that are responsible for data
errors $\mathbf{z}|\hat{\mathbf{z}}$. The \textsc{Source data + SELECT
queries} problem can be expressed as a VCC problem where cells in $P$ form the
input variables $\mathbf{X}$ and each tuple $t_i$ in $D_0$ and the SELECT
query $q_n'$ is translated into a lineage expression $\Phi_i$. In each lineage
expression $\Phi_i$, tuple values in $D_0$ and $D_n^*$ are further converted
into constant variables. By doing so, only variables in $P$ will be selected
to explain data errors. In Example~\ref{fig:example}, lineage expression
$\Phi_3$ for tuple $t_3$ is as following:

{\small
\begin{eqnarray*}
&((3 > p_2 \wedge 3 < p_3) \wedge (3 > p_5 \wedge 3 < p_6) \wedge (5 + p_1+p_4 = 6)) &\\
&\vee ((3 > p_2 \wedge 3 < p_3) \wedge \neg(3 > p_5 \wedge 3 < p_6) \wedge (5 + p_1= 6))& \\
&\vee (\neg(3 > p_2 \wedge 3 < p_3) \wedge (3 > p_5 \wedge 3 < p_6) \wedge (5 + p_4= 6))& \\
&\vee (\neg(3 > p_2 \wedge 3 < p_3) \wedge \neg(3 > p_5 \wedge 3 < p_6) \wedge (5 = 6))&
\end{eqnarray*}
}
Each row corresponding to the satisfactory condition for where clauses in $Q$.
The first row, for example, means the where condition is true for both $q_1$
and $q_2$. The ground truth output values for $\mathbf{\Phi}$ are all set to
\textit{True}. By solving the VCC problem, one can identify incorrect
parameters in the query log. However, as you may notice, the lineage
expressions for tuples in $D_0$ are composed by disjunctive sub-expressions
with exponential cardinality to the number of queries in $Q$. Since the
execution efficiency is highly correlated to the complexity of lineage
expressions, VCC solves the error detection task but cannot solve it
efficiently.

To conclude, none of these highly correlated works could handle both problems
addressed by \sys: ~\cite{meliou2011tracing} solves the error detection
problem with low efficiency; ~\cite{chalamalla2014} may solve the error
detection problem but fail to distinguish data source errors and query errors;
~\cite{Wu13} and ~\cite{roy2014formal} restricted to aggregations, thus cannot
solve either problems.


\comskip

\noindent
\textbf{Comment \#3:} Scalability 
\begin{quote}
\reviewer{
\dots relatively small datasets are tested
\dots I suggest to explicitly test the scalability until the bottlenecks of the solutions become clear (the MILP module?)
}
\end{quote}

We extend our experiment with larger database size in Figure~\ref{f:attr100}
and observe that with all optimization, \sys requires $~300$ sec to solve
problems with $100k$ tuples and a corruption age of $250$ queries.

\alex{I am afraid that we are not answering the second part of this reviewer's
question.}


\comskip

\noindent
\textbf{Comment \#4:} Accuracy metrics
\begin{quote}
\reviewer{
I cannot grasp from the experiments section if the authors are counting the
number of errors correctly identified or correctly fixed.
\dots
this implies that even complete complain sets do NOT guarantee that the query
is correctly fixed and future errors will not occur \dots
}
\end{quote}

In our experiment, we measure \sys's accuracy as whether all complaints are
correctly fixed. In the example, we also consider $income >= 87200$ as a
correct query repair as it resolves all the reported complaints. In addition,
the repaired query is allowed to differ from the true query. However, this is
very rare in practice and we show that \sys always pick the correct query to
fix when the complaint set is complete. Though \sys dos not guarantee to
propose a query log repair that is exactly the same with the original true
query log, it identifies the incorrect query and provides a reasonable
suggestion of how to fix the problem.



\comskip

\noindent
\textbf{Comment \#5:} Error generation
\begin{quote}
\reviewer{
I believe the way that errors are introduced can be improved.
\dots Randomly introducing errors do not guarantee that complex patterns (such as the one in Figure 1) happen 
\dots
Unfortunately having some guarantee on the way that errors are introduced can
be a challenging problem \dots
}
\end{quote}

\alex{We should talk about this here.}


\comskip

\noindent
\textbf{Comment \#6:} Connection to data repair
\begin{quote}
\reviewer{
the MILP formulation reminds me the SAT formulation in [25] and similar works
for data cleaning. In fact def 4 is very close to the minimal repair concept
in data repair. This can be clarified.
}
\end{quote}

\alex{Have we done something for this?}