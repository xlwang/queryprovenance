\section{A Naive/Heuristic Approach}

\xlw{describe decision tree based approach}



\section{Deriving a log repair}
\label{sec:sol}
In this section, we 
introduce a basic solver-based approach to 
resolve the incorrectness reflected by the complaints. 
This approach 
constructs a mixed-integer linear 
programming (MILP) problem by
linearizing and parameterizing the 
original query log over tuples
in the database. 
\subsection{Linearizing \& Parameterizing Single Query}
\label{sec:linearize}
% describe how to linearize a single query
We linearize and parameterize a query $q$ by considering its effects on the 
targeted table $R(A_1, ..., A_m)$: we treat the query $q$ as a 
conditional function over each tuple $t\in R$ and convert the effects of $q$ 
over $t$ into a set of linear inequality constraints. 
\subsubsection{Query as a Conditional Function}
The effect of query $q$ over a tuple $t$ can be expressed in a conditional
function $f_q(t)$ as the following:
\begin{definition} [Conditional Function]
\label{def:cond}
	The conditional function for query $q$ is:
	\[
    f_q(t)= 
\begin{cases}
    f_{q.\mu} (t) ,& \text{if } f_{q.\sigma} (t)\\
    t,              & \text{otherwise}
\end{cases}
\]
$f_{q.\mu}$ is the \textit{update function} consists of
 a set of \textit{update equation(s)};\\
 $f_{q.\sigma}$ is the 
 \textit{condition function} consists of a 
set of \textit{logical expression(s)} in 
disjunctive or conjunctive form.
\end{definition} 

 This conditional function applies 
to all kinds of \emph{update queries} (Section~\ref{sec:model}): \\
\texttt{UPDATE} query: $f_{q.\mu}(t)$, $f_{q.\sigma}(t)$ reflects the 
equations in
set clause and the expressions in where clause over attributes in
$R$. \\
\texttt{INSERT} query: $f_{q.\mu}(t)$ reflects the inserted values in
values clause and 
$f_{q.\sigma}(t)$ is a boolean variable 
reflects the existence of tuple $t$; 
$\vee_{t\in R} f_{q.\sigma}(t)$ 
represents
the existence of this insert query.  \\
\texttt{DELETE} query: $f_{q.\mu}(t)$ reflects the deleted values for 
each attribute and 
$f_{q.\sigma}(t)$ reflects the expressions in where clause (similar to
$f_{q.\sigma}(t)$ in \texttt{UPDATE} query).

\smallskip

We parameterize query $q$ by replacing 
 all numeric values 
in the above conditional function into
undetermined variables (Example~\ref{ex:parameterize}).
\begin{example}\label{ex:parameterize}
Consider a \texttt{UPDATE} query $q$:
\texttt{UPDATE R SET A$_1$ = 3 WHERE A$_2$ $\leq$ 10},
the conditional function of this query is: 
\[
    f_q(t)= 
\begin{cases}
    f_{q.\mu}(t) = \{t.A_1 = 3\} ,& \text{if } f_{q.\sigma}(t) = \wedge\{t.A_2 \leq 10\}\\
    t,              & \text{otherwise}
\end{cases}
\]
The numeric variables in query $q$ including \texttt{3} in $f_{q.\mu}(t)$ and \texttt{10}
in $f_{q.\sigma}(t)$. Thus, we parameterize query $q$ by replacing the value \texttt{3, 10}
by two undetermined variables \texttt{var1, var2}. 
\end{example} 
\subsubsection{Constructing Linear Inequality Constraints}
We linearize query $q$ by transforming its conditional function 
over each tuple $t\in R$ into a set of linear (in)equality constraints. \\
According to Definition~\ref{def:cond}, 
the updated value $t'$ for tuple $t$ in attribute $A_i$ can be expressed as  
\begin{eqnarray}
\label{eq:linearization}
t.A_i' = x\otimes f_{q.\mu(A_i)} (t.A_i) + (1-x)\otimes t.A_i. 
\end{eqnarray} 
Linearizing Equation~\ref{eq:linearization} is equivalent as linearzing
query $q$.  
To linearize Equation~\ref{eq:linearization}, 
we first introduce a boolean variable $x$ to represent the satisfactory 
of tuple $t$ on the condition function of query $q$:
\begin{eqnarray}
\label{eq:x}
x = f_{q.\sigma}(t)
\end{eqnarray}

We then create two set of intermediate variables,
$u=\{u.A_1, ..., u.A_m\}$, $v = \{v.A_1, ..., v.A_m\}$, 
for each attribute $A_i$ in tuple $t$:
\begin{eqnarray}
\label{eq:uv}
u.A_i &\leq & f_{q.\mu(A_i)} (t.A_i) \nonumber\\
u.A_i &\leq & xM \nonumber\\ 
u.A_i &\geq & f_{q.\mu(A_i)} (t.A_i) - (1-x)M ; \nonumber \\\nonumber \\
v.A_i &\leq & t.A_i \nonumber\\
v.A_i &\leq & (1-x)M \nonumber\\
v.A_i &\geq & t.A_i - xM
\end{eqnarray}
Here, $M$ denotes a very large number (greater than the upper bound for the $t.A_i$).
By doing so, we may express Equation~\ref{eq:linearization} as:
\begin{eqnarray}
\label{eq:tnew}
t.A_i' = u.A_i + v.A_i
\end{eqnarray}
By combining these linear (in)equality constraints 
(~\ref{eq:x}~\ref{eq:uv}~\ref{eq:tnew}), we form the transformation 
of a single attribute $A_i$ from a single tuple $t$ 
by a single query $q$. To fully linearize a query, 
similar constraints should be
derived for the other attributes and the other tuples. 

\xlw{Briefly describe what the settings of the variables $u.A_i$ and $v.A_i$ represent in terms of the queries.}

\subsection{A Basic Solver-based Approach}
\label{sec:milp}
Using the linearization
 method in Section~\ref{sec:linearize}, we can further
linearize
the entire query log by converting every tuples in the table $R$. 
During the linearization, we further parameterize each
query in the query log $\mathcal{Q}$ in order by 
derive the log repair. 
The linearized and parameterized query log
should start from and end at clean database states.
To achieve this, we add constraints
by assigning the true initial and end database 
states' values (based on complaints) 
to the corresponding variables. 
Following the above steps 
(Algorithm~\ref{alg:basic}), we convert the 
query log into a collection of constraints 
with newly introduced variables. 
Some of these variables
are introduced to linearize the query log 
and the rest usually
represent the numeric values in the query log
during the paramerization process. 
The latter set of 
variables often involved in the objective function 
according to the pre-defined distance function $d$ 
(as described in Definition~\ref{def:problem}). \\
\xlw{The above is pretty dense.  Describe and justify objective function, outline how to encode database state. }

\begin{algorithm}[htbp]
\caption{$QueryFix_{exh}$ algorithm based based on MILP formulation.}
\label{alg:basic}
\begin{algorithmic}
\REQUIRE {$\mathcal{Q}, D_0, D_n, \mathcal{C}$}
\ENSURE {$\mathcal{Q^*}$}
\STATE $milp\_cons \leftarrow \emptyset$
\FOR {each $t$ in $R$}
\FOR {each $q$ in $\mathcal{Q}$}
\STATE $milp\_cons \leftarrow milp\_cons \cup LinearAndParam(q, t)$
\ENDFOR
\STATE $milp\_cons \leftarrow milp\_cons \cup AssignVals(D_0.t, D_n.t, \mathcal{C})$
\ENDFOR 
\STATE $milp\_obj \leftarrow FormObj (milp\_cons, \mathcal{Q})$
\STATE $solved\_vals \leftarrow MILPSolver(milp\_cons, milp\_obj)$
\STATE $\mathcal{Q}^* \leftarrow ConvertQLog(Q, solved\_vals)$
\STATE Return $\mathcal{Q}^*$
\end{algorithmic}
\end{algorithm}

By constructing linear (in)equality constraints and
defining a objective function, we convert the problem into
 a mixed-integer linear programming (MILP) problem that can be 
 solved by MILP solvers. By solving this MILP problem, we collect the
corrections for the parameterized variables and form the log repair. 
% describe how to linearize the whole querylog with provided
% database states info. 

















