%!TEX root = ../main.tex


\begin{figure*}[t]
    \begin{minipage}[t]{0.28\textwidth}
         \vspace{0pt} 
         \centering
        \begin{tabular}{llll}
            \multicolumn{4}{l}{\emph{Taxes}: $D_0$}\\
            \toprule
            \textbf{ID}  & \textbf{rate}  & \textbf{income}    & \textbf{owed}\\
            \midrule
            $t_1$   & 10    & \$9000    & \$900\\
            $t_2$   & 25    & \$90000   & \$22500\\
            $t_3$   & 25    & \$86000   & \$21500\\
            $t_4$	 & 25	& \$85000	  & \$21250 \\
            $t_5$	 & 10	& \$5000	  & \$500 \\
            \bottomrule
            \\
        \end{tabular}
    \end{minipage}
    \begin{minipage}[t]{0.43\textwidth}
         \vspace{0pt} 
         \centering
        \begin{tabular}{|p{1ex}l|}
            \multicolumn{2}{l}{\emph{Query log}: $\mathcal{Q}$}\\
            % \toprule
            \hline
            $q_1$: & \texttt{\small UPDATE Taxes SET rate = 30}\\
                   & \texttt{\small WHERE income >= \color{red}{8570} \color{black}{and income <=} \color{red}{10000}} \\
            
            $q_2$: & \texttt{\small UPDATE Taxes SET owed = income*rate/100}\\
            
            $q_3$: & \texttt{\small INSERT INTO Taxes}\\ 
                   & \texttt{\small VALUES (4, 30, 86500, 25950)}\\
            \hline
            % \bottomrule
        \end{tabular}
    \end{minipage}
    \begin{minipage}[t]{0.28\textwidth}
         \vspace{0pt} 
         \centering
        \begin{tabular}{llll}
            \multicolumn{4}{l}{\emph{Taxes}: $D_3$}\\
            \toprule
            \textbf{ID}  & \textbf{rate}  & \textbf{income}    & \textbf{owed}\\
            \midrule
            \rowcolor{mid-gray}
            $t_1$   & \color{red}{30}    & \$9000    & \color{red}{\$2700}\\
            \rowcolor{mid-gray}
            $t_2$   & \color{red}{25}    & \$90000   & \color{red}{\$22500}\\
            \rowcolor{mid-gray}
            $t_3$   & \color{red}{25}    & \$86000   & \color{red}{\$21500}\\
            $t_4$	 & 25	& \$85000	  & \$21250\\
            $t_5$	 & 10	& \$5000	  & \$500 \\
            $t_6$   & 30    & \$86500   & \$25950\\
            \bottomrule
        \end{tabular}
    \end{minipage}

    \caption{A recent change in tax rate brackets calls for a tax rate of 30\% for those with income between \$85700 and \$10000. The accounting department issues query $q_1$ to implement the new policy, but the predicate of the WHERE clause condition missing two digits of the income value.  As a result, the tax rate of $t_3$ was increased incorrectly.  Query $q_2$ that calculates the amount owed based on the corresponding tax rate and income propagates the error.  The mistake is further obscured by query $q_3$, which inserts a tuple with similar income and the correct tax rate.
    \alex{I am confused about what happened with this example.  The numbers in the figure don't match the description in Example 3.  This doesn't see to be used later (yet).  Before I fix it, I want to check whether there is a reason for the particular changes...}}\xlw{I changed this example for later optimization sec.6 and noise sec.7 handling explanation. There are two goals: 1. we will demonstrate why we need 2 iterations to reduce noise when we only encode tuples in the complaint set for efficiency; 2. we will show if how we detect and prune false positive complaint $t_5$ by constructing complaint-tuple bipartite graph and searching densest sub-graph. }
    \label{fig:example}
\end{figure*}