%!TEX root = ../main.tex

%
% Notes: 
%
%  exact implementation
%  comparison with rollback at batch N
%  relaxing the solvers
%  dealing with false positives
%
%  running on TPC-C/sanjay's.  equality is easier
%
% Need names for
%  * rollback
%  * exact solution
%  * fixing an individual/batch of queries
%
\section{Experiments}

\ewu{Should say this somewhere: 
Updates are often a small fractice of a query workload -- typically $10\%$~\cite{} in most benchmarks --
thus in practice \sys can work for much larger workloads.  In the experiments, we filtered the workloads to 
only look at modification queries, and specifically UPDATE queries which are hard.
}


In this section, we carefully study the performance and accuracy
characteristics of the naive MILP-based repair algorithm, as well as 
slicing-based optimizations that improve the latency of the system over the naive
approach by over $Y\times$, an incremental algorithm for single query corruptions that
further improves the latency over the naive approach by $>Y\times$, and 
the multi-pass tuple-slicing algorithm that tolerates incomplete complaint sets with minimal loss in accuracy.
Our goals in the evaluation is to understand these trade-offs in
controlled synthetic scenarios, as well as study the effectiveness
in typical database query workloads based on widely used benchmarks.

To this end, our experiments are organized as follows: First, 
we compare the exhaustive MILP algorithm against the different optimizations 
to highlight the value of different optimizations.  We then compare the
repair costs of \texttt{INSERT}, \texttt{DELETE}, or \texttt{UPDATE}-only query logs 
and find that the latter query type is by far the most complicated and costly to repair.
For the subsequent experiments, we focus solely on \texttt{UPDATE}-only synthetic workloads 
to understand how \sys responds to different query logs and databases.  
Finally, we evaluate the efficacy on real-world databases transaction benchmarks,
TPC-C XXX~\cite{tpcc} and AuctionMark~\cite{auctionmark}.
All experiments were run on \ewu{X Core} 2.66 GHz  machines with 16GB RAM running \ewu{OS + Version}.



% establish the quality limitations of existing heuristics and the need for a formal, 
% constraint-based algorithm (\exact).  Second, we study how each of the 
% optimizations described in Section~\ref{s:optimiztaions} improves algorithm scalability.
% Third, we introduce different forms of error in the input complaint sets and study the 
% effectiveness of our noise-handling heuristics.  




%
% NOTE: figures are named <experimentsection>_<subsection>_<xaxis>.pdf
%

\subsection{Experimental Setup}


\begin{table}[t]\small
  \centering
  \begin{tabular}{@{}cll@{}}
  \toprule
  {\bf Param} & {\bf Description} & {\bf Default} \\ \midrule
  $V_d$  & Domain range of the attributes  & $[0, 100]$ \\
  $N_D$  & \# tuples in final database & $1000$ \\
  $N_a$  & \# attributes in database & $10$ \\
  $N_w$  & \# predicates in \texttt{WHERE} clauses & $1$ \\
  $N_s$  & \# \texttt{SET} clauses & $1$ \\
  $N_q$  & \# queries in query log & $300$ \\
  $idx$  & Index of corrupted query & $\{0, 25, 50,$ \\
         & (backwards from most recent) & $100, 200, 250 \}$ \\ %$\frac{N_q}{2}$ \\
  $r$    & Range size of \texttt{UPDATE} queries & 8 (tuples) \\
  $s$    & Zipf $\alpha$ param of query attributes, & $1$ \\ \bottomrule \end{tabular}
  %$set$  & Constant vs relative \texttt{SET} clauses. & const \\ 
         %& power low distribution $P(v) = v^{-s}$ & \\\end{tabular}
  \caption{Experimental Parameters}
  \label{t:params}
\end{table}


\iffalse
  \begin{table}[t]\small
    \centering
    \begin{tabular}{@{}cl@{}}
    \toprule
    {\bf Param} & {\bf Description} \\ \midrule
    $p$ & Precision: \% of repaired tuples that are correct. \\
    $r$ & Recall: \% of full complaint set repaired.\\
    % $t_{prep}$ & Time to construct CPLEX problem \\
    % $t_{send}$ & Time to send CPLEX problem to solver \\
    % $t_{solve}$ & Time for solver to generate a solutions\\
    $t_{total}$ & End-to-end execution time \\ 
    $d_{measure}$ & \red{Some sort of distance measure} \\ \bottomrule \end{tabular}
    \caption{Metrics Compared}
    \label{t:metrics}
  \end{table}
\fi




Each of our experiments follows a consistent procedure. 
We generate a sequence of queries using a synthetic query generator or 
the benchmark program, and corrupt the query log as described below. 
We then execute the original and corrupt query logs on an initial (possibly empty) database,
and perform a tuple-wise comparison between the resulting database states 
to generate a true complaint set.  
We then add noise to the complaint set by 1) picking random tuples not in the true
complaint set to add false positive complaints, and 2) removing true complaints to simulate false negatives.
Finally, we execute the evaluated algorithms on the complaints and compare the fixed
query log with the true query log, as well as the fixed and true
final database states to measure performance and accuracy metrics.

We compare the following algorithms:
$\sys_{exh}$ is the naive, exhaustive algorithma 
$\sys_{S \subseteq \{t,q,a,inc\}}$, where the subscripts $t,q,a,inc$ represent the
application of tuple, query and attribute slicing and single query incremental repairs, 
and $S$ defines the set of optimizations present in the algorithm.
For example, $\sys_{t,q,inc}$ uses both tuple and query slicing along with incremental repair.
When the algorithm uses incremental computation, the superscripts $\sys_{inc}^{1st}$
and $\sys_{inc}^{all}$ variants that either returns the first successful repair,
or the repair with the lowest objective function across the entire query log.

We evaluate the algorithms along several metrics.  Performance is measured as wall clock
time between submitting a query log and the system terminating after retrieving all relevant repairs 
(e.g., the time to the first repairt for $\sys_{inc}^{1st}$, or the time to process the full log for $\sys{inc}^{all}$).  
Experiment~\ref{???} evaluates the time to output {\it each repair} when running the 
incremental variant across the entire query log.  

We also measure the repair's precision (percentage of repaired tuples that were correctly fixed), 
the recall (the percentage of the full complaint set that was repaired), 
and the F1 measure (the harmonic mean of precision and recall).
We note that in the context of complete complaint sets, the recall will always be $1$ if \sys returns a result.
The recall can only degrade due to solver infeasibility (the solver does not find a valid assignment within a loose time bound), 
or in the context of incomplete complaint sets.   Thus our reported metrics are the average across multiple runs.
Finally, Table~\ref{t:params} summarizes the key parameters that we vary throughout our experiments.  
We describe them in the context of the datasets and workloads below.



\subsubsection{Datasets and Workloads}

This subsection describes the query and data generation process in greater detail.

\stitle{Synthetic:} \label{sec:syntheticgen}
We generate an initial database of $N_D$ random tuples.  
The schema contains $N_a=5$ attributes $a_1\ldots a_5$, whose values are
picked from $V_d$ uniformly at random, along with a primary key $id$.
We then generate a sequence of $N_q$ \texttt{UPDATE} queries~\footnote{\scriptsize We focus on
\texttt{UPDATE} only query logs because they are the {\it predominant} cost 
in a query log.  Experiment~\ref{sec:indelup} compares \texttt{INSERT}, \texttt{DELETE} and \texttt{UPDATE}
only workloads to illustrate the differenc \ewu{Experiment could move to appendix}.} where 
\verb|?| is a randomly generated value and \verb|r| is the size of the range predicate: 
{\scriptsize
\begin{verbatim}
  UPDATE SET (a_i = ?),.. WHERE a_j = ? AND ...
  UPDATE SET (a_i = ?),.. WHERE a_j in [?, ?+r] AND ...
\end{verbatim}
}

The \texttt{SET} clause sets the attribute to a random constant value in $V_d$ and 
the \texttt{WHERE} clauses form a conjunction.  
% The $set$ parameter controls whether the \texttt{UPDATE} queries set attributes to random constant values ({\it const}),  
% or relative values ({\it rel}) by incrementing by a random, possibly negative, value.  
In addition, the skew parameter $s$ determines the distribution attributes referenced in the \texttt{WHERE} and \texttt{SET} clauses.  
Each attribute in a query is picked from either a uniform distribution when $s=0$ or a zipfian~\cite{zipf} distribution.
This allows our experiments to vary between a uniform distribution, where each attribute is
equally likely to be picked, and a skewed distribution where nearly all attributes are the same. \\
The \texttt{WHERE} clauses in \texttt{DELETE} queries are generated in an identical fashion, while
\texttt{INSERT} queries insert values picked uniformly at random from $V_d$.



\stitle{TPC-C:} We use the data and query workload over the {\it ORDER} table in TPC-C~\cite{tpcc}.  
We generated a database at scale 1 with one warehouse, and kept the queries that modify the
{\it ORDER} table. The initial table contains 4570 tuples and we ultimately generated a log with
1000 queries, where $530$ are \texttt{UPDATE}s and the rest are \texttt{INSERT}s.

\stitle{AuctionMark:} Auctionmark workload simulates the 
actual action market. We generate a database 
with 8729 tuples and $1000$ queries, with $540$ \texttt{UPDATE}s and $110$ \texttt{INSERT}s.
Both TPC-C and AuctionMark setups were generated using the OLTP-bench project~\cite{oltpbench,oltpbenchgit}.


\stitle{Corrupting Queries:} We corrupt query $q_i$ by replacing it with a randomly
generated query of the same type based on the procedures described above.
To standardize our procedures, we selected a fixed set of indexes $idx$
that are used in all experiments.  Appendix~\ref{app:qidx} presents
experiments that justify the rationale behind our selection.






% \begin{figure}[h]
% \centering
%   \begin{subfigure}[t]{\columnwidth}
%   \includegraphics[width = .95\columnwidth]{figures/tpcc_time}
%   \caption{Performance on TPC-C Benchmark}
%   \label{f:tpcc} 
%   \end{subfigure}
%   \begin{subfigure}[t]{\columnwidth}
%   \includegraphics[width = .95\columnwidth]{figures/auction_time}
%   \caption{Performance on AuctionMark Benchmark}
%   \label{f:auctionmark} 
%   \end{subfigure}
%   \caption{Benchmark Performance}
% \end{figure}

\subsection{Benchmark}

\begin{figure}[h]
\centering
  \includegraphics[width = .95\columnwidth]{figures/benchmark_time}
  \label{f:benchmarks} 
  \vspace*{-.2in}
  \caption{Performance on Benchmarks}
  \vspace*{-.1in}
\end{figure}

Figure~\ref{f:benchmarks} plots the performance of the incremental, optimized \sys algorithm
on both benchmark applications.  In both workloads, the latency to derive a repair was consistenly less
than $5$ seconds, and for TPC-C, on the order of milliseconds.  The precision and recall measures were
$1$ both all settings.   One reason for the good performance is due to the fact that the benchmark queries
are pre-dominantly point update queries.  In this setting, each tuples affects a very small set of 
tuples, which results in $1$ or $2$ complaints on average.  In this case, the tuple and query slicing algorithms
reduce the total number of constraints for almost every setting to less than $100$.  
{\it Takeaway: many workloads in practice are dominated by point update queries.  In these settings,
 \sys is very effective at reducing the number of constraints and can derive repairs in nearly
 interactive latencies.}






  \begin{figure*}[h]
  \centering
    \vspace*{-.2in}
    \begin{subfigure}[t]{.3\textwidth}
    \includegraphics[width = .99\columnwidth]{figures/multi_time}
    \vspace*{-.1in}
    \caption{Performance for multiple corruptions.}
    \label{f:multiquery} 
    \end{subfigure}
    \begin{subfigure}[t]{.3\textwidth}
    \includegraphics[width = .99\columnwidth]{figures/incrementalcompare_time}
    \vspace*{-.1in}
    \caption{Performance for exhaustive and incremental algorithms.}
    \label{f:singlequeryinc_time} 
    \end{subfigure}
    \begin{subfigure}[t]{.3\textwidth}
    \includegraphics[width = .99\columnwidth]{figures/indelup_time}
    \vspace*{-.1in}
    \caption{Performance for different query types.}
    \label{f:indelup_time} 
    \end{subfigure}
    \\
    \begin{subfigure}[t]{.3\textwidth}
    \includegraphics[width = .99\columnwidth]{figures/multi_pr}
    \vspace*{-.1in}
    \caption{Accuracy for multiple corruptions.}
    \label{f:multiquery} 
    \end{subfigure}
    \begin{subfigure}[t]{.3\textwidth}
    \includegraphics[width = .99\columnwidth]{figures/incrementalcompare_acc}
    \vspace*{-.1in}
    \caption{Accuracy for exhaustive and incremental algorithms.}
    \label{f:singlequeryinc_acc} 
    \end{subfigure}
    \begin{subfigure}[t]{.3\textwidth}
    \includegraphics[width = .99\columnwidth]{figures/indelup_acc}
    \vspace*{-.1in}
    \caption{F1-score for different query types.}
    \label{f:indelup_acc} 
    \end{subfigure}
    \vspace*{-.1in}
    \caption{Preliminary Experiments}
  \end{figure*}


\subsection{Preliminaries}
The following set of experiments are designed to establish the rationale for 
our experimental settings in the subsequent experiments.  
Specifically, we compare different slicing-based optimizations of the exhaustive approach
as the number of corrupted queries increases.  
We then evaluate the scalability of the different exhaustive algorithms in the context of a single
corrupted query and highlight the value of the second MILP iteration when using tuple-slicing.
We then establish the difficulty of repairing \texttt{UPDATE} workloads as compared to other query types.


\stitle{Multiple Corrupt Queries:}
In this experiment, we compare the naive exhaustive algorithm with four types of algorithms
$\sys_S$, where $S \in \{t, q, a, tqa\}$.  We use the default settings with $1000$ tuples and
$50$ \texttt{update} queries.  We corrupt every tenth query starting from oldest query $q_0$,
up to $q_{40}$.  For example, when we corrupt $3$ queries, we corrupt $q_{0,10,20}$.
We find that the number of corruptions greatly affects both the scalabality (Figure~\ref{f:multi_time}) 
and the accuracy (Figure~\ref{f:multi_acc}) of the algorithms.  Specifically, as the number increases,
the number of possible assignments of the MILP parameters increases exponentially and the solver often takes
longer than our experimental time limit of $1000$ seconds and returns an infeasibility error.  
This is a predominant reason why the accuracy degrades past $3$ corruptions.  For example, 
when $4$ queries are corrupted and we ignore the infeasible executions, the average execution time is $300$ seconds
and the precision and recall are greater than $0.94$.  Unfortunately, with $5$ queries, all runs exceed the time limit.


\stitle{Single Query Optimization:}
In this experiment, we evaluate the efficacy of the incremental repair algorithm (Section~\ref{sec:incremental})
in the special case when only one query has been corrupted.  We compare the exhaustive 
algorithm against $\sys_{inc}^{1st}$ and $\sys_{inc}^{all}$ using the default parameters.
We also compare against $\sys_{t,inc}^{1st}$ to illustrate the value of the second MILP iteration in tuple slicing.
Figure~\ref{f:singlequeryinc_time} highlights the scalability limitations of the exhaustive algorithm 
even $25$ queries in this experiment.  In contrast, the incremental algorithms perform at nearly the same rate
across the different query indexes.  Figure~\ref{f:singlequeryinc_acc} shows the value of tuple-slicing --
both basic incremental algorithms have severely degraded precision measures due to their propensity to over-generalize
-- however the introduction of the second MILP iteration significantly improves the precision.  
In all algorithms, the recall is near $1$ because the complaint sets are complete and the solver is able to find 
an assignment in nearly all cases.


\stitle{Query Type Experiment:}\label{sec:indelup}
Our final preliminary experiment evaluates the exhaustive, tuple-slicing algorithm 
$\sys_{exh,t}$ on insert, delete, or update-only workloads.
We increase the number of queries from $20$ to $100$ and corrupt the oldest query in the log.  
Figure~\ref{f:indelup_time} shows that while the cost of repairing insert and delete-only workloads
remains relatively constaint, the cost for update-only workloads increases rapidly -- this is
despite a near perfect F1-score for all settings (Figure~\ref{f:indelup_acc}).


{\it Takeaways: 
due to the severe scalability limitation of the non-incremental algorithm, our subsequent synthetic experiments focus
on the case when a single query has been corrupted.
End up picking $queryfix_{inc,t}$, $queryfix_{inc,t,q}$
Justify single corrupt query
}


\subsection{Synthetic Experiments}
In the following set of synthetic experiments, we individually vary 
the parameters skewness $s$, log size $N_q$, database size $N_D$, and
query complexity $N_w$ in order to tease apart the parameters that most affect our metrics.
As described above, we focus on query logs with a single corruption,
and on a small set of representative algorithms \ewu{XXX}.


  \begin{figure*}[h]
    \centering
    \begin{subfigure}[t]{.3\textwidth}
      \includegraphics[width = \columnwidth]{figures/logsize_time}
      \vspace*{-.1in}
      \caption{Query Log Size vs Time.}
      \label{f:scale1} 
    \end{subfigure}
    \begin{subfigure}[t]{.3\textwidth}
      \includegraphics[width = \columnwidth]{figures/dbsize_time}
      \vspace*{-.1in}
      \caption{Database Size vs Time.}
      \label{f:scale2} 
    \end{subfigure}
    \begin{subfigure}[t]{.3\textwidth}
      \includegraphics[width = \columnwidth]{figures/where_time}
      \vspace*{-.1in}
      \caption{Query Complexity vs Time}
      \label{f:scale3} 
    \end{subfigure}
    \\
    \begin{subfigure}[t]{.3\textwidth}
      \includegraphics[width = .95\columnwidth]{figures/logsize_acc}
      \vspace*{-.1in}
      \caption{Query Log Size vs F1-score.}
      \label{f:scale1} 
    \end{subfigure}
    \begin{subfigure}[t]{.3\textwidth}
      \includegraphics[width = .95\columnwidth]{figures/dbsize_acc}
      \vspace*{-.1in}
      \caption{Database Size vs F1-score. }
      \label{f:scale2} 
    \end{subfigure}
    \begin{subfigure}[t]{.3\textwidth}
      \includegraphics[width = .95\columnwidth]{figures/where_acc}
      \vspace*{-.1in}
      \caption{Query Complexity vs F1-score.}
      \label{f:scale3} 
    \end{subfigure}
    \caption{cap}
  \end{figure*}



\stitle{Scalability - Log Size:}

Figure~\ref{f:scale1} shows xxx.

\stitle{Scalability - DB Size:}
We adjust the initial database sizes from 100 to 100,000 and 
while maintain the same range $r$ in each query. As we can see, 
for the same corrupt query index, the MILP-based approaches require
similar amount of time to derive the log repair. 

Figure~\ref{f:scale2} shows xxx.



\stitle{Scalability - Query Complexity:}
In the end, we evaluate the MILP-based approaches by adjust 
the query complexity: number of predicates, $N_w$, in the 
update query while maintain the range $r$ the same. 

Figure~\ref{f:scale3} shows xxx.





\stitle{Skew:} We generate query log at 4 different skewness levels from uniform to 
super skewed with $s = 0, 1e-7, 0.5, 1$. As we can see as query log
is more skewed, the execution time for solving same amount of queries 
increase. This is also due to the fact that increasing skewness also
increase the query dependency, which, in turn, increase the searching
space in the MILP problem.
 
  \begin{figure}[h]
    \centering
    \begin{subfigure}[t]{.9\columnwidth}
    \includegraphics[width = \columnwidth]{figures/skew_time}
    \caption{Skew vs Time}
    \label{f:multiquery} 
    \end{subfigure}
    \begin{subfigure}[t]{.9\columnwidth}
    \includegraphics[width = \columnwidth]{figures/skew_acc}
    \caption{Skew vs F1-score}
    \label{f:multiquery} 
    \end{subfigure}
    \caption{Log size. }
  \end{figure}





\stitle{Incomplete Complaint Sets:}
In this set of experiments, we add false negative complaints to the input
complaint set and evaluate our noise handling algorithms (Section~\ref{sec:noise}).
To generate false negatives ($FN$), we select a random subset of the true complaint set.


  \begin{figure}[h!]
    \centering
    \begin{subfigure}[t]{\columnwidth}
    \includegraphics[width = .9\columnwidth]{figures/noise_fn_time}
    \caption{Time vs Incomplete Complaint Size}
    \label{f:falsenegative_time} 
    \end{subfigure}
    \begin{subfigure}[t]{\columnwidth}
    \includegraphics[width = .9\columnwidth]{figures/noise_fn_acc}
    \caption{F1-score vs Incomplete Complaint Size}
    \label{f:falsenegative_acc} 
    \end{subfigure}
    \caption{cap}
  \end{figure}

Figure~\ref{f:falsenegative} varies $FN \in \{\}$.

\stitle{Incremental Results}
INcremental algorithm, incremental results














