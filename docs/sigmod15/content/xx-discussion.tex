%!TEX root = ../main.tex

\section{Summary and discussion}

In this paper, we presented \sys, a framework for diagnosing and
repairing data errors in the queries that operate on the data.
Datasets are typically dynamic; so even if a dataset starts clean,
updates on the data can introduce new errors. With \sys, we can
analyze query logs to trace reported errors to the queries that
introduced them. This allows us in turn to identify additional errors
in the data that may have been missed and gone unreported.

Our basic algorithm, \naive, uses non-trivial transformation rules to
encode information from the data and the query log as a mixed integer
linear program. Further, we introduce three optimizations that reduce
the size of the problem across three dimensions: tuples, queries, and
attributes, without compromising the accuracy of the produced
solution. In addition, we propose a performance optimization that
analyzes a single query at a time. Our experiments show that this
optimization can achieve significant scaling gains for single-query
errors, without significant reduction in accuracy.

To the best of our knowledge, \sys is the first approach to diagnosis
and repair of errors in queries. Obviously, correcting such errors in
practice poses additional challenges. For example, query errors that
occurred far in the past are more challenging to correct in practice,
even after they have been identified, as rolling back the effects of
old transactions can have cascading implications. Investigating system
solutions that can facilitate such rollback actions or maintain more
detailed state through a dataset's evolution can help \sys operate
more efficiently and effectively.

The current version of \sys focuses on linear functions, so it may not
be able to encode complex updates that involve non-linear
computations. Future versions can target extended models and solvers
to represent richer scenarios. Finally, in our future work, we plan to
investigate additional methods of scaling the constraint analysis, as
well as techniques that can adapt the benefits of single-query
analysis to errors in multiple queries.






