\section{Noisy Complaint Sets}
\label{sec:noise}

As described in the problem setup (Section~\ref{sec:model}),
complaints sets may themselves have two types of error.
The first type are incomplete complaint sets, which are missing complaints
that have not been included in the sets.
In this case, naively encoding the query log and database as in $QueryFix_{exh}$
will compose a problem that is likely to be infeasible to solve.  
For example, consider the query $\textrm{UPDATE T SET a = 1 WHERE b} \in [0,5]$,
the corrupted query $\textrm{UPDATE T SET a = 1 WHERE b} \in [10,15]$,
and initial database state $t_1=(a=2, b=0), t_2=(2, 3), t_3=(2, 5), t_4=(2, 10), t_5=(2, 13)$.
The complete complaint set is thus $c_1((2,0), (1,0)), c_2((2,3), (1,3)), c_3((2,5), (1,5)), c_4((1,10), (2,10)), c_5((1,13), (2,14))$.
However, if the user only submits $c_{1,3,4,5}$ to \sys, then \sys would interpret $(2,3)$
as the correct state of $t_2$ and the correct repair $b \in [0, 5]$ would 
be interpreted as introducing a new error.   Thus the solver will simply return
an infeasibility error and not produce a candidate repair.

In turns out that the Tuple Slicing technique (Section~\ref{sec:tbsize})
can ameliorate this issue.  By only encoding the tuples in the incomplete complaint set,
the encoded problem does not have constraints on the query's effect on other tuples in the database.
In other words, it allows the result to generalize to tuples not in the complaint set.
The second iteration of the MILP execution then uses a soft constraint on the number of
non-complaint tuples that are affected by the repair in order to address the possibilty of over-generalization.

The second type of error are is false positives -- either erroneous complaints that are 
not actually errors, or complaints whose clean tuple $t^*$ is incorrect.
The MILP-based approach suffers from the same infeasibility challenges as in the incomplete complaint set case.
One option is to remove these errors by using one of numerous outlier detection algorithms~\cite{}.
Ultimately, however, we view this as an orthogonal pre-processing step that is worthy of its own study.
Thus, in our experiments, we focus on incomplete complaints sets and assume that
there are not erroneous complaints.


\if{0}
  \subsubsection{False Negatives}
  False negatives are cases when we don't have the full complaint sets, but
  what's provided in the complaint sets are guaranteed as correct. The 
  two-iteration approach in Section~\ref{sec:opt:tbsize} can handle 
  such cases. Refer to Section~\ref{sec:opt:tbsize} for detail.



  \subsubsection{False Positives}
  False positives are cases when we have incorrect information in the complaint sets 
  : they can be a falsely reported tuple which are actually 
  correct, or incorrect suggestions for the erroneous tuples. 
  Including such false positives in the MILP problem may result in a problem
  that is infeasible to solve. Thus, we want to detect and prune these false positives 
  beforehand. However, false positives are hard to 
  detect when there is a lack of information. 
  For example, if we only have few tuples in the complaint
  set, we cannot make any claim about which of them is a false positive 
  complaint. Thus, in this paper, we only considering false positive
  case when the ratio of the number of false positives to the number
  of correctly reported complaints is small.
  A obvious trend for false positives is that they normally are very 
  different or conflict
  with other complaints (Example~\ref{ex:false_positive_1}).
  \begin{example} \label{ex:false_positive_1}
  In Example~r\ref{ex:taxes2}, say there is 
  a false positive complaint on tuple $t_5$ which suggests the correct value
  for $t_5$ at database state $D_3$ should be 
  \{\textbf{ID}:$t_5$, \textbf{rate}: \color{red}{30}
  \color{black}{, \textbf{income}: \$5000, \textbf{owned}
  : }\color{red}{\$1500}\color{black}{\}}. To resolve this complaint, we have
  to guarantee the lower bound of the income range in $q_1$ as 5000. 
  However, the other complaints, $t_1, t_2, t_3$, suggest to
  move this lower bound to at least 9500.0001. In this case, the complaint
  $t_5$ conflicts with the all the other complaints and we may thus
  claim that $t_5$ is likely to be a false positive complaint. 
  \end{example}
  In this section, we introduce a \textbf{Pre-Processing} process that detects 
  false positives effectively. This pre-processing process first searches the 
  best log repair for each complaint separately, it then constructs 
  a bipartite graph between complaints and their impacted tuples and
  searches for the densest sub-graph of the bipartite graph, and finally 
  prunes complaints that are not in the densest sub-graph.
  \begin{itemize}
  \item Using algorithm described in Section~\ref{sec:opt} to solve each 
  complaint individually. Denote the log repair for complaint $c_i$ 
  as $\mathcal{Q}^*(c_i)$, and the impacted tuples of this log repair as
  $T_{c_i}$.
  \item Construct a bipartite graph $G = (\mathcal{C}, T, E)$, where 
  $T = \cup_{i} T_{c_i}$. Note that tuple with same primary key but modified 
  differently are treated as two separate vertices in the bipartite graph. 
  \item Search for densest sub-graph $G' = (\mathcal{C}', T', E')$ in $G$ 
  [] \xlw{cite some papers} and prune complaints in $\mathcal{C} - \mathcal{C}'$. 
  Density[] of a graph $G' = (\mathcal{C}', T', E')$
  is defined by $density(G') = \frac{|E'|}{|\mathcal{C}'|+|T'|}$. 
  \end{itemize}
  For achieve better performance, we can sample tuples
  from the database uniformly when constructing the bipartite graph $G$. 
  We demonstrate how to use this \textbf{Pre-Processing} 
  approach to prune false positive complaint(s) in Example~\ref{ex:false_positive_2}. 
  \begin{example}
  \label{ex:false_positive_2}
  Let's construct the bipartite graph between complaints $t_1, t_2, t_3, t_5$ 
  and tuples in the table for Example~\ref{ex:false_positive_1}. 
  $t_1$ suggests to move the range of income in $q_1$ 
  as $(9000, 10000]$. Similarly, $t_2, t_3, t_5$ suggest $[8570, 90000]$, 
  $[8570, 86000]$, and $[500, 10000]$ respectively. Let's assume tuples are
  uniformly distributed in these ranges. The bipartite graph is shown in 
  Figure~\ref{f:fp}. The density for $t_2, t_3$ is 1.9202 ($\frac{313}{163}$);
  density for $t_2, t_3, t_1$ is 1.9207 ($\frac{315}{164}$); density for 
  $t_1, t_2, t_3, t_5$ is 1.8232 ($\frac{330}{181}$). Thus, $t_5$ is pruned. 
  \begin{figure}[ht]
  \centering
  \includegraphics[width = 0.85\columnwidth]{figures/falsepositive_example}
  \caption{Bipartite graph of complaints and tuples for Example~\ref{ex:taxes2}. }
  \label{f:pf} 
  \end{figure}
  \end{example}

\fi
