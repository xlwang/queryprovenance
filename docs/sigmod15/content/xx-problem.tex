%!TEX root = ../main.tex

\section{Modeling abstractions}
\label{sec:abstractions}

In this section, we introduce a running example inspired from the use-case of
Example~\ref{ex:taxes}, and describe the model abstractions that we use to
formalize the diagnosis problem.


\ewu{Add text to say false positive complaints are an orthogonal problem.}



\begin{figure*}[t]
    \begin{minipage}[t]{0.28\textwidth}
         \vspace{0pt} 
         \centering
        \begin{tabular}{llll}
            \multicolumn{4}{l}{\emph{Taxes}: $D_0$}\\
            \toprule
            \textbf{ID}  & \textbf{rate}  & \textbf{income}    & \textbf{owed}\\
            \midrule
            $t_1$   & 10    & \$9500    & \$950\\
            $t_2$   & 25    & \$90000   & \$22500\\
            $t_3$   & 25    & \$86000   & \$21500\\
            $t_4$   & 25    & \$86500   & \$21625\\
            \bottomrule
            \\
        \end{tabular}
    \end{minipage}
    \begin{minipage}[t]{0.43\textwidth}
         \vspace{0pt} 
         \centering
        \begin{tabular}{|p{1ex}l|}
            \multicolumn{2}{l}{\emph{Query log}: $\mathcal{Q}$}\\
            \hline
            $q_1$: & \texttt{\small UPDATE Taxes SET rate = 30}\\
                   & \texttt{\small WHERE income >= \color{red}{85700}} \\
            
            $q_2$: & \texttt{\small UPDATE Taxes SET owed = income*rate/100}\\
            
            $q_3$: & \texttt{\small INSERT INTO Taxes}\\ 
                   & \texttt{\small VALUES (25, 85800, 21450)}\\
            \hline
        \end{tabular}
    \end{minipage}
    \begin{minipage}[t]{0.28\textwidth}
         \vspace{0pt} 
         \centering
        \begin{tabular}{llll}
            \multicolumn{4}{l}{\emph{Taxes}: $D_3$}\\
            \toprule
            \textbf{ID}  & \textbf{rate}  & \textbf{income}    & \textbf{owed}\\
            \midrule
            $t_1$   & 10    & \$9500    & \$950\\
            $t_2$   & 30    & \$90000   & \$27000\\
            \rowcolor{mid-gray}
            $t_3$   & \color{red}{30}    & \$86000   & \color{red}{\$25800}\\
            \rowcolor{mid-gray}
            $t_4$	 & \color{red}{30}	& \$86500	  & \color{red}{\$25950}\\
            $t_5$	 & 25	& \$87000	  & \$21750\\
            \bottomrule
        \end{tabular}
    \end{minipage}

    \caption{A recent change in tax rate brackets calls for a tax rate of 30\% for those with income above \$87500.  The accounting department issues query $q_1$ to implement the new policy, but the predicate of the WHERE clause condition transposed two digits of the income value.  As a result, the tax rates of $t_3$ and $t_4$ were increased incorrectly.  Query $q_2$ that calculates the amount owed based on the corresponding tax rate and income propagates the error to additional fields.  The mistake is further obscured by query $q_3$, which inserts a tuple with similar income and the correct tax rate.
    }
    \label{fig:example}
\end{figure*}

\begin{example}\label{ex:taxes2}
    
Figure~\ref{fig:example} demonstrates an example tax bracket adjustment in the
spirit of Example~\ref{ex:taxes}. The adjustment sets the tax rate to 30\% for
income levels above \$87,500, and is implemented by query $q_1$. A digit
transposition mistake in the query, results in an incorrect tax rate for tuples
$t_3$ and $t_4$. Query $q_2$ that calculates the amount owed based on the corresponding
tax rate and income propagates the error to other fields. The mistake is
further obscured by query $q_3$, which inserts a tuple with slightly higher
income than $t_3$ and $t_4$ and the correct (lower) tax rate.

\end{example}
% 
While traditional data cleaning techniques seek to identify and correct the
erroneous values in the table \emph{Taxes} directly, our goal is to diagnose
the problem, and understand the reasons for these errors. In this case, the
reason for the data errors is the incorrect predicate value in query $q_1$.

In this paper, we assume that we know \emph{some} errors in the dataset, and
that these errors were caused by erroneous updates. The errors may be
obtained in different ways: traditional data cleaning tools may identify
discrepancies in the data (e.g., a tuple with lower income has higher tax
rate), or errors can be reported directly from users (e.g., customers
reporting discrepancies to customer service). \emph{Our goal is not to correct
the errors directly, but to analyze them as a ``symptom'' and provide a
diagnosis.} The diagnosis can produce a targeted treatment: knowing how the
errors were introduced guides the proper way to trace and resolve them.



\begin{figure}[t]
\centering
{\small
\begin{tabular}{ll}
    \toprule
    \textbf{Notation} & \textbf{Description}\\
    \midrule
    $\mathcal{Q}$& The sequence of executed update queries (log)\\ 
             & $\mathcal{Q}=\{q_1, \dots, q_n\}$ \\
    $D_0$    & Initial database state at beginning of log\\
    $D_n$    & End database state (current) $D_n=\mathcal{Q}(D_0)$\\
    $D_i$    & Database state after query $q_i$: $D_i=q_i(\dots q_1(D_0))$\\
    $c: t\mapsto t^*$ & Complaint: $\mathcal{T}_c(D) = (D_n\setminus\{t\})\cup\{t^*\}$\\
    $\mathcal{C}$ & Complaint set $\mathcal{C}=\{c_1,\dots,c_k\}$\\
    $\mu_q(t)$  & Modifier function of $q$ (e.g., \texttt{SET} clause)\\
    $\sigma_q(t)$   & Conditional function of $q$ (e.g., \texttt{WHERE} clause)\\
    $t_{new}$   & Tuple values introduced in an \texttt{INSERT} query\\
    $\mathcal{Q}^*$& Log repair\\
    $d(\mathcal{Q}, \mathcal{Q}^*)$ & Distance functions between two query logs\\
    \bottomrule
\end{tabular}
}
\vspace{-2mm}
\caption{Summary of notations used in the paper. }
\label{tbl:notation}
\end{figure}

\subsection{Error modeling}
\label{sec:model}

In our setting, the diagnoses are associated with errors in the queries that
operated on the data. In Example~\ref{ex:taxes2}, the errors in the dataset
are due to the digit transposition mistake in the WHERE clause predicate of
query $q_1$. Our goal is to infer the errors in a log of queries
automatically, given a set of incorrect values in the data. We proceed to
describe our modeling abstractions for data, queries, and errors, and how we
use them to define the diagnosis problem.

\subsubsection*{Data and query models}

\noindent
\emph{Query log ($\mathcal{Q}$):}
We define a query log as an ordered sequence of \texttt{UPDATE}, \texttt{INSERT}, and
\texttt{DELETE} queries $\mathcal{Q}=\{q_1,\dots,q_n\}$, that have
operated on a database $D$. In the rest of the paper, we use the term
\emph{update queries}, or just \emph{queries}, to refer to any of the queries in $\mathcal(Q)$,
including insertion and deletion queries.

\smallskip
\noindent
\emph{Query ($q_i$):}  We model each update query as a function over a database $D$, resulting in a new database $D'$:  
\begin{align*}
    D'=q(D)= &\{\mu_{q}(t)\;|\;t\in D, \sigma_{q}(t)\}\\
    &\cup\{t\;|\;t\in D, \neg\sigma_{q}(t)\}\\
    &\cup\{t_{new}\;|\;q\in\texttt{INSERT}\}
\end{align*}
% 
In this definition, the modifier function $\mu_q(t)$ represents the query's update equations, and it transforms a tuple by either deleting it ($\mu_q(t)=\bot$) or changing the values of some of its attributes.
The conditional function $\sigma_q(t)$ is a boolean function that represents the query's condition predicates.  In the example of Figure~\ref{fig:example}:
\begin{align*}
    &\mu_{q_1}(t)=(10, t.income, t.owed)\\
    &\sigma_{q_1}(t)=(t.income\ge 85700)\\
    &\mu_{q_2}(t)=(t.rate, t.income, \frac{t.income\cdot t.rate}{100})\\
    &\sigma_{q_2}(t)=\texttt{true}
\end{align*} 
% 
As an insertion query, $q_3$ has $\sigma_{q_3}(t)=\texttt{false}$ and $t_new=(25, 85800, 21450)$.


\smallskip
\noindent
\emph{Database state ($D_i$):}
We use $D_i$ to represent the state of a database $D$ after the application of
queries $q_1$ through $q_i$ from the log $\mathcal{Q}$. $D_0$ represents the
original database state, and $D_n$ the final, or current, database state. Out
of all the states, the system only maintains $D_0$ and $D_n$. In practice,
$D_0$ can be a checkpoint: a state of the database that we assume is correct;
we cannot diagnose errors before this state. The intermediate states can be
derived by repeating the log: $D_i=q_i(q_{i-1}(\dots q_1(D_0)))$. We also
write $D_n=\mathcal{Q}(D_0)$ to denote that the final database state $D_n$ can
be derived by applying the sequence of queries in the log to the original
database state $D_0$.

\smallskip
\noindent
\emph{True database state ($D_i^*$):}
Queries in $\mathcal{Q}$ are possibly erroneous, introducing errors in the
data. There exists a sequence of \emph{true} database states $\{D_0^*,
D_1^*\dots, D_n^*\}$, with $D_0^*=D_0$, representing the database states that
would have occurred if there had been no errors in the queries.
The true database states are unknown; our goal is to find and correct the errors in $\mathcal{Q}$ and retrieve the correct database state $D_n^*$.

For ease of exposition, in the remainder of the paper we assume that the
database contains a single relation, but this is not a requirement in our
framework.


\subsubsection*{Error models}

Following our motivating examples (Examples~\ref{ex:telco}
and~\ref{ex:taxes}), we model a set of identified or user-reported
data errors as \emph{complaints}. A complaint corresponds to a
particular tuple in the final database state $D_n^*$, and identifies
that tuple's correct value assignment. We formally define complaints
below:

\begin{definition}[Complaint]
    A complaint $c$ is a mapping between two tuples: $c: t\mapsto t^*$, such that $t$ and $t^*$ have the same schema, $t\in D_n$, and $t\neq t^*$. A complaint defines a
    transformation $\mathcal{T}_c$ on a database state $D$: $\mathcal{T}_c(D)
    = (D\setminus\{t\})\cup\{t^*\}$.
\end{definition}

In the example of Figure~\ref{fig:example}, two complaints are reported on the final database state $D_3$: 
$c_1: t_3\mapsto t_3^*$ and
$c_2: t_4\mapsto t_4^*$, where $t_3^*=(25,86000,21500)$
and $t_4^*=(25,86500,21625)$.  For both these cases, each complaint denotes a value correction for a tuple in $D_3$.  Complaints can also model the addition or removal of tuples: $c: \bot\mapsto t^*$ means that $t^*$ should be added to the database, whereas $c: t\mapsto \bot$
means that $t$ should be removed from the database.


\smallskip
\noindent
\emph{Complaint set ($\mathcal{C}$):}
We use $\mathcal{C}$ to denote the set of all known complaints, and we call it
\emph{complaint set}. Each complaint in $\mathcal{C}$ represents a
transformation (addition, deletion, or modification) of a tuple in $D_n$.
Applying all these transformations to $D_n$ results in a new database instance
$D_n'$, and we write $D_n'=\mathcal{T}_\mathcal{C}(D_n)$. If $D_n'$ has no
remaining data errors ($D_n'=D_n^*$), we say that $\mathcal{C}$ is a
\emph{complete complaint set}.
In our work, we do not assume that the complaint set is complete, but, as is
more common in practice, we only know a subset of the errors.

\smallskip
\noindent
\emph{Log repair ($\mathcal{Q}^*$):}
The goal of our framework is to derive a diagnosis as a log repair
$\mathcal{Q}^*=\{q_1^*,\dots, q_n^*\}$, such that
$\mathcal{Q}^*(D_0)=D_n^*$. In this work, we focus on errors produced
by incorrect parameters in queries, so our repairs focus on altering
query constants rather than query structure. Therefore, for each query
$q_i^*\in\mathcal{Q}^*$ $q_i^*$ has the same structure as $q_i$, but
possibly different parameters. For example, a good log repair for the
example of Figure~\ref{fig:example} is
$\mathcal{Q}^*=\{q_1^*,q_2,q_3\}$, where $q_1^*$=\texttt{UPDATE Taxes
SET rate = 30 WHERE income >= 87500}.


\subsubsection*{Problem definition}

We now formalize the definition of the problem of diagnosing data errors using
query histories. A diagnosis is a log repair $\mathcal{Q}^*$ that resolves all
complaints in the set $\mathcal{C}$.

\begin{definition}[Optimal diagnosis]\label{def:problem}
    Given a set of data errors in the form of complaints $\mathcal{C}$, a query log $\mathcal{Q}$, and a distance function $d$, the optimal diagnosis is a log repair $\mathcal{Q}^*$, such that:
    \begin{itemize}[itemsep=0pt, parsep=0pt]
        \item $\mathcal{Q}^*(D_0)\models\mathcal{C}$
        \item $d(\mathcal{Q}, \mathcal{Q}^*)$ is minimized
    \end{itemize}
\end{definition}


\begin{figure}[t]
\centering
\includegraphics[width = 0.75\columnwidth]{figures/probtransform}
\caption{Graphical depiction of the diagnosis problem in our \sys framework.  $D_0$, $D_n$, $\mathcal{Q}$, and $\mathcal{C}$ are given, and \sys uses them to derive the log repair $\mathcal{Q}^*$}
\label{f:probtransform} 
\end{figure}


\deprecate{
\subsection{Naive Formulation}

The most general version of the problem
(depicted in Figure~\ref{f:probtransform}) is to find a sequence of
transformations $T$ that insert, delete, and/or modify queries in $Q_{seq}$ 
such that the resulting sequence, $Q'_{seq} = T(Q_{seq})$, resolves the user's complaint set. 

However this problem is ill-defined because there exist an unbounded set of transformations that
can resolve the user's complaint set.  A naive solution is to append to the query log a statement
that deletes all the records in the database, followed by a query that insert all of the correct records.
Unfortunately this naive solution does not help explain the complaints in any way!

\subsection{Constraints}

For this reason, we constrain the set of possible transformations $\mathcal{T}$ to the following:

\begin{itemize}
\item delete query
\item modify insert statement constants
\item modify constants in WHERE clause
\end{itemize}

Our transformations don't include adding new queries, synthesizing arbitrary queries, or modifying the
number of clauses in a WHERE condition.  We apply these restrictions because we believe it is more likely
for the user to mis-type a constant value as opposed to having an error in the query structure.

Futhermore we define a distance metric between two query logs in order to evaluate
the qulatiy of a transformation.
\xxx{define $\mathcal{T}$ here.}



\subsection{Problem Statements}

In this paper, we present three variants of this problem.

\begin{problem}[Prob-Complete]\label{prob:complete}
Given $C = P_{D_n, D^*_n}$, $Q_{seq}$, and the sequence of database states $D_0,\ldots,D_n$, 
identify a sequence of transformations $T$ such that:
\begin{itemize}
\item $T(Q_{seq})(D_0) = C(D_n)$
\item $|T| = 1$
\item $T$ metric is minimized
\end{itemize}
\end{problem}

This variation of the problme relaxes the constraint that the complaint set must be complete, and allows
for both false positives as well as false negatives.  The goal is the same, however the constraints are relaxed:

\begin{problem}[Prob-Incomplete]\label{prob:incomplete}
Given $C$ where $acc_C < 1$, $Q_{seq}$, and the sequence of database states $D_0,\ldots,D_n$, 
identify a sequence of transformations $T$ such that:
\begin{itemize}
\item $T(Q_{seq})(D_0) = D^*_n$
\item $T$ metric is minimized.
\item $|T| = 1$
\end{itemize}
\end{problem}

Finally, we extend the problem to allow transformations with one or more operations.

\begin{problem}[Prob-MultiQ]\label{prob:multi}
Given $C$ where $acc_C < 1$, $Q_{seq}$, and the sequence of database states $D_0,\ldots,D_n$, 
identify a sequence of transformations $T$ such that:
\begin{itemize}
\item $T(Q_{seq})(D_0) = D^*_n$
\item $T$ metric is minimized.
\end{itemize}
\end{problem}




\subsection{A Naive Approach}

\begin{itemize}
\item roll back complaints to penultimate state using algebraic expressions 
\item perturb each expression in query until the query result matches correct state
\item if an expression cannot be found, iterate
\end{itemize}


Not clear how to roll back complaints

Ways to perturb query expressions is unbounded

}