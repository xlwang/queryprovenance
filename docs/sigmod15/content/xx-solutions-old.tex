\ewu{We don't talk about SET clauses anywhere. Are they trivial?  It's not obvious that conditional functions and linearizing them magically fix them.  Need to justify why we are focusing the text on WHERE clause range predicates}


\section{Basic Solution}

In this section we will introduce a simple approach to solve Prob-Complete.


Given $D'_n = C(D_n)$, able to roll back to fixed intermediate state $D'_i$.


\subsection{Metrics}

Metrics we use to measure how good a fix is.

Used in the optimization problem.




\subsection{Single-Query Case}

In this section, we walk through the core techniques in the case of a query sequence containing a single query.
\xxx{Walk through solution for each error type for each query type.}

Inserts and deletes are straight forward, however modifying WHERE clause is harder because
\xxx{why?}.  To deal with this challenge, we tried three algorithmic approaches.

\stitle{Constraint-solver (CPLEX)}
Describe how to encode problem as CPLEX

\stitle{Bounding Box}


\stitle{Decision Tree}
Describe how to encode as decision tree, what algorithm, and how to interpret learned tree.

Also, how to encode constraint that structure needs to be the same, by simply limiting the possible attribute
choices at each step in the learning algorithm.

\subsubsection{Comparing these approaches}

To understand the tradeoffs between these approachse, we ran a simple experiment for a single query.

Here are the results.


CPLEX takes the longest, however produces exact results.  However, it fails to produce any results if there are conflicting complaints.
Decision tree is fastest, however the quality of the results are poor.
Bounding box is similar to CPLEX in quality, however it generates results even if there are conflicting complaints.








\section{Prob-Complete}

Given the above, the algorithm for solving the complete complaint set problem is straightforward (Algorithm~\ref{alg:basic}).
We can simply try each query and return the one for which the best solution is found.

\subsection{Provenance-based Log Filtering}

Use provenance information (tuple-level or query level) to filter out queries that do not 
affect the complaint tuples at all.





\section{Incomplete Complaints}
\label{s:incomplete-algs}

Incomplete complaints are a challenge because the exesting algorithms fail.  We ran a simple experiment
where the complaint set contains only M\% of the true complaint set, and has N randomly generated erroneous complaints.
Figure~\ref{} shows that none of the algorithms work.

Describe assumptions for why a cleaning-based approach makes sense.

We use a cleaning-based method to deal with false positives, and some magic to deal with false negatives.

\subsection{False Positives}

Describe density, bi-partite graph, consistency, tuples affected scores.  Which ones work well, which ones don't.

NP-completeness of density-based approach.


\subsection{False Negatives}



\section{Multi-query Resolution}

We use a dynamic programming-based algorithm to support multiple queries.




\section{Rolling back the log}

background about why this is a hard problem

\subsection{A Solver-based approach}

Table $R(A,B)$ contains tuple $t_0=(a_0,b_0)$. Query $Q_1$ is applied to the
database: $Q_1 =$ UPDATE R SET $A=u_1$ WHERE $B\leq u_2$. The parameters $u_1$
and $u_2$ are constants specified by the query, but we parameterize them to
allow for their modification based on the complaints.

\smallskip

After the execution of $Q_1$, tuple $t_0$ has become $t_1=(a_1,b_1)$. We will
use attribute-level provenance to represent the resulting values $a_1$ and
$b_1$. For what follows, I use the linearization logic from section 4.4 of the
Tiresias paper.

\smallskip


Let $X$ be a Boolean variable that is true if the WHERE clause of $Q_1$
affects $t_0$: $X=(b_0\leq u_2)$. We will map this to integer variable $x$,
following the logic of the Tiresias provenance representation: $x\otimes
b_0\leq x\otimes u_2$.  We assign new, real variables $v_1= x\otimes
b_0$ and $v_2=x\otimes u_2$.

\smallskip

To linearize the expressions of $v_1$ and $v_2$, we use the following constraints (same as in Tiresias, page 8):

\begin{minipage}{0.7\textwidth}
    \begin{minipage}[t]{0.2\textwidth}
        \begin{align*}
            &v_1\le b_0\\
            &v_1\le xM\\
            &v_1\ge b_0 - (1-x)M
        \end{align*}
    \end{minipage}
    \hspace{4em}
    \begin{minipage}[t]{0.2\textwidth}
        \begin{align*}
            &v_2\le u_2\\
            &v_2\le xM\\
            &v_2\ge u_2 - (1-x)M
        \end{align*}
    \end{minipage}
\end{minipage}

\smallskip

Here, $M$ denotes a very large number (upper bound for the $u_i$).

\smallskip

In $t_1$ the attribute $a_1$ can be expressed: $a_1=x\otimes u_1 +
(1-x)\otimes a_0$, meaning that $a_1$ takes the value $u_1$ if $x=1$, and the
value $a_0$ if $x=0$.  We can now assign: $v_3= x\otimes
u_1$ and $v_4=(1-x)\otimes a_0$, and similarly linearize them:

\begin{minipage}{0.7\textwidth}
    \begin{minipage}[t]{0.2\textwidth}
        \begin{align*}
            &v_3\le u_1\\
            &v_3\le xM\\
            &v_3\ge u_1 - (1-x)M
        \end{align*}
    \end{minipage}
    \hspace{4em}
    \begin{minipage}[t]{0.2\textwidth}
        \begin{align*}
            &v_4\le a_0\\
            &v_4\le (1-x)M\\
            &v_4\ge a_0 - xM
        \end{align*}
    \end{minipage}
\end{minipage}

So, now, we can express the transformation of $t_0$ through query $Q_1$ with a
program with 1 integer variable ($x$) and 8 real variables ($u_1$, $u_2$,
$v_1$, $v_2$, $v_3$, $v_4$, $a_1$, $b_1$). $a_0$ and $b_0$ are constants in
this case, as they are the original values of $t_0$, but they could also be
variables if $t_0$ was derived from another query.

Overall, we have a mixed integer program with the constraints:
\begin{align*}
    &b_1=b_0\\
    &a_1 = v_3+v_4\\
    &0\leq x\leq 1\\
    &v_1\le b_0\\
    &v_1\le xM\\
    &v_1\ge b_0 - (1-x)M\\
    &v_2\le u_2\\
    &v_2\le xM\\
    &v_2\ge u_2 - (1-x)M\\
    &v_3\le u_1\\
    &v_3\le xM\\
    &v_3\ge u_1 - (1-x)M\\
    &v_4\le a_0\\
    &v_4\le (1-x)M\\
    &v_4\ge a_0 - xM
\end{align*}

\smallskip 

This is for the transformation of a single tuple ($t_0$) by a single query.
Similar constraints would be derived for the other tuples, and other queries.
In the end, based on the complaints, we can assign values to the corresponding
variables (e.g., $a_1=5$), and solve the program for the parameters $u_i$,
which will give us a correction. The optimization objective can be set so that
the sum of the differences from the query parameters is minimized.

\smallskip

Since the provenance maintains the transformations, there wouldn't be a need
for rolling back the database instance.



\subsection{Hybrid algorithm}

Describe Exhastive and greedy special cases.


Algorithm:

\begin{verbatim}

querylog = filter(querylog)
batchsize = 1
while i > 0
  batch = [qi to q{i-batchsize}]
  alexandras_solver(batch)
  i -= batchsize

\end{verbatim}
