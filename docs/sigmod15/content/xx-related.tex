%!TEX root = ../main.tex

\section{Related Work}
\label{s:related}

\sys tackles the problem of diagnosis and repair in relational query
histories (query logs). It does not aim to correct errors in the data
directly, but rather to find the underlying reason for reported errors
in the queries that operated on the data. This is in contrast to
traditional data cleaning~\cite{rahm00, Raman01, Kalashnikov06,
Fan2008b} which focuses on identifying and correcting data
``in-place.'' Identifying and correcting errors is an important, and
rightfully well-studied problem. Existing literature has supplied a
variety of tools to capture errors originating from data
integration~\cite{Abiteboul99, Batini1986, Rahm2001, ParentS98},
recognizing the same entities in data~\cite{Koudas2006,
GruenheidDS14}, identifying true facts among conflicting
data~\cite{yin2008truth, DN09, ltm2012}, and language support for
cleaning~\cite{Galhardas2000}. All of these techniques are
complimentary to our work. Their goal is to identify which data is
correct and which data is incorrect, but they don't look for the
sources of these errors in the processes or queries that generate and
modify this data. In fact the output of such methods can be used to
augment the complaint sets used by \sys, which focuses on identifying
errors in the queries that produced the data, rather than the data
itself.


%things about diagnosis and explanations


\cite{Golab2008,GolabKKS10 } 

Studying and explaining
data outcome has been studied in many aspects:
\cite{GebalyAGKS14}
focuses on providing 
explanations to particular data outcome in tables; \cite{wang2015}
provides error diagnosing for general data extraction systems. 


% things  about validating updates before executing/cmmit to database
Related work about validating updates in time: 
\cite{Chen2011} validate correctness of updates before performing the update;


% things for exploring wrong or undetermined query,  why not? 
Related works on deriving desired select query: focus on 
Query by example, \cite{dimitriadou2014explore}  uses machine learning
techniques to explore query that satisfy user interests. 


% things about data provenance and data debugging
\cite{mucslu2013data}



Scorpion explanation uses data to synthesize predicates that explain.

Why not work.

View construction and query by example.

Take a look at temporal databases, the CIDR/arxiv paper has a few good starting points. Aditya also dug these out a while back


The way we return results is more like online aggregation or anytime algorithms -- 
as you run longer, the suggested fixes improve because we examine more of the query log.

\url{http://citeseerx.ist.psu.edu/viewdoc/download?doi=10.1.1.49.3765&rep=rep1&type=pdf}

\url{http://people.cs.aau.dk/~csj/Thesis/pdf/chapter26.pdf}
