%!TEX root = ../main.tex

\section{Related Work}
\label{s:related}

\sys tackles the problem of diagnosis and repair in relational query
histories (query logs). This is in contrast to traditional data
cleaning~\cite{rahm00} which focuses on identifying and correcting
data ``in-place.'' \sys does not aim to correct errors in the data
directly, but rather to find the underlying reason for reported errors
in the queries that operated on the data. 
%things about diagnosis and explanations

Studying and explaining
data outcome has been studied in many aspects: 
\cite{Golab2008, el2014interpretable} focuses on providing 
explanations to particular data outcome in tables; \cite{wang2015}
provides error diagnosing for general data extraction systems. 


% things  about validating updates before executing/cmmit to database
Related work about validating updates in time: 
\cite{Chen2011} validate correctness of updates before performing the update;


% things for exploring wrong or undetermined query,  why not? 
Related works on deriving desired select query: focus on 
Query by example, \cite{dimitriadou2014explore}  uses machine learning
techniques to explore query that satisfy user interests. 


% things about data provenance and data debugging
\cite{mucslu2013data}



Scorpion explanation uses data to synthesize predicates that explain.

Why not work.

View construction and query by example.

Take a look at temporal databases, the CIDR/arxiv paper has a few good starting points. Aditya also dug these out a while back


The way we return results is more like online aggregation or anytime algorithms -- 
as you run longer, the suggested fixes improve because we examine more of the query log.

\url{http://citeseerx.ist.psu.edu/viewdoc/download?doi=10.1.1.49.3765&rep=rep1&type=pdf}

\url{http://people.cs.aau.dk/~csj/Thesis/pdf/chapter26.pdf}
