%!TEX root = main.tex

\documentclass{sig-alternate-05-2015}
% \usepackage{paralist}
% Load basic packages
\usepackage{balance}  % to better equalize the last page
\usepackage{graphicx} % for EPS, load graphicx instead
\usepackage{url}      % llt: nicely formatted URLs
\usepackage{amsmath}
\usepackage{color}
\usepackage{cancel}
\usepackage{listings}
%\usepackage{wrapfig}
\usepackage{graphicx}
\usepackage{subcaption}
\setlength{\abovecaptionskip}{10pt plus 3pt minus 2pt}
% \usepackage[english]{babel}
\usepackage{booktabs}
\usepackage{graphicx}
% \usepackage{subfigure}
% \usepackage{caption}
% \usepackage[scriptsize, it, IT]{subfigure}
% \usepackage[font={scriptsize,it}]{caption}
% \usepackage{tikz}
%\usepackage{MinionPro}
% \usetikzlibrary{arrows,matrix,positioning}
\usepackage[ruled,vlined,algonl,boxed]{algorithm2e}
\usepackage{algorithmic}
\usepackage{wrapfig}
% \usepackage{framed}
\usepackage{enumitem}
\usepackage{xspace}
\usepackage{xcolor}
\usepackage{colortbl}
% \usepackage{newtxmath}
\usepackage[T1]{fontenc}

\usepackage[nocompress]{cite}
\usepackage{microtype}

% llt: Define a global style for URLs, rather that the default one
\makeatletter
\def\url@leostyle{%
  \@ifundefined{selectfont}{\def\UrlFont{\sf}}{\def\UrlFont{\small\bf\ttfamily}}}
\makeatother
\urlstyle{leo}

\makeatletter
%\def\@copyrightspace{\relax}
\makeatother

% To make various LaTeX processors do the right thing with page size.
\def\pprw{8.5in}
\def\pprh{11in}
\special{papersize=\pprw,\pprh}
\setlength{\paperwidth}{\pprw}
\setlength{\paperheight}{\pprh}
\setlength{\pdfpagewidth}{\pprw}
\setlength{\pdfpageheight}{\pprh}

\newtheorem{definition}{Definition}%[section]
\newtheorem{proposition}[definition]{Proposition}
\newtheorem{lemma}[definition]{Lemma}
\newtheorem{remark}[definition]{Remark}
\newtheorem{corollary}[definition]{Corollary}
\newtheorem{claim}[definition]{Claim}
\newtheorem{theorem}[definition]{Theorem}
\newtheorem{heuristic}[definition]{Heuristic}
\newtheorem{example}[definition]{Example}
%\newtheorem{proof}[definition]{Proof}
\newtheorem{dimension}{Dimension}
\newcounter{prob}
\newtheorem{problem}[prob]{Problem}
\newtheorem{conjecture}[definition]{Conjecture}
\newtheorem{reduction}[definition]{Reduction}
\newtheorem{property}[definition]{Property}
\newtheorem{axiom}[definition]{Axiom}
% 
% \tikzset{
%     %Define standard arrow tip
%     >=stealth',
%     %Define style for boxes
%     punkt/.style={
%            rectangle,
%            rounded corners,
%            draw=black, very thick,
%            text width=6.5em,
%            minimum height=2em,
%            text centered},
%     % Define arrow style
%     pil/.style={
%            ->,
%            thick,
%            shorten <=2pt,
%            shorten >=2pt,}
% }


% \setitemize{noitemsep,topsep=0pt,parsep=0pt,partopsep=0pt}

% \def\compactify{\itemsep=0pt \topsep=0pt \partopsep=0pt \parsep=0pt}
% \let\latexusecounter=\usecounter
% \newenvironment{CompactItemize}
%   {\def\usecounter{\latexusecounter}
%    \begin{itemize}[noitemsep,topsep=0pt,parsep=0pt,partopsep=0pt,leftmargin=*]}
%   {\end{itemize}\let\usecounter=\latexusecounter}
% \newenvironment{CompactEnumerate}
%   {\def\usecounter{\compactify\latexusecounter}
%    \begin{enumerate}[leftmargin=*]}
%   {\end{enumerate}\let\usecounter=\latexusecounter}


%%%  What is this?  Use enumitem instead
% 
% \newcommand{\squishlist}{
%    \begin{list}{$\bullet$}
%     { \setlength{\itemsep}{0pt}
%       \setlength{\parsep}{2pt}
%       \setlength{\topsep}{6pt}
%       \setlength{\partopsep}{0pt}
%       \leftmargin=25pt
% \rightmargin=0pt
% \labelsep=5pt
% \labelwidth=10pt
% \itemindent=0pt
% \listparindent=0pt
% \itemsep=\parsep
%     }
% }
% \newcommand{\squishend}{\end{list}}
% 
% 
% \newcommand{\squishframe}{\vspace{-6pt}
% \begin{framed} 
% \vspace{-6pt}}
% 
% \newcommand{\frameend}{\vspace{-6pt}
% \end{framed}
% \vspace{-6pt}}


% create a shortcut to typeset table headings
% \newcommand\tabhead[1]{\small\textbf{#1}}




% Make sure hyperref comes last of your loaded packages, 
% to give it a fighting chance of not being over-written, 
% since its job is to redefine many LaTeX commands.
\usepackage[pdftex]{hyperref}
\hypersetup{
  colorlinks=false,
  linkcolor=darkred,
  citecolor=darkgreen,
  urlcolor=darkblue
}
% \usepackage{cleveref} % After hyperref, listings



% Avoid widows and orphans
\widowpenalty=500
\clubpenalty=500

% % Aggressive figure placement
% \renewcommand{\topfraction}{0.9}
% \renewcommand{\bottomfraction}{0.8}
% \setcounter{topnumber}{2}
% \setcounter{bottomnumber}{2}
% \setcounter{totalnumber}{4}
% \setcounter{dbltopnumber}{2}
% \renewcommand{\dbltopfraction}{0.9}
% \renewcommand{\textfraction}{0.07}
% \renewcommand{\floatpagefraction}{0.7}
% \renewcommand{\dblfloatpagefraction}{0.7}

\definecolor{light-gray}{gray}{0.95}
\definecolor{mid-gray}{gray}{0.85}
\definecolor{darkred}{rgb}{0.7,0.25,0.25}
\definecolor{darkgreen}{rgb}{0.15,0.55,0.15}
\definecolor{darkblue}{rgb}{0.1,0.1,0.5}
\definecolor{blue}{rgb}{0.19,0.58,1}

\newcommand{\red}[1]{\textcolor{red}{#1}}
\newcommand{\green}[1]{\textcolor{green}{#1}}
\newcommand{\blue}[1]{\textcolor{blue}{#1}}
\newcommand{\orange}[1]{\textcolor{orange}{#1}}
\newcommand{\darkred}[1]{\textcolor{darkred}{#1}}
\newcommand{\darkgreen}[1]{\textcolor{darkgreen}{#1}}
\newcommand{\darkblue}[1]{\textcolor{darkblue}{#1}}

\makeatletter
\setlength{\@fptop}{0pt}
\makeatother



\newcommand{\papertext}[1]{#1}
\newcommand{\techreport}[1]{#1}

\newcommand{\alex}[1]{\noindent{\color{darkgreen}{Alexandra: #1}}}
\newcommand{\xlw}[1]{\noindent{\color{blue}{xl: #1}}}
\newcommand{\ewu}[1]{\noindent{\color{red}{EWu: #1}}}
\newcommand{\xxx}[1]{{\fontsize{13pt}{13pt}\selectfont\textcolor{red}{#1}}}
\newcommand{\codesize}{\fontsize{7}{8}}
\newcommand{\stitle}[1]{\vspace{0.5em}\noindent\textbf{#1}}
\newcommand{\calF}[0]{$\cal{F}$}

\newcommand{\ind}{\hspace{\algorithmicindent}}

% \newcommand{\deprecate}[1]{}
\newcommand{\deprecate}[1]{\noindent{\color{light-gray}{#1}}}

\newcommand{\prob}{{\sc Log-Corruption}\xspace}
\newcommand{\exact}{{\sc EXACTSOL}\xspace}
\newcommand{\qfix}{{\sc SingleQueryFix}\xspace}
\newcommand{\density}{{\sc DENSITY}\xspace}


\newcommand{\milpall}{\textsc{MILP-NAIVE}\xspace}
\newcommand{\milptuple}{\textsc{MILP-COMPL}\xspace}
\newcommand{\milptuplestopearly}{\textsc{MILP-COMPL-STOPEARLY}\xspace}
\newcommand{\milpadvtuple}{\textsc{MILP-ADV-TUPLE}\xspace}
\newcommand{\milpadvall}{\textsc{MILP-ADV-ALL}\xspace}
\newcommand{\heurstic}{\textsc{HEURISTIC}\xspace}

\pagenumbering{arabic}

%%Ensuring that equation labels remain normal size, even if we shrink the equation text
\makeatletter
\def\maketag@@@#1{\hbox{\m@th\normalfont\normalsize#1}}
\DeclareRobustCommand*\textsubscript[1]{%
          \@textsubscript{\selectfont#1}}
        \def\@textsubscript#1{%
          {\m@th\ensuremath{_{\mbox{\fontsize\sf@size\z@#1}}}}}
\makeatother

\newcommand{\sysname}{\textsc{QueryFix}}
\newcommand{\sys}{QFix\xspace}
\newcommand{\naive}{\sysname\textsubscript{basic}\xspace}
\newcommand{\tslice}{\sysname\textsubscript{ts}\xspace}
\newcommand{\qslice}{\sysname\textsubscript{qs}\xspace}
\newcommand{\aslice}{\sysname\textsubscript{as}\xspace}
\newcommand{\incremental}{\sysname\textsubscript{inc}\xspace}

% End of preamble. Here it comes the document.
\begin{document}

% for 
\title{{\sys}: Demonstrating error diagnosis in query histories}

\numberofauthors{3}
\author{
  \alignauthor Xiaolan Wang\\
    \affaddr{School of Computer Science}\\
    \affaddr{University of Massachusetts}\\
    \email{xlwang@cs.umass.edu}\\
  \alignauthor Alexandra Meliou\\
  \affaddr{School of Computer Science}\\
    \affaddr{University of Massachusetts}\\
    \email{ameli@cs.umass.edu}\\
  \alignauthor Eugene Wu\\
    \affaddr{Computer Science}\\
    \affaddr{Columbia University}\\
    \email{ewu@cs.columbia.edu}\\
}




{
\maketitle
}

\begin{abstract}

An increasing number of applications in all aspects of society rely on
data. Despite the long line of research in data cleaning and repairs,
data correctness has been an elusive goal. Errors in the data can be
extremely disruptive, and are detrimental to the effectiveness and
proper function of data-driven applications. Even when data is
cleaned, new errors can be introduced by applications and users who
interact with the data. Subsequent valid updates can obscure these
errors and propagate them through the dataset causing more
discrepancies. Any discovered errors tend to be corrected
superficially, on a case-by-case basis, further obscuring the true
underlying cause, and making detection of the remaining errors harder.

In this demo proposal, we outline the design of \sys, a {\it query-centric}
framework that derives explanations and repairs for discrepancies in 
relational data based on potential errors in the queries that operated on the data. 
This is a marked departure from traditional {\it data-centric} techniques that directly fix the data.
We then describe how users will use \sys in a demonstration scenario. 
Participants will be able to select from a number of 
transactional benchmarks, introduce errors into the queries that are executed,
and compare the fixes to the queries proposed by \sys as well as existing alternative algorithms
such as decision trees.

\end{abstract}

%!TEX root = ../main.tex

\section{Introduction}
\label{s:intro}

Poor data quality is estimated to cost the US economy more than \$600 billion
per year~\cite{eckerson2002} and erroneous price data in retail databases
alone cost the US consumers \$2.5 billion each year~\cite{Fan2008}. 
\ewu{
It is possible to use tools to fix existing errors in the database.
However, databases are dynamic -- applications and users constantly generate and execute queries that modify the database.
Any errors in the queries can easily propagate in complex ways and spread errors throughout the database that are hard to reason about.
Fixing the errors that have been found in the current data is not enough because not all of the errors may have been found because the source of the error could be buried deep in the past.
No tools exist to both {\it find} the erroneous queries and provide {\it fixes/explanations}.
}
While data cleaning tools can purge datasets of many errors before the data is
used, applications and users interacting with the data can introduce new
errors. Mistakes in data entry and erroneous updates often affect datasets in
complex ways and result in data errors that are obscure and hard to trace and
correct. Traditional data cleaning approaches are not well-suited for this
purpose: While they provide general-purpose tools to identify and rectify
anomalies in the data, they are not designed to diagnose the causes of these
errors, which are rooted in erroneous updates.

Improving data quality and correcting data errors has been an important focus
for data management research. Yet, handling new errors, introduced during
regular database interactions, has received little attention. Integrity
constraints~\cite{Khoussainova2006} can guard against some improper updates,
but only if the data falls outside rigid, predefined ranges. Certificate-based
verification~\cite{Chen2011} is impractical as it requires users to answer
challenge questions before allowing the updates, and it is not applicable to
updates initiated by applications.


\ewu{
In this paper, we argue for a new type of data error diagnosis called Data Fracking 
that targets data modification queries deep in the database transaction log.
In contrast to existing data cleaning and explanation approaches that aim to
detect, fix, or explain errors in the current database, Data Fracking both identifies the
root causes of the errors as past query transactions that {\it introduced} the errors
into the database, and proposes modifications to those anomalous queries that
fix the current errors.  By introducing a {\it new dimension} to data error diagnosis,
data fracking can lead to the identification of additional discrepancies that would
have otherwise remained undetected.
}
In this paper, we present a specialized data cleaning framework, \sys, that
specifically targets errors in update workloads. In contrast to traditional
data cleaning techniques that aim to identify errors in the data directly, the
goal of \sys is to \emph{explain} how the errors occurred. Our goal is not
simply to provide explanations; identifying how an error was introduced to a
dataset, can lead to the identification of additional discrepancies that would
have otherwise remained undetected.

\begin{example}[Incorrect insurance premiums]
    After negotiations with health insurance companies, an employer achieved
    reductions in the premium rates for the upcoming year, for employees at
    levels 1-4. The administrator who updated the employee database mistakenly
    implemented the new policy for employees at levels 1-3, due to the
    incorrect predicate \texttt{level < 4}.
    
    In subsequent months, level 4 employees who noticed that their premiums
    had not been reduced as expected, notified HR of the error. HR corrected
    employees' records on a case-by-case basis, obscuring the real cause of
    the problem, and making it harder to identify and correct the error for
    other employees.
    
    To make matters worse, subsequent queries that calculated employee
    withholdings, resulted in misestimation of the company's revenue and
    incorrect allocation of bonuses. Correcting errors in the employee
    premiums is no longer sufficient, as these errors have already had a
    larger effect on the data.
    
        \alex{Needs to be more realistic and convincing.  Ideally grounded on a real scenario.  Do we have more specific info for the use-cases?}
\end{example}

\ewu{In reality, the problems presented to HR were simply a small sample
of the errors stemming from the anomalous query.  Once HR fixes the errors,
there are still a potentially large number of employees whose premiums are incorrect that have
simply not complained to HR yet.
it is thus desirable to, given HR's fixes, identify candidate queries that may have been the source
of the error, propose fixes, and subsequently identify the other employees whos premiums are in error.

This problem is fundamentally challenging because ??? and must solve several challenging subproblems.
1) rollback  2) fix generation  3) fix picking (if there are multiple possible fixes) 4) performance.
}



% these characterize the problem
\ewu{Would say these nicely characterize the problem, not necessarily challenges.}
Diagnosing data errors stemming from incorrect updates raises three major
challenges that make existing data cleaning techniques not applicable.



\begin{description}[leftmargin=*, topsep=0mm, itemsep=0mm]
    
    \item[Obscurity.] Detection of the resulting errors in the data often
    leads to partial fixes that further complicate the eventual diagnosis and
    resolution of the problem. For example, a transaction implementing a
    change in the state tax law updated tax rates using the wrong rate,
    affecting a large number of consumers. This causes a large number of
    complaints to a call center, but each customer agent usually fixes each
    problem individually, which ends up obscuring the source of the problem.
    
    \item[Large impact.] Erroneous queries cause errors in a large scale. The
    potential impact of the errors is high, as manifested in several
    real-world cases~\cite{Yates10, Grady13, sakalerrors}. Further, errors
    that remain undetected for a significant amount of time can instigate
    additional errors, even through valid updates. This increases both their
    impact, and their obscurity.
    
    \item[Systemic errors.] The errors created by bad queries are
    \emph{systemic}: they have common characteristics, as they share the same
    cause. The link between the resulting data errors is the query that
    created them; cleaning techniques should leverage this connection to
    diagnose and fix the problem. Diagnosing the cause of the errors, will
    achieve systematic fixes that will correct all relevant errors, even if
    they have not been explicitly identified.
    
\end{description}
% 
\sys addresses these challenges by analyzing the queries that operated on a
dataset in an efficient and scalable manner. More concretely, we make the
following contributions:


% The goal of this paper is to design effective query
% diagnosis techniques and identify possible fixes for query errors. We
% model the problem assuming a log of update workloads over a database,
% and a set of complaints that identify errors in the final database
% state. We organize our contributions as follows:

\ewu{really like this organization}.

\begin{itemize}[leftmargin=*, topsep=0mm, itemsep=0mm]      
    \item We formalize the Data Fracking problem of diagnosing a set of errors using log
    histories of updates that operated on the data. Given a set of 
    \emph{complaints} as representations of data discrepancies in the current
    state of a dataset, \sys identifies the queries in the log that require the  minimal
    amount of changes that would resolve all of the complaints (Section~\ref{???}).
      
    \item We provide an exact error-diagnosis solution through a non-trivial
    transformation of the problem to a mixed integer program (MIP) that
    analyzes the data provenance of the erroneous tuples at the attribute
    level. Our approach employs state-of-the-art MIP solvers to identify
    optimal diagnoses that are guaranteed to resolve all complaints, and is
    tolerant to incomplete information (missing complaints)
    (Section~\ref{???}).
    
    \item While modern solvers can handle large numbers of variables and
    constraints, eventually, they fail to scale to very large datasets and
    large query logs. We present several performance optimizations that allow
    our diagnostic methods to scale, in many cases without affecting the
    quality of the produced solutions (Section~\ref{???}).
    
    \item We extend our framework to also handle false positives: complaints
    that mistakenly identify data as erroneous. We define the notion of
    complaint \emph{density}, which is a query-driven measure of closeness of
    a complaint to other complaints. The main intuition of our approach is
    that complaints of low density are likely false positives and thus can be
    safely ignored (Section~\ref{???}).
    
    \item We experimentally evaluate the effectiveness and efficiency of our
    methods against real-world and synthetic datasets and query logs. We
    demonstrate that \xxx{... to be completed when we know what we have}
    (Section~\ref{???}).
\end{itemize}


\deprecate{
While exploring data, its natural to come across surprising or unexpected data.
For example, visual data analysis explores the current state of the database and users may be surprised by outliers in a visualization.
Similarly, enterprise customers (e.g., billing) may find outliers in their monthly bills and be surprised by the amount they are asked to pay.

When presented with these surprises, users want to better understand the reasons behind the anomalies.
A recent wave of research focuses on deriving predicate-based explanations for outliers for statistical aggregation queries.
For example, if the user wants to understand why the total sales in the past few months have gone up, these systems can general explanations such as ``most related to customers in California between the ages of 12 to 18.''
However, these approaches simply generate predicates that describe {\it current state} of the database, and do not resolve {\it how} the anomalous data came to be.

Specifically, the user may also be interested to understand which past database modification was responsible for these explanations.
Describe why this makes sense to want.  In this form of the problem, we are interested in historical database queries whose modifications, when propogated to the current database state, 
The goal is to provide diagnostic tools that can peer into past transactions.


In this paper, we approach anomaly explanation from the persepctive of the query log and seek to
both {\it identify}  historical database modification queries that most likely caused user complaints 
in the current state of the database, and suggest replacement queries that will resolve these complaints.
We call this problem the {\it Query-based Complaint-Satisfaction Problem}.

Given a database query log and a set of {\it complaints} (e.g., tuple 1's attribute B should be 20\% lower) about records in the current state of the database,
we seek to identify the subset of queries in the log that, by modifying their parameters and propogating the new effects of the queries, 
will best resolve the complaints.  

One way to solve this problem is to try modifying the most recent query until it fixes the complaints.  
If not, then try the second most recent query.  
The problem with this approach is the number of possible modifications is unbounded.

Our contributions include

\begin{enumerate}
\item Developing and formalizing the problem of Query-oriented explanation in contrast to data-oriented explanation
\item Prove that the general problem is impossible.
\item Designing alogirthems to solve the problem for complete complaint sets
\item Extending the algorithms to support incomplete complaint sets
\item Extending to support multiple queries
\end{enumerate}

%%%%%%%%%%%%%%% Clean up.... %%%%%%%%%%%%%%

}


\section{{\Large\textbf{\sys}}: architecture}


\begin{figure}[t]
    \centering
        \includegraphics[scale=0.35]{figures/architecture}
    \caption{\sys processes data anomalies in the form of complaints and analyzes logged query histories to identify the causes of error. In the heart of the system, transformation algorithms express the diagnosis problem as a mixed integer program, and optimization modules ensure that the MIP programs can be evaluated efficiently.}
    \label{fig:architecture}
\end{figure}


\alex{Do we want to actually describe the architecture in a small section, or should we just include the figure in the front page?}



\section{{\large\textbf{\sys}}: architecture}

\begin{figure}[h]
    \centering
        \includegraphics[scale=0.35]{figures/architecture}
    \caption{\sys processes data anomalies in the form of complaints and analyzes logged query histories to identify the causes of error. In the heart of the system, transformation algorithms express the diagnosis problem as a mixed integer program, and optimization modules ensure that the MIP programs can be evaluated efficiently.}
    \label{fig:architecture}
\end{figure}


Figure~\ref{fig:architecture} shows \sys's major components.  \sys takes as input 
a query log containing UPDATE, INSERT and DELETE queries, the database, along with a
set of identified data errors (called {\it complaints}).  These complaints pairs
of tuple ids of tuples that are wrong, along with an estimate of their correct values (e.g., $30$ or $30\%$ higher).
\sys uses this information to trace the causes of the errors and output the most likely set of 
queries in the log ({\it diagnoses}), along with proposed {\it repairs} of these queries.

To achieve this, \sys first performs an optional outlier removal step to deal with potential
false positives in the complaints.  Then the {\it MILP Encoding} component transforms the
query diagnosis problem into a Mixed Integer Linear Program (MILP) that is further optimized
through slicing and incremental repair techniques, before being sent
to an industrial MILP solver.  The output of the solver constitutes solutions to the query diagnosis
problem.

%!TEX root = ../main.tex

\section{Modeling abstractions}
\label{sec:abstractions}

In this section, we introduce a running example inspired from the use-case of
Example~\ref{ex:taxes}, and describe the model abstractions that we use to
formalize the diagnosis problem.


% \ewu{Add text to say false positive complaints are an orthogonal problem.}

%!TEX root = ../main.tex
\begin{example}[Salary Update Error]\label{ex:telco}
  A manager updates the tax amount with $30\%$ tax rate for high income employees that earn more than $\$87500$.
  She submits this through a form in the salary accounting application, 
  but due to keyboard slip, incorrectly types $\$85700$ for the income threshold.  
  Later queries that insert new paychecks, compute tax calculations,
  aggregate department salaries end up propogating this error throughout other records in the database, leading to
  employee dissatifaction.  Figure~\ref{fig:example} illustrates a concrete example where $Q_1$, $Q_2$, and $Q_3$ are executed on an 
  initial salary database $D_0$.  The error in $Q_1$ that incorrectly sets some of the tax rates is propogated to other fields in the table.
\end{example}

\begin{figure}[t]
    \centering
        \includegraphics[width=0.45\textwidth]{figures/example2}
    \caption{$Q_1$ updates the tax amount with $30\%$ tax rate 
      for high income employees using an incorrect predicate.  
      The error is propogated by $Q_2$ to the $pay$ field in the database.
      Finally, a benign insert query $Q_3$ inserts correct salary information. 
      The final database state contains a mixture of incorrect and correct salary data.
    }
    \label{fig:example}
\end{figure}


In this example, by the time errors in the database have been detected, 
perhaps by employees that report their encorrect paystubs, it is difficult 
to both identify all of the other errors in the database, and to trace these errors back to the erroneous query to correct it.
This example can occur in any data processing systems where manual input is used to generate queries that modify the database --
this could be in the form of adhoc queries executed by a system administrator, or web-based forms that construct queries based
on user input, or even stored procedures that use human input to fill the parameter values.


\begin{example}\label{ex:taxes2}
Figure~\ref{fig:example} demonstrates an example tax bracket adjustment in the
spirit of Example~\ref{ex:taxes}. The adjustment sets the tax rate to 30\% for
income levels above \$87,500, and is implemented by query $q_1$. A digit
transposition mistake in the query, results in an incorrect owed amount for tuples
$t_3$ and $t_4$. Query $q_2$, which inserts a tuple with slightly higher
income than $t_3$ and $t_4$ and the correct (lower) tax rate, obscures this mistake.
This mistake is further propagated by query $q_3$, which calculates the pay 
check amount based on the corresponding
tax rate and income. 
\iffalse    
Figure~\ref{fig:example} demonstrates an example tax bracket adjustment in the
spirit of Example~\ref{ex:taxes}. The adjustment sets the tax rate to 30\% for
income levels above \$87,500, and is implemented by query $q_1$. A digit
transposition mistake in the query, results in an incorrect tax rate for tuples
$t_3$ and $t_4$. Query $q_2$ that calculates the amount owed based on the corresponding
tax rate and income propagates the error to other fields. The mistake is
further obscured by query $q_3$, which inserts a tuple with slightly higher
income than $t_3$ and $t_4$ and the correct (lower) tax rate.
\fi
\end{example}
\vspace*{-0.07in}
While traditional data cleaning techniques seek to identify and correct the
erroneous values in the table \emph{Taxes} directly, our goal is to diagnose
the problem, and understand the reasons for these errors. In this case, the
reason for the data errors is the incorrect predicate value in query $q_1$.

In this paper, we assume that we know \emph{some} errors in the dataset, and
that these errors were caused by erroneous updates. The errors may be
obtained in different ways: traditional data cleaning tools may identify
discrepancies in the data (e.g., a tuple with lower income has higher tax
rate), or errors can be reported directly from users (e.g., customers
reporting discrepancies to customer service). \emph{Our goal is not to correct
the errors directly in the data, but to analyze them as a ``symptom'' and provide a
diagnosis.} The diagnosis can produce a targeted treatment: knowing how the
errors were introduced guides the proper way to trace and resolve them.



\begin{figure}[t]
\centering
{\small
\begin{tabular}{ll}
    \toprule
    \textbf{Notation} & \textbf{Description}\\
    \midrule
    $\mathcal{Q}$& The sequence of executed update queries (log)\\ 
             & $\mathcal{Q}=\{q_1, \dots, q_n\}$ \\
    $D_0$    & Initial database state at beginning of log\\
    $D_n$    & End database state (current) $D_n=\mathcal{Q}(D_0)$\\
    $D_i$    & Database state after query $q_i$: $D_i=q_i(\dots q_1(D_0))$\\
    $c: t\mapsto t^*$ & Complaint: $\mathcal{T}_c(D) = (D_n\setminus\{t\})\cup\{t^*\}$\\
    $\mathcal{C}$ & Complaint set $\mathcal{C}=\{c_1,\dots,c_k\}$\\
    $\mu_q(t)$  & Modifier function of $q$ (e.g., \texttt{SET} clause)\\
    $\sigma_q(t)$   & Conditional function of $q$ (e.g., \texttt{WHERE} clause)\\
    $t_{new}$   & Tuple values introduced in an \texttt{INSERT} query\\
    $\mathcal{Q}^*$& Log repair\\
    $d(\mathcal{Q}, \mathcal{Q}^*)$ & Distance functions between two query logs\\
    \bottomrule
\end{tabular}
}
\vspace{-2mm}
\caption{Summary of notations used in the paper. }
\label{tbl:notation}
\end{figure}

\subsection{Error modeling}
\label{sec:model}

In our setting, the diagnoses are associated with errors in the queries that
operated on the data. In Example~\ref{ex:taxes2}, the errors in the dataset
are due to the digit transposition mistake in the WHERE clause predicate of
query $q_1$. Our goal is to infer the errors in a log of queries
automatically, given a set of incorrect values in the data. We proceed to
describe our modeling abstractions for data, queries, and errors, and how we
use them to define the diagnosis problem.

\subsubsection*{Data and query models}
\label{sec:models}

\noindent
\emph{Query log ($\mathcal{Q}$):}
We define a query log that update the database 
as an ordered sequence of \texttt{UPDATE}, \texttt{INSERT}, and
\texttt{DELETE} queries $\mathcal{Q}=\{q_1,\dots,q_n\}$, that have
operated on a database $D$. In the rest of the paper, we use the term
\emph{update queries}, or just \emph{queries}, to refer to any of the queries in $\mathcal(Q)$,
including insertion and deletion queries.

\smallskip
\noindent
\emph{Query ($q_i$):} We model each query as a function over a
database $D$, resulting in a new database $D'$. For \texttt{INSERT}
queries, $D'=q(D)=D\cup\{t_{new}\}$.
We model \texttt{UPDATE} and \texttt{DELETE} queries as follows:  
\begin{align*}
    D'=q(D)= &\{\mu_{q}(t)\;|\;t\in D, \sigma_{q}(t)\}%\\
    % &
    \cup\{t\;|\;t\in D, \neg\sigma_{q}(t)\}%\\
    % &\cup\{t_{new}\;|\;q\in\texttt{INSERT}\}
\end{align*}
% 
In this definition, the modifier function $\mu_q(t)$ represents the query's update equations, and it transforms a tuple by either deleting it ($\mu_q(t)=\bot$) or changing the values of some of its attributes.
The conditional function $\sigma_q(t)$ is a boolean function that represents the query's condition predicates.  In the example of Figure~\ref{fig:example}:
\begin{align*}
    &\mu_{q_1}(t)=(t.income, t.income*0.3, t.pay)\\
    &\sigma_{q_1}(t)=(t.income\ge 85700)\\
    &\mu_{q_3}(t)=(t.income, t.owed, t.income-t.owed)\\
    &\sigma_{q_2}(t)=\texttt{true}
\end{align*} 
\iffalse
\begin{align*}
    &\mu_{q_1}(t)=(30, t.income, t.owed)\\
    &\sigma_{q_1}(t)=(t.income\ge 85700)\\
    &\mu_{q_2}(t)=(t.rate, t.income, \frac{t.income\cdot t.rate}{100})\\
    &\sigma_{q_2}(t)=\texttt{true}
\end{align*} 
\fi
Note that in this paper, we only consider query without sub-query or aggregation. 
% 
% As an insertion query, $q_3$ has $\sigma_{q_3}(t)=\texttt{false}$ and $t_{new}=(25, 85800, 21450)$.
% \ewu{why does it return false?  should it only be true if input is $\bot$?}


\smallskip
\noindent
\emph{Database state ($D_i$):}
We use $D_i$ to represent the state of a database $D$ after the application of
queries $q_1$ through $q_i$ from the log $\mathcal{Q}$. $D_0$ represents the
original database state, and $D_n$ the final, or current, database state. Out
of all the states, the system only maintains $D_0$ and $D_n$. In practice,
$D_0$ can be a checkpoint: a state of the database that we assume is correct;
we cannot diagnose errors before this state. The intermediate states can be
derived by executing the log: $D_i=q_i(q_{i-1}(\dots q_1(D_0)))$. We also
write $D_n=\mathcal{Q}(D_0)$ to denote that the final database state $D_n$ can
be derived by applying the sequence of queries in the log to the original
database state $D_0$.

\smallskip
\noindent
\emph{True database state ($D_i^*$):}
Queries in $\mathcal{Q}$ are possibly erroneous, introducing errors in the
data. There exists a sequence of \emph{true} database states $\{D_0^*,
D_1^*\dots, D_n^*\}$, with $D_0^*=D_0$, representing the database states that
would have occurred if there had been no errors in the queries.
The true database states are unknown; our goal is to find and correct the errors in $\mathcal{Q}$ and retrieve the correct database state $D_n^*$.

For ease of exposition, in the remainder of the paper we assume that the
database contains a single relation with attributes $A_i,\ldots,A_m$,
but the single table is not a requirement in our framework.


\subsubsection*{Error models}

Following the terminology in Examples~\ref{ex:telco}
and~\ref{ex:taxes}, we model a set of identified or user-reported
data errors as \emph{complaints}. A complaint corresponds to a
particular tuple in the final database state $D_n^*$, and identifies
that tuple's correct value assignment. We formally define complaints
below:

\begin{definition}[Complaint]
    A complaint $c$ is a mapping between two tuples: $c: t\mapsto t^*$, such that $t$ and $t^*$ have the same schema, $t\in D_n\cup\{\bot\}$, and $t\neq t^*$. A complaint defines a
    transformation $\mathcal{T}_c$ on a database state $D$: $\mathcal{T}_c(D)
    = (D\setminus\{t\})\cup\{t^*\}$.
\end{definition}

In the example of Figure~\ref{fig:example}, two complaints are reported on the final database state $D_3$: 
$c_1: t_3\mapsto t_3^*$ and
$c_2: t_4\mapsto t_4^*$, 
where $t_3^*=(86000,21500,64500)$ and $t_4^*=(86500,21625,64875)$. 
%where $t_3^*=(25,86000,21500)$ and $t_4^*=(25,86500,21625)$.  
For both these cases, each complaint denotes a \textbf{value correction} for a tuple in $D_3$.  Complaints can also model the \textbf{addition} or \textbf{removal} of tuples: $c: \bot\mapsto t^*$ means that $t^*$ should be added to the database, whereas $c: t\mapsto \bot$
means that $t$ should be removed from the database.


\smallskip
\noindent
\emph{Complaint set ($\mathcal{C}$):}
We use $\mathcal{C}$ to denote the set of all known complaints
$\mathcal{C}=\{c_1,\dots,c_k\}$, and we call it the \emph{complaint set}.
Each complaint in $\mathcal{C}$ represents a transformation (addition,
deletion, or modification) of a tuple in $D_n$. We assume that the
complaint set is consistent, i.e., there are no two complaints that
propose different transformations to the same tuple $t\in D_n$.
Applying all these transformations to $D_n$ results in a new database
instance
$D_n'=\mathcal{T}_{c_1}(\mathcal{T}_{c_2}(\dots\mathcal{T}_{c_k}(D_n)))$.\footnote{Since
the complaint set is consistent, it is easy to see that the order of
transformations is inconsequential.} $\mathcal{C}$ is \emph{complete}
if it contains a complaint for each error in $D_n$. In that case,
$D_n'=D_n^*$. In our work, we do not assume that the complaint set is
complete, but, as is more common in practice, we only know a subset of
the errors (incomplete complaint set). Further, we focus our analysis
on \emph{valid} complaints; we briefly discuss dealing with invalid
complaints (complaints identifying a correct value as an error) in
Section~\ref{sec:noise}, but these techniques are beyond the scope of this paper.

\smallskip
\noindent
\emph{Log repair ($\mathcal{Q}^*$):}
The goal of our framework is to derive a diagnosis as a log repair
$\mathcal{Q}^*=\{q_1^*,\dots, q_n^*\}$, such that
$\mathcal{Q}^*(D_0)=D_n^*$. In this work, we focus on errors produced
by incorrect parameters in queries, so our repairs focus on altering
query constants rather than query structure. Therefore, for each query
$q_i^*\in\mathcal{Q}^*$, $q_i^*$ has the same structure as $q_i$
(e.g., the same number of predicates and the same variables in the \texttt{WHERE} clause), 
but possibly different parameters. For example, a good log repair for the
example of Figure~\ref{fig:example} is
$\mathcal{Q}^*=\{q_1^*,q_2,q_3\}$, where $q_1^*$=\texttt{UPDATE Taxes
SET owed=income*0.3 WHERE income >= 87500}.


\subsubsection*{Problem definition}

We now formalize the problem definition for diagnosing data
errors using query logs. A diagnosis is a log repair
$\mathcal{Q}^*$ that resolves all complaints in the set $\mathcal{C}$
and leads to a correct database state $D_n^*$.

\begin{definition}[Optimal diagnosis]\label{def:problem}
    Given database states $D_0$ and $D_n$, a query log $\mathcal{Q}$ such that $\mathcal{Q}(D_0)=D_n$, a set of complaints $\mathcal{C}$ on $D_n$,  and a distance function $d$, the optimal diagnosis is a log repair $\mathcal{Q}^*$, such that:
    \begin{itemize}[itemsep=0pt, parsep=0pt]
        \item $\mathcal{Q}^*(D_0)=D_n^*$, where $D_n^*$ has no errors
        \item $d(\mathcal{Q}, \mathcal{Q}^*)$ is minimized
    \end{itemize}
\end{definition}

More informally, we seek the minimum changes to the log $\mathcal{Q}$
that would result in a clean database state $D_n^*$. Obviously, a
challenge is that $D_n^*$ is unknown, unless we know that the
complaint set is complete. 

In Section~\ref{sec:sol}, we describe our basic method, 
which
uses a constraint programming formulation that expresses this
diagnosis problem as a mixed integer linear program (MILP). 
% We justify
% using this constraint formulation as opposed to methods, such as
%classification, that can analyze one query at a time in Section~\ref{sec:heuristic}. We show that the latter, heuristic
% approach is flawed, and one needs to encode the constraints in the entire log.
In Section~\ref{sec:opt}, we present several optimization
techniques that extend the basic method, allowing \sys to 
(1)~handle
cases of incomplete information (incomplete complaint set), and
(2)~scale to large data and log sizes. Specifically, the proposed \sys algorithm with incremental
repair and tuple slicing optimization (Section~\ref{sec:incremental}), can
handle $10\times$ compared to the basic MILP approach. 

Although we emphasis on solving the query logs with multiple queries, 
an alternative approach, which leveraging on classification tools and linear system equations for 
fixing the WHERE/SET clauses, applies for simpler scenario where the query log only contains one query. 
In Appendix~\ref{sec:heuristic}, we provide detailed description and experimental comparison 
for this alternative approach and demonstrate that it is not compatible with the \sys even in the simple 
single query scenario. 

% \begin{figure}[t]
% \centering
% \includegraphics[width = 0.75\columnwidth]{figures/probtransform}
% \caption{Graphical depiction of the diagnosis problem in our \sys framework.  $D_0$, $D_n$, $\mathcal{Q}$, and $\mathcal{C}$ are given, and \sys uses them to derive the log repair $\mathcal{Q}^*$.
% \alex{not sure if this figure is actually useful.}}
% \label{f:probtransform} 
% \end{figure}


% \deprecate{
% \subsection{Naive Formulation}
% 
% The most general version of the problem
% (depicted in Figure~\ref{f:probtransform}) is to find a sequence of
% transformations $T$ that insert, delete, and/or modify queries in $Q_{seq}$ 
% such that the resulting sequence, $Q'_{seq} = T(Q_{seq})$, resolves the user's complaint set. 
% 
% However this problem is ill-defined because there exist an unbounded set of transformations that
% can resolve the user's complaint set.  A naive solution is to append to the query log a statement
% that deletes all the records in the database, followed by a query that insert all of the correct records.
% Unfortunately this naive solution does not help explain the complaints in any way!
% 
% \subsection{Constraints}
% 
% For this reason, we constrain the set of possible transformations $\mathcal{T}$ to the following:
% 
% \begin{itemize}
% \item delete query
% \item modify insert statement constants
% \item modify constants in WHERE clause
% \end{itemize}
% 
% Our transformations don't include adding new queries, synthesizing arbitrary queries, or modifying the
% number of clauses in a WHERE condition.  We apply these restrictions because we believe it is more likely
% for the user to mis-type a constant value as opposed to having an error in the query structure.
% 
% Futhermore we define a distance metric between two query logs in order to evaluate
% the qulatiy of a transformation.
% \xxx{define $\mathcal{T}$ here.}
% 
% 
% 
% \subsection{Problem Statements}
% 
% In this paper, we present three variants of this problem.
% 
% \begin{problem}[Prob-Complete]\label{prob:complete}
% Given $C = P_{D_n, D^*_n}$, $Q_{seq}$, and the sequence of database states $D_0,\ldots,D_n$, 
% identify a sequence of transformations $T$ such that:
% \begin{itemize}
% \item $T(Q_{seq})(D_0) = C(D_n)$
% \item $|T| = 1$
% \item $T$ metric is minimized
% \end{itemize}
% \end{problem}
% 
% This variation of the problme relaxes the constraint that the complaint set must be complete, and allows
% for both false positives as well as false negatives.  The goal is the same, however the constraints are relaxed:
% 
% \begin{problem}[Prob-Incomplete]\label{prob:incomplete}
% Given $C$ where $acc_C < 1$, $Q_{seq}$, and the sequence of database states $D_0,\ldots,D_n$, 
% identify a sequence of transformations $T$ such that:
% \begin{itemize}
% \item $T(Q_{seq})(D_0) = D^*_n$
% \item $T$ metric is minimized.
% \item $|T| = 1$
% \end{itemize}
% \end{problem}
% 
% Finally, we extend the problem to allow transformations with one or more operations.
% 
% \begin{problem}[Prob-MultiQ]\label{prob:multi}
% Given $C$ where $acc_C < 1$, $Q_{seq}$, and the sequence of database states $D_0,\ldots,D_n$, 
% identify a sequence of transformations $T$ such that:
% \begin{itemize}
% \item $T(Q_{seq})(D_0) = D^*_n$
% \item $T$ metric is minimized.
% \end{itemize}
% \end{problem}
% 
% 
% 
% 
% \subsection{A Naive Approach}
% 
% \begin{itemize}
% \item roll back complaints to penultimate state using algebraic expressions 
% \item perturb each expression in query until the query result matches correct state
% \item if an expression cannot be found, iterate
% \end{itemize}
% 
% 
% Not clear how to roll back complaints
% 
% Ways to perturb query expressions is unbounded
% 
% }
\section{Demonstration Outline}
\label{sec:demo}

The objective of this demonstration is to show how \sys can quickly and accurately detect 
and propose fixes to errors in a query log, and compare its results to alternatives
that use existing techniques.


\begin{figure}[t]
\centering
  \includegraphics[width = .75\columnwidth]{figures/demo1_exp2}
  \caption{Users select and introduce errors to benchmark query workloads.}
  \label{f:demo1} 
\end{figure}

Figures~\ref{f:demo1} and~\ref{f:demo2} show screenshots of the initial and results pages.
Each step is annotated with a circled number, which we detail below.

\noindent {\bf Step 1 (Select Dataset):} Participants may first choose from a dropdown menu containing
a number of transaction workload generators from the benchmarks in OLTPBench~\cite{oltpbench}.
Since most transactional benchmarks focus on point update queries, we additionally include a 
synthetic workload generator that includes range updates, as well as insert and delete queries.
The text box on the right side allows users to additionally specify the number of queries to generate
in the workload.  

\noindent {\bf Step 2 (Corrupt the Query Log):} 
Once the workload generator is specified, the Query Log component of the interface
renders a scrollable list containing all of the queries.  Users can either let the system to 
inject errors randomly by clicking the ``Random Error'' button or manually add errors. 
To introduce errors, the interface allows users to select any editable query in the log and shows the 
selected query in an editable popup so that users can edit the queries.  
For example, in the figure, the user has edited query $Q_2$ and reduced the threshold from 
$income < 10,000$ to $income < 100$.
%For example, in the figure, the user has edited query $Q_2$ by changing the \texttt{SET}
%clause from $income = 1.1 * income$ to $income = 1.5 * income$. 

\noindent {\bf Step 3 (Form a Complaint Set):} 
The modified query cause the state of the database at the end of the workload
to differ from the result of the original workload.  The candidate complaints table lists the tuples
that are different and highlights the attribute values in those tuples as red text.  For instance,
$t_1.tax$, $t_1.income$ and $t_1.pay$ are all errors introduced by the modified query. 
Users can select individual attribute values or entire tuples to add to the complaint set that is used as input to the \sys algorithms.  
When she is satisfied, the user clicks \texttt{Run QFix} to execute the \sys and alternative algorithms.


\begin{figure}[t]
\centering
  \includegraphics[width = .85\columnwidth]{figures/demo2_exp2}
  \caption{Comparisons between proposed fixes.}
  \label{f:demo2} 
\end{figure}

\noindent {\bf Step 4 (View Log Repairs):}  The result page lists the original query ID and text at the top.  The ID 
is important because some proposed fixes may identify an incorrect query.  Below the original query,
the interface shows each of the proposed fixes as columns.  For example, Figure~\ref{f:demo2} shows 
that both the \sys and alternative fixes identified the correct query $Q_2$, however \sys only took $0.2$ seconds
to run, and correctly fixed $Q_2$, whereas the alternative took $10$ sec, 
incorrectly selected $Q_5$, and proposed an incorrect fix.

\noindent {\bf Step 5 (Validate Repairs):} The bottom tables show the effects of the fixes on the complaints from
step 3.  Correctly fixed attribute values are highlighted in blue, unfixed errors are shown as red text, while incorrectly
fixed values are highlighted with a red background.  Finally, it is possible for proposed fixes to 
{\it introduce} new errors, which are shown as entire rows that are highlighted with a red background.








%%!TEX root = ../main.tex

\section{Related Work}
\label{s:related}

\sys tackles the problem of diagnosis and repair in relational query
histories (query logs). This is in contrast to traditional data
cleaning~\cite{rahm00} which focuses on identifying and correcting
data ``in-place.'' \sys does not aim to correct errors in the data
directly, but rather to find the underlying reason for reported errors
in the queries that operated on the data.



Scorpion explanation uses data to synthesize predicates that explain.

Why not work.

View construction and query by example.

Take a look at temporal databases, the CIDR/arxiv paper has a few good starting points. Aditya also dug these out a while back


The way we return results is more like online aggregation or anytime algorithms -- 
as you run longer, the suggested fixes improve because we examine more of the query log.

\url{http://citeseerx.ist.psu.edu/viewdoc/download?doi=10.1.1.49.3765&rep=rep1&type=pdf}

\url{http://people.cs.aau.dk/~csj/Thesis/pdf/chapter26.pdf}


\balance


{
% If you want to use smaller typesetting for the reference list,
% uncomment the following line:
%\small
%\bibliographystyle{acm-sigchi}
\bibliographystyle{abbrv}
\bibliography{main}
}

%\techreport{%!TEX root=../main.tex

\appendix

\section{Heuristic Approaches}
\label{sec:heuristic}

In this
section, we examine alternative, simpler models that process a single
query at a time, and demonstrate why they are insufficient.

\noindent
\textbf{WHERE repairs through classification:}
The \texttt{WHERE} clause of an update query is equivalent to a
rule-based binary classifier that splits tuples into two groups:
(1)~tuples that satisfy the conditions in the \texttt{WHERE} clause
and (2)~tuples that do not. Thus, by training a classifier,
such as decision trees \cite{quinlan1987} to learn
the correct classification rules rules for the \texttt{WHERE} clause.

\noindent
\textbf{SET repairs:}
This alternative approach constructs 
a simple linear system of equations to solve the parameters in the \texttt{SET}
when errors persist after fixing the \texttt{WHERE} clause:
For each expression in the \texttt{SET} clause we create a
linear equation, using unknown variables to represent any parameters
in the \texttt{SET} expression. 
  \begin{figure}[h]
  \centering
    \includegraphics[width = .6\columnwidth]{figures/heuristicacc}
    \vspace*{-.1in}
    \caption{Heuristic Approach vs. \sys on Single-Query. }
    \label{f:heuristic_acc} 
  \end{figure}
  \vspace*{-0.1in}
  
The na\"ive approach that we just described is heuristic in nature. It
is simple and fast, but it can only process a single incorrect query. As shown in
Figure~\ref{f:heuristic_acc}, the F-1 score of na\"ive heuristic approach is less 
than 0.6 while \sys maintain high accuracy in solving single query problem with
above 0.9 F-1 score across all database sizes. 

  
\iffalse
\xlw{CUT FROM THIS POINT}
\section{Effect of Index of Corrupted Query}
\label{app:qidx}

A key parameter for our experiments is the location of the corrupted query ($idx$).  
\alex{Have we discussed anywhere yet that we focus on single errors?}
This parameter determines the number of queries \sys must consider when searching for a fix,
and affects the size of the complaint set.  
\alex{It won't be clear to the reader how this relates to the size of the complaint set.}
Both of these characteristics directly impact \sys's 
runtime performance. For this reason, it is undesirable to randomly pick and corrupt queries
throughout the query log, as the performance and accuracy results may not be comparable. 
To better understand the relationship between $idx$ and the size of the complaint set, we ran
simulations using a database with $20$ attributes, and a query log of size $1000$ containing
either all $set = const$ or $set = rel$ \texttt{UPDATE} queries.
We varied  $idx$ uniformly throughout the query log, and additionally varied
the skew $s$ and range $r$ parameters to study how they affect the size of the complaint sets.


  \begin{figure}[h]
  \centering
  \includegraphics[width = 3.5in]{figures/qidxsimulation/qidx_v_ncomplaints_20attrs_const}
  \caption{Query index vs complaint set size for $set = const$.}
  \label{f:qidx_v_ncomplaints_const} 
  \end{figure}


Figure~\ref{f:qidx_v_ncomplaints_const} plots a representative set of parameters.  We plot one point
for each corrupted query index that results in a complaint set with at least one complaint. 
These results highlight several interesting trends.  When queries do not overlap ($r = 1$, leftmost column),
the size of the complaint sets are relatively small, and their frequency is constant across the possible query indices.
However as the possibility of overlap increases (e.g., $r$ increases), more recent queries are more likely to result in
very large complaint sets (at times the size of the database).   
This effect is a symptom of the fact that queries with large ranges will set groups of tuples to the same value,
and over time, skew the distribution of tuple values to a small number of possible values.
Thus, more recent corruptions that affect a large cluster of similar tuples will result in a large complaint set.
We find that increasing the skew parameter also exacerbates this effect.  
In addition, high skew increases the likelihood that queries will share the same \texttt{WHERE} and \texttt{SET} clause 
attributes as a corrupted query, thus overwriting the error introduced by the corrupted query.  
This is why the frequency of non-empty complaint sets decreases significantly as $s$ increases.


\begin{figure}[h]
\centering
\includegraphics[width = 3.5in]{figures/qidxsimulation/qidx_v_ncomplaints_20attrs_rel}
\caption{Query index vs complaint set size for $set = rel$.}
\label{f:qidx_v_ncomplaints_rel} 
\end{figure}

In contrast to $set=const$ queries, Figure~\ref{f:qidx_v_ncomplaints_rel} executes the 
same experiment using $set=rel$ queries.  In this setting, we find that the trend is
reversed, and older corruptions tend to result in larger complaint sets.  This is because,
subsequent \texttt{UPDATE} queries increment or decrement the attribute value, rather than
overwriting it with a constant value.  The clustering of data values due to query overlap
then increases the number of other tuples affected.


\ewu{summarize findings and implications to experiments here.}
not all corruptions result in complaint sets.
In constant SET clause workloads, larger complaints sets are more likely to
result from more recent corrupted queries -- particularly if the queries are range updates or
the updated attributes are skewed.
For this reason, our experiments corrupt the query log at six positions 
$idx \in \{0, 25, 50, 100, 200, 250\}$ , relative 
to the most recent query (e.g., the most recent query, the $25^th$ most recent query, and so on).

% \begin{figure}[h]
% \centering
% \includegraphics[width = 3.5in]{figures/qidxsimulation/numinrange}
% \caption{.}
% \label{f:numinrange} 
% \end{figure}


As we observed from Figure~\ref{f:multiquery}, \milpall maintains high accuracy when errors
happen more recent, however it does not scale when the error locate further from the most
recent query. \milptuple scales better than \milpall, but ignoring tuples not 
in the complaint set apparently hurts the precision. \milptuplestopearly run times faster
than \milpall and \milptuple, however the aggressive strategy greatly reduce the 
precision. In the end, \milpadvtuple significantly improves the precision with very limited
time cost compare to \milptuple. \\
Based on these observations, we only include the performance of \milpadvtuple and \milpadvall
in the rest of the experiments. 

%!TEX root = ../main.tex

\subsection{A plausible (but bad) alternative}
\label{sec:dt}

% \sys analyzes query logs and complaints by producing a mathematical
% formulation of the constraints that need to be satisfied. The
% constraint problem can then be evaluated by dedicated external tools.

The MIP models generated by \sys can grow large as the sizes of the
data and the log increase. However, modeling all present constraints
from the beginning to the end of the log is necessary; in this
section, we examine alternative, simpler models that process a single
query at a time, and demonstrate why they are insufficient.

\smallskip
\noindent
\textbf{WHERE repairs through classification:}
The \texttt{WHERE} clause of an update query is equivalent to a
rule-based binary classifier that splits tuples into two groups:
(1)~tuples that satisfy the conditions in the \texttt{WHERE} clause
and (2)~tuples that do not. A mistake in a query predicate can then
result in misclassification: some tuples get classified into the wrong
group, which in turn translates to errors in the data. Therefore,
repairing the mistake corresponds to repairing the imprecise
classification. This works as follows: For an incorrect query $q$, let
$D_0$ be the database state before $q$, and $D_1^*$ the \emph{correct}
database state that should result after $q$.
We use each tuple $t \in D_0$ as an element in the input training data
for the classifier where the values (of each attribute) of $t$ define
the feature vector and the label for $t$:
	\[
    label(t)= 
    \begin{cases}
    true ,& D_0.t \neq D_1^*.t\\
    false,              & \text{otherwise}
    \end{cases}
\]
We then train a classifier, such as decision trees \cite{???} to learn
the correct classification rules rules for the \texttt{WHERE} clause.


\smallskip
\noindent
\textbf{SET repairs:}
After repairing the \texttt{WHERE} clause through learning a
rule-based classifier, some complaints may still persist. This
indicates a possible error in the \texttt{SET} clause. The errors can
be modeled and solved by constructing a simple linear system of
equations: For each expression in the \texttt{SET} clause we create a
linear equation, using unknown variables to represent any parameters
in the \texttt{SET} expression. Solving for these variables then
provides a repair for the \texttt{SET} expression.


\smallskip
\noindent
\textbf{Why it does not work:}
The na\"ive approach that we just described is heuristic in nature. It
is simple and fast, but it can only process a single incorrect query.
This results in several shortcomings that make it insufficient in
practice:
\begin{itemize}[itemsep=1pt, leftmargin=5mm]
    
\item In principle, one could attempt to apply this technique to the
entire log one-query-at-a-time. However, this is not possible in
practice: to learn a classifier on the \texttt{WHERE} clause of query
$q_i$, one needs to know the correct classification output, which
corresponds to $D_i^*$. Unfortunately, even with a complete complaint
set, which can derive the correct database $D_n^*$, there is no
obvious way to ``rollback'' this state to derive $D_i^*$.

\item The classifier may derive a clause that is structurally very
different from the original one (different attributes or number of
conditions). This is problematic in general, as it corresponds to a
larger-scale mistake in the query, which is a less likely scenario.

\item Classifiers try to avoid overfitting, which is problematic for
queries with high selectivity (e.g., single-tuple updates), as the
classifier is unlikely to generate any rules.

\end{itemize}


Therefore, while examining one query at a time superficially appears
to be a reasonable and efficient alternative, the reality is that one
has to model all constraints and transformations through the entire
log history. In the following section, we propose several
optimizations to our initial approach that make scaling to large data
and log sizes feasible. 
% \red{Add graph comparing naive and d-trees here.}


\iffalse
In this
section, we examine alternative, simpler models that process a single
query at a time, and demonstrate why they are insufficient.

\noindent
\textbf{WHERE repairs through classification:}
The \texttt{WHERE} clause of an update query is equivalent to a
rule-based binary classifier that splits tuples into two groups:
(1)~tuples that satisfy the conditions in the \texttt{WHERE} clause
and (2)~tuples that do not. Thus, by training a classifier,
such as decision trees \cite{quinlan1987} to learn
the correct classification rules rules for the \texttt{WHERE} clause.

\noindent
\textbf{SET repairs:}
This alternative approach constructs 
a simple linear system of equations to solve the parameters in the \texttt{SET}
when errors persist after fixing the \texttt{WHERE} clause:
For each expression in the \texttt{SET} clause we create a
linear equation, using unknown variables to represent any parameters
in the \texttt{SET} expression. 
  \begin{figure}[h]
  \centering
    \includegraphics[width = .6\columnwidth]{figures/heuristicacc}
    \vspace*{-.1in}
    \caption{Heuristic Approach vs. \sys on Single-Query. }
    \label{f:heuristic_acc} 
  \end{figure}
  \vspace*{-0.1in}
  
The na\"ive approach that we just described is heuristic in nature. It
is simple and fast, but it can only process a single incorrect query. As shown in
Figure~\ref{f:heuristic_acc}, the F-1 score of na\"ive heuristic approach is less 
than 0.6 while \sys maintain high accuracy in solving single query problem with
above 0.9 F-1 score across all database sizes. 
\fi

\fi
}






\end{document}
