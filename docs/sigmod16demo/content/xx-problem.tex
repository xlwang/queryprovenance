%!TEX root = ../main.tex

\section{Modeling abstractions}
\label{sec:abstractions}

In this section, we introduce a running example inspired from the use-case of
Example~\ref{ex:taxes}, and sketch the problem description as well as our
MILP-based solution.



\begin{figure*}[t]
    \begin{minipage}[t]{0.28\textwidth}
         \vspace{0pt} 
         \centering
        \begin{tabular}{llll}
            \multicolumn{4}{l}{\emph{Taxes}: $D_0$}\\
            \toprule
            \textbf{ID}  & \textbf{rate}  & \textbf{income}    & \textbf{owed}\\
            \midrule
            $t_1$   & 10    & \$9500    & \$950\\
            $t_2$   & 25    & \$90000   & \$22500\\
            $t_3$   & 25    & \$86000   & \$21500\\
            $t_4$   & 25    & \$86500   & \$21625\\
            \bottomrule
            \\
        \end{tabular}
    \end{minipage}
    \begin{minipage}[t]{0.43\textwidth}
         \vspace{0pt} 
         \centering
        \begin{tabular}{|p{1ex}l|}
            \multicolumn{2}{l}{\emph{Query log}: $\mathcal{Q}$}\\
            \hline
            $q_1$: & \texttt{\small UPDATE Taxes SET rate = 30}\\
                   & \texttt{\small WHERE income >= \color{red}{85700}} \\
            
            $q_2$: & \texttt{\small UPDATE Taxes SET owed = income*rate/100}\\
            
            $q_3$: & \texttt{\small INSERT INTO Taxes}\\ 
                   & \texttt{\small VALUES (25, 85800, 21450)}\\
            \hline
        \end{tabular}
    \end{minipage}
    \begin{minipage}[t]{0.28\textwidth}
         \vspace{0pt} 
         \centering
        \begin{tabular}{llll}
            \multicolumn{4}{l}{\emph{Taxes}: $D_3$}\\
            \toprule
            \textbf{ID}  & \textbf{rate}  & \textbf{income}    & \textbf{owed}\\
            \midrule
            $t_1$   & 10    & \$9500    & \$950\\
            $t_2$   & 30    & \$90000   & \$27000\\
            \rowcolor{mid-gray}
            $t_3$   & \color{red}{30}    & \$86000   & \color{red}{\$25800}\\
            \rowcolor{mid-gray}
            $t_4$	 & \color{red}{30}	& \$86500	  & \color{red}{\$25950}\\
            $t_5$	 & 25	& \$87000	  & \$21750\\
            \bottomrule
        \end{tabular}
    \end{minipage}

    \caption{A recent change in tax rate brackets calls for a tax rate of 30\% for those with income above \$87500.  The accounting department issues query $q_1$ to implement the new policy, but the predicate of the WHERE clause condition transposed two digits of the income value.  As a result, the tax rates of $t_3$ and $t_4$ were increased incorrectly.  Query $q_2$ that calculates the amount owed based on the corresponding tax rate and income propagates the error to additional fields.  The mistake is further obscured by query $q_3$, which inserts a tuple with similar income and the correct tax rate.
    }
    \label{fig:example}
\end{figure*}

\begin{example}\label{ex:taxes2}
    
Figure~\ref{fig:example} demonstrates an example tax bracket adjustment in the
spirit of Example~\ref{ex:taxes}. The adjustment sets the tax rate to 30\% for
income levels above \$87,500, and is implemented by query $q_1$. A digit
transposition mistake in the query, results in an incorrect tax rate for tuples
$t_3$ and $t_4$. Query $q_2$ that calculates the amount owed based on the corresponding
tax rate and income propagates the error to other fields. The mistake is
further obscured by query $q_3$, which inserts a tuple with slightly higher
income than $t_3$ and $t_4$ and the correct (lower) tax rate.

\end{example}
% 
While traditional data cleaning techniques seek to identify and correct the
erroneous values in the table \emph{Taxes} directly, our goal is to diagnose
the problem, and understand the reasons for these errors. In this case, the
reason for the data errors is the incorrect predicate value in query $q_1$.

In this paper, we assume that we know \emph{some} errors in the dataset, and
that these errors were caused by erroneous updates. The errors may be
obtained in different ways: traditional data cleaning tools may identify
discrepancies in the data (e.g., a tuple with lower income has higher tax
rate), or errors can be reported directly from users (e.g., customers
reporting discrepancies to customer service). \emph{Our goal is not to correct
the errors directly in the data, but to analyze them as a ``symptom'' and provide a
diagnosis.} The diagnosis can produce a targeted treatment: knowing how the
errors were introduced guides the proper way to trace and resolve them.



\subsection{Problem Outline}

We assume a query log $Q$ containing a sequence of update, insert and delete queries 
$q_1, \ldots q_n$  that have been executed on an initial database state $D_0$.  
For simplicity, we assume that the database is a single relation with attributes 
$A_i,\ldots,A_m$, though this restriction is not a requirement for \sys.
Let $D_i = q_i(\ldots q_1(D_0))$ be the the database state output after executing queries $q_1$ though $q_i$ on $D_0$.
Thus, the final database state is simply the application of the query log to
the initial state: $D_n = Q(D_0)$.
We assume that \texttt{UPDATE} and \texttt{DELETE} queries consist of conjunctive range clauses
of the form $A_j\ op\ C$, where $op \in \{=, >, <, \le, \ge, \ne\}$ and $C$ is a constant value,
and that \texttt{SET} clauses of the form $A_j\ =\ C$.

Queries in $Q$ are possibly erroneous and introduce errors in the data.  We assume there is 
a true sequence of queries $Q^* = \{q^*_1,\ldots,q^*_n\}$ that generate a true sequence of database states
$\{D_0, D^*_1,\ldots,D^*_n\}$.  The true query log and database states are unknown, and our goal is to 
find and correct errors in $Q$ to retrieve the correct database state $D^*_n$>

To do so, \sys takes as input a set of identified or user-reported
data errors, or a {\it complaint set} denoted as $\mathcal{C}$. 
A complaint $c$ corresponds to a tuple in $D^*_n$ along with its true attribute value assignments. 
For example, Figure~\ref{} shows several complaints. $c_1 = (t_3, \{rate=25, owed=21500\})$ and 
$c_2 = (t_4, \{rate=25, owed=21625\})$ show that tuples $t_3$ and $t_4$ 
have incorrect values for attributes {\it rate} and {\it owed}.
A complaint can also model addition or removal of tuples. 
For example, $c = (\bot, t^*)$ means that $t^*$ should be added to the database, 
while $c = (t_i, \bot)$ denotes that $t_i$ should be removed.

Our goal is to derive a diagnosis as a log repair
$Q'=\{q_1',\dots, q_n'\}$, such that
$Q'(D_0)=D_n^* $. In this work, we focus on errors produced
by incorrect parameters in queries, so our repairs focus on altering
query constants rather than query structure. Therefore, each query
$q_i'\in Q'$ has the same structure as $q_i$
(e.g., the same number of predicates and the same variables in the \texttt{WHERE} clause), 
but possibly different parameters. For example, a good log repair for the
example of Figure~\ref{fig:example} is
$Q'=\{q_1',q_2,q_3\}$, where $q_1'$=\texttt{UPDATE Taxes
SET rate = 30 WHERE income >= 87500}.


\subsubsection*{Problem definition}

We now formalize the problem definition for diagnosing data
errors using query logs. A diagnosis is a log repair
$Q'$ that resolves all complaints in the set $\mathcal{C}$
and leads to a correct database state $D_n^*$.

\begin{definition}[Optimal diagnosis]\label{def:problem}
    Given database states $D_0$ and $D_n$, a query log $Q$ 
    such that $Q(D_0)=D_n$, a set of complaints $\mathcal{C}$ on $D_n$,  
    and a distance function $d$, the optimal diagnosis is a 
    log repair $Q'$, such that:
    \begin{itemize}[itemsep=0pt, parsep=0pt]
        \item $Q'(D_0)=D_n^*$, where $D_n^*$ has no errors
        \item $d(\mathcal{Q}, \mathcal{Q}^*)$ is minimized
    \end{itemize}
\end{definition}

More informally, we seek the minimum changes to the log $\mathcal{Q}$
that would result in a clean database state $D_n^*$. Obviously, a
challenge is that $D_n^*$ is unknown, unless we know that the
complaint set is complete. 


\subsection{Solution Sketch}

Our general strategy is to translate the starting
and ending database states, $D_0$ and $D_n$, the query log $Q$, and
complaint set $\mathcal{C}$ into a mixed-integer linear program (MILP)
and solve the resulting program using a generic solver.  
Briefly, a linear program is a minimization problem consisting of
a set of constraints in the form of linear equations, along with
an objective function that is minimized; an MILP is a linear program
where only a subset of the undetermined variables are required to
be integers, while the rest are real valued.  For example, in the following
example, $b$ and $z$ are undetermined integer and real valued variables, respectively,
whereas $x$ is determined to be $1$.  The objective function is $b+z$, 
resulting in the optimal solution $b = z = 1$.

\[
\begin{array}{l l l}
minimize      & b + z\\
\mbox{s.t.}   &x &= b * z \\
              &x &= 1\\
              &b &\in \{0, 1, 2, 3\}\\
              &z &\le 6\\
              &z &\ge 0\\
\end{array}
\]


To achieve this translation,
we model each query as a linear equation that computes the output
tuple values from the inputs and to transform the equation into a
set of of linear constraints. In addition, the constant values in
the queries are parameterized into a set of undetermined variables,
while the database state is encoded as constraints on the initial
and final tuple values. Finally, the undetermined variables are
used to construct an objective function that prefers value assignments
that minimize both the amount that the queries change and the number
of non-complaint tuples that are affected.

Due to space constraints, we will walk through an example of how the query
\texttt{UPDATE Taxes SET rate = 30 WHERE income > 85700}, when applied to 
$t_3^{old}$ to produce $t_3^{new}$, is translated into a set of constraints
as detailed below.

First, we can rewrite the query as a conditional statement:

\[
\begin{array}{l l l}
t_3^{new}.rate &= 30              &\mathrm{if}\ t_3^{old}.income > 85700 \\
               &= t_3^{old}.rate  &\mathrm{otherwise} \\
\end{array}
\]

By instantiating a variable $p$ to stand for the conditional predicate,
the statement can be rewritten as:

\[
\begin{array}{l l l}
t_3^{new}.rate &= 30              &\mathrm{if}\ p \\
               &= t_3^{old}.rate  &\mathrm{if}\ 1-p\\
p              &= t_3^{old}.income > 85700
\end{array}
\]

This is equivalent to the following linearized form.
In addition, we add constraints on the starting and ending value for the $rate$ attribute:
If $t_3^{new}.rate$ was specified in $\mathcal{C}$, then the provided value
is used instead of the value in the database:

\[
\begin{array}{l l l}
t_3^{new}.rate &= 30\times p + t_3^{old}.rate\times (1-p)\\
p              &= t_3^{old}.income > 85700\\
t_3^{old}.rate &= 30 \\
t_3^{new}.rate &= 25
\end{array}
\]

In the above formulation, all variables are determined and is trivially
solvable.  Instead, we replace
the constaints $30$ and $85700$ with undetermined variables $v_1$ and $v_2$,
so that solving the constraints will reassign those query constants 
to new values that result in the desired value for $t_3^{new}.rate$.  

\[
\begin{array}{l l l}
t_3^{new}.rate &= v_1\times p + t_3^{old}.rate\times (1-p)\\
p              &= t_3^{old}.income > v_2\\
t_3^{old}.rate &= 30 \\
t_3^{new}.rate &= 25 \\
v_1            &\in [minval, maxval]\\
v_2            &\in [minval, maxval]\\
\end{array}
\]

Extending this process to all tuples and all queries in the log 
describes the naive encoding procedure that solves the
\textsc{Optimal Diagnosis} problem.
However, the size of the resulting constraint problem increases 
at a rate of $O(|D|\times |Q|\times \mathrm{\#attributes})$, rendering
it infeasible for all but the smallest databases and query logs.


\sys uses four additional optimizations not presented in this paper 
to scale to large query log and database sizes.  The first three are
called {\it Slicing} optimizations that reduce each of the components in the problem size:
{\it Query-slicing}; {\it Attribute-slicing}; {\it Tuple-slicing}. 
MILP solvers typically (though not guaranteed) take a much longer time as the size of the MILP problem increases, thus
each of the slicing techniques server to reduce the problem and speed up the solver time.

The final optimization serves to reduce the number of undetermined variables that the MILP solver must
provide a solution for.  The cost of the solver, in our experiments, increases exponentially with the number of 
undetermined variables.  To this end, \sys uses an incremental algorithm that tries to fix the queries in the query
log one at a time.

With these optimizations, for common transaction benchmarks, \sys can propose a solution within several seconds
for thousands of queries and tuples.


















% 
\begin{figure}[t]
\centering
{\small
\begin{tabular}{ll}
    \toprule
    \textbf{Notation} & \textbf{Description}\\
    \midrule
    $\mathcal{Q}$& The sequence of executed update queries (log)\\ 
             & $\mathcal{Q}=\{q_1, \dots, q_n\}$ \\
    $D_0$    & Initial database state at beginning of log\\
    $D_n$    & End database state (current) $D_n=\mathcal{Q}(D_0)$\\
    $D_i$    & Database state after query $q_i$: $D_i=q_i(\dots q_1(D_0))$\\
    $c: t\mapsto t^*$ & Complaint: $\mathcal{T}_c(D) = (D_n\setminus\{t\})\cup\{t^*\}$\\
    $\mathcal{C}$ & Complaint set $\mathcal{C}=\{c_1,\dots,c_k\}$\\
    $\mu_q(t)$  & Modifier function of $q$ (e.g., \texttt{SET} clause)\\
    $\sigma_q(t)$   & Conditional function of $q$ (e.g., \texttt{WHERE} clause)\\
    $t_{new}$   & Tuple values introduced in an \texttt{INSERT} query\\
    $\mathcal{Q}^*$& Log repair\\
    $d(\mathcal{Q}, \mathcal{Q}^*)$ & Distance functions between two query logs\\
    \bottomrule
\end{tabular}
}
\vspace{-2mm}
\caption{Summary of notations used in the paper. }
\label{tbl:notation}
\end{figure}

\if{0}
\subsection{Error modeling}
\label{sec:model}

In our setting, the diagnoses are associated with errors in the queries that
operated on the data. In Example~\ref{ex:taxes2}, the errors in the dataset
are due to the digit transposition mistake in the WHERE clause predicate of
query $q_1$. Our goal is to infer the errors in a log of queries
automatically, given a set of incorrect values in the data. We proceed to
describe our modeling abstractions for data, queries, and errors, and how we
use them to define the diagnosis problem.

\subsubsection*{Data and query models}
\label{sec:models}

\noindent
\emph{Query log ($\mathcal{Q}$):}
We define a query log as an ordered sequence of \texttt{UPDATE}, \texttt{INSERT}, and
\texttt{DELETE} queries $\mathcal{Q}=\{q_1,\dots,q_n\}$, that have
operated on a database $D$. In the rest of the paper, we use the term
\emph{update queries}, or just \emph{queries}, to refer to any of the queries in $\mathcal(Q)$,
including insertion and deletion queries.

\smallskip
\noindent
\emph{Query ($q_i$):} We model each query as a function over a
database $D$, resulting in a new database $D'$. For \texttt{INSERT}
queries, $D'=q(D)=D\cup\{t_{new}\}$.
We model \texttt{UPDATE} and \texttt{DELETE} queries as follows:  
\begin{align*}
    D'=q(D)= &\{\mu_{q}(t)\;|\;t\in D, \sigma_{q}(t)\}%\\
    % &
    \cup\{t\;|\;t\in D, \neg\sigma_{q}(t)\}%\\
    % &\cup\{t_{new}\;|\;q\in\texttt{INSERT}\}
\end{align*}
% 
In this definition, the modifier function $\mu_q(t)$ represents the query's update equations, and it transforms a tuple by either deleting it ($\mu_q(t)=\bot$) or changing the values of some of its attributes.
The conditional function $\sigma_q(t)$ is a boolean function that represents the query's condition predicates.  In the example of Figure~\ref{fig:example}:
\begin{align*}
    &\mu_{q_1}(t)=(30, t.income, t.owed)\\
    &\sigma_{q_1}(t)=(t.income\ge 85700)\\
    &\mu_{q_2}(t)=(t.rate, t.income, \frac{t.income\cdot t.rate}{100})\\
    &\sigma_{q_2}(t)=\texttt{true}
\end{align*} 
% 
% As an insertion query, $q_3$ has $\sigma_{q_3}(t)=\texttt{false}$ and $t_{new}=(25, 85800, 21450)$.
% \ewu{why does it return false?  should it only be true if input is $\bot$?}


\smallskip
\noindent
\emph{Database state ($D_i$):}
We use $D_i$ to represent the state of a database $D$ after the application of
queries $q_1$ through $q_i$ from the log $\mathcal{Q}$. $D_0$ represents the
original database state, and $D_n$ the final, or current, database state. Out
of all the states, the system only maintains $D_0$ and $D_n$. In practice,
$D_0$ can be a checkpoint: a state of the database that we assume is correct;
we cannot diagnose errors before this state. The intermediate states can be
derived by executing the log: $D_i=q_i(q_{i-1}(\dots q_1(D_0)))$. We also
write $D_n=\mathcal{Q}(D_0)$ to denote that the final database state $D_n$ can
be derived by applying the sequence of queries in the log to the original
database state $D_0$.

\smallskip
\noindent
\emph{True database state ($D_i^*$):}
Queries in $\mathcal{Q}$ are possibly erroneous, introducing errors in the
data. There exists a sequence of \emph{true} database states $\{D_0^*,
D_1^*\dots, D_n^*\}$, with $D_0^*=D_0$, representing the database states that
would have occurred if there had been no errors in the queries.
The true database states are unknown; our goal is to find and correct the errors in $\mathcal{Q}$ and retrieve the correct database state $D_n^*$.

For ease of exposition, in the remainder of the paper we assume that the
database contains a single relation with attributes $A_i,\ldots,A_m$,
but the single table is not a requirement in our framework.


\subsubsection*{Error models}

Following the terminology in Examples~\ref{ex:telco}
and~\ref{ex:taxes}, we model a set of identified or user-reported
data errors as \emph{complaints}. A complaint corresponds to a
particular tuple in the final database state $D_n^*$, and identifies
that tuple's correct value assignment. We formally define complaints
below:

\begin{definition}[Complaint]
    A complaint $c$ is a mapping between two tuples: $c: t\mapsto t^*$, such that $t$ and $t^*$ have the same schema, $t\in D_n\cup\{\bot\}$, and $t\neq t^*$. A complaint defines a
    transformation $\mathcal{T}_c$ on a database state $D$: $\mathcal{T}_c(D)
    = (D\setminus\{t\})\cup\{t^*\}$.
\end{definition}

In the example of Figure~\ref{fig:example}, two complaints are reported on the final database state $D_3$: 
$c_1: t_3\mapsto t_3^*$ and
$c_2: t_4\mapsto t_4^*$, where $t_3^*=(25,86000,21500)$
and $t_4^*=(25,86500,21625)$.  For both these cases, each complaint denotes a value correction for a tuple in $D_3$.  Complaints can also model the addition or removal of tuples: $c: \bot\mapsto t^*$ means that $t^*$ should be added to the database, whereas $c: t\mapsto \bot$
means that $t$ should be removed from the database.


\smallskip
\noindent
\emph{Complaint set ($\mathcal{C}$):}
We use $\mathcal{C}$ to denote the set of all known complaints
$\mathcal{C}=\{c_1,\dots,c_k\}$, and we call it the \emph{complaint set}.
Each complaint in $\mathcal{C}$ represents a transformation (addition,
deletion, or modification) of a tuple in $D_n$. We assume that the
complaint set is consistent, i.e., there are no two complaints that
propose different transformations to the same tuple $t\in D_n$.
Applying all these transformations to $D_n$ results in a new database
instance
$D_n'=\mathcal{T}_{c_1}(\mathcal{T}_{c_2}(\dots\mathcal{T}_{c_k}(D_n)))$.\footnote{Since
the complaint set is consistent, it is easy to see that the order of
transformations is inconsequential.} $\mathcal{C}$ is \emph{complete}
if it contains a complaint for each error in $D_n$. In that case,
$D_n'=D_n^*$. In our work, we do not assume that the complaint set is
complete, but, as is more common in practice, we only know a subset of
the errors (incomplete complaint set). Further, we focus our analysis
on \emph{valid} complaints; we briefly discuss dealing with invalid
complaints (complaints identifying a correct value as an error) in
Section~\ref{sec:noise}, but these techniques are beyond the scope of this paper.

\smallskip
\noindent
\emph{Log repair ($\mathcal{Q}^*$):}
The goal of our framework is to derive a diagnosis as a log repair
$\mathcal{Q}^*=\{q_1^*,\dots, q_n^*\}$, such that
$\mathcal{Q}^*(D_0)=D_n^*$. In this work, we focus on errors produced
by incorrect parameters in queries, so our repairs focus on altering
query constants rather than query structure. Therefore, for each query
$q_i^*\in\mathcal{Q}^*$, $q_i^*$ has the same structure as $q_i$
(e.g., the same number of predicates and the same variables in the \texttt{WHERE} clause), 
but possibly different parameters. For example, a good log repair for the
example of Figure~\ref{fig:example} is
$\mathcal{Q}^*=\{q_1^*,q_2,q_3\}$, where $q_1^*$=\texttt{UPDATE Taxes
SET rate = 30 WHERE income >= 87500}.


\subsubsection*{Problem definition}

We now formalize the problem definition for diagnosing data
errors using query logs. A diagnosis is a log repair
$\mathcal{Q}^*$ that resolves all complaints in the set $\mathcal{C}$
and leads to a correct database state $D_n^*$.

\begin{definition}[Optimal diagnosis]\label{def:problem}
    Given database states $D_0$ and $D_n$, a query log $\mathcal{Q}$ such that $\mathcal{Q}(D_0)=D_n$, a set of complaints $\mathcal{C}$ on $D_n$,  and a distance function $d$, the optimal diagnosis is a log repair $\mathcal{Q}^*$, such that:
    \begin{itemize}[itemsep=0pt, parsep=0pt]
        \item $\mathcal{Q}^*(D_0)=D_n^*$, where $D_n^*$ has no errors
        \item $d(\mathcal{Q}, \mathcal{Q}^*)$ is minimized
    \end{itemize}
\end{definition}

More informally, we seek the minimum changes to the log $\mathcal{Q}$
that would result in a clean database state $D_n^*$. Obviously, a
challenge is that $D_n^*$ is unknown, unless we know that the
complaint set is complete. 

In Section~\ref{sec:sol}, we describe our basic method \naive, which
uses a constraint programming formulation that expresses this
diagnosis problem as a mixed integer linear program (MILP). We justify
using this constraint formulation as opposed to methods, such as
classification, that can analyze one query at a time in
Section~\ref{sec:heuristic}. We show that the latter, heuristic
approach is flawed, and one needs to encode the constraints in the
entire log. In Section~\ref{sec:opt}, we present several optimization
techniques that extend our basic method, allowing \sys to (1)~handle
cases of incomplete information (incomplete complaint set), and
(2)~scale to large data and log sizes. Specifically, our incremental
repair method (\incremental, Section~\ref{sec:incremental}), can
handle $10\times$ compared to the basic MILP approach.





\fi


% \begin{figure}[t]
% \centering
% \includegraphics[width = 0.75\columnwidth]{figures/probtransform}
% \caption{Graphical depiction of the diagnosis problem in our \sys framework.  $D_0$, $D_n$, $\mathcal{Q}$, and $\mathcal{C}$ are given, and \sys uses them to derive the log repair $\mathcal{Q}^*$.
% \alex{not sure if this figure is actually useful.}}
% \label{f:probtransform} 
% \end{figure}


% \deprecate{
% \subsection{Naive Formulation}
% 
% The most general version of the problem
% (depicted in Figure~\ref{f:probtransform}) is to find a sequence of
% transformations $T$ that insert, delete, and/or modify queries in $Q_{seq}$ 
% such that the resulting sequence, $Q'_{seq} = T(Q_{seq})$, resolves the user's complaint set. 
% 
% However this problem is ill-defined because there exist an unbounded set of transformations that
% can resolve the user's complaint set.  A naive solution is to append to the query log a statement
% that deletes all the records in the database, followed by a query that insert all of the correct records.
% Unfortunately this naive solution does not help explain the complaints in any way!
% 
% \subsection{Constraints}
% 
% For this reason, we constrain the set of possible transformations $\mathcal{T}$ to the following:
% 
% \begin{itemize}
% \item delete query
% \item modify insert statement constants
% \item modify constants in WHERE clause
% \end{itemize}
% 
% Our transformations don't include adding new queries, synthesizing arbitrary queries, or modifying the
% number of clauses in a WHERE condition.  We apply these restrictions because we believe it is more likely
% for the user to mis-type a constant value as opposed to having an error in the query structure.
% 
% Futhermore we define a distance metric between two query logs in order to evaluate
% the qulatiy of a transformation.
% \xxx{define $\mathcal{T}$ here.}
% 
% 
% 
% \subsection{Problem Statements}
% 
% In this paper, we present three variants of this problem.
% 
% \begin{problem}[Prob-Complete]\label{prob:complete}
% Given $C = P_{D_n, D^*_n}$, $Q_{seq}$, and the sequence of database states $D_0,\ldots,D_n$, 
% identify a sequence of transformations $T$ such that:
% \begin{itemize}
% \item $T(Q_{seq})(D_0) = C(D_n)$
% \item $|T| = 1$
% \item $T$ metric is minimized
% \end{itemize}
% \end{problem}
% 
% This variation of the problme relaxes the constraint that the complaint set must be complete, and allows
% for both false positives as well as false negatives.  The goal is the same, however the constraints are relaxed:
% 
% \begin{problem}[Prob-Incomplete]\label{prob:incomplete}
% Given $C$ where $acc_C < 1$, $Q_{seq}$, and the sequence of database states $D_0,\ldots,D_n$, 
% identify a sequence of transformations $T$ such that:
% \begin{itemize}
% \item $T(Q_{seq})(D_0) = D^*_n$
% \item $T$ metric is minimized.
% \item $|T| = 1$
% \end{itemize}
% \end{problem}
% 
% Finally, we extend the problem to allow transformations with one or more operations.
% 
% \begin{problem}[Prob-MultiQ]\label{prob:multi}
% Given $C$ where $acc_C < 1$, $Q_{seq}$, and the sequence of database states $D_0,\ldots,D_n$, 
% identify a sequence of transformations $T$ such that:
% \begin{itemize}
% \item $T(Q_{seq})(D_0) = D^*_n$
% \item $T$ metric is minimized.
% \end{itemize}
% \end{problem}
% 
% 
% 
% 
% \subsection{A Naive Approach}
% 
% \begin{itemize}
% \item roll back complaints to penultimate state using algebraic expressions 
% \item perturb each expression in query until the query result matches correct state
% \item if an expression cannot be found, iterate
% \end{itemize}
% 
% 
% Not clear how to roll back complaints
% 
% Ways to perturb query expressions is unbounded
% 
% }