
Due to space constraints, we will walk through an example of how the query
\texttt{UPDATE Taxes SET tax = 0.3 * income WHERE income > 85700}, when applied to 
$t_3^{old}$ to produce $t_3^{new}$, is translated into a set of constraints
as detailed below.

First, we can rewrite the query as a conditional statement:
\[
t_3^{new}.tax = \begin{cases}
				  0.3*t_3^{old}.rate              & \mathrm{if}\ p  \\
                  t_3^{old}.tax  & \mathrm{1-p}
               \end{cases}
\]
\vspace*{-0.03in}
where 
\vspace*{-0.03in}
\[p = t_3^{old}.income > 85700\]

This is equivalent to the following linearized form.
In addition, we add constraints on the starting and ending value for the $rate$ attribute:
If $t_3^{new}.tax$ was specified in $\mathcal{C}$, then the provided value
is used instead of the value in the database:
\vspace*{-0.03in}
\[
\begin{array}{l l l}
t_3^{new}.tax &= (0.3*t_3^{old}.income)\times p + t_3^{old}.tax\times (1-p)\\
p &= t_3^{old}.income > 85700\\
t_3^{old}.tax &= 30 \\
t_3^{new}.tax &= 25
\end{array}
\]
\vspace*{-0.03in}
In the above formulation, all variables are determined and is trivially
solvable.  Instead, we replace
the constaints $0.3$ and $85700$ with undetermined variables $v_1$ and $v_2$,
so that solving the constraints will reassign those query constants 
to new values that result in the desired value for $t_3^{new}.rate$.  
\iffalse
\[
\begin{array}{l l l}
t_3^{new}.tax &= (v_1*t_3^{old}.income)\times p + t_3^{old}.tax\times (1-p)\\
p              &= t_3^{old}.income > v_2\\
t_3^{old}.tax 30 \\
t_3^{new}.tax &= 25 \\
v_1            &\in [minval, maxval]\\
v_2            &\in [minval, maxval]\\
\end{array}
\]
\fi