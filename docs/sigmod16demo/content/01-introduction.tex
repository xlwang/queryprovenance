%!TEX root = ../main.tex

\section{Introduction}
\label{s:intro}

Data errors are a perrenial, expensive, and time consuming problem that afflict the vast majority of data-driven applications.
For example, errors in detail price data are estimated to cost US consumers \$2.5 billion each year~\cite{Fan2008}.
In aggregate, data errors are estimated to cost the US economy >\$600 billion per year~\cite{eckerson2002}.
Meanwhile, it is estimated that a significant fraction of time performing data analysis --
up to 80\%~\cite{kandel2012} -- is devoted to cleaning and
wrangling~\cite{kandel2011wrangler} the data into a structured and sufficiently clean form to use in downstream applications.

In response, both industry and academia has focused on data cleaning solutions to mitigate this problem.
ETL-type systems~\cite{systemt,informatica} focus on cleansing the data before it is loaded into the database
using a set of pre-defined rules; outlier and anomaly detection algorithms~\cite{} are used to identify errors
in the database after the data has been loaded; while recent approaches use downstream applications
such as interactive visualizations, application queries, or data-products to facilitate error detection and correction algorithms.
In each of these cases, the focus of error diagnosis and cleaning has been {\it data centric}, in the sense that
the process is meant to identify and directly fix data values.

Despite these efforts, they largely ignored an important source of data errors -- data modification {\it queries}.
Mistakes in such queries can introduce errors that spread throughout the database due to subsequent update queries.
Consider the following salary management example:

\begin{example}[Salary Update Error]\label{ex:telco}
A manager updates the tax rate to $30\%$ for high income employees that earn more than $\$87500$.
She submits this through a form in the salary accounting application, and due to keyboard slip, incorrectly types $\$85700$
for the income threshold.  Later queries that insert new paychecks, compute tax calculations,
aggregate department salaries end up propogating this error throughout other records in the database, leading to
employee dissatifaction.
\end{example}

In this example, by the time errors in the database have been detected, 
perhaps by employees that report their encorrect paystubs, it is difficult 
to both identify all of the other errors in the database, and to trace these errors back to the erroneous query to correct it.
This example can occur in any data processing systems where manual input is used to generate queries that modify the database --
this could be in the form of adhoc queries executed by a system administrator, or web-based forms that construct queries based
on user input, or even stored procedures that use human input to fill the parameter values.

Although existing data-centric cleaning techniques may help identify and correct these reported errors directly, 
this is suboptimal because it treats the symptom -- the errors in the current database state -- rather than the anomalous
queries that are the underlying cause.  In practice, only a subset of the paystub errors will be reported, thus fixing
the reported errors on a case-by-case basis will more likely than not obscure the root problem, making it more difficult to
find both the erroneous query and the other affected data.  
Furthermore, a data-centry approach must fix {\it all errors} -- a unfixed value such as the incorrect tax rate
may continue to introduce errors in the database due to future queries that 
e.g., insert incorrectly computed paychecks based on the unfixed tax rate.

For these reasons, traditional data cleaning approaches may be helpful for finding errors in the data, but are
not designed to diagnose causes of the errors when they are rooted in incorrect queries.
There has been some work, but not directly for this problme.  
Integrety constraints.
Certificate-based verification.
Structural sources of mistakes, but not identify the specific faulty query and how it is wrong.
In more restricted domains, data explanation finds predicates to exclude from qureies, but only work for a single, simple  aggregation query.

This problem is challenging because an error introduced by a query can be masked by, or propogated throughout the database,
by subsequent queries, This alludes to several factors that make even identifying problematic queries dificult:

\begin{enumerate}
\item {\bf Butterfly Effect: } 
An error in even a single query can affect a large number of records, as documented in several real-world
cases~\cite{Yates10, Grady13, sakalerrors}.  However, even if a single record is incorrect,
its value may be used as part of the \texttt{WHERE} or \texttt{SET} clauses of 
subsequent valid queries that introduce additional errors that are seemingly unrelated.

\item {\bf Partial Information:}  We often cannot rely on processes that can identify all possible
errors in the database -- for example, not all employees will complain about their incorrect paystubs.  
More likely, we only have access to a subset of the data errors, and must use them to extrapolate 
the queries that afflicted the entire set of errors.  A diagnostic tool that can reduce
the entire transaction log to the most likely candidate queries and propose fixes
is needed to make this process managable.


\item {\bf Multiple Types of Errors:} 
An erroneous query can cause multiple types of data errors.  For example, a record that should not exist may have been accidentally inserted, or conversely a record that 
should exist was unintentionally deleted.  Similarly, attribute values may be incorrectly updated,
updated when they should not have been, or not updated when they should have.  
Any combination of these error types may be present in the current state of the database,
and although they may not be obviously related to each other, they must be addressed in a holistic manner.  

\end{enumerate}


In this demo proposal, we outline the design of \sys, a framework that derives explanations
and repairs for discrepancies in relational data based on potential errors in
the queries that operated on the data, and describe how users will use \sys
in a demonstration scenario. 
In contrast to existing approaches in data
cleaning that aim to detect and correct errors in the data directly, the goal
of \sys is to identify  the problematic queries that introduced errors into the
database. These diagnoses both \emph{explain} how errors were introduced to a
dataset, and also lead to the identification of additional discrepancies in
the data that would have otherwise remained undetected.
Participants will be able to 
select from a number of transactional benchmarks to generate a query workload,
introduce errors into the queries that are executed,
and compare the fixes to the queries proposed by \sys as well as existing alternative algorithms such as decision trees.



% Poor data quality is a hard and persistent problem.  It is estimated to cost the US economy more than \$600 billion
% per year~\cite{eckerson2002} and erroneous price data in retail databases
% alone cost the US consumers \$2.5 billion each year~\cite{Fan2008}. Although data
% cleaning tools can purge many errors from a dataset before downstream 
% applications use the data, databases are constantly changing as applications
% and users execute queries that modify the data.
% Mistakes in these queries can introduce errors that spread through the database
% due to subsequent update queries; 
% by the time errors are detected, it is difficult to trace those errors back to the 
% erroneous query and correct it.
% Identifying and correcting errors in the data directly is suboptimal: it treats the symptom,
% rather than the underlying cause. Fixing the manifested data errors on a
% case-by-case basis often obscures the root of the problem and other data that may have been
% affected. Therefore, traditional data cleaning approaches are not well-suited
% for this setting: While they provide general-purpose tools to identify and
% rectify anomalies in the data, they are not designed to diagnose the causes of
% errors that are rooted in erroneous updates.
% Some data cleaning systems try to identify structural sources of
% mistakes~\cite{wang2015}, but they are unable to trace the source of
% the mistakes to particular faulty queries.
% 
% While improving data quality and correcting data errors has been an important
% focus for data management research, handling new errors, introduced during
% regular database interactions, has received little attention. Most work in
% this direction focuses on \emph{guarding against} erroneous updates. For
% example, integrity constraints~\cite{Khoussainova2006} reject some improper
% updates, but only if the data falls outside rigid, predefined ranges.
% Certificate-based verification~\cite{Chen2011} is less rigid, but it is
% impractical and non-scalable as it requires users to answer challenge
% questions before allowing the updates, and it is not applicable to updates
% initiated by applications.





% \begin{example}[Wireless discount policies]\label{ex:telco}
% 
% A large US-based wireless provider offers company discounts as incentives for
% corporate customers. There are different types of discounts (flat, percentage,
% fee-based), and their details are specific to corporate agreements. The large
% number of policies and complexities in their rules frequently cause policies
% to be set incorrectly, leading to errors in the application of discounts to
% customers' accounts.
% 
% Customers who notice billing errors contact the provider, but the call centers
% do not have the capacity or ability to investigate the complaints deeply. The
% standard course of action is to correct billing mistakes on a case-by-case
% basis for each complaint. As a result, unreported errors remain in the
% database for a long time, or they never get corrected, and their cause becomes
% harder to trace as the source of the errors is obscured.\footnote{This is a real-life scenario, provided to us by a popular US-based wireless provider.}
% 
% \end{example}

% \begin{example}[Tax bracket adjustment]\label{ex:taxes}
%     
% Tax brackets, which determine tax rates for different income levels, are
% often adjusted. Accounting firms implement these changes to their
% databases by appropriately updating the tax rates of their customers. Mistakes
% in these update queries (e.g., Figure~\ref{fig:example}) result in errors in
% the tax rates and computed taxes. 
% 
% \end{example}
% 
% 
% In this application, data errors are typically reported to a
% customer service department, which does not have the resources nor the
% capability to investigate the errors more broadly. Instead, errors are
% resolved on a case-by-case basis. In practice, customer service only deals
% with a small portion of the actual errors. Once these errors are resolved,
% there will still be a large number of incorrect records that were not
% identified. The goal of \sys is to identify the query or queries that caused
% the errors, propose corrections to those queries, and use the modified queries
% to identify other errors.
% 
% Diagnosing data errors stemming from incorrect updates is fundamentally
% challenging: the search space of possible mistakes and fixes is large, and the
% amount of information (number of known errors) may be limited. The problem has
% the following important characteristics that render traditional data cleaning
% methods unsuitable:



% \begin{description}[leftmargin=*, topsep=0mm, itemsep=0mm]
%     
%     \item[Obscurity.] Detection of the resulting errors in the data often
%     leads to partial fixes that further complicate the eventual diagnosis and
%     resolution of the problem. For example, a transaction implementing a
%     change in the state tax law updated tax rates using the wrong rate,
%     affecting a large number of consumers. This causes a large number of
%     complaints to a call center, but each customer agent usually fixes each
%     problem individually, which ends up obscuring the source of the problem.
%     
%     \item[Large impact.] Erroneous queries cause errors at a large scale. The
%     potential impact of the errors is high, as manifested in several
%     real-world cases~\cite{Yates10, Grady13, sakalerrors}. Further, errors
%     that remain undetected for a significant amount of time can instigate
%     additional errors, even through valid updates. This increases both their
%     impact, and their obscurity.
%     
%     \item[Systemic errors.] The errors created by bad queries are
%     \emph{systemic}: they have common characteristics, as they share the same
%     cause. The link between the resulting data errors is the query that
%     created them; cleaning techniques should leverage this connection to
%     diagnose and fix the problem. Diagnosing the cause of the errors, will
%     achieve systematic fixes that will correct all relevant errors, even if
%     they have not been explicitly identified.
%    
% \end{description}
% 
% \sys addresses these challenges by analyzing the queries that operated on a
% dataset in an efficient and scalable manner. More concretely, we make the
% following contributions:


% The goal of this paper is to design effective query
% diagnosis techniques and identify possible fixes for query errors. We
% model the problem assuming a log of update workloads over a database,
% and a set of complaints that identify errors in the final database
% state. We organize our contributions as follows:

% \ewu{really like this organization}

% \ewu{add: special case optimizations for single query case.}


% \begin{itemize}[leftmargin=*, topsep=0mm, itemsep=0mm]      
%     \item We formalize the problem of diagnosing a set of errors using log
%     histories of updates that operated on the data. Given a set of 
%     \emph{complaints} as representations of data discrepancies in the current
%     state of a dataset, \sys identifies the queries in the log that require the  minimal
%     amount of changes that would resolve all of the complaints (Section~\ref{sec:abstractions}).
%       
%     \item We provide an exact error-diagnosis solution through a non-trivial
%     transformation of the problem to a mixed integer linear program (MILP) that
%     encodes the data provenance of the erroneous tuples. Our approach employs state-of-the-art MILP solvers to identify
%     optimal diagnoses that are guaranteed to resolve all complaints without introducing new errors to the data
%     (Section~\ref{sec:sol}).
%     
%     \item We present several optimizations to our basic diagnostic
%     method, which reduce the problem size without affecting the
%     quality of the produced solutions. Further, we propose an
%     incremental repair method that targets the cases where the log
%     contains a single corrupted query (or the search focuses on a
%     single repair). This incremental analysis of the log allows us to
%     scale to very large datasets and large query logs. Further, we
%     show that our optimization techniques have the additional
%     advantage of tolerating incomplete information, such as unreported
%     errors (Section~\ref{sec:opt}).
% 
%     
%     % \item We extend our framework to also handle false positives: complaints
%     % that mistakenly identify data as erroneous. We define the notion of
%     % complaint \emph{density}, which is a query-driven measure of closeness of
%     % a complaint to other complaints. The main intuition of our approach is
%     % that complaints of low density are likely false positives and thus can be
%     % safely ignored (Section~\ref{???}).
%     
%     \item We experimentally evaluate the effectiveness and efficiency of our
%     methods against real-world and synthetic datasets and query logs. 
%     (Section~\ref{sec:experiments}). 
% \end{itemize}
