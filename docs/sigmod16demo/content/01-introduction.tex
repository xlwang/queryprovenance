%!TEX root = ../main.tex

\section{Introduction}
\label{s:intro}

data quality and errors are an issue.
examples are why it's an issue.

Exsiting techniques are data centric.
Detection, prevention, filtering, cleaning solutions.
ETL.

However, this is a limittion, because they ignore a common source of errors -- user generated qureies.
Adhoc queries.
Even stored procedures are suseptible through web based interfaces that fill in procedure parameters.  Think SAP

Consider an example representative of a large wireless carrier: call center example.

Consider another interactive data analysis example.  User executes queries that update different portions of her database.  Then after visualizing, finds problematic data.
Although she knows the correct values, she ran multiple queries.  Which one does she need to fix?

This source is challenging because an error introduced by a query can be masked by subsequent queries,
or propagated throughout the database by subsequent queries.  This alludes to several factors that make
even identifying problematic queries dificult:

\begin{enumerate}
\item {\bf butterfly effect: } An error in a handful of records in the past can cascade and affect many records that are seemingly unrelated

\item {\bf partial information:}  not all customers will call in to complain and complaints may not be complete

\item {\bf multiple types of errors:} Record was modified/inserted/deleted incorrectly
Record should have been changed, was not changed

\end{enumerate}

Solving this would be good.  It can identify systemitc sources of errors. 


\begin{figure}[t]
    \centering
        \includegraphics[scale=0.35]{figures/logs}
    \caption{}
    \label{fig:logs}
\end{figure}




In this demo proposal, we outline the design of \sys, a framework that derives explanations
and repairs for discrepancies in relational data based on potential errors in
the queries that operated on the data, and describe how users will use \sys
in a demonstration scenario. 
In contrast to existing approaches in data
cleaning that aim to detect and correct errors in the data directly, the goal
of \sys is to identify errors in the queries that introduced errors in the
data. These diagnoses both \emph{explain} how errors were introduced to a
dataset, and also lead to the identification of additional discrepancies in
the data that would have otherwise remained undetected.
Participants will be able to select from a number of 
transactional benchmarks, introduce errors into the queries that are executed,
and compare the fixes to the queries proposed by \sys as well as existing alternative algorithms
such as decision trees.



% Poor data quality is a hard and persistent problem.  It is estimated to cost the US economy more than \$600 billion
% per year~\cite{eckerson2002} and erroneous price data in retail databases
% alone cost the US consumers \$2.5 billion each year~\cite{Fan2008}. Although data
% cleaning tools can purge many errors from a dataset before downstream 
% applications use the data, databases are constantly changing as applications
% and users execute queries that modify the data.
% Mistakes in these queries can introduce errors that spread through the database
% due to subsequent update queries; 
% by the time errors are detected, it is difficult to trace those errors back to the 
% erroneous query and correct it.
% Identifying and correcting errors in the data directly is suboptimal: it treats the symptom,
% rather than the underlying cause. Fixing the manifested data errors on a
% case-by-case basis often obscures the root of the problem and other data that may have been
% affected. Therefore, traditional data cleaning approaches are not well-suited
% for this setting: While they provide general-purpose tools to identify and
% rectify anomalies in the data, they are not designed to diagnose the causes of
% errors that are rooted in erroneous updates.
% Some data cleaning systems try to identify structural sources of
% mistakes~\cite{wang2015}, but they are unable to trace the source of
% the mistakes to particular faulty queries.
% 
% While improving data quality and correcting data errors has been an important
% focus for data management research, handling new errors, introduced during
% regular database interactions, has received little attention. Most work in
% this direction focuses on \emph{guarding against} erroneous updates. For
% example, integrity constraints~\cite{Khoussainova2006} reject some improper
% updates, but only if the data falls outside rigid, predefined ranges.
% Certificate-based verification~\cite{Chen2011} is less rigid, but it is
% impractical and non-scalable as it requires users to answer challenge
% questions before allowing the updates, and it is not applicable to updates
% initiated by applications.





% \begin{example}[Wireless discount policies]\label{ex:telco}
% 
% A large US-based wireless provider offers company discounts as incentives for
% corporate customers. There are different types of discounts (flat, percentage,
% fee-based), and their details are specific to corporate agreements. The large
% number of policies and complexities in their rules frequently cause policies
% to be set incorrectly, leading to errors in the application of discounts to
% customers' accounts.
% 
% Customers who notice billing errors contact the provider, but the call centers
% do not have the capacity or ability to investigate the complaints deeply. The
% standard course of action is to correct billing mistakes on a case-by-case
% basis for each complaint. As a result, unreported errors remain in the
% database for a long time, or they never get corrected, and their cause becomes
% harder to trace as the source of the errors is obscured.\footnote{This is a real-life scenario, provided to us by a popular US-based wireless provider.}
% 
% \end{example}

\begin{example}[Tax bracket adjustment]\label{ex:taxes}
    
Tax brackets, which determine tax rates for different income levels, are
often adjusted. Accounting firms implement these changes to their
databases by appropriately updating the tax rates of their customers. Mistakes
in these update queries (e.g., Figure~\ref{fig:example}) result in errors in
the tax rates and computed taxes. 

\end{example}


In this application, data errors are typically reported to a
customer service department, which does not have the resources nor the
capability to investigate the errors more broadly. Instead, errors are
resolved on a case-by-case basis. In practice, customer service only deals
with a small portion of the actual errors. Once these errors are resolved,
there will still be a large number of incorrect records that were not
identified. The goal of \sys is to identify the query or queries that caused
the errors, propose corrections to those queries, and use the modified queries
to identify other errors.

Diagnosing data errors stemming from incorrect updates is fundamentally
challenging: the search space of possible mistakes and fixes is large, and the
amount of information (number of known errors) may be limited. The problem has
the following important characteristics that render traditional data cleaning
methods unsuitable:



% \begin{description}[leftmargin=*, topsep=0mm, itemsep=0mm]
%     
%     \item[Obscurity.] Detection of the resulting errors in the data often
%     leads to partial fixes that further complicate the eventual diagnosis and
%     resolution of the problem. For example, a transaction implementing a
%     change in the state tax law updated tax rates using the wrong rate,
%     affecting a large number of consumers. This causes a large number of
%     complaints to a call center, but each customer agent usually fixes each
%     problem individually, which ends up obscuring the source of the problem.
%     
%     \item[Large impact.] Erroneous queries cause errors at a large scale. The
%     potential impact of the errors is high, as manifested in several
%     real-world cases~\cite{Yates10, Grady13, sakalerrors}. Further, errors
%     that remain undetected for a significant amount of time can instigate
%     additional errors, even through valid updates. This increases both their
%     impact, and their obscurity.
%     
%     \item[Systemic errors.] The errors created by bad queries are
%     \emph{systemic}: they have common characteristics, as they share the same
%     cause. The link between the resulting data errors is the query that
%     created them; cleaning techniques should leverage this connection to
%     diagnose and fix the problem. Diagnosing the cause of the errors, will
%     achieve systematic fixes that will correct all relevant errors, even if
%     they have not been explicitly identified.
%    
% \end{description}
% 
% \sys addresses these challenges by analyzing the queries that operated on a
% dataset in an efficient and scalable manner. More concretely, we make the
% following contributions:


% The goal of this paper is to design effective query
% diagnosis techniques and identify possible fixes for query errors. We
% model the problem assuming a log of update workloads over a database,
% and a set of complaints that identify errors in the final database
% state. We organize our contributions as follows:

% \ewu{really like this organization}

% \ewu{add: special case optimizations for single query case.}


% \begin{itemize}[leftmargin=*, topsep=0mm, itemsep=0mm]      
%     \item We formalize the problem of diagnosing a set of errors using log
%     histories of updates that operated on the data. Given a set of 
%     \emph{complaints} as representations of data discrepancies in the current
%     state of a dataset, \sys identifies the queries in the log that require the  minimal
%     amount of changes that would resolve all of the complaints (Section~\ref{sec:abstractions}).
%       
%     \item We provide an exact error-diagnosis solution through a non-trivial
%     transformation of the problem to a mixed integer linear program (MILP) that
%     encodes the data provenance of the erroneous tuples. Our approach employs state-of-the-art MILP solvers to identify
%     optimal diagnoses that are guaranteed to resolve all complaints without introducing new errors to the data
%     (Section~\ref{sec:sol}).
%     
%     \item We present several optimizations to our basic diagnostic
%     method, which reduce the problem size without affecting the
%     quality of the produced solutions. Further, we propose an
%     incremental repair method that targets the cases where the log
%     contains a single corrupted query (or the search focuses on a
%     single repair). This incremental analysis of the log allows us to
%     scale to very large datasets and large query logs. Further, we
%     show that our optimization techniques have the additional
%     advantage of tolerating incomplete information, such as unreported
%     errors (Section~\ref{sec:opt}).
% 
%     
%     % \item We extend our framework to also handle false positives: complaints
%     % that mistakenly identify data as erroneous. We define the notion of
%     % complaint \emph{density}, which is a query-driven measure of closeness of
%     % a complaint to other complaints. The main intuition of our approach is
%     % that complaints of low density are likely false positives and thus can be
%     % safely ignored (Section~\ref{???}).
%     
%     \item We experimentally evaluate the effectiveness and efficiency of our
%     methods against real-world and synthetic datasets and query logs. 
%     (Section~\ref{sec:experiments}). 
% \end{itemize}
