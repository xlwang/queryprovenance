\documentclass{article}
\usepackage{amsmath}
\usepackage{newtxmath}


\begin{document}

The problem of the one-pass algorithm is that the objective targets overall
minimum changes, and leads the algorithm to sometimes pick multi-query fixes.
We can add some additional constraints to the one-pass program to prevent multi-query fixes.

The intuition is to introduce a new \emph{integer} variable $y_i$ for each
query. The value of the variable $y_i$ will represent whether the query was changed by the solver. For example, take the following query:

\[
\text{SET } a=u_0^* \text{ WHERE } b_1\leq u_1^* \text{ AND } b_2\geq u_2^*
\]

In the above query, $u_0^*$, $u_1^*$, and $u_2^*$ are constants, but they are replaced with
variables $u_0$, $u_1$, and $u_2$ during our linearization. If $u_1$ is assigned by
the solver a value different than $u_1^*$, that means that the corresponding predicate in the WHERE clause  was
modified.  Basically, if any of these variables are assigned by the solver to different values than their original ones, then the query was ``fixed''.

So, we can set the variable $y$ that corresponds to this query to:
\[
y= (u_0= u_0^* \text{ AND } u_1= u_1^* \text{ AND } u_2=u_2^*)
\]

This will give $y=1$ if no variables in the query where assigned values
different than their original ones, and we will get $y=0$ if any of them
changed.

So, if we have one $y$ variable for each query, we can set the constraint:

\[
\sum_i(1-y_i)\leq 1
\]

This will enforce the condition that at most one $y$ can be 0, which means at most one query will have been modified. 

We can generalize this to allow fixes of up to $k$ queries, by setting the constraint to:

\[
\sum_i(1-y_i)\leq k
\]


\end{document}