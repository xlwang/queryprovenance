\newpage
\appendix

\section{A Learning-based Approach}
\label{sec:heuristic}
  
A drawback of the MILP approach is that the generated models grow with the 
size of the database and query log.
However, we argue that the encoded information is necessary in order to generate a sufficient set of constraints that result in a good repair.
In this section, we examine an alternative, simpler, decision tree-based approach called \dt. 
We show that even in a simple case of a single query log and a complete complaint set, it is expected to perform poorly.
We will first describe how to model the repair process using a decision tree,
and then we will present and discuss experimental results that illustrate its limitations.

\subsection{Modeling Repairs with Decision Trees}

Rule-based learners are used in classification tasks to generate a set of rules, or conjunctive predicates that best classify a group of labeled tuples.
The rules are non-overlapping, and each is associated with a label---a tuple that matches a given rule is assigned the corresponding label.
These rules exhibit a natural parallel with SQL \texttt{WHERE} clauses, 
which can be viewed as labeling selected tuples with a positive label and rejected tuples with a negative label.
Similarly, the structure of the rules is identical to those that \sys is designed to repair.
Thus, given the database tuples labeled to describe the errors, we may use a rule-based learner to
generate the most appropriate \texttt{WHERE} clause.
We focus our attention on rule-based learners;
specifically, we experiment with the C4.5~\cite{quinlan1987} decision tree learner, which is an 
exemplar of rule-based learners.

A core limitation of this classification-based approach is that there is no means to 
repair \texttt{SET} clauses, which modify data values rather than simply label them.
We resolve this with a two step approach.
We first use the decision tree to generate a repair for the
\texttt{WHERE} clause, and then use the modified query to identify repairs for the \texttt{SET} clause.
The need for this two step procedure limits this approach to encoding and repairing at most one query
at a time.

\noindent
\textbf{Repairing the WHERE Clause:}
The \texttt{WHERE} clause of an update query is equivalent to a
rule-based binary classifier that splits tuples into two groups:
(1)~tuples that satisfy the conditions in the \texttt{WHERE} clause
and (2)~tuples that do not. A mistake in a query predicate can 
cause a subset of the tuples to be misclassified, and in turn,
translate into data errors. 
Therefore, repairing the complaints corresponds to repairing the imprecise classification. 

The repair works as follows: For an incorrect query $q$, let
$D_0$ be the database state before $q$, and $D_1^*$ the \emph{correct}
database state that should have been the result after $q$, if $q$ were correct.
We use each tuple $t \in D_0$ as an element in the input training data
for the classifier where the values (of each attribute) of $t$ define
the feature vector and the label for $t$:
	\[
    label(t)= 
    \begin{cases}
    true & \textrm{if\ }D_0.t \neq D_1^*.t\\
    false              & \text{otherwise}
    \end{cases}
\]
The \texttt{true} rules generated by the decision tree trained on this labeled dataset 
forms a disjunction of rules that constitute the repaired \texttt{WHERE} clause.


\noindent
\textbf{Repairing the SET Clause:}
The \texttt{WHERE} clause repair proposed by the classifier may not completely repair 
the complaints if there was also an error in the \texttt{SET} clause. 
In this case, we execute a second repair step.
\begin{figure}[!htb]
\centering
  \begin{subfigure} [t]{.75\columnwidth}
  \includegraphics[width = \columnwidth]{figures/heuristictime}
  \caption{Comparison on Performance.}
  \label{f:heuristic_time} 
  \end{subfigure}\\

  \begin{subfigure} [t]{.75\columnwidth}
  \includegraphics[width = \columnwidth]{figures/heuristicacc}
  \caption{Comparison on Accuracy.}
  \label{f:heuristic_acc} 
  \end{subfigure}
 \caption{\dt compared with \sys}
 \label{f:heuristic}
\end{figure}

We model the errors as a simple linear system of equations: 
each expression in the \texttt{SET} clause is translated into a
linear equation in the same fashion as described in Section~\ref{sec:sol}.
Directly solving the system of equations for the undetermined variables 
will generate the desired repair for the \texttt{SET} expression.



\subsection{Experimental Results}

To illustrate these shortcomings, we compare \dt with \sys using a simplified version of the setup from Section~\ref{sec:experiments} that favors \dt.
We restrict the query log to contain a single query that is corrupted, use a complete complaint set  and vary the database size.
We use the following query template, where all \texttt{SET} clauses assign the attributes to constants,
and the \texttt{WHERE} clauses consist of range predicates:

{\scriptsize
\begin{verbatim}
  UPDATE table
  SET  (a_i=?), ...
  WHERE a_j in [?,?+r] AND ...
\end{verbatim}
}

Figure~\ref{f:heuristic_time} shows that although the runtime performance of \dt is better than \sys by small a constant factor ($\sim 2.5 \times$),
both runtimes degrade exponentially.
In addition, the \dt repairs are effectively unusable as their accuracy is low: the F1-score starts at $0.5$ and rapidly degrades towards $0$.
From these empirical results, we find that \dt generates low-quality repairs even under the simplest conditions---an approach
that applies \dt over more queries is expected to have little hope of succeeding.



There are three important reasons why \dt, and any approach that focuses on a single query at a 
time\footnote{Although our incremental approach tries to generate a repair for a single
query at a time, it encodes all subsequent queries in the log.}, will not perform well.

\begin{itemize}[itemsep=1pt, leftmargin=5mm]
    
\item \textbf{Single Query Limitation: }
In principle, one could attempt to apply this technique to the
entire log one query at a time, starting from the most recent query.
Even ignoring the low repair accuracy shown in Figure~\ref{f:heuristic_acc},
this approach is infeasible.
Consider that we generate a labeled training dataset to repair $q_i$ 
using the query's input and output database states $D_{i-1}$ and $D_i^*$.
Note that $D_i^*$ is the theoretically \emph{correct} database state assuming no errors in the query log.
We would need to derive $D_i^*$ by applying the complaint set to $D_n$ to create $D_n^*$, and roll back the database state.
Unfortunately, \texttt{UPDATE} queries are commonly surjective such
that their inverses are ambiguous, which means that it is often
impossible to derive $D_i^*$. In contrast, the incremental version of
\sys can bypass this problem by encoding subsequent queries in the log
in a MILP representation.


\item \textbf{Structurally Different \texttt{WHERE} Clause Results: } 
The basic classifier approach simply learns a set of rules to minimize
classification error, and can derive a clause whose struture is arbitrarily 
different from the original query's \texttt{WHERE} clause.
Although it may be possible to incorporate a distance measure as part of the decision tree
splitting criteria, it is likely to be a heuristic with no guarantees.


\item \textbf{High Selectivity, Low Precision: }
Classifiers try to avoid overfitting by balancing the complexity of the rules with classification accuracy.
This is problematic for highly selective queries (e.g., primary key updates), because the classifier
may simply ignore the single incorrect record and generate a rule such as \texttt{FALSE}.
In fact, this form of severely imbalanced data continues to be a challenge for most major classification algorithms~\cite{he2009learning, galar2012review}. 
Thus, we believe that alternative classification algorithms would not improve on these results. 
Compound with the fact that many workloads are primarily composed of
key update queries~\cite{oltpbench} this issue severely limits the
applicability of learning-based approaches.

\end{itemize}








\section{MILP Solver Performance}
\label{app:solvtime}
\subsection{Factors Relevant to Solver Performance}

\ewu{reference other constraint performance papers}
\xlw{
In this section, we study factors that influence the MILP solver solving time. In this paper, we
use IBM CPLEX~\cite{cplex2014v12} as a black box to solve the constructed MILP problems. Similar to many solver performance studies~\cite{atamturk2005integer, meindl2012analysis, gearhart2013comparison}, 
we first study the solver behavior and the connection between solver performance (on \texttt{UPDATE} workload) and multiple
 factors: the number of constraints, undetermined variables, and complaints.  We create \texttt{UPDATE}-only workloads using our synthetic data generator with 300 queries where the \texttt{SET} clauses assign the attribute to constant and \texttt{WHERE} clause select tuples by range predicates: }

{\scriptsize
\begin{verbatim}
  UPDATE table
  SET  (a_i=?), ...
  WHERE a_j in [?,?+r] AND ...
\end{verbatim}
}
\xlw{We further select two query indexes, $q_{50}$ and $q_{200}$, to create the dirty query log. }

\ewu{XL: more detail on how these experiments are run.  See my edits in previous section for example}
  
  \begin{figure*}[t]
  \centering
      \begin{subfigure} [t]{.3\textwidth}
    \includegraphics[width = .99\columnwidth]{figures/num_cons_time}
    \vspace*{-.25in}
    \caption{\# of constraints vs. solver solving time.}
    \vspace*{-.1in}
    \label{f:cons_vs_time} 
    \end{subfigure}
    \begin{subfigure} [t]{.3\textwidth}
    \includegraphics[width = .99\columnwidth]{figures/num_vars_time}
    \vspace*{-.25in}
    \caption{\# of undetermined variables vs. solver solving time.}
    \vspace*{-.1in}
    \label{f:var_vs_time} 
    \end{subfigure}
    \begin{subfigure} [t]{.3\textwidth}
    \includegraphics[width = .99\columnwidth]{figures/num_compl_time}
    \vspace*{-.25in}
    \caption{\# of compl. vs. solver solving time.}
    \vspace*{-.1in}
    \label{f:compl_vs_time} 
    \end{subfigure} 
    \iffalse
    \begin{subfigure} [t]{.3\textwidth}
    \includegraphics[width = .99\columnwidth]{figures/num_compl_cons}
    \vspace*{-.25in}
    \caption{\# of constraints vs.\# of compl.}
    \vspace*{-.1in}
    \label{f:compl_vs_cons} 
    \end{subfigure}
    \begin{subfigure} [t]{.3\textwidth}
    \includegraphics[width = .99\columnwidth]{figures/num_compl_var}
    \vspace*{-.25in}
    \caption{\# of undetermined variables vs. \# of compl.}
    \vspace*{-.1in}
    \label{f:compl_vs_var}
    \end{subfigure}
    \fi
   \caption{Solver solving time grows with all three factors: number of constraints, number of undermined variables and number of complaints. }
   \vspace*{-.1in}
   \label{f:soltime}
  \end{figure*}

  \begin{figure*}[ht]
  \centering
  \begin{subfigure} [t]{.3\textwidth}
    \includegraphics[width = .99\columnwidth]{figures/indelup_acc_idx}
    \vspace*{-.25in}
    \caption{Query types vs. rate of ``true'' fix}
    \vspace*{-.1in}
    \label{f:qtype_acc} 
    \end{subfigure}
    \begin{subfigure} [t]{.3\textwidth}
    \includegraphics[width = .99\columnwidth]{figures/dbsize_acc_idx}
    \vspace*{-.25in}
    \caption{Database sizes vs. rate of ``true'' fix}
    \vspace*{-.1in}
    \label{f:size_acc} 
    \end{subfigure}
    \begin{subfigure} [t]{.3\textwidth}
    \includegraphics[width = .99\columnwidth]{figures/noise_fn_acc_idx}
    \vspace*{-.25in}
    \caption{False negatives vs. rate of ``true'' fix}
    \vspace*{-.1in}
    \label{f:fn_acc} 
    \end{subfigure}
   \caption{Rate of ``true'' fix for \sys varies on different types of workloads and false negative rate but remain 
   stable for with increasing number of tuples on \texttt{UPDATE} workload.}
   \vspace*{-.1in}
   \label{fig:truerate}
  \end{figure*}

\smallskip
\emph{Number of Constraints: } We first study the relation between solver solving time and number of constraints and observe that solver solving time is roughly positive proportional, with tiny fluctuation, to the number of constraints in both corrupt indexes (Figure~\ref{f:cons_vs_time}). This is expected because increasing the number of constraints by large scale usually accompany with significantly greater amount of variables in the MILP problem, which lead to harder MILP problems by natural. However, MILP problems with slightly larger amount of constraints sometimes could be even easier to solve since the additional constraints help pruning the searching space for similar amount of variables.

\smallskip
\emph{Number of Undetermined Variables: } In our experiments we found that increasing the number of constraints by increasing the database, query log, or complaint set sizes aversely affects the solver performance,
  solver performance has been shown to \emph{improve} as constraints that effectively reduce the problem space are added to the problem.  
  Thus, we now focus on the effects on the number of undetermined variables, which strictly increases the problem space.
  We keep the number of constraints fixed, however increase the number of undetermined variables by \ewu{XL fill in}. \xlw{need to rerun exp to get required numbers. We need to gradually increase the batch size to achieve this. }

\smallskip
\emph{Number of Complaints: } In Figure~\ref{f:soltime}(b) we compare the solver solving 
time with increasing number of complaints. As illustrated in Section~\ref{sec:sol}, 
the size of number of constraints constructed MILP problem grows linearly with the number of complaints. Thus, solver solving time grows with the number of complaints. 
\ewu{Can you plot the number of complaints vs num constraints and num undetermined variables?  That's what really matters right? Same for num queries}

\xlw{\sout{
\smallskip
\emph{Number of Queries: }
\ewu{Can cut, since overlaps with experiments in main paper}
We finally compare solver solving time with the number of queries encoded in the MILP problem and find the solver solving time grows exponentially with the number of queries. This again is due to the way we construct the MILP problem: by increasing the number of queries, more variables are introduced, which increase the hardness in solving the problem. }}

\subsection{Factors in Synthetic Data Generator}
In this section, we further studies the parameters in synthetic generator that control the number of complaints in synthetic datasets. 






\section{Repairing the Correct Query}
\label{app:index}

\ewu{
The experimental section (Section~\ref{sec:experiments}) primarily focused on performance and accuracy measures of the repairs, and for space constraints, ignored 
whether the proposed repair was actually the correct query (it is possible we fix an incorrect query in a way that resolves the complaint set).
Although we found \ewu{describe if we picked the right queries in tnhe paper experiments}, we now turn our attention to better understand 
the conditions when we may fix the incorrect query.   We find that under most conditions, we can expect to identify and repair the correct query.
}

\ewu{how many runs are you plotting? May be good to have more than 20 runs, or are you re-using existing experimental runs from the paper?}

\ewu{
To do this we \ewu{Describe experimental setup}.  We use two evaluation metrics to summarize our results over \ewu{X} runs.
The first $p_{correct}$ computes the ratio between the number of runs where \sys correctly repairs the corrupt query and where \sys
repairs the wrong query.  The second metric $topk$ measures the number of repairs that \sys proposes before repairing the true corrupt query.
This metric is useful to quantify the extent that \sys's repairs are false positives (though they may correctly resolve the complaint set).
A smaller $topk$ value suggests that \sys is still effective at providing repair \emph{recommendations} as part of a diagnostic interface.
}


\textbf{Rate of ``true'' fix vs. query types: } In Figure~\ref{fig:truerate}(a), we first evaluate the ``true'' fix rate on three types of workloads: \sys maintains constant and 
\ewu{what does nearly perfect mean?  Quantify how many we missed! e.g., 1 out of 1000 runs}
nearly perfect ``true'' fix rate for \texttt{UPDATE} and \texttt{INSERT} workloads but unstable and much lower rate for \texttt{DELETE} workload. This is expected since more information of the actual incorrectness get maintained through \texttt{UPDATE} queries by the updated tuple values. However, such information get lost and obscured for \texttt{DELETE} queries as those deleted tuples no longer exist in the database. As a result, actual incorrect queries in \texttt{UPDATE} workloads are much easier to identify than those in \texttt{DELETE} workloads. \texttt{INSERT} workloads are easy to fix by natural since each \texttt{INSERT} query only touches one tuple. 
\ewu{talk about negative vs positive examples, why DELETE is harder.  Basically negative examples don't help a learner/solver correctly generalize, so you it is easy to overfit with a delet query that is similar enough.  Positive examples generate actual constraints that help reduce the set of queries that can be modified to repair the complaints. }

\textbf{top k}: TO FILL IN


\textbf{Rate of ``true'' fix vs. database sizes: } We next narrow down to \texttt{UPDATE} workload and evaluate \sys's behavior with increasing amount of tuples in the database. We observe nearly perfect rate on different corrupt indexes in all database sizes (Figure~\ref{fig:truerate}(b)). 

\textbf{Rate of ``true'' fix vs. false negatives: } We finally evaluate \sys on \texttt{UPDATE} workload with increasing false negative rates. Similar to the quality drop in Figure~\ref{f:falsenegative_acc}, we observe the ``true'' fix rate also suffers from less submitted complaints when errors happen early in the query log. However, \sys maintain nearly perfect ``true'' fix rate for more recent errors (corrupt index is 300 and 200). 


\section{Effect of Index of Corrupted Query}
\label{app:qidx}


A key parameter for our experiments is the location of the corrupted query ($idx$).  
This parameter determines the number of queries \sys must consider when searching for a fix,
and affects the size of the complaint set.  
Both of these characteristics directly impact \sys's 
runtime performance. For this reason, it is undesirable to randomly pick and corrupt queries
throughout the query log, as the performance and accuracy results may not be comparable. 
To better understand the relationship between $idx$ and the size of the complaint set, we ran
simulations using a database with $20$ attributes, and a query log of size $1000$ containing
either all $set = const$ or $set = rel$ \texttt{UPDATE} queries.
We varied  $idx$ uniformly throughout the query log, and additionally varied
the skew $s$ and range $r$ parameters to study how they affect the size of the complaint sets.


  \begin{figure}[h]
  \centering
  \includegraphics[width = 3.5in]{figures/qidxsimulation/qidx_v_ncomplaints_20attrs_const}
  \caption{Query index vs complaint set size for $set = const$.}
  \label{f:qidx_v_ncomplaints_const} 
  \end{figure}


Figure~\ref{f:qidx_v_ncomplaints_const} plots a representative set of parameters.  We plot one point
for each corrupted query index that results in a complaint set with at least one complaint. 
These results highlight several interesting trends.  When queries do not overlap ($r = 1$, leftmost column),
the size of the complaint sets are relatively small, and their frequency is constant across the possible query indices.
However as the possibility of overlap increases (e.g., $r$ increases), more recent queries are more likely to result in
very large complaint sets (at times the size of the database).   
This effect is a symptom of the fact that queries with large ranges will set groups of tuples to the same value,
and over time, skew the distribution of tuple values to a small number of possible values.
Thus, more recent corruptions that affect a large cluster of similar tuples will result in a large complaint set.
We find that increasing the skew parameter also exacerbates this effect.  
In addition, high skew increases the likelihood that queries will share the same \texttt{WHERE} and \texttt{SET} clause 
attributes as a corrupted query, thus overwriting the error introduced by the corrupted query.  
This is why the frequency of non-empty complaint sets decreases significantly as $s$ increases.


\begin{figure}[h]
\centering
\includegraphics[width = 3.5in]{figures/qidxsimulation/qidx_v_ncomplaints_20attrs_rel}
\caption{Query index vs complaint set size for $set = rel$.}
\label{f:qidx_v_ncomplaints_rel} 
\end{figure}

In contrast to $set=const$ queries, Figure~\ref{f:qidx_v_ncomplaints_rel} executes the 
same experiment using $set=rel$ queries.  In this setting, we find that the trend is
reversed, and older corruptions tend to result in larger complaint sets.  This is because,
subsequent \texttt{UPDATE} queries increment or decrement the attribute value, rather than
overwriting it with a constant value.  The clustering of data values due to query overlap
then increases the number of other tuples affected.


\ewu{summarize findings and implications to experiments here.}
not all corruptions result in complaint sets.
In constant SET clause workloads, larger complaints sets are more likely to
result from more recent corrupted queries -- particularly if the queries are range updates or
the updated attributes are skewed.
For this reason, our experiments corrupt the query log at six positions 
$idx \in \{0, 25, 50, 100, 200, 250\}$ , relative 
to the most recent query (e.g., the most recent query, the $25^th$ most recent query, and so on).

% \begin{figure}[h]
% \centering
% \includegraphics[width = 3.5in]{figures/qidxsimulation/numinrange}
% \caption{.}
% \label{f:numinrange} 
% \end{figure}


As we observed from Figure~\ref{f:multiquery}, \milpall maintains high accuracy when errors
happen more recent, however it does not scale when the error locate further from the most
recent query. \milptuple scales better than \milpall, but ignoring tuples not 
in the complaint set apparently hurts the precision. \milptuplestopearly run times faster
than \milpall and \milptuple, however the aggressive strategy greatly reduce the 
precision. In the end, \milpadvtuple significantly improves the precision with very limited
time cost compare to \milptuple. \\
Based on these observations, we only include the performance of \milpadvtuple and \milpadvall
in the rest of the experiments. 

